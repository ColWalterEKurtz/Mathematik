\begin{exercise}
      {ID-44e961ad7fb10c7eadf2439d5789c4034ab8cac3}
      {Kegel in einer Kugel}
  \ifproblem\problem
    \ifthenelse{\isundefined{\linecalc}}{\newlength{\linecalc}}{\relax}%
    \setlength{\linecalc}{\linewidth}%
    \addtolength{\linecalc}{-68mm}%
    \begin{minipage}[T]{\linecalc}
      In eine Kugel mit dem Radius $R$ ist ein gerader Kreiskegel mit dem Radius $r$
      und der Höhe $h$ einbeschrieben.
      \begin{enumerate}[a)]
        \item Ermitteln Sie den Term $V(\varphi)$, der das Kegelvolumen in
              Abhängigkeit von der Größe $\varphi$ des Winkels zwischen
              der Kegelhöhe und einer Mantellinie darstellt.
              Wie groß muss dieser Winkel sein, damit das Kegelvolumen
              maximal wird?
        \item Drücken Sie die Kegelvolumenfunktion durch andere Variablen aus
              und diskutieren Sie eine dieser Funktionen.
      \end{enumerate}
    \end{minipage}\hfill
    \begin{minipage}[T]{61mm}
      \raggedleft
      \begin{tikzpicture}
        \coordinate (M)   at (  0.0000,   0.0000);
        \coordinate (m)   at (  0.0000,  -1.9800);
        \coordinate (A)   at ( -2.2538,  -1.9800);
        \coordinate (B)   at (  2.2538,  -1.9800);
        \coordinate (C)   at (  0.0000,   3.0000);
        \coordinate (phi) at ( -0.1687,   2.2180);
        \draw (M) circle[radius=3cm];
        \draw (m) ellipse[x radius=2.2538, y radius=0.3300];
        \fill (m) circle[radius=1pt];
        \draw (C) -- (A);
        \draw (C) -- (B);
        \draw (C) -- node[right]{{\small$h$}} (m);
        \begin{scope}
          \clip (A) -- (m) -- (C) -- cycle;
          \draw (C) circle[radius=1.1200];
          \node at (phi) {{\small$\varphi$}};
        \end{scope}
        \draw (m) -- node[above]{{\small$r$}} ++(300:2.2538 and 0.3300);
      \end{tikzpicture}
    \end{minipage}
  \fi
  %\ifoutline\outline
  %\fi
  %\ifoutcome\outcome
  %\fi
\end{exercise}

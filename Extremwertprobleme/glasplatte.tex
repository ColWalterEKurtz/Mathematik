% 2021-11-16
\begin{exercise}
      {ID-57f45e9ca6ce3bccc30110ee5b44d510f1fc632c}
      {Glasplatte}
  \ifproblem\problem\par
    % <PROBLEM>
      \ifthenelse{\isundefined{\linecalc}}{\newlength{\linecalc}}{\relax}%
      \setlength{\linecalc}{\linewidth}%
      \addtolength{\linecalc}{-72mm}%
      \begin{minipage}[b]{\linecalc}
        Von einer rechteckigen Glasplatte mit den Seitenlängen $a$ und $b$
        ist an einer Ecke ein Stück von der Form eines rechtwinkligen
        Dreiecks abgebrochen.
        Gegeben sind die Maße $a=\simeter{1}$, $b=\sicm{60}$, $c=\sicm{10}$ und
        $d=\sicm{4}$.
        Wie sind $e$ und $f$ zu wählen, damit aus dem Rest eine
        möglichst große, rechteckige Platte geschnitten werden kann?
      \end{minipage}\hspace*{\fill}%
      \begin{minipage}[b]{65mm}
        \raggedleft
        \raisebox{\baselineskip}[7\baselineskip][0pt]{%
        \begin{tikzpicture}[scale=0.05]
          % um die Punkte verschieben zu koennen
          \newcommand{\pdef}
          {
            \coordinate (A) at (  0.0000,   0.0000);
            \coordinate (B) at (100.0000,   0.0000);
            \coordinate (C) at (100.0000,  60.0000);
            \coordinate (D) at (  0.0000,  60.0000);
            \coordinate (E) at (100.0000,  35.0000);
            \coordinate (F) at ( 60.0000,  60.0000);
            \coordinate (G) at ( 80.0000,   0.0000);
            \coordinate (H) at ( 80.0000,  47.5000);
            \coordinate (I) at (  0.0000,  47.5000);
          }
          % die unverschobene Variante
          \pdef
          % abgebrochene Ecke
          \filldraw[fill=black!25!white] (E) -- (C) -- (F);
          % Reststueck
          \draw (A) -- (B) -- (E) -- (F) -- (D) -- cycle;
          % inneres Rechteck
          \draw (G) -- node[left]{{\small$f$}} (H)
                    -- node[below]{{\small$e$}}  (I);
          % Beschriftung oben
          \begin{scope}[yshift=200]
            \pdef
            \draw[|<->|, >=stealth] (D) -- node[above]{{\small$a-c$}} (F);
            \draw[<->|, >=stealth] (F) -- node[above]{{\small$c$}} (C);
          \end{scope}
          % Beschriftung rechts
          \begin{scope}[xshift=200]
            \pdef
            \draw[|<->|, >=stealth] (B) -- node[right]{{\small$b-d$}} (E);
            \draw[<->|, >=stealth] (E) -- node[right]{{\small$d$}} (C);
          \end{scope}
        \end{tikzpicture}}
      \end{minipage}
    % </PROBLEM>
  \fi
  %\ifoutline\outline\par
    % <OUTLINE>
    % </OUTLINE>
  %\fi
  \ifoutcome\outcome\par
    % <OUTCOME>
    Gesucht wird eine möglichst große rechteckige
    Fläche mit den Seitenlängen $e$ und $f$.
    Die zu maximierende Funktion lautet also:
    \begin{equation*}
      A(e,f)=e\cdot f
    \end{equation*}
    Die größte Fläche ergibt sich, wenn die rechte
    obere Ecke der neuen Glasplatte genau auf die
    Bruchkante gelegt wird.
    Die Variablen $e$ und $f$ sind also nicht
    unabhängig von einander, sondern müssen
    auf einander abgestimmt werden:
    Zum Beispiel definiert zu einer frei wählbaren
    Breite $e$ die Bruchkante die zugehörige Höhe $f$.
    \par
    Wenn es nun gelänge, die Gleichung einer Geraden
    aufzustellen, welche die Bruchkante als
    Teilstrecke enthält, hätte man einen
    funktionalen Zusammenhang gefunden, mit
    dem sich die Abhängigkeit zwischen $e$ und
    $f$ quantitativ beschreiben lässt.
    \par
    Bettet man dazu die Glasplatte in ein
    Koordinatensystem ein, dessen Urprung zum
    Beispiel in die untere linke Ecke der
    Glasplatte gelegt wird, erhalten die Enden
    der Bruchkante folgende Koordinaten:
    \begin{equation*}
      E_1\left(a-c\;\middle|\;b\right)
      \quad\text{bzw.}\quad
      E_2\left(a\;\middle|\;b-d\right)
    \end{equation*}
    Mit den Angaben aus der Aufgabenstellung
    ergeben sich dann die Zahlenwerte:
    \begin{equation*}
      E_1\left(90\;\middle|\;60\right)
      \quad\text{bzw.}\quad
      E_2\left(100\;\middle|\;56\right)
    \end{equation*}
    Da durch zwei verschiedene Punkte eine Gerade
    genau festgelegt ist, und diese im gewählten
    Koordinatensystem nicht parallel zur $y$-Achse
    verläuft, lässt sich deren Funktionsgleichung
    nun wie folgt bestimmen:
    \begin{equation*}
      \begin{split}
        g(x)&=m\cdot x+b
        \\
        m&=\frac{y_2-y_1}{x_2-x_1}
          =\frac{56-60}{100-90}
          =\frac{-4}{10}
          =\num{-0.4}
        \\
        b&=g(x)-m\cdot x
          =56-(\num{-0.4})\cdot100
          =\num{96}
        \\[1ex]
        g(x)&=\num{-0.4}x+96
      \end{split}
    \end{equation*}
    Setzt man in diese Geradengleichung einen
    Wert zwischen \num{90} und \num{100} ein,
    liefert die Rechnung genau den passenden
    Wert, mit dem ein Punkt auf der Bruchkante
    erreicht wird.
    Zu einem gewählten $e$ erhält man also
    das passende $f$.
    \par
    Mit dem gefundenen Zusammenhang $f=g(e)$
    erhält die zu maximierende Funktion $A$
    nun folgende Form:
    \begin{equation*}
      f=\num{-0.4}e+\num{96}
      \quad\Rightarrow\quad
      A(e,f)=e\cdot f
            =e\cdot(\num{-0.4}e+\num{96})
            =\num{-0.4}e^2+\num{96}e
            =A(e)
    \end{equation*}
    Damit ist sie nur noch von einer Variablen
    abhängig und die Extremstellen können mit
    den bekannten Verfahren der Analysis bestimmt
    werden:
    \begin{alignat*}{3}
      \relax&\quad
      &
      A'(e)&=\num{-0.8}e+\num{96}
      &
      \quad&\relax
      \\
      \relax&\quad
      &
      0&=\num{-0.8}e+\num{96}
      &
      \quad&|-\num{96}
      \\
      \Leftrightarrow&\quad
      &
      \num{-96}&=\num{-0.8}e
      &
      \quad&|:(\num{-0.8})
      \\
      \Leftrightarrow&\quad
      &
      120&=e
      &
      \quad&\relax
    \end{alignat*}
    Die einzige Nullstelle der Ableitung liegt
    außerhalb des Definitionsbereichs von $e$.
    Das gesuchte Maximum kann also nur ein
    Randwert des Definitionsbereichs sein:
    \begin{equation*}
      \begin{split}
        A(90)&=\num{-0.4}\cdot90^2+96\cdot90=\num{5400}
        %-0.4*90^2+96*90
        \\
        A(100)&=\num{-0.4}\cdot100^2+96\cdot100=\num{5600}
        %-0.4*100^2+96*100
      \end{split}
    \end{equation*}
    Die größtmögliche rechteckige Glasplatte, die
    man aus dem beschädigten Original noch
    schneiden kann, hat also eine Breite von
    \SI{100}{\centi\metre} und eine Höhe von
    \SI{56}{\centi\metre}.
    % </OUTCOME>
  \fi
\end{exercise}

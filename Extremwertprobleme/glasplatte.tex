\begin{exercise}
      {ID-3f2907460f3f3f644e1ebc02f5798964143cafe4}
      {Glasplatte}
  \ifproblem\problem\par
    \ifthenelse{\isundefined{\linecalc}}{\newlength{\linecalc}}{\relax}%
    \setlength{\linecalc}{\linewidth}%
    \addtolength{\linecalc}{-72mm}%
    \begin{minipage}[b]{\linecalc}
      Von einer rechteckigen Glasplatte mit den Seitenlängen $a$ und $b$
      ist an einer Ecke ein Stück von der Form eines rechtwinkligen
      Dreiecks abgebrochen.
      Gegeben sind die Maße $a=\simeter{1}$, $b=\sicm{60}$, $c=\sicm{10}$ und
      $d=\sicm{4}$.
      Wie sind $e$ und $f$ zu wählen, damit aus dem Rest eine
      möglichst große, rechteckige Platte geschnitten werden kann?
    \end{minipage}\hspace*{\fill}%
    \begin{minipage}[b]{65mm}
      \raggedleft
      \raisebox{\baselineskip}[7\baselineskip][0pt]{%
      \begin{tikzpicture}[scale=0.05]
        % um die Punkte verschieben zu koennen
        \newcommand{\pdef}
        {
          \coordinate (A) at (  0.0000,   0.0000);
          \coordinate (B) at (100.0000,   0.0000);
          \coordinate (C) at (100.0000,  60.0000);
          \coordinate (D) at (  0.0000,  60.0000);
          \coordinate (E) at (100.0000,  35.0000);
          \coordinate (F) at ( 60.0000,  60.0000);
          \coordinate (G) at ( 80.0000,   0.0000);
          \coordinate (H) at ( 80.0000,  47.5000);
          \coordinate (I) at (  0.0000,  47.5000);
        }
        % die unverschobene Variante
        \pdef
        % abgebrochene Ecke
        \filldraw[fill=black!25!white] (E) -- (C) -- (F);
        % Reststueck
        \draw (A) -- (B) -- (E) -- (F) -- (D) -- cycle;
        % inneres Rechteck
        \draw (G) -- node[left]{{\small$f$}} (H)
                  -- node[below]{{\small$e$}}  (I);
        % Beschriftung oben
        \begin{scope}[yshift=200]
          \pdef
          \draw[|<->|, >=stealth] (D) -- node[above]{{\small$a-c$}} (F);
          \draw[<->|, >=stealth] (F) -- node[above]{{\small$c$}} (C);
        \end{scope}
        % Beschriftung rechts
        \begin{scope}[xshift=200]
          \pdef
          \draw[|<->|, >=stealth] (B) -- node[right]{{\small$b-d$}} (E);
          \draw[<->|, >=stealth] (E) -- node[right]{{\small$d$}} (C);
        \end{scope}
      \end{tikzpicture}}
    \end{minipage}
  \fi
  %\ifoutline\outline\par
  %\fi
  %\ifoutcome\outcome\par
  %\fi
\end{exercise}

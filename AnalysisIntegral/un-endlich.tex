% 2022-01-04
\begin{exercise}
      {ID-bdbf8ab09e6506a7833d9745fb169fcebe1a30c4}
      {(Un-)Endlich?}
  \ifproblem\problem\par
    % <PROBLEM>
    Gegeben sei die Funktion
    $f:\mathbb{R}\setminus\{0\}\to\mathbb{R}$ mit:
    \begin{equation*}
      f(x)=\frac{1}{x}
    \end{equation*}
    \begin{enumerate}[a)]
      \item Skizzieren Sie den Graphen von $f$.
      \item Bestimmen Sie die Gleichung der Tangente
            an den Graphen von $f$ im Punkt
            $P\left(1\;\middle|\;f(1)\right)$.
      \item Der Graph von $f$ und die $x$-Achse
            schließen für $x>1$ eine nach rechts
            unbeschränkte Fläche ein.
            Zeigen Sie, dass der Flächeninhalt dieser
            Fläche nicht begrenzt ist.
      \item Der Graph von $f$ soll über dem Intervall
            $[1;z]$ um die $x$-Achse rotieren.
            Bestimmen Sie $z$ so, dass der dabei
            entstehende Rotationskörper das Volumen
            $\frac{3}{4}\pi$ besitzt.
            Zeigen Sie, dass das Volumen des
            Rotationskörpers für $z\to\infty$
            endlich ist.
    \end{enumerate}
    % </PROBLEM>
  \fi
  %\ifoutline\outline\par
    % <OUTLINE>
    % </OUTLINE>
  %\fi
  %\ifoutcome\outcome\par
    % <OUTCOME>
    % </OUTCOME>
  %\fi
\end{exercise}

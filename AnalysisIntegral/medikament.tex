% 2022-01-04
\begin{exercise}
      {ID-bde640557ba4027407d994850b36ef0bdf1ea836}
      {Medikament}
  \ifproblem\problem\par
    % <PROBLEM>
    Durch die Funktionenschar $f_a:\mathbb{R}\to\mathbb{R}$ mit
    \begin{equation*}
      f_a(t)=20a\cdot t\cdot e^{\num{-0.5}t}
      \quad\text{und}\quad
      a>0
    \end{equation*}
    wird für $t\in[\num{0},\num{12}]$ eine Konzentration
    eines Medikaments im Blut eines Patienten beschrieben
    ($t$ in Stunden seit der Einnahme und $f(t)$ in
    \si[per-mode=symbol]{\milli\gram\per\litre}).
    Der Parameter $a$ hängt von der Höhe der
    Wirkstoffdosis ab.
    \begin{enumerate}[a)]
      \item Zeigen Sie, dass der Zeitpunkt, zu dem
            die Konzentration des Medikaments am
            größten ist, nicht vom Parameter $a$
            abhängt und berechnen Sie diesen
            Zeitpunkt.
            Berechnen Sie, für welchen Wert von
            $a$ die maximale Konzentration im Blut
            \SI[per-mode=symbol]{20}{\milli\gram\per\litre}
            beträgt.
      \item Berechnen Sie den Zeitpunkt, zu dem die
            Wirkstoffkonzentration am stärksten abfällt.
            Hängt das Ergebnis vom Parameter $a$ ab?
      \item Zeigen Sie, dass $F_a$ mit
            $F_a(t)=-(40at+80a)\cdot e^{\num{-0.5}t}$
            eine Stammfunktion von $f_a$ ist.
            Berechnen Sie
            \begin{equation*}
              \frac{1}{12}\cdot\int_{0}^{12}f_a(t)\,\mathrm{d}t
            \end{equation*}
            und interpretieren Sie das Ergebnis im
            Sachzusammenhang.
      \item Für $t>12$ kann die Konzentration des
            Medikaments durch die Tangente an den
            Graphen von $f$ in $P\left(12\;\middle|\;f_a(12)\right))$
            beschrieben werden.
            Bestimmen Sie die Gleichung der Tangente und
            berechnen Sie, wann das Medikament vollständig
            abgebaut ist. Hängt das Ergebnis von $a$ ab?
    \end{enumerate}
    % </PROBLEM>
  \fi
  %\ifoutline\outline\par
    % <OUTLINE>
    % </OUTLINE>
  %\fi
  %\ifoutcome\outcome\par
    % <OUTCOME>
    % </OUTCOME>
  %\fi
\end{exercise}

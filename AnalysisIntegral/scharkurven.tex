% 2022-01-05
\begin{exercise}
      {ID-f8021628875a81d19a6f399de8c15f6d6839cb09}
      {Scharkurven}
  \ifproblem\problem\par
    % <PROBLEM>
    Gegeben sei die Funktionenschar $f_a:\mathbb{R}\to\mathbb{R}$ mit
    \begin{equation*}
      f_a(x)=\frac{x+a}{e^x}
      \quad\text{und}\quad
      a\in\mathbb{R}.
    \end{equation*}
    \begin{enumerate}[a)]
      %\setlength{\itemsep}{-1ex}%
      %\setcounter{enumi}{0}%
      \item Skizzieren Sie die Graphen von $f_1$ und $f_3$.
      \item Die beiden gezeichneten Kurven, die
            $x$-Achse und die Gerade $x=c$ mit $c>-1$
            begrenzen eine Fläche.
            Berechnen Sie ihren Inhalt $A(c)$.
            Zeigen Sie, dass $A(c)$ mit $c$ streng
            monoton wächst und berechnen Sie
            $\displaystyle\lim_{c\to\infty}A(c)$.
      \item Bestimmen Sie den maximalen Abstand von
            $f_1$ und $f_3$ für $x>0$.
      \item Zeigen Sie, dass Scharkurven zu verschiedenen
            Parameterwerten keine gemeinsamen Punkte
            besitzen.
    \end{enumerate}
    % </PROBLEM>
  \fi
  %\ifoutline\outline\par
    % <OUTLINE>
    % </OUTLINE>
  %\fi
  %\ifoutcome\outcome\par
    % <OUTCOME>
    % </OUTCOME>
  %\fi
\end{exercise}

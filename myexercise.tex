% the exercise's parts to typeset
\newif\ifproblem
\newif\ifoutline
\newif\ifoutcome

% set default values
\problemtrue
\outlinetrue
\outcometrue

% the number of thecurrent exercise
\newcounter{excounter}[chapter]
\setcounter{excounter}{0}

% names of the different parts
\newcommand{\problemtag}{Aufgabe}%
\newcommand{\outlinetag}{Ansatz}%
\newcommand{\outcometag}{Ergebnis}%

% prefixes of the different label IDs
\newcommand{\problemprefix}{problem}%
\newcommand{\outlineprefix}{outline}%
\newcommand{\outcomeprefix}{outcome}%

% the label of the current exercise
\newcommand{\exlabel}{}

% the name of the current exercise
\newcommand{\extitle}{}

% --------
% exercise
% --------
%
% #1  label of this exercise
% #2  title of this exercise
%
\newenvironment{exercise}[2]
{%
  % increase counter
  \refstepcounter{excounter}%
  % set current label
  \renewcommand{\exlabel}{#1}%
  % set current title
  \renewcommand{\extitle}{#2}%
}%
{%
  % nothing
}

% -------
% problem
% -------
%
% sets the caption of the 'problem' part
%
\newcommand{\problem}
{%
  % start caption
  \paragraph{\problemtag~\theexcounter}%
  % get title width
  \setbox0=\hbox{\extitle\unskip}%
  % check width
  \ifdim\wd0=0pt%
    % enable \par
    \leavevmode
  \else
    % add title
    \textit{(\extitle)}%
  \fi
  % check if outline is labeled
  \ifcsname r@\outlineprefix.\exlabel\endcsname
    % check if outcome is labeled
    \ifcsname r@\outcomeprefix.\exlabel\endcsname
      \begingroup
        \footnotesize\sffamily\hspace*{\fill}%
        (\outlinetag~S.\,\pageref{\outlineprefix.\exlabel};
         \outcometag~S.\,\pageref{\outcomeprefix.\exlabel})%
      \endgroup
    \else
      \begingroup
        \footnotesize\sffamily\hspace*{\fill}%
        (\outlinetag~S.\,\pageref{\outlineprefix.\exlabel})%
      \endgroup
    \fi
  \else
    % check if outcome is labeled
    \ifcsname r@\outcomeprefix.\exlabel\endcsname
      \begingroup
        \footnotesize\sffamily\hspace*{\fill}%
        (\outcometag~S.\,\pageref{\outcomeprefix.\exlabel})%
      \endgroup
    \else
      \relax
    \fi
  \fi
  % update current label
  \label{\problemprefix.\exlabel}%
  \par
}

% -------
% outline
% -------
%
% sets the caption of the 'outline' part
%
\newcommand{\outline}
{%
  % start caption
  \paragraph{\outlinetag~\theexcounter}%
  % get title width
  \setbox0=\hbox{\extitle\unskip}%
  % check width
  \ifdim\wd0=0pt%
    % enable \par
    \leavevmode
  \else
    % add title
    \textit{(\extitle)}%
  \fi
  % update current label
  \label{\outlineprefix.\exlabel}%
  \par
}

% -------
% outcome
% -------
%
% sets the caption of the 'outcome' part
%
\newcommand{\outcome}
{%
  % start caption
  \paragraph{\outcometag~\theexcounter}%
  % get title width
  \setbox0=\hbox{\extitle\unskip}%
  % check width
  \ifdim\wd0=0pt%
    % enable \par
    \leavevmode
  \else
    % add title
    \textit{(\extitle)}%
  \fi
  % update current label
  \label{\outcomeprefix.\exlabel}%
  \par
}

% ------------
% readproblems
% ------------
%
% read all exercises from the given file and show problems only
%
% #1  filename
%
\newcommand{\readproblems}[1]
{%
  % show only 'problem' parts
  \problemtrue
  \outlinefalse
  \outcomefalse
  % read exercises from given file
  \input{#1}
}

% ------------
% readoutlines
% ------------
%
% read all exercises from the given file and show outlines only
%
% #1  filename
%
\newcommand{\readoutlines}[1]
{%
  % show only 'approach' parts
  \problemfalse
  \outlinetrue
  \outcomefalse
  % read exercises from given file
  \input{#1}
}

% ------------
% readoutcomes
% ------------
%
% read all exercises from the given file and show outcomes only
%
% #1  filename
%
\newcommand{\readoutcomes}[1]
{%
  % show only 'solution' parts
  \problemfalse
  \outlinefalse
  \outcometrue
  % read exercises from given file
  \input{#1}
}

% -------
% readall
% -------
%
% read all exercises from the given file and show all parts
%
% #1  filename
%
\newcommand{\readall}[1]
{%
  % show only 'solution' parts
  \problemtrue
  \outlinetrue
  \outcometrue
  % read exercises from given file
  \input{#1}
}


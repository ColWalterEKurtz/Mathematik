\begin{exercise}
      {ID-2ac2584f5ad6fc8d2f0460198e380c2b7831adf2}
      {Drohne}
  \ifproblem\problem\par
    % <PROBLEM>
    Ein Amateurfilmer möchte mit seiner Drohne spektakuläre Landschaftsaufnahnem
    eines Bergpanoramas machen. Etwa auf halber Höhe des Berges startet er
    die Drohne. Sie ist so programmiert, dass sie zunächst \SI{20}{\metre}
    bis auf Punkt
    $A\left(\num{-1516}\;\middle|\;\num{112}\;\middle|\;\num{960}\right)$
    steigt, um dann geradlinig mit konstanter Geschwindigkeit in
    \num{5} Minuten zu Punkt
    $B\left(\num{-616}\;\middle|\;\num{1312}\;\middle|\;\num{960}\right)$
    zu fliegn. Sobald sie Punkt $B$ erreicht hat, kehrt sie um, und fliegt auf
    demselben Weg mit derselben Geschwindigkeit zum Ausgangspunkt $A$ zurück.
    Dort wird das abschließende Landemanöver
    eingeleitet.
    \begin{enumerate}[a)]
      \item Geben Sie eine Parametergleichung
            der Geraden an, mit der man die
            Flugbahn der Drohne zwischen den
            Punkten $A$ und $B$ beschreiben kann
            (ohne Start und Landung).
      \item Alle Koordinaten sind in der Einheit
            Meter angegebem. Mit welcher Geschwindigkeit
            fliegt die Drohne auf dem programmierten Kurs
            von $A$ nach $B$?
    \end{enumerate}
    Auf dem Berghang, den die Drohne überfliegen soll,
    befindet sich eine Seilbahn.
    Eine Minute bevor die Drohne in Punkt $A$
    ihre Kamerafahrt startet, verlassen zwei Gondeln der
    Seilbahn die Tal- bzw. Bergstation.
    Sie bewegen sich auf folgenden Bahnen:
    \begin{equation*}
      \vec{x}_{\text{auf}}=
      \begin{pmatrix}\num{0}\\\num{0}\\\num{0}\end{pmatrix}
      +r\cdot
      \begin{pmatrix}\num{-101}\\\num{132}\\\num{120}\end{pmatrix}
      \quad\text{bzw.}\quad
      \vec x_{\text{ab}}=
      \begin{pmatrix}\num{-1503}\\\num{1996}\\\num{1800}\end{pmatrix}
      +s\cdot
      \begin{pmatrix}\num{101}\\\num{-132}\\\num{-120}\end{pmatrix}
    \end{equation*}
    \begin{enumerate}[a)]
      \setcounter{enumi}{2}%
      \item Die Bergstation liegt \SI{1800}{\metre}
            höher als die Talstation. Wie lang dauert
            eine Fahrt zwischen den Stationen, wenn
            sich die Gondeln mit \SI{205}{\metre\per\minute}
            bewegen?
      \item Wie lang ist die Seilbahnstrecke?
      \item Wie groß ist der Steigungswinkel der
            Seilbahnstrecke?
      \item Schneidet die Flugbahn der Drohne eine oder
            beide Bahnen der Gondeln? Geben Sie
            eventuelle Schnittpunkte am.
      \item Kommt es zu einer Kollision?
    \end{enumerate}
    % </PROBLEM>
  \fi
  %\ifoutline\outline\par
    % <OUTLINE>
    % </OUTLINE>
  %\fi
  %\ifoutcome\outcome\par
    % <OUTCOME>
    % </OUTCOME>
  %\fi
\end{exercise}

\begin{exercise}
      {ID-db574718b06b2fad98ca1e47a7369c3e5dd13c9b}
      {Walmdach}
  \ifproblem\problem\par
    Das abgebildete Haus ist quaderförmig mit einem aufgesetzten Walmdach.
    Die Koordinaten in der Zeichnung besitzen alle die Einheit Meter.
    \begin{center}
      \begin{tikzpicture}[scale=1.75]
        \coordinate (O) at (  0.0000,   0.0000);
        \coordinate (X1) at ( -3.8400,  -0.5648);
        \coordinate (X2) at (  2.1600,  -0.5648);
        \coordinate (X3) at (  0.0000,   3.0594);
        \coordinate (A) at ( -2.4960,  -0.6025);
        \coordinate (B) at ( -0.5760,  -0.7907);
        \coordinate (C) at (  0.6240,  -0.3201);
        \coordinate (D) at ( -1.2960,  -0.1318);
        \coordinate (E) at ( -2.4960,   1.2802);
        \coordinate (F) at ( -0.5760,   1.0920);
        \coordinate (G) at (  0.6240,   1.5627);
        \coordinate (H) at ( -1.2960,   1.7509);
        \coordinate (I) at ( -1.2960,   2.2216);
        \coordinate (J) at ( -0.5760,   2.5040);
        \draw[line width=0.75pt, ->, >=stealth] (O) -- (X1) node[below]{{\small$x$}};
        \draw[line width=0.75pt, ->, >=stealth] (O) -- (X2) node[below]{{\small$y$}};
        \draw[line width=0.75pt, ->, >=stealth] (O) -- (X3) node[right]{{\small$z$}};
        \fill[fill=black] (A) circle[radius=0.75pt] node[below]{{\small$A$}};
        \fill[fill=black] (B) circle[radius=0.75pt] node[below]{{\small$B$}};
        \fill[fill=black] (C) circle[radius=0.75pt] node[below]{{\small$C$}};
        \fill[fill=black] (D) circle[radius=0.75pt] node[above left]{{\small$D$}};
        \fill[fill=black] (E) circle[radius=0.75pt] node[below right]{{\small$E$}};
        \fill[fill=black] (F) circle[radius=0.75pt] node[below right]{{\small$F$}};
        \fill[fill=black] (G) circle[radius=0.75pt] node[below right]{{\small$G$}};
        \fill[fill=black] (H) circle[radius=0.75pt] node[below right]{{\small$H$}};
        \fill[fill=black] (I) circle[radius=0.75pt] node[above]{{\small$I$}};
        \fill[fill=black] (J) circle[radius=0.75pt] node[above]{{\small$J$}};
        \draw[line width=0.6pt, style=solid, join=bevel] (F) -- (G) -- (C) -- (B) -- (F) -- (E) -- (A) -- (B);
        \draw[line width=0.6pt, style=solid, join=bevel] (E) -- (I) -- (F);
        \draw[line width=0.6pt, style=solid, join=bevel] (I) -- (J) -- (G);
        \draw[line width=0.6pt, style=dotted] (A) -- (D) -- (C);
        \draw[line width=0.6pt, style=dotted] (E) -- (H) -- (G);
        \draw[line width=0.6pt, style=dotted] (J) -- (H) -- (D);
        \node[right=4em] at (G)
        {%
          \begin{minipage}{7em}
            \setlength{\abovedisplayskip}{0pt}%
            \begin{equation*}
              \begin{split}
                A\;&(16 \mid 4\mid 0) \\
                B\;&(12 \mid 12\mid 0) \\
                C\;&(2 \mid 7\mid 0) \\
                G\;&(2 \mid 7\mid 8) \\
                I\;&(12 \mid 7\mid 12) \\
                J\;&(6 \mid 4\mid 12)
              \end{split}
            \end{equation*}
          \end{minipage}%
        };
      \end{tikzpicture}
    \end{center}
    \begin{enumerate}[a)]
      \item Die Punkte $E$, $F$ und $I$ legen Ebene $E_{1}$ fest.
            Die Punkte $F$, $G$, $I$ und $J$ liegen in Ebene $E_{2}$.
            Gib die beiden Ebenen jeweils in Parameter- und Normalenform an.
      \item Berechne den Schnittwinkel der beiden Dachebenen $E_{1}$ und $E_{2}$.
      \item Die Punkte $G$, $H$ und $J$ legen Ebene $E_{3}$ fest.
            Ermittle, in welcher Höhe über dem Boden sich die Ebenen
            $E_{1}$ und $E_{3}$ schneiden.
      \item Über dem Haus fliegt ein Vogel, dessen Flugbahn durch die Gerade
            \begin{equation*}
              g:\vec{x}=
              \begin{pmatrix} 9 \\ 0 \\ 11 \end{pmatrix}
              +t\cdot
              \begin{pmatrix} 0 \\ 1 \\ 1 \end{pmatrix}
            \end{equation*}
            beschrieben werden kann. Bestimme den minimalen Abstand des
            Vogels von der Geraden durch $I$ und $J$ und den minimalen
            Abstand des Vogels von der Ecke $J$.
      \item Ermittle den Ort, an dem der Vogel aus d) landen wird.
            Bestimme auch den Winkel, in dem der Vogel auf die Erde trifft.
      \item Ein weiterer Vogel fliegt auf der Flugbahn mit der Gleichung
            \begin{equation*}
              h_{\alpha}:\vec{x}=
              \begin{pmatrix} 5 \\ 11 \\ 8 \end{pmatrix}
              +t\cdot
              \begin{pmatrix} 2 \\ \alpha \\ 3 \end{pmatrix}
            \end{equation*}
            Ermittle denjenigen Wert von $\alpha$, für den sich die beiden
            Flugbahnen schneiden.
    \end{enumerate}
  \fi
  %\ifoutline\outline\par
  %\fi
  %\ifoutcome\outcome\par
  %\fi
\end{exercise}

% 2021-12-02
\begin{exercise}
      {ID-0c7f7a4af04575fa5349f38ef6b1df18a86b1f27}
      {Wolfspopulationen}
  \ifproblem\problem\par
    % <PROBLEM>
    Im Folgenden berachten wir die Entwicklung von
    Wolfspopulationen. Dabei beschränken wir uns
    ausschließlich auf die weiblichen Tiere einer
    Popuöation. die aus Welpen (w), jungen Fähen (j)
    sowie ausgewachsenen Fähen (a) bestehen soll.
    Alle Fähen sind vermehrungsfähig. Die Welpen
    entwickeln sich ein Jahr nach der Geburt zu
    jungen Fähen und ein Jahr später zu
    ausgewachsenen Fähen.
    \par
    Die folgenden Tabellen zeigen die Verteilung
    der weiblichen Tiere innerhalb einer in der
    Wildnis lebenden Population für die Jahre 2013
    und 2014, sowie die Übergänge zwischen den
    einzelnen Lebensphasen, mit denen die
    Entwicklung der Population modellhaft
    beschieben werden kann:
    \begin{center}
      \begingroup
        \renewcommand{\arraystretch}{1.2}%
        \begin{tabular}{|c|r|r|}
          \hline
            & 2013 & 2014 \\
          \hline
          w &   65 &   52 \\
          \hline
          j &    8 &   26 \\
          \hline
          a &   20 &   16 \\
          \hline
        \end{tabular}%
        \hspace{3em}%
        \begin{tabular}{|l|c|c|c|}
          \hline
                 & von w   & von j     & von a     \\
          \hline
          nach w & \num{0} & \num{1.5} & \num{2}   \\
          \hline
          nach j & $b$     & \num{0}   & \num{0}   \\
          \hline
          nach a & \num{0} & \num{0.5} & \num{0.6} \\
          \hline
        \end{tabular}
      \endgroup
    \end{center}
    \begin{enumerate}[a)]
      \item Stellen Sie die Übergangsmatrix $A$ auf,
            die zu diesem Entwicklungsprozess gehört,
            und begründen Sie mit den Daten aus der
            Tabelle, dass $b=\num{0.4}$ gilt.
      \item Berechnen Sie die Verteilungen, die nach
            diesem Modell in den Jahen 2015 und 2016
            zu erwarten sind.
      \item Bestimmen Sie die Verteilung, die nach
            diesem Modell im Jahr 2012 vorgelegen
            haben müsste.
      \item Zeigen Sie, dass sich in diesem Modell
            die Population aus 2011 nicht bestimmen
            lässt.
      \item Ein Biologe behauptet, dass weniger als
            \SI{15}{\percent} aller Welpen ein Alter
            von mindestens 3 Jahren erreichen.
            Prüfen Sie, ob nach vorliegender
            Modellierung die Behauptung des Biologen
            zutrifft.
    \end{enumerate}
    Wölfe, die in einem Tierpark leben, haben andere
    Überlebens- und Fortpflanzungsraten. Für einen
    Tierpark kann die Entwicklung seiner
    Wolfspopulation durch folgende Matrix modelliert
    werden:
    \begin{equation*}
      B=
      \begin{pmatrix}
        \num{0}   & \num{1}    & \num{0.1} \\
        \num{0.8} & \num{0}    & \num{0}   \\
        \num{0}   & \num{0.75} & \num{0.7}
      \end{pmatrix}
    \end{equation*}
    \begin{enumerate}[a)]
      \setcounter{enumi}{5}%
      \item Beschreiben Sie die Bedeutung der
            Einträge in der zweiten Spalte von
            Matrix $B$ im Sachzusammengang und
            interpretieren Sie die Unterschiede,
            die sich im Vergleich mit der zweiten
            Spalte von Matrix $A$ zeigen.
      \item Wegen der räumlichen Beschränkung will
            die Tierparkleitung die Gesamtzahl der
            Wölfe konstant halten.
            Eine Verteilung, die sich beim Übergang
            in den Folgezustand nicht ändert, wird
            \emph{stationäre} Verteilung genannt.
            Zeigen Sie, dass für die Population im
            Tierpark eine nicht-triviale
            stationäre Verteilung existiert.
      \item Wie groß muss die Anzahl der weiblichen
            Individuen in der Population mindestens
            sein, und wie müssen sie sich auf die
            verschiedenen Entwicklungsstufen
            verteilen, damit sie eine stationäre
            Verteilung bilden.
            Geben Sie eine ganzzahlige Lösung an,
            die sich vom Nullvektor unterscheidet.
    \end{enumerate}
    % </PROBLEM>
  \fi
  %\ifoutline\outline\par
    % <OUTLINE>
    % </OUTLINE>
  %\fi
  \ifoutcome\outcome
    % <OUTCOME>
    \begin{enumerate}[a)]
      %\setlength{\itemsep}{-1ex}%
      %\setcounter{enumi}{0}%
      \item Aus der rechten Tabelle erhält man
            folgende Übergangsmatrix:
            \begin{equation*}
              A=
              \begin{pmatrix}
                \num{0} & \num{1.5} & \num{2}   \\
                b       & \num{0}   & \num{0}   \\
                \num{0} & \num{0.5} & \num{0.6}
              \end{pmatrix}
            \end{equation*}
            Den Wert für $b$ erhält man, indem
            man die Matrixmultiplikation für die
            zweite Zeile durchführt und die
            entstehende Gleichung nach $b$
            auflöst:
            \begin{equation*}
              \begin{split}
                A\cdot n_{2013}=n_{2014}
                \quad&\Rightarrow\quad
                b\cdot65+0\cdot8+0\cdot20=26
                \quad|:65
                \\
                &\Rightarrow\quad
                b=\num{0.4}
              \end{split}
            \end{equation*}
      \item Die $n$-fache Multiplikation mit der
            Übergangsmatrix ergibt die entsprexhene
            Verteilung:
            \begin{equation*}
              n_{2015}=A\cdot n_{2014}=
              \begin{pmatrix}
                \num{71}   \\
                \num{20.8} \\
                \num{22.6}
              \end{pmatrix}
              \qquad
              n_{2016}=A^2\cdot n_{2014}=
              \begin{pmatrix}
                \num{76.4}  \\
                \num{28.4}  \\
                \num{23.96}
              \end{pmatrix}
            \end{equation*}
      \item Zum Berechnen früherer Zustände
            in stochastischen Ptozessen
            benötigt man die inverse
            Übergangsmatrix $A^{-1}$.
            Für den vorliegenden Prozess gilt:
            \begin{equation*}
              A^{-1}=
              \begin{pmatrix}
                \num{0}  & \num{2.5} & \num{0}   \\
                -\num{6} & \num{0}   & \num{20}  \\
                \num{5}  & \num{0}   & -\num{15}
              \end{pmatrix}
            \end{equation*}
            Für das Jahr 2012 ergibt sich damit
            folgende Verteilung:
            \begin{equation*}
              \begin{pmatrix}
                \num{0}  & \num{2.5} & \num{0}   \\
                -\num{6} & \num{0}   & \num{20}  \\
                \num{5}  & \num{0}   & -\num{15}
              \end{pmatrix}
              \cdot
              \begin{pmatrix}
                \num{65} \\
                \num{8}  \\
                \num{20}
              \end{pmatrix}
              =
              \begin{pmatrix}
                \num{20} \\
                \num{10} \\
                \num{25}
              \end{pmatrix}
            \end{equation*}
      \item Wenn man von 2013 zwei Jahre auf 2011
            zurück rechnet, erhält man:
            \begin{equation*}
              \begin{pmatrix}
                \num{0}  & \num{2.5} & \num{0}   \\
                -\num{6} & \num{0}   & \num{20}  \\
                \num{5}  & \num{0}   & -\num{15}
              \end{pmatrix}^2
              \cdot
              \begin{pmatrix}
                \num{65} \\
                \num{8}  \\
                \num{20}
              \end{pmatrix}
              =
              \begin{pmatrix}
                \num{25}  \\
                \num{380} \\
                -\num{275}
              \end{pmatrix}
            \end{equation*}
            Da der negative Wert im Sachzusammengang
            keinen Sinn ergibt, ist dieses Modell
            ungeeignet, um die Verteilung aus dem
            Jahr 2011 zu berechnen. Auch die
            Verteilungen aus noch früheren Jahren
            sind durch dieses Modell nicht
            zugänglich.
      \item Um ein Alter von mindestens 3 Jahren
            zu erreichen sind in diesem Modell
            drei Übergänge erforderlich.
            Die entsprechende Übergangsmatrix
            lautet:
            \begin{equation*}
              A^3=
              \begin{pmatrix}
                \num{0.4}  & \num{1.5}  & \num{1.92}  \\
                \num{0.24} & \num{0.4}  & \num{0.48}  \\
                \num{0.12} & \num{0.48} & \num{0.616}
              \end{pmatrix}
            \end{equation*}
            Man erkennt am Eintrag $A_{31}=\num{0.12}$,
            dass \SI{12}{\percent} der Welpen drei
            Übergänge (Jahre) überleben.
            Die Aussage des Biologen trifft also zu.
      \item TODO
      \item Eine Übergangsmatrix $U$ besitzt eine
            (nicht-triviale) stationäre Verteilung
            $x$, wenn in dem Gleichungssystem
            $U\cdot x=x$ identische Zeilen auftreten.
            \par
            Mit den Daten aus dem Tierpark ergibt
            sich damit folgender Ansatz:
            \begin{equation*}
              B\cdot x=x
              \quad\Rightarrow\quad
              \begin{pmatrix}
                \num{0}   & \num{1}    & \num{0.1} \\
                \num{0.8} & \num{0}    & \num{0}   \\
                \num{0}   & \num{0.75} & \num{0.7}
              \end{pmatrix}
              \cdot
              \begin{pmatrix}
                w \\
                j \\
                a
              \end{pmatrix}
              =
              \begin{pmatrix}
                w \\
                j \\
                a
              \end{pmatrix}
            \end{equation*}
            Diese Gleichung führt auf das
            äquivalente Gleichungssystem:
            \begingroup
              \renewcommand{\arraycolsep}{2pt}%
              \begin{alignat*}{1}
                &
                \begin{array}{r|rcrcrcll}
                  \text{I}\;\;&\;   \num{0}w & + &    \num{1}j & + & \num{0.1}a & = & w & \quad|-w \\
                 \text{II}\;\;&\; \num{0.8}w & + &    \num{0}j & + &   \num{0}a & = & j & \quad|-j \\
                \text{III}\;\;&\;   \num{0}w & + & \num{0.75}j & + & \num{0.7}a & = & a & \quad|-a
                \end{array}
                \\[1ex] &
                \begin{array}{r|rcrcrcll}
                  \text{I}\;\;&\;         -w & + &           j & + & \num{0.1}a & = & 0 & \quad|\cdot(-1)                  \\
                 \text{II}\;\;&\; \num{0.8}w & - &           j &   &            & = & 0 & \quad|\rightleftarrows\text{III} \\
                \text{III}\;\;&\;            &   & \num{0.75}j & - & \num{0.3}a & = & 0 & \quad|\rightleftarrows\text{II}
                \end{array}
                \\[1ex] &
                \begin{array}{r|rcrcrcll}
                  \text{I}\;\;&\;          w & - &           j & - & \num{0.1}a & = & 0 & \quad\relax       \\
                 \text{II}\;\;&\;            &   & \num{0.75}j & - & \num{0.3}a & = & 0 & \quad|:\num{0.75} \\
                \text{III}\;\;&\; \num{0.8}w & - &           j &   &            & = & 0 & \quad|:\num{0.8}
                \end{array}
                \\[1ex] &
                \begin{array}{r|rcrcrcll}
                  \text{I}\;\;&\; w & - &           j & - & \num{0.1}a & = & 0 & \quad\relax \\
                 \text{II}\;\;&\;   &   &           j & - & \num{0.4}a & = & 0 & \quad\relax \\
                \text{III}\;\;&\; w & - & \num{1.25}j &   &            & = & 0 & \quad|-\text{I}
                \end{array}
                \\[1ex] &
                \begin{array}{r|rcrcrcll}
                  \text{I}\;\;&\; w & - &            j & - & \num{0.1}a & = & 0 & \quad\relax \\
                 \text{II}\;\;&\;   &   &            j & - & \num{0.4}a & = & 0 & \quad\relax \\
                \text{III}\;\;&\;   &   & -\num{0.25}j & + & \num{0.1}a & = & 0 & \quad|:(-\num{0.25})
                \end{array}
                \\[1ex] &
                \begin{array}{r|rcrcrcll}
                  \text{I}\;\;&\; w & - & j & - & \num{0.1}a & = & 0 & \quad\relax \\
                 \text{II}\;\;&\;   &   & j & - & \num{0.4}a & = & 0 & \quad\relax \\
                \text{III}\;\;&\;   &   & j & - & \num{0.4}a & = & 0 & \quad\relax
                \end{array}
              \end{alignat*}
            \endgroup
            Da die Zeilen II und III des Gleichungssystems
            übereinstimmen, existiert für die Übergangsmatrix
            $B$ eine nicht-triviale stationäre Verteilung.
      \item Die stationäre Verteilung erhält man nun,
            indem man das Gleichungssystem löst, das
            sich durch Weglassen aller redundanten
            Zeilen aus dem vorherigen ergibt:
            \begingroup
              \renewcommand{\arraycolsep}{2pt}%
              \begin{alignat*}{1}
                &
                \begin{array}{r|rcrcrcll}
                 \text{I}\;\;&\; w & - & j & - & \num{0.1}a & = & 0 & \quad\relax \\
                \text{II}\;\;&\;   &   & j & - & \num{0.4}a & = & 0 & \quad|+\num{0.4} a
                \end{array}
                \\[1ex] &
                \begin{array}{r|rcrcrcll}
                 \text{I}\;\;&\; w & - & j & - & \num{0.1}a & = & 0          & \quad|\leftarrow\text{II} \\
                \text{II}\;\;&\;   &   & j &   &            & = & \num{0.4}a & \quad\relax
                \end{array}
                \\[1ex] &
                \begin{array}{r|rcrcrcll}
                 \text{I}\;\;&\; w & - & \num{0.4}a & - & \num{0.1}a & = & 0          & \quad\relax \\
                \text{II}\;\;&\;   &   &          j &   &            & = & \num{0.4}a & \quad\relax
                \end{array}
                \\[1ex] &
                \begin{array}{r|rcrcll}
                 \text{I}\;\;&\; w & - & \num{0.5}a & = & 0          & \quad|+\num{0.5}a \\
                \text{II}\;\;&\;   &   &          j & = & \num{0.4}a & \quad\relax
                \end{array}
                \\[1ex] &
                \begin{array}{r|rcl}
                 \text{I}\;\;&\; w & = & \num{0.5}a \\
                \text{II}\;\;&\; j & = & \num{0.4}a
                \end{array}
              \end{alignat*}
            \endgroup
            Alle stationären Verteilungen der Matrix
            $B$ haben damit die Form:
            \begin{equation*}
              x=\begin{pmatrix}\num{0.5}a\\\num{0.4}a\\a\end{pmatrix}
               =a\cdot\begin{pmatrix}\num{0.5}\\\num{0.4}\\\num{1}\end{pmatrix}
            \end{equation*}
            Die kleinstmögliche Lösung mit
            ganzzahligen Werten ergibt sich
            für $a=\num{10}$ und man
            erhält:
            \begin{equation*}
              x=10\cdot\begin{pmatrix}\num{0.5}\\\num{0.4}\\\num{1}\end{pmatrix}
               =\begin{pmatrix}5\\4\\10\end{pmatrix}
              \quad\text{mit}\quad
              |x|=\num{19}
            \end{equation*}
            In der Anfangspopulation sollten also
            \num{19} Fähen vorhanden sein: \num{5}
            als Welpen, \num{4} als junge und
            \num{10} als ausgewachsene Tiere.
    \end{enumerate}
    % </OUTCOME>
  \fi
\end{exercise}

\begin{exercise}
      {ID-d175c5cfaef70166600d7fbfceb9139726dd8972}
      {Ebene mit einem Parameter}
  \ifproblem\problem\par
    % <PROBLEM>
    Bestimmen Sie den Wert für $a\in\mathbb{R}$ so, dass
    \begin{enumerate}[a)]
      \item die Ebene
            \begin{equation*}
              E:\vec{x}=
              \begin{pmatrix}5\\6\\a\end{pmatrix}
              +r\cdot
              \begin{pmatrix}2\\3\\-1\end{pmatrix}
              +s\cdot
              \begin{pmatrix}1\\0\\4\end{pmatrix}
              ,\quad r,s\in\mathbb{R}
            \end{equation*}
            durch den Ursprung verläuft.
      \item die Gleichung
            \begin{equation*}
              E:\vec{x}=
              \begin{pmatrix}1\\2\\-1\end{pmatrix}
              +r\cdot
              \begin{pmatrix}-3\\-6\\a\end{pmatrix}
              +s\cdot
              \begin{pmatrix}a\\2a\\a\end{pmatrix}
              ,\quad r,s\in\mathbb{R}
            \end{equation*}
            keine Ebene beschreibt.
    \end{enumerate}
    % </PROBLEM>
  \fi
  %\ifoutline\outline\par
    % <OUTLINE>
    % </OUTLINE>
  %\fi
  \ifoutcome\outcome\par
    % <OUTCOME>
    \begin{enumerate}[a)]
      \item Wenn der Ursprung $O\left(0\;\middle|\;0\;\middle|\;0\right)$
            auf der Ebene $E$ liegen soll, muss folgende Gleichung
            erfüllt sein:
            \begin{equation*}
              \begin{pmatrix}0\\0\\0\end{pmatrix}
              =
              \begin{pmatrix}5\\6\\a\end{pmatrix}
              +r\cdot
              \begin{pmatrix}2\\3\\-1\end{pmatrix}
              +s\cdot
              \begin{pmatrix}1\\0\\4\end{pmatrix}
            \end{equation*}
            Diese Vektorgleichung führt zu dem Gleichungssystem:
            \begin{align*}
              &
              \begin{array}{r|rcrcrcrl}
                  \text{I.} & 0 & = & 5 & + & 2r & + &  s & \quad|-5 \\
                 \text{II.} & 0 & = & 6 & + & 3r &   &    & \quad|-6 \\
                \text{III.} & 0 & = & a & - &  r & + & 4s & \quad
              \end{array}
              \\[2ex]
              &
              \begin{array}{r|rcrcrcrl}
                  \text{I.} & -5 & = &   &   & 2r & + &  s & \quad    \\
                 \text{II.} & -6 & = &   &   & 3r &   &    & \quad|:3 \\
                \text{III.} &  0 & = & a & - &  r & + & 4s & \quad
              \end{array}
              \\[2ex]
              &
              \begin{array}{r|rcrcrcrl}
                  \text{I.} & -5 & = &   &   & 2r & + &  s & \quad|\;r=-2 \\
                 \text{II.} & -2 & = &   &   &  r &   &    & \quad        \\
                \text{III.} &  0 & = & a & - &  r & + & 4s & \quad|\;r=-2
              \end{array}
              \\[2ex]
              &
              \begin{array}{r|rcrcrcrl}
                  \text{I.} & -5 & = &   &   & -4 & + &  s & \quad|+4 \\
                 \text{II.} & -2 & = &   &   &  r &   &    & \quad    \\
                \text{III.} &  0 & = & a & + &  2 & + & 4s & \quad|-2
              \end{array}
              \\[2ex]
              &
              \begin{array}{r|rcrcrcrl}
                  \text{I.} & -1 & = &   & &   &   &  s & \quad \\
                 \text{II.} & -2 & = &   & & r &   &    & \quad \\
                \text{III.} & -2 & = & a & &   & + & 4s & \quad|\;s=-1
              \end{array}
              \\[2ex]
              &
              \begin{array}{r|rcrcrcrl}
                  \text{I.} & -1 & = &   & &   &   & s & \quad \\
                 \text{II.} & -2 & = &   & & r &   &   & \quad \\
                \text{III.} & -2 & = & a & &   & - & 4 & \quad|+4
              \end{array}
              \\[2ex]
              &
              \begin{array}{r|rcrcrcrl}
                  \text{I.} & -1 & = &   & &   & & s & \quad \\
                 \text{II.} & -2 & = &   & & r & &   & \quad \\
                \text{III.} &  2 & = & a & &   & &   & \quad
              \end{array}
            \end{align*}
            Wenn man $a=2$ wählt, kann man den
            Ursprung mit $r=-2$ und $s=-1$ erreichen.
      \item Damit die Lösungsmenge der Gleichung
            keine Ebene beschreibt, müssen die
            Spannvektoren linear abhängig sein,
            also:
            \begin{equation*}
              \begin{pmatrix}0\\0\\0\end{pmatrix}
              =
              r\cdot
              \begin{pmatrix}-3\\-6\\a\end{pmatrix}
              +s\cdot
              \begin{pmatrix}a\\2a\\a\end{pmatrix}
              \quad\Rightarrow\quad
              \exists\,\{r;s\}\neq\{0\}
            \end{equation*}
            Diese Vektorgleichung führt zu dem Gleichungssystem:
            \begin{align*}
              &
              \begin{array}{r|rcrcrl}
                  \text{I.} & 0 & = & -3r & + &  as &\quad    \\
                 \text{II.} & 0 & = & -6r & + & 2as &\quad|:2 \\
                \text{III.} & 0 & = &  ar & + &  as &\quad
              \end{array}
              \\[2ex]
              &
              \begin{array}{r|rcrcrl}
                  \text{I.} & 0 & = & -3r & + & as &\quad \\
                 \text{II.} & 0 & = & -3r & + & as &\quad \\
                \text{III.} & 0 & = &  ar & + & as &\quad
              \end{array}
              \intertext{Die Zeilen I. und II. stimmen überein, so dass man eine davon streichen kann:}
              &
              \begin{array}{r|rcrcrl}
                 \text{I.} & 0 & = & -3r & + & as &\quad \\
                \text{II.} & 0 & = &  ar & + & as &\quad
              \end{array}
              \intertext{Mit der Festlegung $\mathbb{D}_a=\mathbb{R}\setminus\{0\}$ kann fortgesetzt werden:}
              &
              \begin{array}{r|rcrcrl}
                 \text{I.} & 0 & = & -3r & + & as &\quad \\
                \text{II.} & 0 & = &  ar & + & as &\quad|:a
              \end{array}
              \\[2ex]
              &
              \begin{array}{r|rcrcrl}
                 \text{I.} & 0 & = & -3r & + & as &\quad|\;r=-s \\
                \text{II.} & 0 & = &   r & + &  s &\quad
              \end{array}
              \\[2ex]
              &
              \begin{array}{r|rcrcrl}
                 \text{I.} & 0 & = & 3s & + & as &\quad \\
                \text{II.} & 0 & = &  r & + &  s &\quad
              \end{array}
              \\[2ex]
              &
              \begin{array}{r|rcrcrl}
                 \text{I.} & 0 & = &   &   & (3+a)\cdot s &\quad \\
                \text{II.} & 0 & = & r & + &            s &\quad
              \end{array}
            \end{align*}
            Mit $-3\neq a\in\mathbb{D}_a$ besitzt das
            Gleichungssystem nur die triviale Lösung
            $r=s=0$.
            Die Spannvektoren wären also linear unabhängig
            und würden eine Ebene aufspannen.
            Für $a=-3$ erhält man die nicht-trivialen
            Lösungen $r=-s$, und damit linear
            abhängige Vektoren, die keine Ebene
            mehr aufspannen.\par
            Da $0\not\in\mathbb{D}_a$, trifft das
            Gleichungssystem keine Aussage für
            diesen Fall, und man muss ihn gesondert
            betrachten.
            Mit $a=0$ entartet der zweite Richtungsvektor
            der Gleichung zum Nullvektor, und es
            ergibt sich ebenfalls keine Ebene als
            Lösungsmenge mehr.
    \end{enumerate}
    % </OUTCOME>
  \fi
\end{exercise}

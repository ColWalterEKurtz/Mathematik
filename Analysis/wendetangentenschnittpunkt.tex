\begin{exercise}
      {ID-d840964f507587b576bf442f8b034c9b6d7b9184}
      {Wendetangentenschnittpunkt}
  \ifproblem\problem\par
    % <PROBLEM>
    Bestimmen Sie den Schnittpunkt der Wendetangenten der
    Funktion $f$ mit
    \begin{equation*}
      %<OCTAVE>
      f(x)=\frac{\num{1}}{\num{256}}x^{4}-\frac{\num{1}}{\num{64}}x^{3}-\frac{\num{45}}{\num{128}}x^{2}+\frac{\num{47}}{\num{64}}x+\frac{\num{673}}{\num{256}}
      %</OCTAVE>
      %f = [1/256 -1/64 -45/128 47/64 673/256];
      %printf("f(x)=%s\n", mypolystr(f, "x", [0 0 0 0 1]));
    \end{equation*}
    % </PROBLEM>
  \fi
  %\ifoutline\outline\par
    % <OUTLINE>
    % </OUTLINE>
  %\fi
  \ifoutcome\outcome\par
    % <OUTCOME>
    Zur Bestimmung der Wendepunkte, wird zunächst
    die zweite Ableitung benötigt:
    \begin{equation*}
      \begin{split}
      %<OCTAVE>
      f(x)&=\frac{\num{1}}{\num{256}}x^{4}-\frac{\num{1}}{\num{64}}x^{3}-\frac{\num{45}}{\num{128}}x^{2}+\frac{\num{47}}{\num{64}}x+\frac{\num{673}}{\num{256}}\\[2ex]
      f'(x)&=\frac{\num{1}}{\num{64}}x^{3}-\frac{\num{3}}{\num{64}}x^{2}-\frac{\num{45}}{\num{64}}x+\frac{\num{47}}{\num{64}}\\[2ex]
      f''(x)&=\frac{\num{3}}{\num{64}}x^{2}-\frac{\num{3}}{\num{32}}x-\frac{\num{45}}{\num{64}}
      %</OCTAVE>
      %f = [1/256 -1/64 -45/128 47/64 673/256];
      %df = polyder(f);
      %d2f = polyder(df);
      %printf("f(x)&=%s\\\\[2ex]\n", mypolystr(f, "x", [0 0 0 0 1]));
      %printf("f'(x)&=%s\\\\[2ex]\n", mypolystr(df, "x", [0 0 0 0 1]));
      %printf("f''(x)&=%s\\\\[2ex]\n", mypolystr(d2f, "x", [0 0 0 0 1]));
      \end{split}
    \end{equation*}
    Eine Nullstelle in $f''$ ist notwendige Bedingung
    für die Existenz eines Wendepunktes in $f$:
    \begin{alignat*}{3}
      \relax&\quad
      &
      0&=\frac{\num{3}}{\num{64}}x^{2}-\frac{\num{3}}{\num{32}}x-\frac{\num{45}}{\num{64}}
      &
      \quad&|:\frac{3}{64}
      \\[2ex]
      \Leftrightarrow&\quad
      &
      0&=x^2-2x-15
      &
      \quad&|\;\text{$pq$-Formel}
      \\[2ex]
      \Leftrightarrow&\quad
      &
      x_{1,2}&=-\frac{-2}{2}\pm\sqrt{\left(\frac{-2}{2}\right)^2-(-15)}
      &
      \quad&\relax
      \\[2ex]
      \relax&\quad
      &
      &=1\pm\sqrt{16}=1\pm4
      &
      \quad&\relax
      \\[2ex]
      \Leftrightarrow&\quad
      &
      x&\in\{-3;5\}
      &
      \quad&\relax
    \end{alignat*}
    Wenn eine quadratische Funktion zwei
    Nullstellen besitzt, dann findet an beiden
    Nullstellen ein Vorzeichenwechsel statt.
    Die Funktion $f$ enthält also genau zwei
    Wendepunkte:
    \begin{equation*}
      \begin{split}
        W_1&=\left(W_{1x}\;\middle|\;W_{1y}\right)
            =\left(-3\;\middle|\;f(-3)\right)
            =\left(-3\;\middle|\;-2\right)
        \\
        W_2&=\left(W_{2x}\;\middle|\;W_{2y}\right)
            =\left(5\;\middle|\;f(5)\right)
            =\left(5\;\middle|\;-2\right)
        %f = [1/256 -1/64 -45/128 47/64 673/256];
        %polyval(f, -3)
        %polyval(f, 5)
      \end{split}
    \end{equation*}
    Die zugehörigen Steigungen erhält man aus der
    ersten Ableitung:
    \begin{equation*}
      \begin{split}
        m_1&=f'(-3)=2
        \\
        m_2&=f'(5)=-2
        %f = [1/256 -1/64 -45/128 47/64 673/256];
        %df = polyder(f);
        %polyval(df, -3)
        %polyval(df, 5)
      \end{split}
    \end{equation*}
    Mit diesen Informationen lassen sich jetzt die
    Tangentengleichungen aufstellen:
    \begin{equation*}
      \begin{split}
        t_1(x)&=m_1\cdot x+b_1
        \qquad
        b_1=W_{1y}-m_1\cdot W_{1x}
           =-2-2\cdot(-3)
           =4
        \qquad
        t_1(x)=2x+4
        \\
        t_2(x)&=m_2\cdot x+b_2
        \qquad
        b_2=W_{2y}-m_2\cdot W_{2x}
           =-2-(-2)\cdot5
           =8
        \qquad
        t_2(x)=-2x+8
      \end{split}
    \end{equation*}
    Ein Gleichsetzen der Tangentengleichungen liefert
    die $x$-Koordinate des Schnittpunktes:
    \begin{alignat*}{3}
      \relax&\quad
      &
      t_1(x)&=t_2(x)
      &
      \quad&\relax
      \\
      \Leftrightarrow&\quad
      &
      2x+4&=-2x+8
      &
      \quad&|+2x
      \\
      \Leftrightarrow&\quad
      &
      4x+4&=8
      &
      \quad&|-4
      \\
      \Leftrightarrow&\quad
      &
      4x&=4
      &
      \quad&|:4
      \\
      \Leftrightarrow&\quad
      &
      x&=1
      &
      \quad&\relax
    \end{alignat*}
    Die $y$-Koordinate ergibt sich aus dem
    Einsetzen des eben gefundenen $x$-Wertes:
    \begin{equation*}
      t_{1}(1)=t_2(1)=6
    \end{equation*}
    Die beiden Wendetangenten schneiden sich also
    im Punkt $S\left(1\;\middle|\;6\right)$.
    % </OUTCOME>
  \fi
\end{exercise}

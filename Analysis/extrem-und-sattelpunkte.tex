\begin{exercise}
      {ID-bfd94ce82bc953becdd54cffa81b08f870ebc7ee}
      {Extrem- und Sattelpunkte}
  \ifproblem\problem\par
    % <PROBLEM>
    Bestimmen Sie rechnerisch alle Extrem- und Sattelpunkte des Graphen
    von $f$. Skizzieren Sie anschließend den groben Verlauf des Graphen
    von $f$ mithilfe dieser Informationen und entscheiden Sie jeweils
    für die Extrempunkte, ob es sich um ein lokales oder globales
    Extremum handelt. Überprüfen Sie Ihre Lösungen mit dem GTR.
    \begin{align*}
      \text{a)}\;\;f(x)&=\frac{\num{1}}{\num{2}}x^{2}-x-\frac{\num{3}}{\num{2}}
      &
      \text{e)}\;\;f(x)&=-\frac{\num{1}}{\num{24}}x^{4}-\frac{\num{1}}{\num{12}}x^{3}+\frac{\num{1}}{\num{2}}x^{2}
      \\[1ex]
      \text{b)}\;\;f(x)&=\frac{\num{1}}{\num{4}}x^{3}-\num{3}x
      &
      \text{f)}\;\;f(x)&=\frac{\num{1}}{\num{60}}x^{5}-\frac{\num{1}}{\num{12}}x^{4}
      \\[1ex]
      \text{c)}\;\;f(x)&=\frac{\num{1}}{\num{32}}x^{4}-\frac{\num{1}}{\num{6}}x^{3}
      &
      \text{g)}\;\;f(x)&=\frac{\num{1}}{\num{25}}x^{5}-\frac{\num{5}}{\num{12}}x^{3}+\num{1}
      \\[1ex]
      \text{d)}\;\;f(x)&=\frac{\num{1}}{\num{36}}x^{4}-\frac{\num{1}}{\num{3}}x^{3}+x^{2}
      &
      \text{h)}\;\;f(x)&=-\frac{\num{1}}{\num{125}}x^{5}+\frac{\num{1}}{\num{6}}x^{3}-\frac{\num{25}}{\num{16}}x
    \end{align*}
    % </PROBLEM>
  \fi
  \ifoutline\outline\par
    % <OUTLINE>
    Eine ganzrationale Funktion $f$ besitzt
    in ihren Extrem- und Sattelpunkten
    horizontale Tangenten.
    Ändert sich an einer dieser Stellen
    zusätzlich das Vorzeichen von $f'$,
    handelt es sich dort um ein Extremum,
    andernfalls um einen Sattelpunkt.
    % </OUTLINE>
  \fi
  \ifoutcome\outcome\par
    % <OUTCOME>
Teilaufgabe a)
\begin{equation*}
  \begin{split}
    f(x)&=\frac{\num{1}}{\num{2}}x^{2}-x-\frac{\num{3}}{\num{2}}
    \\[1ex]
    f'(x)&=x-\num{1}
  \end{split}
\end{equation*}
Horizontale Tangenten besitzen die Steigung 0, also
müssen zunächst die Nullstellen der ersten Ableitung
bestimmt werden:
\begin{equation*}
  f'(x)=0=x-\num{1}
  \quad\Rightarrow\quad
  x=1
\end{equation*}
Um die Extem- von den Sattelpunkten unterscheiden zu
können, nutzt man die Tatsache, dass sich in Sattelpunkten
das Monotonieverhalten einer Funktion nicht ändert.
Nullstellen der Ableitung ohne Vorzeichenwechsel
identifizieren also Sattelpunkte.
\begin{center}
  \renewcommand{\arraystretch}{1.25}%
  \begin{tabular}{r|c|c|c}
    Punkt           & $P_{1}$      & $P_{2}$       & $P_{3}$      \\
    \hline
    $x$             & $\num{0}$    & $\num{1}$     & $\num{2}$    \\
    \hline
    $f(x)$          & $-\num{1.5}$ & $-\num{2}$    & $-\num{1.5}$ \\
    \hline
    $f'(x)$         & $-\num{1}$   & $\num{0}$     & $\num{1}$    \\
    \hline
    Steigung in $P$ & $\searrow$   & $\rightarrow$ & $\nearrow$   \\
    \hline
    Klassifikation  &              & TP            &             
  \end{tabular}
\end{center}
Die Funktion $f$ besitzt in Punkt $\left(\num{1}\;\middle|\;-\num{2}\right)$ ein globales Minimum.
\begin{center}
\begin{tikzpicture}[scale=0.500]
  % grid
  \draw[draw=black!50!white] (-4.000, -5.000) grid[step=0.5] (6.000, 5.000);
  % x-axis
  \draw[line width=0.6pt, ->, >=stealth] (-4.000, 0) -- (6.000, 0) node[below left] {\small$x$};
  % y-axis
  \draw[line width=0.6pt, ->, >=stealth] (0, -5.000) -- (0, 5.000) node[below left] {\small$y$};
  % function: f(x)=\num{0.5}x^{2}-x-\num{1.5}
  \begin{scope}[line width=0.7pt]
    \clip (-4.000, -5.000) rectangle (6.000, 5.000);
    \draw plot[smooth] coordinates
    {
      ( -4.000,   8.000) ( -3.900,   8.000) ( -3.800,   8.000)
      ( -3.700,   8.000) ( -3.600,   8.000) ( -3.500,   8.000)
      ( -3.400,   7.680) ( -3.300,   7.245) ( -3.200,   6.820)
      ( -3.100,   6.405) ( -3.000,   6.000) ( -2.900,   5.605)
      ( -2.800,   5.220) ( -2.700,   4.845) ( -2.600,   4.480)
      ( -2.500,   4.125) ( -2.400,   3.780) ( -2.300,   3.445)
      ( -2.200,   3.120) ( -2.100,   2.805) ( -2.000,   2.500)
      ( -1.900,   2.205) ( -1.800,   1.920) ( -1.700,   1.645)
      ( -1.600,   1.380) ( -1.500,   1.125) ( -1.400,   0.880)
      ( -1.300,   0.645) ( -1.200,   0.420) ( -1.100,   0.205)
      ( -1.000,   0.000) ( -0.900,  -0.195) ( -0.800,  -0.380)
      ( -0.700,  -0.555) ( -0.600,  -0.720) ( -0.500,  -0.875)
      ( -0.400,  -1.020) ( -0.300,  -1.155) ( -0.200,  -1.280)
      ( -0.100,  -1.395) (  0.000,  -1.500) (  0.100,  -1.595)
      (  0.200,  -1.680) (  0.300,  -1.755) (  0.400,  -1.820)
      (  0.500,  -1.875) (  0.600,  -1.920) (  0.700,  -1.955)
      (  0.800,  -1.980) (  0.900,  -1.995) (  1.000,  -2.000)
      (  1.100,  -1.995) (  1.200,  -1.980) (  1.300,  -1.955)
      (  1.400,  -1.920) (  1.500,  -1.875) (  1.600,  -1.820)
      (  1.700,  -1.755) (  1.800,  -1.680) (  1.900,  -1.595)
      (  2.000,  -1.500) (  2.100,  -1.395) (  2.200,  -1.280)
      (  2.300,  -1.155) (  2.400,  -1.020) (  2.500,  -0.875)
      (  2.600,  -0.720) (  2.700,  -0.555) (  2.800,  -0.380)
      (  2.900,  -0.195) (  3.000,   0.000) (  3.100,   0.205)
      (  3.200,   0.420) (  3.300,   0.645) (  3.400,   0.880)
      (  3.500,   1.125) (  3.600,   1.380) (  3.700,   1.645)
      (  3.800,   1.920) (  3.900,   2.205) (  4.000,   2.500)
      (  4.100,   2.805) (  4.200,   3.120) (  4.300,   3.445)
      (  4.400,   3.780) (  4.500,   4.125) (  4.600,   4.480)
      (  4.700,   4.845) (  4.800,   5.220) (  4.900,   5.605)
      (  5.000,   6.000) (  5.100,   6.405) (  5.200,   6.820)
      (  5.300,   7.245) (  5.400,   7.680) (  5.500,   8.000)
      (  5.600,   8.000) (  5.700,   8.000) (  5.800,   8.000)
      (  5.900,   8.000) (  6.000,   8.000)
    };
  \end{scope}
\end{tikzpicture}
\end{center}

Teilaufgabe b)
\begin{equation*}
  \begin{split}
    f(x)&=\frac{\num{1}}{\num{4}}x^{3}-\num{3}x
    \\[1ex]
    f'(x)&=\frac{\num{3}}{\num{4}}x^{2}-\num{3}
  \end{split}
\end{equation*}
Horizontale Tangenten besitzen die Steigung 0, also
müssen zunächst die Nullstellen der ersten Ableitung
bestimmt werden:
\begin{equation*}
  f'(x)=0=\frac{\num{3}}{\num{4}}x^{2}-\num{3}
  \quad\Rightarrow\quad
  x\in\{-2;2\}
\end{equation*}
Um die Extem- von den Sattelpunkten unterscheiden zu
können, nutzt man die Tatsache, dass sich in Sattelpunkten
das Monotonieverhalten einer Funktion nicht ändert.
Nullstellen der Ableitung ohne Vorzeichenwechsel
identifizieren also Sattelpunkte.
\begin{center}
  \renewcommand{\arraystretch}{1.25}%
  \begin{tabular}{r|c|c|c|c|c}
    Punkt           & $P_{1}$      & $P_{2}$       & $P_{3}$    & $P_{4}$       & $P_{5}$       \\
    \hline
    $x$             & $-\num{3}$   & $-\num{2}$    & $\num{0}$  & $\num{2}$     & $\num{3}$     \\
    \hline
    $f(x)$          & $\num{2.25}$ & $\num{4}$     & $\num{0}$  & $-\num{4}$    & $-\num{2.25}$ \\
    \hline
    $f'(x)$         & $\num{3.75}$ & $\num{0}$     & $-\num{3}$ & $\num{0}$     & $\num{3.75}$  \\
    \hline
    Steigung in $P$ & $\nearrow$   & $\rightarrow$ & $\searrow$ & $\rightarrow$ & $\nearrow$    \\
    \hline
    Klassifikation  &              & HP            &            & TP            &              
  \end{tabular}
\end{center}
Die Funktion $f$ besitzt in Punkt $\left(-\num{2}\;\middle|\;\num{4}\right)$ ein lokales Maximum und in Punkt $\left(\num{2}\;\middle|\;-\num{4}\right)$ ein lokales Minimum.
\begin{center}
\begin{tikzpicture}[scale=0.500]
  % grid
  \draw[draw=black!50!white] (-5.000, -5.000) grid[step=0.5] (5.000, 5.000);
  % x-axis
  \draw[line width=0.6pt, ->, >=stealth] (-5.000, 0) -- (5.000, 0) node[below left] {\small$x$};
  % y-axis
  \draw[line width=0.6pt, ->, >=stealth] (0, -5.000) -- (0, 5.000) node[below left] {\small$y$};
  % function: f(x)=\num{0.25}x^{3}-\num{3}x
  \begin{scope}[line width=0.7pt]
    \clip (-5.000, -5.000) rectangle (5.000, 5.000);
    \draw plot[smooth] coordinates
    {
      ( -5.000,  -8.000) ( -4.900,  -8.000) ( -4.800,  -8.000)
      ( -4.700,  -8.000) ( -4.600,  -8.000) ( -4.500,  -8.000)
      ( -4.400,  -8.000) ( -4.300,  -6.977) ( -4.200,  -5.922)
      ( -4.100,  -4.930) ( -4.000,  -4.000) ( -3.900,  -3.130)
      ( -3.800,  -2.318) ( -3.700,  -1.563) ( -3.600,  -0.864)
      ( -3.500,  -0.219) ( -3.400,   0.374) ( -3.300,   0.916)
      ( -3.200,   1.408) ( -3.100,   1.852) ( -3.000,   2.250)
      ( -2.900,   2.603) ( -2.800,   2.912) ( -2.700,   3.179)
      ( -2.600,   3.406) ( -2.500,   3.594) ( -2.400,   3.744)
      ( -2.300,   3.858) ( -2.200,   3.938) ( -2.100,   3.985)
      ( -2.000,   4.000) ( -1.900,   3.985) ( -1.800,   3.942)
      ( -1.700,   3.872) ( -1.600,   3.776) ( -1.500,   3.656)
      ( -1.400,   3.514) ( -1.300,   3.351) ( -1.200,   3.168)
      ( -1.100,   2.967) ( -1.000,   2.750) ( -0.900,   2.518)
      ( -0.800,   2.272) ( -0.700,   2.014) ( -0.600,   1.746)
      ( -0.500,   1.469) ( -0.400,   1.184) ( -0.300,   0.893)
      ( -0.200,   0.598) ( -0.100,   0.300) (  0.000,   0.000)
      (  0.100,  -0.300) (  0.200,  -0.598) (  0.300,  -0.893)
      (  0.400,  -1.184) (  0.500,  -1.469) (  0.600,  -1.746)
      (  0.700,  -2.014) (  0.800,  -2.272) (  0.900,  -2.518)
      (  1.000,  -2.750) (  1.100,  -2.967) (  1.200,  -3.168)
      (  1.300,  -3.351) (  1.400,  -3.514) (  1.500,  -3.656)
      (  1.600,  -3.776) (  1.700,  -3.872) (  1.800,  -3.942)
      (  1.900,  -3.985) (  2.000,  -4.000) (  2.100,  -3.985)
      (  2.200,  -3.938) (  2.300,  -3.858) (  2.400,  -3.744)
      (  2.500,  -3.594) (  2.600,  -3.406) (  2.700,  -3.179)
      (  2.800,  -2.912) (  2.900,  -2.603) (  3.000,  -2.250)
      (  3.100,  -1.852) (  3.200,  -1.408) (  3.300,  -0.916)
      (  3.400,  -0.374) (  3.500,   0.219) (  3.600,   0.864)
      (  3.700,   1.563) (  3.800,   2.318) (  3.900,   3.130)
      (  4.000,   4.000) (  4.100,   4.930) (  4.200,   5.922)
      (  4.300,   6.977) (  4.400,   8.000) (  4.500,   8.000)
      (  4.600,   8.000) (  4.700,   8.000) (  4.800,   8.000)
      (  4.900,   8.000) (  5.000,   8.000)
    };
  \end{scope}
\end{tikzpicture}
\end{center}

Teilaufgabe c)
\begin{equation*}
  \begin{split}
    f(x)&=\frac{\num{1}}{\num{32}}x^{4}-\frac{\num{1}}{\num{6}}x^{3}
    \\[1ex]
    f'(x)&=\frac{\num{1}}{\num{8}}x^{3}-\frac{\num{1}}{\num{2}}x^{2}
  \end{split}
\end{equation*}
Horizontale Tangenten besitzen die Steigung 0, also
müssen zunächst die Nullstellen der ersten Ableitung
bestimmt werden:
\begin{equation*}
  f'(x)=0
  =\frac{\num{1}}{\num{8}}x^{3}-\frac{\num{1}}{\num{2}}x^{2}
  =\frac{\num{1}}{\num{8}}x^{2}\cdot(x-4)
  \quad\Rightarrow\quad
  x\in\{0;4\}
\end{equation*}
Um die Extem- von den Sattelpunkten unterscheiden zu
können, nutzt man die Tatsache, dass sich in Sattelpunkten
das Monotonieverhalten einer Funktion nicht ändert.
Nullstellen der Ableitung ohne Vorzeichenwechsel
identifizieren also Sattelpunkte.
\begin{center}
  \renewcommand{\arraystretch}{1.25}%
  \begin{tabular}{r|c|c|c|c|c}
    Punkt           & $P_{1}$       & $P_{2}$       & $P_{3}$        & $P_{4}$        & $P_{5}$     \\
    \hline
    $x$             & $-\num{2}$    & $\num{0}$     & $\num{2}$      & $\num{4}$      & $\num{6}$   \\
    \hline
    $f(x)$          & $\num{1.833}$ & $\num{0}$     & $-\num{0.833}$ & $-\num{2.667}$ & $\num{4.5}$ \\
    \hline
    $f'(x)$         & $-\num{3}$    & $\num{0}$     & $-\num{1}$     & $\num{0}$      & $\num{9}$   \\
    \hline
    Steigung in $P$ & $\searrow$    & $\rightarrow$ & $\searrow$     & $\rightarrow$  & $\nearrow$  \\
    \hline
    Klassifikation  &               & SP            &                & TP             &            
  \end{tabular}
\end{center}
Die Funktion $f$ besitzt in Punkt $\left(\num{0}\;\middle|\;\num{0}\right)$ einen Sattelpunkt und in Punkt $\left(\num{4}\;\middle|\;-\frac{\num{8}}{\num{3}}\right)$ ein globales Minimum.
\begin{center}
\begin{tikzpicture}[scale=0.500]
  % grid
  \draw[draw=black!50!white] (-3.000, -5.000) grid[step=0.5] (7.000, 5.000);
  % x-axis
  \draw[line width=0.6pt, ->, >=stealth] (-3.000, 0) -- (7.000, 0) node[below left] {\small$x$};
  % y-axis
  \draw[line width=0.6pt, ->, >=stealth] (0, -5.000) -- (0, 5.000) node[below left] {\small$y$};
  % function: f(x)=\num{0.03125}x^{4}-\num{0.1666667}x^{3}
  \begin{scope}[line width=0.7pt]
    \clip (-3.000, -5.000) rectangle (7.000, 5.000);
    \draw plot[smooth] coordinates
    {
      ( -3.000,   7.031) ( -2.900,   6.275) ( -2.800,   5.579)
      ( -2.700,   4.941) ( -2.600,   4.357) ( -2.500,   3.825)
      ( -2.400,   3.341) ( -2.300,   2.902) ( -2.200,   2.507)
      ( -2.100,   2.151) ( -2.000,   1.833) ( -1.900,   1.550)
      ( -1.800,   1.300) ( -1.700,   1.080) ( -1.600,   0.887)
      ( -1.500,   0.721) ( -1.400,   0.577) ( -1.300,   0.455)
      ( -1.200,   0.353) ( -1.100,   0.268) ( -1.000,   0.198)
      ( -0.900,   0.142) ( -0.800,   0.098) ( -0.700,   0.065)
      ( -0.600,   0.040) ( -0.500,   0.023) ( -0.400,   0.011)
      ( -0.300,   0.005) ( -0.200,   0.001) ( -0.100,   0.000)
      (  0.000,   0.000) (  0.100,  -0.000) (  0.200,  -0.001)
      (  0.300,  -0.004) (  0.400,  -0.010) (  0.500,  -0.019)
      (  0.600,  -0.032) (  0.700,  -0.050) (  0.800,  -0.073)
      (  0.900,  -0.101) (  1.000,  -0.135) (  1.100,  -0.176)
      (  1.200,  -0.223) (  1.300,  -0.277) (  1.400,  -0.337)
      (  1.500,  -0.404) (  1.600,  -0.478) (  1.700,  -0.558)
      (  1.800,  -0.644) (  1.900,  -0.736) (  2.000,  -0.833)
      (  2.100,  -0.936) (  2.200,  -1.043) (  2.300,  -1.153)
      (  2.400,  -1.267) (  2.500,  -1.383) (  2.600,  -1.501)
      (  2.700,  -1.620) (  2.800,  -1.738) (  2.900,  -1.855)
      (  3.000,  -1.969) (  3.100,  -2.079) (  3.200,  -2.185)
      (  3.300,  -2.283) (  3.400,  -2.375) (  3.500,  -2.456)
      (  3.600,  -2.527) (  3.700,  -2.585) (  3.800,  -2.629)
      (  3.900,  -2.657) (  4.000,  -2.667) (  4.100,  -2.656)
      (  4.200,  -2.624) (  4.300,  -2.567) (  4.400,  -2.485)
      (  4.500,  -2.373) (  4.600,  -2.231) (  4.700,  -2.055)
      (  4.800,  -1.843) (  4.900,  -1.593) (  5.000,  -1.302)
      (  5.100,  -0.967) (  5.200,  -0.586) (  5.300,  -0.155)
      (  5.400,   0.328) (  5.500,   0.867) (  5.600,   1.463)
      (  5.700,   2.122) (  5.800,   2.845) (  5.900,   3.637)
      (  6.000,   4.500) (  6.100,   5.438) (  6.200,   6.455)
      (  6.300,   7.554) (  6.400,   8.000) (  6.500,   8.000)
      (  6.600,   8.000) (  6.700,   8.000) (  6.800,   8.000)
      (  6.900,   8.000) (  7.000,   8.000)
    };
  \end{scope}
\end{tikzpicture}
\end{center}

Teilaufgabe d)
\begin{equation*}
  \begin{split}
    f(x)&=\frac{\num{1}}{\num{36}}x^{4}-\frac{\num{1}}{\num{3}}x^{3}+x^{2}
    \\[1ex]
    f'(x)&=\frac{\num{1}}{\num{9}}x^{3}-x^{2}+\num{2}x
  \end{split}
\end{equation*}
Horizontale Tangenten besitzen die Steigung 0, also
müssen zunächst die Nullstellen der ersten Ableitung
bestimmt werden:
\begin{equation*}
  f'(x)=0
  =\frac{\num{1}}{\num{9}}x^{3}-x^{2}+\num{2}x
  =\frac{1}{9}x\cdot(x^2-9x+18)
\end{equation*}
\begingroup
  \newcommand{\vstrut}{\vphantom{\left(f_0^0\right)}}%
  \newcommand{\noeq}{\phantom{\Leftrightarrow}\vstrut&\quad}%
  \newcommand{\iseq}{\Leftrightarrow\vstrut&\quad}%
  \newcommand{\impl}{\Rightarrow\vstrut&\quad}%
  \newcommand{\nomod}{\quad&\phantom{|}}%
  \newcommand{\domod}[1]{\quad&|#1}%
  \begin{alignat*}{3}
    \noeq
    &
    \num{0}&=x^{2}-\num{9}x+\num{18}
    &
    \domod{\;\text{$pq$-Formel}}
    \\
    \noeq
    &
    p&=-\num{9}
    &
    \nomod
    \\
    \noeq
    &
    q&=\num{18}
    &
    \nomod
    \\
    \noeq
    &
    x_{1,2}&=-\frac{p}{2}\pm\sqrt{\left(\frac{p}{2}\right)^2-q}
    &
    \nomod
    \\
    \noeq
    &
    &=\frac{\num{9}}{\num{2}}\pm\sqrt{\left(-\frac{\num{9}}{\num{2}}\right)^2-\num{18}}
    &
    \nomod
    \\
    \noeq
    &
    &=\frac{\num{9}}{\num{2}}\pm\sqrt{\frac{\num{9}}{\num{4}}}
    &
    \nomod
    \\
    \noeq
    &
    &=\frac{\num{9}}{\num{2}}\pm\frac{\num{3}}{\num{2}}
    &
    \nomod
    \\
    \noeq
    &
    x_1&=\num{3}
    &
    \nomod
    \\
    \noeq
    &
    x_2&=\num{6}
    &
    \nomod
  \end{alignat*}
\endgroup
  Es gibt also insgesamt drei Nullstellen:
  \begin{equation*}
    f'(x)=0=\frac{\num{1}}{\num{9}}x^{3}-x^{2}+\num{2}x\quad\Rightarrow\quad x\in\{0;3;6\}
  \end{equation*}
Um die Extem- von den Sattelpunkten unterscheiden zu
können, nutzt man die Tatsache, dass sich in Sattelpunkten
das Monotonieverhalten einer Funktion nicht ändert.
Nullstellen der Ableitung ohne Vorzeichenwechsel
identifizieren also Sattelpunkte.
\begin{center}
  \renewcommand{\arraystretch}{1.25}%
  \begin{tabular}{r|c|c|c|c|c|c|c}
    Punkt           & $P_{1}$        & $P_{2}$       & $P_{3}$       & $P_{4}$       & $P_{5}$        & $P_{6}$       & $P_{7}$       \\
    \hline
    $x$             & $-\num{1}$     & $\num{0}$     & $\num{1}$     & $\num{3}$     & $\num{5}$      & $\num{6}$     & $\num{7}$     \\
    \hline
    $f(x)$          & $\num{1.361}$  & $\num{0}$     & $\num{0.694}$ & $\num{2.25}$  & $\num{0.694}$  & $\num{0}$     & $\num{1.361}$ \\
    \hline
    $f'(x)$         & $-\num{3.111}$ & $\num{0}$     & $\num{1.111}$ & $\num{0}$     & $-\num{1.111}$ & $\num{0}$     & $\num{3.111}$ \\
    \hline
    Steigung in $P$ & $\searrow$     & $\rightarrow$ & $\nearrow$    & $\rightarrow$ & $\searrow$     & $\rightarrow$ & $\nearrow$    \\
    \hline
    Klassifikation  &                & TP            &               & HP            &                & TP            &              
  \end{tabular}
\end{center}
Die Funktion $f$ besitzt in den Punkten $\left(\num{0}\;\middle|\;\num{0}\right)$ und $\left(\num{6}\;\middle|\;\num{0}\right)$ zwei globale Minima und in Punkt $\left(\num{3}\;\middle|\;\frac{\num{9}}{\num{4}}\right)$ ein lokales Maximum.
\begin{center}
\begin{tikzpicture}[scale=0.500]
  % grid
  \draw[draw=black!50!white] (-3.000, -2.000) grid[step=0.5] (9.000, 6.000);
  % x-axis
  \draw[line width=0.6pt, ->, >=stealth] (-3.000, 0) -- (9.000, 0) node[below left] {\small$x$};
  % y-axis
  \draw[line width=0.6pt, ->, >=stealth] (0, -2.000) -- (0, 6.000) node[below left] {\small$y$};
  % function: f(x)=\num{0.0277778}x^{4}-\num{0.3333333}x^{3}+x^{2}
  \begin{scope}[line width=0.7pt]
    \clip (-3.000, -2.000) rectangle (9.000, 6.000);
    \draw plot[smooth] coordinates
    {
      ( -3.000,   9.000) ( -2.900,   9.000) ( -2.800,   9.000)
      ( -2.700,   9.000) ( -2.600,   9.000) ( -2.500,   9.000)
      ( -2.400,   9.000) ( -2.300,   9.000) ( -2.200,   9.000)
      ( -2.100,   8.037) ( -2.000,   7.111) ( -1.900,   6.258)
      ( -1.800,   5.476) ( -1.700,   4.760) ( -1.600,   4.107)
      ( -1.500,   3.516) ( -1.400,   2.981) ( -1.300,   2.502)
      ( -1.200,   2.074) ( -1.100,   1.694) ( -1.000,   1.361)
      ( -0.900,   1.071) ( -0.800,   0.822) ( -0.700,   0.611)
      ( -0.600,   0.436) ( -0.500,   0.293) ( -0.400,   0.182)
      ( -0.300,   0.099) ( -0.200,   0.043) ( -0.100,   0.010)
      (  0.000,   0.000) (  0.100,   0.010) (  0.200,   0.037)
      (  0.300,   0.081) (  0.400,   0.139) (  0.500,   0.210)
      (  0.600,   0.292) (  0.700,   0.382) (  0.800,   0.481)
      (  0.900,   0.585) (  1.000,   0.694) (  1.100,   0.807)
      (  1.200,   0.922) (  1.300,   1.037) (  1.400,   1.152)
      (  1.500,   1.266) (  1.600,   1.377) (  1.700,   1.484)
      (  1.800,   1.588) (  1.900,   1.686) (  2.000,   1.778)
      (  2.100,   1.863) (  2.200,   1.941) (  2.300,   2.012)
      (  2.400,   2.074) (  2.500,   2.127) (  2.600,   2.171)
      (  2.700,   2.205) (  2.800,   2.230) (  2.900,   2.245)
      (  3.000,   2.250) (  3.100,   2.245) (  3.200,   2.230)
      (  3.300,   2.205) (  3.400,   2.171) (  3.500,   2.127)
      (  3.600,   2.074) (  3.700,   2.012) (  3.800,   1.941)
      (  3.900,   1.863) (  4.000,   1.778) (  4.100,   1.686)
      (  4.200,   1.588) (  4.300,   1.484) (  4.400,   1.377)
      (  4.500,   1.266) (  4.600,   1.152) (  4.700,   1.037)
      (  4.800,   0.922) (  4.900,   0.807) (  5.000,   0.694)
      (  5.100,   0.585) (  5.200,   0.481) (  5.300,   0.382)
      (  5.400,   0.292) (  5.500,   0.210) (  5.600,   0.139)
      (  5.700,   0.081) (  5.800,   0.037) (  5.900,   0.010)
      (  6.000,   0.000) (  6.100,   0.010) (  6.200,   0.043)
      (  6.300,   0.099) (  6.400,   0.182) (  6.500,   0.293)
      (  6.600,   0.436) (  6.700,   0.611) (  6.800,   0.822)
      (  6.900,   1.071) (  7.000,   1.361) (  7.100,   1.694)
      (  7.200,   2.074) (  7.300,   2.502) (  7.400,   2.981)
      (  7.500,   3.516) (  7.600,   4.107) (  7.700,   4.760)
      (  7.800,   5.476) (  7.900,   6.258) (  8.000,   7.111)
      (  8.100,   8.037) (  8.200,   9.000) (  8.300,   9.000)
      (  8.400,   9.000) (  8.500,   9.000) (  8.600,   9.000)
      (  8.700,   9.000) (  8.800,   9.000) (  8.900,   9.000)
      (  9.000,   9.000)
    };
  \end{scope}
\end{tikzpicture}
\end{center}

Teilaufgabe e)
\begin{equation*}
  \begin{split}
    f(x)&=-\frac{\num{1}}{\num{24}}x^{4}-\frac{\num{1}}{\num{12}}x^{3}+\frac{\num{1}}{\num{2}}x^{2}
    \\[1ex]
    f'(x)&=-\frac{\num{1}}{\num{6}}x^{3}-\frac{\num{1}}{\num{4}}x^{2}+x
  \end{split}
\end{equation*}
Horizontale Tangenten besitzen die Steigung 0, also
müssen zunächst die Nullstellen der ersten Ableitung
bestimmt werden:
\begin{equation*}
%%f'(x)=0=-1/6x^3 - 1/4x^2 + x
  f'(x)=0
  =-\frac{\num{1}}{\num{6}}x^{3}-\frac{\num{1}}{\num{4}}x^{2}+x
  =-\frac{1}{6}x\cdot\left(x^2+\frac{3}{2}x-6\right)
\end{equation*}
\begingroup
  \newcommand{\vstrut}{\vphantom{\left(f_0^0\right)}}%
  \newcommand{\noeq}{\phantom{\Leftrightarrow}\vstrut&\quad}%
  \newcommand{\iseq}{\Leftrightarrow\vstrut&\quad}%
  \newcommand{\impl}{\Rightarrow\vstrut&\quad}%
  \newcommand{\nomod}{\quad&\phantom{|}}%
  \newcommand{\domod}[1]{\quad&|#1}%
  \begin{alignat*}{3}
    \noeq
    &
    \num{0}&=x^{2}+\frac{\num{3}}{\num{2}}x-\num{6}
    &
    \domod{\;\text{$pq$-Formel}}
    \\[1ex]
    \noeq
    &
    p&=\frac{\num{3}}{\num{2}}
    &
    \nomod
    \\
    \noeq
    &
    q&=-\num{6}
    &
    \nomod
    \\
    \noeq
    &
    x_{1,2}&=-\frac{p}{2}\pm\sqrt{\left(\frac{p}{2}\right)^2-q}
    &
    \nomod
    \\[1ex]
    \noeq
    &
    &=-\frac{\num{3}}{\num{4}}\pm\sqrt{\left(\frac{\num{3}}{\num{4}}\right)^2-\left(-\num{6}\right)}
    &
    \nomod
    \\[1ex]
    \noeq
    &
    &=-\frac{\num{3}}{\num{4}}\pm\sqrt{\frac{\num{105}}{\num{16}}}
    &
    \nomod
    \\[1ex]
    \noeq
    &
    x_1&=-\frac{\num{3}}{\num{4}}-\frac{\sqrt{\num{105}}}{\num{4}}
    \approx
    %<OCTAVE>
    -\num{3.312}
    %</OCTAVE>
    %myn2s(-3/4-sqrt(105)/4, 3, 0, 0, 0, 1)
    &
    \nomod
    \\[1ex]
    \noeq
    &
    x_2&=-\frac{\num{3}}{\num{4}}+\frac{\sqrt{\num{105}}}{\num{4}}
    \approx
    %<OCTAVE>
    \num{1.812}
    %</OCTAVE>
    %myn2s(-3/4+sqrt(105)/4, 3, 0, 0, 0, 1)
    &
    \nomod
  \end{alignat*}
\endgroup
Es gibt also insgesamt drei Nullstellen:
\begin{equation*}
  f'(x)=0=-\frac{\num{1}}{\num{6}}x^{3}-\frac{\num{1}}{\num{4}}x^{2}+x
  \quad\Rightarrow\quad
  x\in
  \left\{
    \frac{-3-\sqrt{105}}{4};
    0;
    \frac{-3+\sqrt{105}}{4}
  \right\}
\end{equation*}
Um die Extem- von den Sattelpunkten unterscheiden zu
können, nutzt man die Tatsache, dass sich in Sattelpunkten
das Monotonieverhalten einer Funktion nicht ändert.
Nullstellen der Ableitung ohne Vorzeichenwechsel
identifizieren also Sattelpunkte.
\begin{center}
  \renewcommand{\arraystretch}{1.25}%
  \begin{tabular}{r|c|c|c|c|c|c|c}
    Punkt           & $P_{1}$        & $P_{2}$        & $P_{3}$        & $P_{4}$       & $P_{5}$       & $P_{6}$       & $P_{7}$        \\
    \hline
    $x$             & $-\num{5}$     & $-\num{3.312}$ & $-\num{1}$     & $\num{0}$     & $\num{1}$     & $\num{1.812}$ & $\num{3}$      \\
    \hline
    $f(x)$          & $-\num{3.125}$ & $\num{3.499}$  & $\num{0.542}$  & $\num{0}$     & $\num{0.375}$ & $\num{0.697}$ & $-\num{1.125}$ \\
    \hline
    $f'(x)$         & $\num{9.583}$  & $\num{0}$      & $-\num{1.083}$ & $\num{0}$     & $\num{0.583}$ & $\num{0}$     & $-\num{3.75}$  \\
    \hline
    Steigung in $P$ & $\nearrow$     & $\rightarrow$  & $\searrow$     & $\rightarrow$ & $\nearrow$    & $\rightarrow$ & $\searrow$     \\
    \hline
    Klassifikation  &                & HP             &                & TP            &               & HP            &               
  \end{tabular}
\end{center}
Die Funktion $f$ besitzt näherungsweise in Punkt
$\left(-\num{3.312}\;\middle|\;\num{3.499}\right)$ ein globales Maximum, in Punkt
$\left(\num{0}\;\middle|\;\num{0}\right)$ ein lokales Minimum und in Punkt
$\left(\num{1.812}\;\middle|\;\num{0.697}\right)$ ein lokales Maximum.
\begin{center}
\begin{tikzpicture}[scale=0.500]
  % grid
  \draw[draw=black!50!white] (-6.000, -5.000) grid[step=0.5] (5.000, 5.000);
  % x-axis
  \draw[line width=0.6pt, ->, >=stealth] (-6.000, 0) -- (5.000, 0) node[below left] {\small$x$};
  % y-axis
  \draw[line width=0.6pt, ->, >=stealth] (0, -5.000) -- (0, 5.000) node[below left] {\small$y$};
  % function: f(x)=-\num{0.0416667}x^{4}-\num{0.0833333}x^{3}+\num{0.5}x^{2}
  \begin{scope}[line width=0.7pt]
    \clip (-6.000, -5.000) rectangle (5.000, 5.000);
    \draw plot[smooth] coordinates
    {
      ( -6.000,  -8.000) ( -5.900,  -8.000) ( -5.800,  -8.000)
      ( -5.700,  -8.000) ( -5.600,  -8.000) ( -5.500,  -8.000)
      ( -5.400,  -7.727) ( -5.300,  -6.426) ( -5.200,  -5.228)
      ( -5.100,  -4.129) ( -5.000,  -3.125) ( -4.900,  -2.211)
      ( -4.800,  -1.382) ( -4.700,  -0.635) ( -4.600,   0.035)
      ( -4.500,   0.633) ( -4.400,   1.162) ( -4.300,   1.626)
      ( -4.200,   2.029) ( -4.100,   2.374) ( -4.000,   2.667)
      ( -3.900,   2.909) ( -3.800,   3.105) ( -3.700,   3.257)
      ( -3.600,   3.370) ( -3.500,   3.445) ( -3.400,   3.487)
      ( -3.300,   3.498) ( -3.200,   3.482) ( -3.100,   3.440)
      ( -3.000,   3.375) ( -2.900,   3.290) ( -2.800,   3.188)
      ( -2.700,   3.071) ( -2.600,   2.941) ( -2.500,   2.799)
      ( -2.400,   2.650) ( -2.300,   2.493) ( -2.200,   2.331)
      ( -2.100,   2.166) ( -2.000,   2.000) ( -1.900,   1.834)
      ( -1.800,   1.669) ( -1.700,   1.506) ( -1.600,   1.348)
      ( -1.500,   1.195) ( -1.400,   1.049) ( -1.300,   0.909)
      ( -1.200,   0.778) ( -1.100,   0.655) ( -1.000,   0.542)
      ( -0.900,   0.438) ( -0.800,   0.346) ( -0.700,   0.264)
      ( -0.600,   0.193) ( -0.500,   0.133) ( -0.400,   0.084)
      ( -0.300,   0.047) ( -0.200,   0.021) ( -0.100,   0.005)
      (  0.000,   0.000) (  0.100,   0.005) (  0.200,   0.019)
      (  0.300,   0.042) (  0.400,   0.074) (  0.500,   0.112)
      (  0.600,   0.157) (  0.700,   0.206) (  0.800,   0.260)
      (  0.900,   0.317) (  1.000,   0.375) (  1.100,   0.433)
      (  1.200,   0.490) (  1.300,   0.543) (  1.400,   0.591)
      (  1.500,   0.633) (  1.600,   0.666) (  1.700,   0.688)
      (  1.800,   0.697) (  1.900,   0.690) (  2.000,   0.667)
      (  2.100,   0.623) (  2.200,   0.557) (  2.300,   0.465)
      (  2.400,   0.346) (  2.500,   0.195) (  2.600,   0.011)
      (  2.700,  -0.210) (  2.800,  -0.470) (  2.900,  -0.774)
      (  3.000,  -1.125) (  3.100,  -1.526) (  3.200,  -1.980)
      (  3.300,  -2.491) (  3.400,  -3.063) (  3.500,  -3.701)
      (  3.600,  -4.406) (  3.700,  -5.185) (  3.800,  -6.041)
      (  3.900,  -6.978) (  4.000,  -8.000) (  4.100,  -8.000)
      (  4.200,  -8.000) (  4.300,  -8.000) (  4.400,  -8.000)
      (  4.500,  -8.000) (  4.600,  -8.000) (  4.700,  -8.000)
      (  4.800,  -8.000) (  4.900,  -8.000) (  5.000,  -8.000)
    };
  \end{scope}
\end{tikzpicture}
\end{center}

Teilaufgabe f)
\begin{equation*}
  \begin{split}
    f(x)&=\frac{\num{1}}{\num{60}}x^{5}-\frac{\num{1}}{\num{12}}x^{4}
    \\[1ex]
    f'(x)&=\frac{\num{1}}{\num{12}}x^{4}-\frac{\num{1}}{\num{3}}x^{3}
  \end{split}
\end{equation*}
Horizontale Tangenten besitzen die Steigung 0, also
müssen zunächst die Nullstellen der ersten Ableitung
bestimmt werden:
\begin{equation*}
  f'(x)=0
  =\frac{\num{1}}{\num{12}}x^{4}-\frac{\num{1}}{\num{3}}x^{3}
  =\frac{1}{12}x^3\cdot(x-4)
  \quad\Rightarrow\quad
  x\in\{0;4\}
\end{equation*}
Um die Extem- von den Sattelpunkten unterscheiden zu
können, nutzt man die Tatsache, dass sich in Sattelpunkten
das Monotonieverhalten einer Funktion nicht ändert.
Nullstellen der Ableitung ohne Vorzeichenwechsel
identifizieren also Sattelpunkte.
\begin{center}
  \renewcommand{\arraystretch}{1.25}%
  \begin{tabular}{r|c|c|c|c|c}
    Punkt           & $P_{1}$        & $P_{2}$       & $P_{3}$        & $P_{4}$        & $P_{5}$      \\
    \hline
    $x$             & $-\num{2}$     & $\num{0}$     & $\num{2}$      & $\num{4}$      & $\num{6}$    \\
    \hline
    $f(x)$          & $-\num{1.867}$ & $\num{0}$     & $-\num{0.8}$   & $-\num{4.267}$ & $\num{21.6}$ \\
    \hline
    $f'(x)$         & $\num{4}$      & $\num{0}$     & $-\num{1.333}$ & $\num{0}$      & $\num{36}$   \\
    \hline
    Steigung in $P$ & $\nearrow$     & $\rightarrow$ & $\searrow$     & $\rightarrow$  & $\nearrow$   \\
    \hline
    Klassifikation  &                & HP            &                & TP             &             
  \end{tabular}
\end{center}
Die Funktion $f$ besitzt in Punkt
$\left(\num{0}\;\middle|\;\num{0}\right)$ ein lokales Maximum und in Punkt
$\left(\num{4}\;\middle|\;-\frac{\num{64}}{\num{15}}\right)$ ein lokales Minimum.
\begin{center}
\begin{tikzpicture}[scale=0.500]
  % grid
  \draw[draw=black!50!white] (-4.000, -5.000) grid[step=0.5] (6.000, 5.000);
  % x-axis
  \draw[line width=0.6pt, ->, >=stealth] (-4.000, 0) -- (6.000, 0) node[below left] {\small$x$};
  % y-axis
  \draw[line width=0.6pt, ->, >=stealth] (0, -5.000) -- (0, 5.000) node[below left] {\small$y$};
  % function: f(x)=\num{0.0166667}x^{5}-\num{0.0833333}x^{4}
  \begin{scope}[line width=0.7pt]
    \clip (-4.000, -5.000) rectangle (6.000, 5.000);
    \draw plot[smooth] coordinates
    {
      ( -4.000,  -8.000) ( -3.900,  -8.000) ( -3.800,  -8.000)
      ( -3.700,  -8.000) ( -3.600,  -8.000) ( -3.500,  -8.000)
      ( -3.400,  -8.000) ( -3.300,  -8.000) ( -3.200,  -8.000)
      ( -3.100,  -8.000) ( -3.000,  -8.000) ( -2.900,  -8.000)
      ( -2.800,  -7.991) ( -2.700,  -6.820) ( -2.600,  -5.788)
      ( -2.500,  -4.883) ( -2.400,  -4.092) ( -2.300,  -3.405)
      ( -2.200,  -2.811) ( -2.100,  -2.301) ( -2.000,  -1.867)
      ( -1.900,  -1.499) ( -1.800,  -1.190) ( -1.700,  -0.933)
      ( -1.600,  -0.721) ( -1.500,  -0.548) ( -1.400,  -0.410)
      ( -1.300,  -0.300) ( -1.200,  -0.214) ( -1.100,  -0.149)
      ( -1.000,  -0.100) ( -0.900,  -0.065) ( -0.800,  -0.040)
      ( -0.700,  -0.023) ( -0.600,  -0.012) ( -0.500,  -0.006)
      ( -0.400,  -0.002) ( -0.300,  -0.001) ( -0.200,  -0.000)
      ( -0.100,  -0.000) (  0.000,   0.000) (  0.100,  -0.000)
      (  0.200,  -0.000) (  0.300,  -0.001) (  0.400,  -0.002)
      (  0.500,  -0.005) (  0.600,  -0.010) (  0.700,  -0.017)
      (  0.800,  -0.029) (  0.900,  -0.045) (  1.000,  -0.067)
      (  1.100,  -0.095) (  1.200,  -0.131) (  1.300,  -0.176)
      (  1.400,  -0.230) (  1.500,  -0.295) (  1.600,  -0.371)
      (  1.700,  -0.459) (  1.800,  -0.560) (  1.900,  -0.673)
      (  2.000,  -0.800) (  2.100,  -0.940) (  2.200,  -1.093)
      (  2.300,  -1.259) (  2.400,  -1.438) (  2.500,  -1.628)
      (  2.600,  -1.828) (  2.700,  -2.037) (  2.800,  -2.254)
      (  2.900,  -2.475) (  3.000,  -2.700) (  3.100,  -2.924)
      (  3.200,  -3.146) (  3.300,  -3.360) (  3.400,  -3.564)
      (  3.500,  -3.752) (  3.600,  -3.919) (  3.700,  -4.061)
      (  3.800,  -4.170) (  3.900,  -4.241) (  4.000,  -4.267)
      (  4.100,  -4.239) (  4.200,  -4.149) (  4.300,  -3.989)
      (  4.400,  -3.748) (  4.500,  -3.417) (  4.600,  -2.985)
      (  4.700,  -2.440) (  4.800,  -1.769) (  4.900,  -0.961)
      (  5.000,   0.000) (  5.100,   1.128) (  5.200,   2.437)
      (  5.300,   3.945) (  5.400,   5.669) (  5.500,   7.626)
      (  5.600,   8.000) (  5.700,   8.000) (  5.800,   8.000)
      (  5.900,   8.000) (  6.000,   8.000)
    };
  \end{scope}
\end{tikzpicture}
\end{center}

Teilaufgabe g)
\begin{equation*}
  \begin{split}
    f(x)&=\frac{\num{1}}{\num{25}}x^{5}-\frac{\num{5}}{\num{12}}x^{3}+\num{1}
    \\[1ex]
    f'(x)&=\frac{\num{1}}{\num{5}}x^{4}-\frac{\num{5}}{\num{4}}x^{2}
  \end{split}
\end{equation*}
Horizontale Tangenten besitzen die Steigung 0, also
müssen zunächst die Nullstellen der ersten Ableitung
bestimmt werden:
\begin{equation*}
  \begin{split}
  f'(x)&=0
  =\frac{\num{1}}{\num{5}}x^{4}-\frac{\num{5}}{\num{4}}x^{2}
  =\frac{1}{5}x^2\cdot\left(x^2-\frac{25}{4}\right)
  =\frac{1}{5}x^2\cdot\left(x-\frac{5}{2}\right)\cdot\left(x+\frac{5}{2}\right)
  \\[1ex]
  \Rightarrow\;
  x&\in\left\{-\frac{5}{2};0;\frac{5}{2}\right\}
  \end{split}
\end{equation*}
Um die Extem- von den Sattelpunkten unterscheiden zu
können, nutzt man die Tatsache, dass sich in Sattelpunkten
das Monotonieverhalten einer Funktion nicht ändert.
Nullstellen der Ableitung ohne Vorzeichenwechsel
identifizieren also Sattelpunkte.
\begin{center}
  \renewcommand{\arraystretch}{1.25}%
  \begin{tabular}{r|c|c|c|c|c|c|c}
    Punkt           & $P_{1}$         & $P_{2}$       & $P_{3}$       & $P_{4}$       & $P_{5}$       & $P_{6}$        & $P_{7}$        \\
    \hline
    $x$             & $-\num{4}$      & $-\num{2.5}$  & $-\num{1}$    & $\num{0}$     & $\num{1}$     & $\num{2.5}$    & $\num{4}$      \\
    \hline
    $f(x)$          & $-\num{13.293}$ & $\num{3.604}$ & $\num{1.377}$ & $\num{1}$     & $\num{0.623}$ & $-\num{1.604}$ & $\num{15.293}$ \\
    \hline
    $f'(x)$         & $\num{31.2}$    & $\num{0}$     & $-\num{1.05}$ & $\num{0}$     & $-\num{1.05}$ & $\num{0}$      & $\num{31.2}$   \\
    \hline
    Steigung in $P$ & $\nearrow$      & $\rightarrow$ & $\searrow$    & $\rightarrow$ & $\searrow$    & $\rightarrow$  & $\nearrow$     \\
    \hline
    Klassifikation  &                 & HP            &               & SP            &               & TP             &               
  \end{tabular}
\end{center}
Die Funktion $f$ besitzt in Punkt
$\left(-\frac{\num{5}}{\num{2}}\;\middle|\;\frac{\num{173}}{\num{48}}\right)$
ein lokales Maximum, in Punkt
$\left(\num{0}\;\middle|\;\num{1}\right)$
einen Sattelpunkt und in Punkt
$\left(\frac{\num{5}}{\num{2}}\;\middle|\;-\frac{\num{77}}{\num{48}}\right)$
ein lokales Minimum.
\begin{center}
\begin{tikzpicture}[scale=0.500]
  % grid
  \draw[draw=black!50!white] (-5.000, -4.000) grid[step=0.5] (5.000, 6.000);
  % x-axis
  \draw[line width=0.6pt, ->, >=stealth] (-5.000, 0) -- (5.000, 0) node[below left] {\small$x$};
  % y-axis
  \draw[line width=0.6pt, ->, >=stealth] (0, -4.000) -- (0, 6.000) node[below left] {\small$y$};
  % function: f(x)=\num{0.04}x^{5}-\num{0.4166667}x^{3}+\num{1}
  \begin{scope}[line width=0.7pt]
    \clip (-5.000, -4.000) rectangle (5.000, 6.000);
    \draw plot[smooth] coordinates
    {
      ( -5.000,  -7.000) ( -4.900,  -7.000) ( -4.800,  -7.000)
      ( -4.700,  -7.000) ( -4.600,  -7.000) ( -4.500,  -7.000)
      ( -4.400,  -7.000) ( -4.300,  -7.000) ( -4.200,  -7.000)
      ( -4.100,  -7.000) ( -4.000,  -7.000) ( -3.900,  -7.000)
      ( -3.800,  -7.000) ( -3.700,  -5.632) ( -3.600,  -3.746)
      ( -3.500,  -2.144) ( -3.400,  -0.798) ( -3.300,   0.320)
      ( -3.200,   1.232) ( -3.100,   1.961) ( -3.000,   2.530)
      ( -2.900,   2.958) ( -2.800,   3.263) ( -2.700,   3.462)
      ( -2.600,   3.571) ( -2.500,   3.604) ( -2.400,   3.575)
      ( -2.300,   3.495) ( -2.200,   3.375) ( -2.100,   3.225)
      ( -2.000,   3.053) ( -1.900,   2.867) ( -1.800,   2.674)
      ( -1.700,   2.479) ( -1.600,   2.287) ( -1.500,   2.103)
      ( -1.400,   1.928) ( -1.300,   1.767) ( -1.200,   1.620)
      ( -1.100,   1.490) ( -1.000,   1.377) ( -0.900,   1.280)
      ( -0.800,   1.200) ( -0.700,   1.136) ( -0.600,   1.087)
      ( -0.500,   1.051) ( -0.400,   1.026) ( -0.300,   1.011)
      ( -0.200,   1.003) ( -0.100,   1.000) (  0.000,   1.000)
      (  0.100,   1.000) (  0.200,   0.997) (  0.300,   0.989)
      (  0.400,   0.974) (  0.500,   0.949) (  0.600,   0.913)
      (  0.700,   0.864) (  0.800,   0.800) (  0.900,   0.720)
      (  1.000,   0.623) (  1.100,   0.510) (  1.200,   0.380)
      (  1.300,   0.233) (  1.400,   0.072) (  1.500,  -0.103)
      (  1.600,  -0.287) (  1.700,  -0.479) (  1.800,  -0.674)
      (  1.900,  -0.867) (  2.000,  -1.053) (  2.100,  -1.225)
      (  2.200,  -1.375) (  2.300,  -1.495) (  2.400,  -1.575)
      (  2.500,  -1.604) (  2.600,  -1.571) (  2.700,  -1.462)
      (  2.800,  -1.263) (  2.900,  -0.958) (  3.000,  -0.530)
      (  3.100,   0.039) (  3.200,   0.768) (  3.300,   1.680)
      (  3.400,   2.798) (  3.500,   4.144) (  3.600,   5.746)
      (  3.700,   7.632) (  3.800,   9.000) (  3.900,   9.000)
      (  4.000,   9.000) (  4.100,   9.000) (  4.200,   9.000)
      (  4.300,   9.000) (  4.400,   9.000) (  4.500,   9.000)
      (  4.600,   9.000) (  4.700,   9.000) (  4.800,   9.000)
      (  4.900,   9.000) (  5.000,   9.000)
    };
  \end{scope}
\end{tikzpicture}
\end{center}

Teilaufgabe h)
\begin{equation*}
  \begin{split}
    f(x)&=-\frac{\num{1}}{\num{125}}x^{5}+\frac{\num{1}}{\num{6}}x^{3}-\frac{\num{25}}{\num{16}}x
    \\[1ex]
    f'(x)&=-\frac{\num{1}}{\num{25}}x^{4}+\frac{\num{1}}{\num{2}}x^{2}-\frac{\num{25}}{\num{16}}
  \end{split}
\end{equation*}
Horizontale Tangenten besitzen die Steigung 0, also
müssen zunächst die Nullstellen der ersten Ableitung
bestimmt werden:
\begin{equation*}
  \begin{split}
  f'(x)=0&=-\frac{\num{1}}{\num{25}}x^{4}+\frac{\num{1}}{\num{2}}x^{2}-\frac{\num{25}}{\num{16}}
  \qquad|\cdot(-25)
  \\[1ex]
  \Leftrightarrow\qquad
  0&=x^4-\frac{25}{2}x^2+\frac{625}{16}
  \end{split}
\end{equation*}
Diese Gleichung löst man mit der Substitution $z\defeq x^2$:
%<OCTAVE>
\begingroup
  \newcommand{\vstrut}{\vphantom{\left(f_0^0\right)}}%
  \newcommand{\noeq}{\phantom{\Leftrightarrow}\vstrut&\quad}%
  \newcommand{\iseq}{\Leftrightarrow\vstrut&\quad}%
  \newcommand{\impl}{\Rightarrow\vstrut&\quad}%
  \newcommand{\nomod}{\quad&\phantom{|}}%
  \newcommand{\domod}[1]{\quad&|#1}%
  \begin{alignat*}{3}
    \noeq
    &
    \num{0}&=z^{2}-\frac{\num{25}}{\num{2}}z+\frac{\num{625}}{\num{16}}
    &
    \domod{\;\text{$pq$-Formel}}
    \\
    \noeq
    &
    p&=-\frac{\num{25}}{\num{2}}
    &
    \nomod
    \\
    \noeq
    &
    q&=\frac{\num{625}}{\num{16}}
    &
    \nomod
    \\
    \noeq
    &
    z_{1,2}&=-\frac{p}{2}\pm\sqrt{\left(\frac{p}{2}\right)^2-q}
    &
    \nomod
    \\
    \noeq
    &
    &=\frac{\num{25}}{\num{4}}\pm\sqrt{\left(-\frac{\num{25}}{\num{4}}\right)^2-\frac{\num{625}}{\num{16}}}
    &
    \nomod
    \\
    \noeq
    &
    &=\frac{\num{25}}{\num{4}}\pm\sqrt{\num{0}}
    &
    \nomod
    \\
    \impl
    &
    z_1&=z_2=\frac{\num{25}}{\num{4}}=\num{6.25}
    &
    \nomod
  \end{alignat*}
\endgroup
%</OCTAVE>
%myqsolve(1, -25/2, 625/16, 0, "z", [0 0 0 0 1]);
Die zugehörigen $x$-Werte erhält man dann durch
folgende Rücksubstitutionen:
\begin{equation*}
  x_{1,2}=\pm\sqrt{z_1}
  \qquad\text{und}\qquad
  x_{3,4}=\pm\sqrt{z_2}
\end{equation*}
In diesem Fall besitzt die erste Ableitung also die beiden Nullstellen:
\begin{equation*}
  x_{1,2}=x_{3,4}=\pm\sqrt{\frac{25}{4}}
  \quad\Rightarrow\quad
  x\in\left\{-\frac{5}{2};\frac{5}{2}\right\}
\end{equation*}
Um die Extem- von den Sattelpunkten unterscheiden zu
können, nutzt man die Tatsache, dass sich in Sattelpunkten
das Monotonieverhalten einer Funktion nicht ändert.
Nullstellen der Ableitung ohne Vorzeichenwechsel
identifizieren also Sattelpunkte.
\begin{center}
  \renewcommand{\arraystretch}{1.25}%
  \begin{tabular}{r|c|c|c|c|c}
    Punkt           & $P_{1}$        & $P_{2}$       & $P_{3}$        & $P_{4}$        & $P_{5}$        \\
    \hline
    $x$             & $-\num{4}$     & $-\num{2.5}$  & $\num{0}$      & $\num{2.5}$    & $\num{4}$      \\
    \hline
    $f(x)$          & $\num{3.775}$  & $\num{2.083}$ & $\num{0}$      & $-\num{2.083}$ & $-\num{3.775}$ \\
    \hline
    $f'(x)$         & $-\num{3.803}$ & $\num{0}$     & $-\num{1.562}$ & $\num{0}$      & $-\num{3.803}$ \\
    \hline
    Steigung in $P$ & $\searrow$     & $\rightarrow$ & $\searrow$     & $\rightarrow$  & $\searrow$     \\
    \hline
    Klassifikation  &                & SP            &                & SP             &               
  \end{tabular}
\end{center}
Die Funktion $f$ besitzt in den Punkten
$\left(-\frac{\num{5}}{\num{2}}\;\middle|\;\frac{\num{25}}{\num{12}}\right)$
und
$\left(\frac{\num{5}}{\num{2}}\;\middle|\;-\frac{\num{25}}{\num{12}}\right)$
jeweils einen Sattelpunkt.
\begin{center}
\begin{tikzpicture}[scale=0.500]
  % grid
  \draw[draw=black!50!white] (-5.000, -5.000) grid[step=0.5] (5.000, 5.000);
  % x-axis
  \draw[line width=0.6pt, ->, >=stealth] (-5.000, 0) -- (5.000, 0) node[below left] {\small$x$};
  % y-axis
  \draw[line width=0.6pt, ->, >=stealth] (0, -5.000) -- (0, 5.000) node[below left] {\small$y$};
  % function: f(x)=-\num{0.008}x^{5}+\num{0.1666667}x^{3}-\num{1.5625}x
  \begin{scope}[line width=0.7pt]
    \clip (-5.000, -5.000) rectangle (5.000, 5.000);
    \draw plot[smooth] coordinates
    {
      ( -5.000,   8.000) ( -4.900,   8.000) ( -4.800,   8.000)
      ( -4.700,   8.000) ( -4.600,   7.442) ( -4.500,   6.606)
      ( -4.400,   5.871) ( -4.300,   5.228) ( -4.200,   4.670)
      ( -4.100,   4.188) ( -4.000,   3.775) ( -3.900,   3.425)
      ( -3.800,   3.131) ( -3.700,   2.887) ( -3.600,   2.686)
      ( -3.500,   2.525) ( -3.400,   2.397) ( -3.300,   2.298)
      ( -3.200,   2.223) ( -3.100,   2.169) ( -3.000,   2.132)
      ( -2.900,   2.107) ( -2.800,   2.093) ( -2.700,   2.086)
      ( -2.600,   2.084) ( -2.500,   2.083) ( -2.400,   2.083)
      ( -2.300,   2.081) ( -2.200,   2.075) ( -2.100,   2.064)
      ( -2.000,   2.048) ( -1.900,   2.024) ( -1.800,   1.992)
      ( -1.700,   1.951) ( -1.600,   1.901) ( -1.500,   1.842)
      ( -1.400,   1.773) ( -1.300,   1.695) ( -1.200,   1.607)
      ( -1.100,   1.510) ( -1.000,   1.404) ( -0.900,   1.289)
      ( -0.800,   1.167) ( -0.700,   1.038) ( -0.600,   0.902)
      ( -0.500,   0.761) ( -0.400,   0.614) ( -0.300,   0.464)
      ( -0.200,   0.311) ( -0.100,   0.156) (  0.000,   0.000)
      (  0.100,  -0.156) (  0.200,  -0.311) (  0.300,  -0.464)
      (  0.400,  -0.614) (  0.500,  -0.761) (  0.600,  -0.902)
      (  0.700,  -1.038) (  0.800,  -1.167) (  0.900,  -1.289)
      (  1.000,  -1.404) (  1.100,  -1.510) (  1.200,  -1.607)
      (  1.300,  -1.695) (  1.400,  -1.773) (  1.500,  -1.842)
      (  1.600,  -1.901) (  1.700,  -1.951) (  1.800,  -1.992)
      (  1.900,  -2.024) (  2.000,  -2.048) (  2.100,  -2.064)
      (  2.200,  -2.075) (  2.300,  -2.081) (  2.400,  -2.083)
      (  2.500,  -2.083) (  2.600,  -2.084) (  2.700,  -2.086)
      (  2.800,  -2.093) (  2.900,  -2.107) (  3.000,  -2.132)
      (  3.100,  -2.169) (  3.200,  -2.223) (  3.300,  -2.298)
      (  3.400,  -2.397) (  3.500,  -2.525) (  3.600,  -2.686)
      (  3.700,  -2.887) (  3.800,  -3.131) (  3.900,  -3.425)
      (  4.000,  -3.775) (  4.100,  -4.188) (  4.200,  -4.670)
      (  4.300,  -5.228) (  4.400,  -5.871) (  4.500,  -6.606)
      (  4.600,  -7.442) (  4.700,  -8.000) (  4.800,  -8.000)
      (  4.900,  -8.000) (  5.000,  -8.000)
    };
  \end{scope}
\end{tikzpicture}
\end{center}
    % </OUTCOME>
  \fi
\end{exercise}

\begin{exercise}
      {ID-3cbc9dea2421a4c62fafeb3cac3bde9a82da7caa}
      {Stausee}
  \ifproblem\problem\par
    Die Wassermenge eines Stausees ändert sich, indem Wasser entweder
    hineinfließt oder herausläuft. Die Zuflussratenfunktion ist für den
    hier betrachteten Stausee gegeben durch:
    \begin{equation*}
      z(x)=\left(x^2-10x+24\right)\cdot e^{\frac{1}{2}x}
    \end{equation*}
    Dabei wird $x$ in Tagen und $z(x)$ in \num{1000} Kubikmetern pro Tag angegeben.
    Betrachtet wird das Intervall $0\leq x\leq\num{6.5}$.
    Folgende Abbildung zeigt den Graphen von $z(x)$, der zur besseren
    Veranschaulichung in $y$-Richtung mit den Faktor \num{0.125} gestaucht wurde:
    \begin{center}
      \begin{tikzpicture}
        % Koordinatensystem
        \draw[line width=0.6pt, ->, >=stealth] (-1.0,  0.0) -- (7.0, 0.0) node[below]      {{\small$x$}};
        \draw[line width=0.6pt, ->, >=stealth] ( 0.0, -2.0) -- (0.0, 4.0) node[below left] {{\small$y$}};
        % Ursprung
        \node[below left] at (0, 0) {{\small$0$}};
        % Graph
        \begin{scope}
          \clip (0.000, -2.000) rectangle (6.000, 3.500);
          \draw[line width=0.5pt] plot[smooth] coordinates
          {
            (  0.000,   3.000)
            (  0.100,   3.024)
            (  0.200,   3.045)
            (  0.300,   3.063)
            (  0.400,   3.078)
            (  0.500,   3.090)
            (  0.600,   3.098)
            (  0.700,   3.102)
            (  0.800,   3.103)
            (  0.900,   3.099)
            (  1.000,   3.091)
            (  1.100,   3.079)
            (  1.200,   3.061)
            (  1.300,   3.039)
            (  1.400,   3.011)
            (  1.500,   2.977)
            (  1.600,   2.938)
            (  1.700,   2.892)
            (  1.800,   2.841)
            (  1.900,   2.783)
            (  2.000,   2.718)
            (  2.100,   2.647)
            (  2.200,   2.569)
            (  2.300,   2.483)
            (  2.400,   2.390)
            (  2.500,   2.291)
            (  2.600,   2.183)
            (  2.700,   2.069)
            (  2.800,   1.946)
            (  2.900,   1.817)
            (  3.000,   1.681)
            (  3.100,   1.537)
            (  3.200,   1.387)
            (  3.300,   1.230)
            (  3.400,   1.067)
            (  3.500,   0.899)
            (  3.600,   0.726)
            (  3.700,   0.549)
            (  3.800,   0.368)
            (  3.900,   0.185)
            (  4.000,   0.000)
            (  4.100,  -0.184)
            (  4.200,  -0.367)
            (  4.300,  -0.547)
            (  4.400,  -0.722)
            (  4.500,  -0.889)
            (  4.600,  -1.047)
            (  4.700,  -1.193)
            (  4.800,  -1.323)
            (  4.900,  -1.434)
            (  5.000,  -1.523)
            (  5.100,  -1.585)
            (  5.200,  -1.616)
            (  5.300,  -1.610)
            (  5.400,  -1.562)
            (  5.500,  -1.466)
            (  5.600,  -1.316)
            (  5.700,  -1.102)
            (  5.800,  -0.818)
            (  5.900,  -0.454)
            (  6.000,   0.000)
          };
        \end{scope}
      \end{tikzpicture}
    \end{center}
    Ohne eigene Herleitung dürfen Sie für die zweite und dritte Ableitung
    im Weiteren folgende Funktionen verwenden:
    \begin{equation*}
      \begin{split}
         z''(x)&=\left(\frac{1}{4}x^2-\frac{1}{2}x-2\right)\cdot e^{\frac{1}{2}x} \\[2ex]
        z'''(x)&=\left(\frac{1}{8}x^2+\frac{1}{4}x-\frac{3}{2}\right)\cdot e^{\frac{1}{2}x}
      \end{split}
    \end{equation*}
    \begin{enumerate}[a)]
      \item Berechnen Sie die Zeitpunkte, zu denen das Wasser weder ein- noch abfließt.
            Geben Sie die Zeitintervalle an, in denen Wasser zu- bzw. abläuft.
      \item Bestimmen Sie den Zeitpunkt, zu welchem die Zuflussrate im betrachteten
            Intervall maximal ist.
      \item Welche Aussagen über die Änderung der Wassermenge zum Zeitpunkt
            $x=5$ sind möglich?
      \item Bestimmen Sie den Zeitpunkt, zu dem sich die Zuflussrate am stärksten ändert.
      \item Entscheiden Sie, ob es einen Zeitpunkt gibt, zu dem sich im Stausee wieder
            die Anfangswassermenge befindet. Die Begründung soll ohne Rechnung erfolgen.
      \item Im Stausee ist eine bestimmte Bakteriensorte aufgetreten.
            Zum Zeitpunkt $x=0$ befinden sich bereits \num{5000} Bakterien im Stausee.
            Die Wachstumsratenfunktion der Bakterien ist gegeben durch:
            \begin{equation*}
              w(x)=x^3-12x^2+35x
            \end{equation*}
            Dabei wird $x$ wieder in Tagen angegeben und $w(x)$ in \num{10000} Bakterien pro Tag.
            Ermitteln Sie die Anzahl der Bakterien, die nach 3 Tagen im Stausee vorhanden sind.
    \end{enumerate}
  \fi
  %\ifoutline\outline\par
  %\fi
  %\ifoutcome\outcome\par
  %\fi
\end{exercise}

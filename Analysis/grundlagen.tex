\begin{exercise}
      {ID-6bd5c0ce3c9ca56f18b875a693dc8ab24de09529}
      {Grundlagen}
  \ifproblem\problem\par
    % <PROBLEM>
    Gegeben sei die Funktion $f:\mathbb{R}\to\mathbb{R}$ mit:
    \begin{equation*}
      %<OCTAVE>
      f(x)=\frac{\num{1}}{\num{4}}x^{4}-x^{3}-\frac{\num{1}}{\num{2}}x^{2}+\num{3}x
      %</OCTAVE>
      %p = [1/4 0 -2 0 2];
      %q = mypolyshift(p, 1, -1/4);
      %printf("f(x)=%s\n", mypolystr(q, "x", [0 0 0 0 1]));
    \end{equation*}
    \begin{enumerate}[a)]
      \item Bestätigen Sie durch eine Rechnung, dass die
            Funktion $f$ bei $x=2$ eine Nullstelle besitzt.
      \item Bestimmen Sie auch alle anderen Nullstellen von $f$.
      \item Besitzt die Funktion $f$ an der Stelle $x=1$
            einen Hoch-, Tief- oder Sattelpunkt?
      \item Bestimmen Sie auch alle anderen Extrem- und
            Sattelpunkte von $f$.
      \item Ermitteln Sie die Wendepunkte von $f$.
      \item Geben Sie für jeden Wendepunkt die Gleichung
            der Wendetangente an.
    \end{enumerate}
    % </PROBLEM>
  \fi
  %\ifoutline\outline\par
    % <OUTLINE>
    % </OUTLINE>
  %\fi
  %\ifoutcome\outcome\par
    % <OUTCOME>
    % </OUTCOME>
  %\fi
\end{exercise}

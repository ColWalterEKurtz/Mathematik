\begin{exercise}
      {ID-ed0edbda2fd9ce5c413eded318ea97887781e700}
      {Schmuggler}
  \ifproblem\problem\par
    % <PROBLEM>
    Unter 11 Reisenden befinden sich 4 Schmuggler. Ein Zollbeamter wählt drei
    Personen zur Kontrolle aus. Wie groß ist die Wahrscheinlichkeit, dass sich
    alle drei als Schmuggler entpuppen?
    % </PROBLEM>
  \fi
  %\ifoutline\outline\par
    % <OUTLINE>
    % </OUTLINE>
  %\fi
  \ifoutcome\outcome\par
    % <OUTCOME>
    Wenn man die Schmuggler mit \glqq S\grqq{} und
    die normalen Touristen mit \glqq T\grqq{} abkürzt,
    lässt sich das Zufallsexperiment mit folgendem
    Baumdiagramm beschreiben:
    \begin{center}
      %<OCTAVE>
      \begin{tikzpicture}[line width=0.6pt]
        % tree
        \begin{scope}
          % some default colors
          \newcommand{\colr}{Red};%
          \newcommand{\colg}{ForestGreen};%
          \newcommand{\colb}{Cerulean};%
          \newcommand{\coly}{YellowOrange};%
          \newcommand{\cola}{Black!35!White};%
          \newcommand{\cole}{Black!55!White};%
          % size settings
          \newcommand{\radius}{3mm}%
          \newcommand{\xscale}{4}%
          \newcommand{\yscale}{4}%
          % background color of nodes
          \newcommand{\colora}{white}%
          \newcommand{\colorb}{white}%
          % default node text
          \newcommand{\ntexta}{S}%
          \newcommand{\ntextb}{T}%
          % default edge text
          \newcommand{\etexta}{}%
          \newcommand{\etextb}{}%
          % geometry
          \coordinate (Z)   at ( 3.500*\xscale*\radius,  3.000*\yscale*\radius);
          \coordinate (A)   at ( 1.500*\xscale*\radius,  2.000*\yscale*\radius);
          \coordinate (B)   at ( 5.500*\xscale*\radius,  2.000*\yscale*\radius);
          \coordinate (AA)  at ( 0.500*\xscale*\radius,  1.000*\yscale*\radius);
          \coordinate (AB)  at ( 2.500*\xscale*\radius,  1.000*\yscale*\radius);
          \coordinate (BA)  at ( 4.500*\xscale*\radius,  1.000*\yscale*\radius);
          \coordinate (BB)  at ( 6.500*\xscale*\radius,  1.000*\yscale*\radius);
          \coordinate (AAA) at ( 0.000*\xscale*\radius,  0.000*\yscale*\radius);
          \coordinate (AAB) at ( 1.000*\xscale*\radius,  0.000*\yscale*\radius);
          \coordinate (ABA) at ( 2.000*\xscale*\radius,  0.000*\yscale*\radius);
          \coordinate (ABB) at ( 3.000*\xscale*\radius,  0.000*\yscale*\radius);
          \coordinate (BAA) at ( 4.000*\xscale*\radius,  0.000*\yscale*\radius);
          \coordinate (BAB) at ( 5.000*\xscale*\radius,  0.000*\yscale*\radius);
          \coordinate (BBA) at ( 6.000*\xscale*\radius,  0.000*\yscale*\radius);
          \coordinate (BBB) at ( 7.000*\xscale*\radius,  0.000*\yscale*\radius);
          % edges
          \draw  (Z) -- (A);
          \draw  (Z) -- (B);
          \draw  (A) -- (AA);
          \draw  (A) -- (AB);
          \draw  (B) -- (BA);
          \draw  (B) -- (BB);
          \draw (AA) -- (AAA);
          \draw (AA) -- (AAB);
          \draw (AB) -- (ABA);
          \draw (AB) -- (ABB);
          \draw (BA) -- (BAA);
          \draw (BA) -- (BAB);
          \draw (BB) -- (BBA);
          \draw (BB) -- (BBB);
          % root
          \fill[fill=black] (Z) circle[radius=2pt];
          % nodes
          \filldraw[fill=\colora, draw=black] (A)   circle[radius=\radius] node{\ntexta};
          \filldraw[fill=\colorb, draw=black] (B)   circle[radius=\radius] node{\ntextb};
          \filldraw[fill=\colora, draw=black] (AA)  circle[radius=\radius] node{\ntexta};
          \filldraw[fill=\colorb, draw=black] (AB)  circle[radius=\radius] node{\ntextb};
          \filldraw[fill=\colora, draw=black] (BA)  circle[radius=\radius] node{\ntexta};
          \filldraw[fill=\colorb, draw=black] (BB)  circle[radius=\radius] node{\ntextb};
          \filldraw[fill=\colora, draw=black] (AAA) circle[radius=\radius] node{\ntexta};
          \filldraw[fill=\colorb, draw=black] (AAB) circle[radius=\radius] node{\ntextb};
          \filldraw[fill=\colora, draw=black] (ABA) circle[radius=\radius] node{\ntexta};
          \filldraw[fill=\colorb, draw=black] (ABB) circle[radius=\radius] node{\ntextb};
          \filldraw[fill=\colora, draw=black] (BAA) circle[radius=\radius] node{\ntexta};
          \filldraw[fill=\colorb, draw=black] (BAB) circle[radius=\radius] node{\ntextb};
          \filldraw[fill=\colora, draw=black] (BBA) circle[radius=\radius] node{\ntexta};
          \filldraw[fill=\colorb, draw=black] (BBB) circle[radius=\radius] node{\ntextb};
          % label macros
          \newcommand{\rlabel}[4]%
          {%
            \coordinate (TEMP) at ($(#1)!0.5!(#2)$);
            \coordinate (TEMP) at ($(TEMP)!#3!270:(#2)$);
            \node at (TEMP) {#4};
          }%
          \newcommand{\llabel}[4]{\rlabel{#2}{#1}{#3}{#4}};
          % edge labels
          \rlabel{Z}{A}{4mm}{$\frac{4}{11}$};
          \llabel{Z}{B}{4mm}{$\frac{7}{11}$};
          \rlabel{A}{AA}{3.5mm}{$\frac{3}{10}$};
          \llabel{A}{AB}{3.5mm}{$\frac{7}{10}$};
          \rlabel{B}{BA}{3.5mm}{$\frac{4}{10}$};
          \llabel{B}{BB}{3.5mm}{$\frac{6}{10}$};
          \rlabel{AA}{AAA}{3mm}{$\frac{2}{9}$};
          \llabel{AA}{AAB}{3mm}{$\frac{7}{9}$};
          \rlabel{AB}{ABA}{3mm}{$\frac{3}{9}$};
          \llabel{AB}{ABB}{3mm}{$\frac{6}{9}$};
          \rlabel{BA}{BAA}{3mm}{$\frac{3}{9}$};
          \llabel{BA}{BAB}{3mm}{$\frac{6}{9}$};
          \rlabel{BB}{BBA}{3mm}{$\frac{4}{9}$};
          \llabel{BB}{BBB}{3mm}{$\frac{5}{9}$};
        \end{scope}
      \end{tikzpicture}
      %</OCTAVE>
      %mytree(2, 3)
    \end{center}
    Zur Beantwortung der Frage reicht schon ein
    Pfad des Baumdiagramms aus:
    \begin{equation*}
      \begin{split}
        E&=\text{\glqq alle drei sind Schmuggler\grqq}
        =\{(\text{SSS})\}
        \\[1ex]
        P(E)&=\frac{4}{11}\cdot\frac{3}{10}\cdot\frac{2}{9}
        =\frac{4}{165}
        \approx\num{0.024}
        =\pc{2.4}
        %rats(4/11 * 3/10 * 2/9)
      \end{split}
    \end{equation*}
    Mit etwa \pc{2.4}-iger Wahrscheinlichkeit entpuppen
    sich alle drei ausgewählten Reisenden als Schmuggler.
    % </OUTCOME>
  \fi
\end{exercise}

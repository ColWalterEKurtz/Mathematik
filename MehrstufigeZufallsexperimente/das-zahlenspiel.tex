% 2021-11-10
\begin{exercise}
      {ID-24161ee77a4665a7a943c0a0da8fdf751fe347f8}
      {Das Zahlenspiel}
  \ifproblem\problem\par
    % <PROBLEM>
    Bei einem Spiel wird eine Urne eingesetzt, in der
    sich 4 Kugeln mit der Aufschrift \glqq 0\grqq,
    2 Kugeln mit der Aufschrift \glqq 1\grqq, 3 Kugeln
    mit der Aufschrift \glqq 2\grqq{} und eine Kugel
    mit der Aufschrift \glqq 3\grqq{} befinden.
    Man zahlt \eur{1} als Einsatz und zieht zufällig
    eine Kugel. Dann erhält man der Beschriftung
    entsprechend Geld zurück. Zieht man z.\,B. eine
    Kugel mit aufgedruckter \glqq 2\grqq{} erhält man
    \eur{2} zurück.
    \begin{enumerate}[a)]
      \item Gib den Ergebnisraum $\Omega$ an.
      \item Gib den Ereignisraum $\mathcal{P}(\Omega)$ an.
      \item Berechne die Mächtigkeit des Ereignisraums.
      \item Gib eine Wahrscheinlichkeitsverteilung für
            das Spiel an.
      \item Überprüfe das Spiel auf Fairness.
    \end{enumerate}
    Ein anderes Spiel nutzt die gleiche Urne mit den
    gleichen Kugeln, es darf aber zweimal gezogen werden.
    \begin{enumerate}[a)]
      \setcounter{enumi}{5}
      \item Zeichne ein Baumdiagramm.
      \item Berechne die Wahrscheinlichkeiten für folgende
            Ereignisse:
            \begin{enumerate}[i.]
              \item Es werden nur Kuglen mit der Aufschrift
                    \glqq 0\grqq{} gezogen.
              \item Es wird mindestens eine Kugel mit
                    der Aufschrift \glqq 0\grqq{}
                    gezogen.
              \item Es wird keine Kugel mit der
                    Aufschrift \glqq 0\grqq{}
                    gezogen.
            \end{enumerate}
    \end{enumerate}
    % </PROBLEM>
  \fi
  %\ifoutline\outline\par
    % <OUTLINE>
    % </OUTLINE>
  %\fi
  %\ifoutcome\outcome\par
    % <OUTCOME>
    % </OUTCOME>
  %\fi
\end{exercise}

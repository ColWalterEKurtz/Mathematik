\begin{exercise}
      {ID-cedaf88441a4867bf4957ed340dce2b0f55b3e28}
      {Socken}
  \ifproblem\problem\par
    % <PROBLEM>
    In einem dunklen Zimmer befinden sich in einer Schublade 4 blaue, 2 rote
    und 6 weiße Socken. Vier werden zufällig ausgewählt. Wie groß ist die
    Wahrscheinlichkeit dafür, dass zwei Paare mit gleicher Farbe gewählt werden?
    % </PROBLEM>
  \fi
  %\ifoutline\outline\par
    % <OUTLINE>
    % </OUTLINE>
  %\fi
  \ifoutcome\outcome\par
    % <OUTCOME>
    Das vollständige Baumdiagramm zu diesem
    Zufallsexperiment besteht aus $3^4=81$
    Pfaden, d.\,h. zum Zeichnen ist es viel
    zu groß. Um die gefragte Wahrscheinlichkeit
    zu bestimme, ist es also sinnvoll sich
    auf die relevanten Pfade zu beschränken.
    \par
    Zwei gleichfarbige Paare erhält man auf
    folgenden 20 Pfaden:
    \begin{equation*}
      \begin{split}
        A&=\{(\text{bbbb});(\text{wwww})\}
        \\
        B_1&=\{(\text{bbrr});(\text{brrb});(\text{brbr})\}
        \\
        B_2&=\{(\text{rrbb});(\text{rbbr});(\text{rbrb})\}
        \\
        C_1&=\{(\text{bbww});(\text{bwwb});(\text{bwbw})\}
        \\
        C_2&=\{(\text{wwbb});(\text{wbbw});(\text{wbwb})\}
        \\
        D_1&=\{(\text{rrww});(\text{rwwr});(\text{rwrw})\}
        \\
        D_2&=\{(\text{wwrr});(\text{wrrw});(\text{wrwr})\}
      \end{split}
    \end{equation*}
    Alle Pfadwahrscheinlichkeiten haben den Nenner:
    \begin{equation*}
      12\cdot11\cdot10\cdot9=\num{11880}
      %12*11*10*9
    \end{equation*}
    Wegen des Kommutativgesetzes besitzen alle
    Pfade in $B_1$ und $B_2$ denselben Zähler:
    \begin{equation*}
      4\cdot3\cdot2\cdot1=\num{24}
      %4*3*2*1
    \end{equation*}
    Gleiches gilt für die Pfade in $C_1$ und $C_2$:
    \begin{equation*}
      4\cdot3\cdot6\cdot5=\num{360}
      %4*3*6*5
    \end{equation*}
    Sowie für die Pfade in $D_1$ und $D_2$:
    \begin{equation*}
      2\cdot1\cdot6\cdot5=\num{60}
      %2*1*6*5
    \end{equation*}
    Damit lässt sich die Wahrscheinlichkeit für
    zwei gleichfarbige Paare berechnen:
    \begin{equation*}
      \begin{split}
        E&=A\cup B_1\cup B_2\cup C_1\cup C_2\cup D_1\cup D_2
        \\[1ex]
        P(E)&=\frac{4}{12}\cdot\frac{3}{11}\cdot\frac{2}{10}\cdot\frac{1}{9}
             +\frac{6}{12}\cdot\frac{5}{11}\cdot\frac{4}{10}\cdot\frac{3}{9}
             +6\cdot\frac{24}{\num{11880}}
             +6\cdot\frac{360}{\num{11880}}
             +6\cdot\frac{60}{\num{11880}}
        \\[1ex]
        &=\frac{\num{3048}}{\num{11880}}
        =\frac{127}{495}
        \approx\num{0.257}
        =\pc{25.7}
        %a = 4*3*2*1 + 6*5*4*3 + 6*24 + 6*360 + 6*60
        %b = 12*11*10*9
        %rats(a/b)
        %a/b
        %a/b * 100
      \end{split}
    \end{equation*}
    Mit ca. \pc{25.7}-iger Wahrscheinlichkeit sind
    bei vier zufällig ausgewählten Socken zwei
    gleichfarbige Paare dabei.
    % </OUTCOME>
  \fi
\end{exercise}

\begin{exercise}
      {ID-017ccdbff65746aeb6f2aa3c7f24983702ff5fa9}
      {Ballett}
  \ifproblem\problem\par
    % <PROBLEM>
    Von den 500 Schülerinnen und Schülern einer Schule sind 350 Mädchen.
    Für Ballett interessieren sich \pc{80} der Mädchen, aber nur \pc{10}
    der Jungen. Eine Freikarte für eine Ballettaufführung wird ausgelost.
    Mit welcher Wahrscheinlichkeit erhält sie
    \begin{enumerate}[a)]
      \squeeze
      \item ein ballettinteressiertes Mädchen?
      \item ein an Ballett uninteressiertes Mädchen?
      \item ein ballettinteressierter Junge?
      \item ein an Ballett uninteressierter Junge?
    \end{enumerate}
    % </PROBLEM>
  \fi
  %\ifoutline\outline\par
    % <OUTLINE>
    % </OUTLINE>
  %\fi
  \ifoutcome\outcome
    % <OUTCOME>
    \begin{center}
      %<OCTAVE>
      \begin{tikzpicture}[line width=0.6pt]
        % tree
        \begin{scope}
          % some default colors
          \newcommand{\colr}{Red};%
          \newcommand{\colg}{ForestGreen};%
          \newcommand{\colb}{Cerulean};%
          \newcommand{\coly}{YellowOrange};%
          \newcommand{\cola}{Black!35!White};%
          \newcommand{\cole}{Black!55!White};%
          % size settings
          \newcommand{\radius}{3mm}%
          \newcommand{\xscale}{5}%
          \newcommand{\yscale}{4}%
          % background color of nodes
          \newcommand{\colora}{white}%
          \newcommand{\colorb}{white}%
          % default node text
          \newcommand{\ntexta}{}%
          \newcommand{\ntextb}{}%
          % default edge text
          \newcommand{\etexta}{}%
          \newcommand{\etextb}{}%
          % geometry
          \coordinate (Z)  at ( 1.500*\xscale*\radius,  2.000*\yscale*\radius);
          \coordinate (A)  at ( 0.500*\xscale*\radius,  1.000*\yscale*\radius);
          \coordinate (B)  at ( 2.500*\xscale*\radius,  1.000*\yscale*\radius);
          \coordinate (AA) at ( 0.000*\xscale*\radius,  0.000*\yscale*\radius);
          \coordinate (AB) at ( 1.000*\xscale*\radius,  0.000*\yscale*\radius);
          \coordinate (BA) at ( 2.000*\xscale*\radius,  0.000*\yscale*\radius);
          \coordinate (BB) at ( 3.000*\xscale*\radius,  0.000*\yscale*\radius);
          % edges
          \draw (Z) -- (A);
          \draw (Z) -- (B);
          \draw (A) -- (AA);
          \draw (A) -- (AB);
          \draw (B) -- (BA);
          \draw (B) -- (BB);
          % root
          \fill[fill=black] (Z) circle[radius=2pt];
          % nodes
          \filldraw[fill=\colora, draw=black] (A)  circle[radius=\radius] node{J};
          \filldraw[fill=\colorb, draw=black] (B)  circle[radius=\radius] node{M};
          \filldraw[fill=\colora, draw=black] (AA) circle[radius=\radius] node{i};
          \filldraw[fill=\colorb, draw=black] (AB) circle[radius=\radius] node{u};
          \filldraw[fill=\colora, draw=black] (BA) circle[radius=\radius] node{i};
          \filldraw[fill=\colorb, draw=black] (BB) circle[radius=\radius] node{u};
          % label macros
          \newcommand{\rlabel}[4]%
          {%
            \coordinate (TEMP) at ($(#1)!0.5!(#2)$);
            \coordinate (TEMP) at ($(TEMP)!#3!270:(#2)$);
            \node at (TEMP) {#4};
          }%
          \newcommand{\llabel}[4]{\rlabel{#2}{#1}{#3}{#4}};
          % edge labels
          \rlabel{Z}{A}{4mm}{$\frac{150}{500}$};
          \llabel{Z}{B}{4mm}{$\frac{350}{500}$};
          \rlabel{A}{AA}{5mm}{\rule{0pt}{2.5ex}\pc{10}};
          \llabel{A}{AB}{5mm}{\rule{0pt}{2.5ex}\pc{90}};
          \rlabel{B}{BA}{5mm}{\rule{0pt}{2.5ex}\pc{80}};
          \llabel{B}{BB}{5mm}{\rule{0pt}{2.5ex}\pc{20}};
        \end{scope}
      \end{tikzpicture}
      %</OCTAVE>
      %mytree(2,2)
    \end{center}
    \begin{enumerate}[a)]
      \squeeze
      \item Wahrscheinlichkeit für ein ballettinteressiertes Mädchen:
            \begin{equation*}
              P(\text{M,i})
              =\frac{350}{500}\cdot\pc{80}
              =\frac{350}{500}\cdot\frac{80}{100}
              =\frac{14}{25}
              =\num{0.56}
              =\pc{56}
              %rats(350/500 * 80/100)
            \end{equation*}
      \item Wahrscheinlichkeit für ein an Ballett uninteressiertes Mädchen:
            \begin{equation*}
              P(\text{M,u})
              =\frac{350}{500}\cdot\pc{20}
              =\frac{350}{500}\cdot\frac{20}{100}
              =\frac{7}{50}
              =\num{0.14}
              =\pc{14}
              %rats(350/500 * 20/100)
            \end{equation*}
      \item Wahrscheinlichkeit für einen ballettinteressierten Jungen:
            \begin{equation*}
              P(\text{J,i})
              =\frac{150}{500}\cdot\pc{10}
              =\frac{150}{500}\cdot\frac{10}{100}
              =\frac{3}{100}
              =\num{0.03}
              =\pc{3}
              %rats(150/500 * 10/100)
            \end{equation*}
      \item Wahrscheinlichkeit für einen an Ballett uninteressierten Jungen:
            \begin{equation*}
              P(\text{J,u})
              =\frac{150}{500}\cdot\pc{90}
              =\frac{150}{500}\cdot\frac{90}{100}
              =\frac{27}{100}
              =\num{0.27}
              =\pc{27}
              %rats(150/500 * 90/100)
            \end{equation*}
    \end{enumerate}
    % </OUTCOME>
  \fi
\end{exercise}

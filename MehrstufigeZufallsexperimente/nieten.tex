\begin{exercise}
      {ID-56645a0929240ebd444b862b2b7e53fc2444a06b}
      {Nieten}
  \ifproblem\problem\par
    % <PROBLEM>
    In einem Karton liegen 50 Lose, davon sind 5 Gewinne, der Rest Nieten.
    \xxa{} zieht zwei Lose. Mit welcher Wahrscheinlichkleit zieht \xxa{}
    zwei Nieten?
    % </PROBLEM>
  \fi
  %\ifoutline\outline\par
    % <OUTLINE>
    % </OUTLINE>
  %\fi
  \ifoutcome\outcome\par
    % <OUTCOME>
    Die gezogenen Lose werden nach dem Ziehen
    nicht wieder zurück in den Karton gelegt.
    Also ändern sich nach jedem Zug die Wahrscheinlichkeiten.
    \begin{center}
      %<OCTAVE>
      \begin{tikzpicture}[line width=0.6pt]
        % tree
        \begin{scope}
          % some default colors
          \newcommand{\colr}{Red};%
          \newcommand{\colg}{ForestGreen};%
          \newcommand{\colb}{Cerulean};%
          \newcommand{\coly}{YellowOrange};%
          \newcommand{\cola}{Black!35!White};%
          \newcommand{\cole}{Black!55!White};%
          % size settings
          \newcommand{\radius}{3mm}%
          \newcommand{\xscale}{4}%
          \newcommand{\yscale}{4}%
          % background color of nodes
          \newcommand{\colora}{white}%
          \newcommand{\colorb}{white}%
          % default node text
          \newcommand{\ntexta}{G}%
          \newcommand{\ntextb}{N}%
          % default edge text
          \newcommand{\etexta}{}%
          \newcommand{\etextb}{}%
          % geometry
          \coordinate (Z)  at ( 1.500*\xscale*\radius,  2.000*\yscale*\radius);
          \coordinate (A)  at ( 0.500*\xscale*\radius,  1.000*\yscale*\radius);
          \coordinate (B)  at ( 2.500*\xscale*\radius,  1.000*\yscale*\radius);
          \coordinate (AA) at ( 0.000*\xscale*\radius,  0.000*\yscale*\radius);
          \coordinate (AB) at ( 1.000*\xscale*\radius,  0.000*\yscale*\radius);
          \coordinate (BA) at ( 2.000*\xscale*\radius,  0.000*\yscale*\radius);
          \coordinate (BB) at ( 3.000*\xscale*\radius,  0.000*\yscale*\radius);
          % edges
          \draw (Z) -- (A);
          \draw (Z) -- (B);
          \draw (A) -- (AA);
          \draw (A) -- (AB);
          \draw (B) -- (BA);
          \draw (B) -- (BB);
          % root
          \fill[fill=black] (Z) circle[radius=2pt];
          % nodes
          \filldraw[fill=\colora, draw=black] (A)  circle[radius=\radius] node{\ntexta};
          \filldraw[fill=\colorb, draw=black] (B)  circle[radius=\radius] node{\ntextb};
          \filldraw[fill=\colora, draw=black] (AA) circle[radius=\radius] node{\ntexta};
          \filldraw[fill=\colorb, draw=black] (AB) circle[radius=\radius] node{\ntextb};
          \filldraw[fill=\colora, draw=black] (BA) circle[radius=\radius] node{\ntexta};
          \filldraw[fill=\colorb, draw=black] (BB) circle[radius=\radius] node{\ntextb};
          % label macros
          \newcommand{\rlabel}[4]%
          {%
            \coordinate (TEMP) at ($(#1)!0.5!(#2)$);
            \coordinate (TEMP) at ($(TEMP)!#3!270:(#2)$);
            \node at (TEMP) {#4};
          }%
          \newcommand{\llabel}[4]{\rlabel{#2}{#1}{#3}{#4}};
          % edge labels
          \rlabel{Z}{A}{3mm}{$\frac{5}{50}$};
          \llabel{Z}{B}{3mm}{$\frac{45}{50}$};
          \rlabel{A}{AA}{3mm}{$\frac{4}{49}$};
          \llabel{A}{AB}{3mm}{$\frac{45}{49}$};
          \rlabel{B}{BA}{3mm}{$\frac{5}{49}$};
          \llabel{B}{BB}{3mm}{$\frac{44}{49}$};
        \end{scope}
      \end{tikzpicture}
      %</OCTAVE>
      %mytree(2,2)
    \end{center}
    Summen- und Pfadregel (Produktregel) gelten unverändert.
    Damit lässt sich die Wahrscheinlichkeit für das Ziehen
    zweier Nieten wie folgt berechnen:
    \begin{equation*}
      \begin{split}
        E&=\text{\glqq zweimal Niete\grqq}
        =\{(\text{NN})\}
        \\[1ex]
        P(E)&=\frac{45}{50}\cdot\frac{44}{49}
        =\frac{9}{10}\cdot\frac{44}{49}
        =\frac{9}{5}\cdot\frac{22}{49}
        =\frac{198}{245}
        \approx\num{0.808}
        \approx\pc{81}
      \end{split}
    \end{equation*}
    \xxa{} zieht also mit etwa \pc{81}-iger
    Wahrscheinlichkeit zwei Nieten.
    % </OUTCOME>
  \fi
\end{exercise}

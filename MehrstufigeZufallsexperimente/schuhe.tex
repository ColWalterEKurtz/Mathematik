\begin{exercise}
      {ID-e836cefc276c6d753e050cd699cd82eff0dbac10}
      {Schuhe}
  \ifproblem\problem\par
    % <PROBLEM>
    In einem dunklen Keller befinden sich in einem
    Schrank 2 weiße, 6 schwarze und 8 braune Schuhe.
    Zwei Schuhe werden zufällig ausgewählt. Bereche
    die Wahrscheinlichkeit dafür, dass beide Schuhe
    die gleiche Farbe haben.
    % </PROBLEM>
  \fi
  %\ifoutline\outline\par
    % <OUTLINE>
    % </OUTLINE>
  %\fi
  \ifoutcome\outcome
    % <OUTCOME>
    \begin{center}
      %<OCTAVE>
      \begin{tikzpicture}[line width=0.6pt]
        % tree
        \begin{scope}
          % some default colors
          \newcommand{\colr}{Red};%
          \newcommand{\colg}{ForestGreen};%
          \newcommand{\colb}{Cerulean};%
          \newcommand{\coly}{YellowOrange};%
          \newcommand{\cola}{Black!35!White};%
          \newcommand{\cole}{Black!55!White};%
          % size settings
          \newcommand{\radius}{3mm}%
          \newcommand{\xscale}{4}%
          \newcommand{\yscale}{5}%
          % background color of nodes
          \newcommand{\colora}{white}%
          \newcommand{\colorb}{white}%
          \newcommand{\colorc}{white}%
          % default node text
          \newcommand{\ntexta}{w}%
          \newcommand{\ntextb}{s}%
          \newcommand{\ntextc}{b}%
          % default edge text
          \newcommand{\etexta}{}%
          \newcommand{\etextb}{}%
          \newcommand{\etextc}{}%
          % geometry
          \coordinate (Z)  at ( 4.000*\xscale*\radius,  2.000*\yscale*\radius);
          \coordinate (A)  at ( 1.000*\xscale*\radius,  1.000*\yscale*\radius);
          \coordinate (B)  at ( 4.000*\xscale*\radius,  1.000*\yscale*\radius);
          \coordinate (C)  at ( 7.000*\xscale*\radius,  1.000*\yscale*\radius);
          \coordinate (AA) at ( 0.000*\xscale*\radius,  0.000*\yscale*\radius);
          \coordinate (AB) at ( 1.000*\xscale*\radius,  0.000*\yscale*\radius);
          \coordinate (AC) at ( 2.000*\xscale*\radius,  0.000*\yscale*\radius);
          \coordinate (BA) at ( 3.000*\xscale*\radius,  0.000*\yscale*\radius);
          \coordinate (BB) at ( 4.000*\xscale*\radius,  0.000*\yscale*\radius);
          \coordinate (BC) at ( 5.000*\xscale*\radius,  0.000*\yscale*\radius);
          \coordinate (CA) at ( 6.000*\xscale*\radius,  0.000*\yscale*\radius);
          \coordinate (CB) at ( 7.000*\xscale*\radius,  0.000*\yscale*\radius);
          \coordinate (CC) at ( 8.000*\xscale*\radius,  0.000*\yscale*\radius);
          % edges
          \draw (Z) -- (A);
          \draw (Z) -- (B);
          \draw (Z) -- (C);
          \draw (A) -- (AA);
          \draw (A) -- (AB);
          \draw (A) -- (AC);
          \draw (B) -- (BA);
          \draw (B) -- (BB);
          \draw (B) -- (BC);
          \draw (C) -- (CA);
          \draw (C) -- (CB);
          \draw (C) -- (CC);
          % root
          \fill[fill=black] (Z) circle[radius=2pt];
          % nodes
          \filldraw[fill=\colora, draw=black] (A)  circle[radius=\radius] node{\ntexta};
          \filldraw[fill=\colorb, draw=black] (B)  circle[radius=\radius] node{\ntextb};
          \filldraw[fill=\colorc, draw=black] (C)  circle[radius=\radius] node{\ntextc};
          \filldraw[fill=\colora, draw=black] (AA) circle[radius=\radius] node{\ntexta};
          \filldraw[fill=\colorb, draw=black] (AB) circle[radius=\radius] node{\ntextb};
          \filldraw[fill=\colorc, draw=black] (AC) circle[radius=\radius] node{\ntextc};
          \filldraw[fill=\colora, draw=black] (BA) circle[radius=\radius] node{\ntexta};
          \filldraw[fill=\colorb, draw=black] (BB) circle[radius=\radius] node{\ntextb};
          \filldraw[fill=\colorc, draw=black] (BC) circle[radius=\radius] node{\ntextc};
          \filldraw[fill=\colora, draw=black] (CA) circle[radius=\radius] node{\ntexta};
          \filldraw[fill=\colorb, draw=black] (CB) circle[radius=\radius] node{\ntextb};
          \filldraw[fill=\colorc, draw=black] (CC) circle[radius=\radius] node{\ntextc};
          % label macros
          \newcommand{\rlabel}[4]%
          {%
            \coordinate (TEMP) at ($(#1)!0.5!(#2)$);
            \coordinate (TEMP) at ($(TEMP)!#3!270:(#2)$);
            \node at (TEMP) {#4};
          }%
          \newcommand{\llabel}[4]{\rlabel{#2}{#1}{#3}{#4}};
          % edge labels
          \rlabel{Z}{A}{3.5mm}{$\frac{2}{16}$};
          \llabel{Z}{B}{2mm}{$\frac{6}{16}$};
          \llabel{Z}{C}{3.5mm}{$\frac{8}{16}$};
          \rlabel{A}{AA}{3.5mm}{$\frac{1}{15}$};
          \llabel{A}{AB}{2mm}{$\frac{6}{15}$};
          \llabel{A}{AC}{3.5mm}{$\frac{8}{15}$};
          \rlabel{B}{BA}{3.5mm}{$\frac{2}{15}$};
          \llabel{B}{BB}{2mm}{$\frac{5}{15}$};
          \llabel{B}{BC}{3.5mm}{$\frac{8}{15}$};
          \rlabel{C}{CA}{3.5mm}{$\frac{2}{15}$};
          \llabel{C}{CB}{2mm}{$\frac{6}{15}$};
          \llabel{C}{CC}{3.5mm}{$\frac{7}{15}$};
        \end{scope}
      \end{tikzpicture}
      %</OCTAVE>
      %mytree(3, 2)
    \end{center}
    Das zu untersuchende Ereignis besteht aus drei Pfaden:
    \begin{equation*}
      \begin{split}
        E&=\text{\glqq zwei gleichfarbige Schuge\grqq}
        =\{(\text{ww});(\text{ss});(\text{bb})\}
        \\[1ex]
        P(E)&=\frac{2}{16}\cdot\frac{1}{15}
             +\frac{6}{16}\cdot\frac{5}{15}
             +\frac{8}{16}\cdot\frac{7}{15}
        =\frac{1}{120}+\frac{1}{8}+\frac{7}{30}
        =\frac{11}{30}
        \approx\num{0.367}
        =\pc{36.7}
        %rats(2/16 * 1/15)
        %rats(6/16 * 5/15)
        %rats(8/16 * 7/15)
        %rats(1/120 + 1/8 + 7/30)
      \end{split}
    \end{equation*}
    Mit etwa \pc{36.7}-iger Wahrscheinlichkeit
    entnimmt man dem Schrank zwei gleichfarbige
    Schuhe.
    % </OUTCOME>
  \fi
\end{exercise}

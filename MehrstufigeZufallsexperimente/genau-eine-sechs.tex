% 2021-11-16
\begin{exercise}
      {ID-c74ee3f12b8ca075cbc7ae00fa6cdb3a5f6ded88}
      {Genau eine Sechs}
  % dicepair
  \newcommand{\dicepair}[2]%
  {%
    \begin{scope}[xshift=-0.75\dicewidth]
      \dicenum{#1}%
    \end{scope}
    \node at (0, 0)
    {%
      \raisebox{-0.5\dicewidth}[0pt][0pt]{,}%
    };
    \begin{scope}[xshift=0.85\dicewidth]
      \dicenum{#2}%
    \end{scope}
  }%
  % mathdice
  \newcommand{\mathdice}[2]
  {%
    \Big(%
    \text
    {%
      \raisebox{-0.6ex}{%
      \begin{tikzpicture}%
        \dicepair{#1}{#2}%
      \end{tikzpicture}}%
    }%
    \Big)%
  }%
  \ifproblem\problem\par
    % <PROBLEM>
    Du hast mit einem Würfelbecher zwei Würfel
    geworfen. Ein Mitspieler schaut unter den
    Becher und verrät dir, dass beide Zahlen
    gerade sind.
    \par
    Mit welcher Wahrscheinlichkeit ist genau
    eine der beiden Zahlen eine Sechs?
    % </PROBLEM>
  \fi
  %\ifoutline\outline\par
    % <OUTLINE>
    % </OUTLINE>
  %\fi
  \ifoutcome\outcome\par
    % <OUTCOME>
    Die Ergebnismenge $\Omega$ beim Werfen von
    zwei Würfeln besteht aus folgenden Paaren:
    \begin{equation*}
      \setlength{\arraycolsep}{0.1em}%
      \begin{array}{rccccl}
        \Omega=\Big\{%
        \mathdice{1}{1}, & \mathdice{1}{2}, & \mathdice{1}{3}, & \mathdice{1}{4}, & \mathdice{1}{5}, & \mathdice{1}{6}, \\
        \mathdice{2}{1}, & \mathdice{2}{2}, & \mathdice{2}{3}, & \mathdice{2}{4}, & \mathdice{2}{5}, & \mathdice{2}{6}, \\
        \mathdice{3}{1}, & \mathdice{3}{2}, & \mathdice{3}{3}, & \mathdice{3}{4}, & \mathdice{3}{5}, & \mathdice{3}{6}, \\
        \mathdice{4}{1}, & \mathdice{4}{2}, & \mathdice{4}{3}, & \mathdice{4}{4}, & \mathdice{4}{5}, & \mathdice{4}{6}, \\
        \mathdice{5}{1}, & \mathdice{5}{2}, & \mathdice{5}{3}, & \mathdice{5}{4}, & \mathdice{5}{5}, & \mathdice{5}{6}, \\
        \mathdice{6}{1}, & \mathdice{6}{2}, & \mathdice{6}{3}, & \mathdice{6}{4}, & \mathdice{6}{5}, & \mathdice{6}{6}\Big\}
      \end{array}
    \end{equation*}
    Das Ereignis \glqq beide Zahlen sind
    gerade\grqq{} wird durch folgende Teilmenge
    $A\subset\Omega$ beschrieben:
    \begin{equation*}
      \setlength{\arraycolsep}{0.1em}%
      \begin{array}{rcl}
        A=\Big\{%
        \mathdice{2}{2}, & \mathdice{2}{4}, & \mathdice{2}{6}, \\
        \mathdice{4}{2}, & \mathdice{4}{4}, & \mathdice{4}{6}, \\
        \mathdice{6}{2}, & \mathdice{6}{4}, & \mathdice{6}{6}\Big\}
      \end{array}
    \end{equation*}
    Da jedes Ergebnis mit der gleichen
    Wahrscheinlichkeit eintritt,
    folgt die Wahrscheinlichkeit des Ereignisses
    $A$ den Regeln der Laplace-Wahrscheinlichkeit:
    \begin{equation*}
      P(A)=\frac{|A|}{|\Omega|}=\frac{9}{36}=\frac{1}{4}
    \end{equation*}
    Das Ereignis \glqq genau ein Würfel zeigt eine
    Sechs\grqq{} wird durch folgende Teilmenge
    $B\subset\Omega$ beschrieben:
    \begin{equation*}
      \setlength{\arraycolsep}{0.1em}%
      \begin{array}{rcccl}
        B=\Big\{%
        \mathdice{1}{6}, & \mathdice{2}{6}, & \mathdice{3}{6}, & \mathdice{4}{6}, & \mathdice{5}{6}, \\
        \mathdice{6}{1}, & \mathdice{6}{2}, & \mathdice{6}{3}, & \mathdice{6}{4}, & \mathdice{6}{5}\Big\}
      \end{array}
    \end{equation*}
    Das Ereignis $A\cap B$ kann mit \glqq beide Würfel
    zeigen eine gerade Zahl und genau eine der beiden
    Zahlen ist eine Sechs\grqq{} übersetzt werden.
    Als Menge enthält dieses Ereignis folgende Ergebnisse:
    \begin{equation*}
      \setlength{\arraycolsep}{0.1em}%
      \begin{array}{rccl}
        A\cap B=\Big\{%
        \mathdice{2}{6}, & \mathdice{4}{6}, & \mathdice{6}{2}, & \mathdice{6}{4}\Big\}
      \end{array}
    \end{equation*}
    Auch für dieses Ereignis ergibt sich die
    Wahrscheinlichkeit nach \textit{Laplace}:
    \begin{equation*}
      P(A\cap B)=\frac{|A\cap B|}{|\Omega|}
      =\frac{4}{36}
      =\frac{1}{9}
    \end{equation*}
    Da der Mitspieler allerdings unter den
    Würfelbecher geschaut und verraten hat,
    dass Ereignis $A$ bereits eingetreten ist,
    beschreibt $P(A\cap B)$ nicht die hier
    gesuchte Wahrscheinlichkeit.
    Die hier gesuchte Wahrscheinlichkeit ist
    die Wahrscheinlichkeit für das Werfen genau
    einer Sechs unter der Voraussetzung, dass
    beide Würfel eine gerade Zahl zeigen, und
    ergibt sich damit als bedingte Wahrscheinlichkeit
    nach folgendem Zusammenhang:
    \begin{equation*}
      P_{A}(B)=\frac{P(A\cap B)}{P(A)}
      =\frac{1/9}{1/4}
      =\frac{4}{9}
      \approx\SI{44.44}{\percent}
    \end{equation*}
    % </OUTCOME>
  \fi
\end{exercise}

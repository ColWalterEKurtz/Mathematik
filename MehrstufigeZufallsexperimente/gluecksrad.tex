\begin{exercise}
      {ID-13ad0ca691a242e47d568c79816250e9c8f26cda}
      {Glücksrad}
  \ifproblem\problem\par
    % <PROBLEM>
    Für ein Spiel wird ein Glücksrad verwendet, das drei farbigen Sektoren
    besitzt. Der Tabelle können die Farben der Sektoren und die Größen der
    zugehörigen Mittelpunktswinkel entnommen werden.
    \begin{center}
      \renewcommand{\arraystretch}{1.25}%
      \begin{tabular}{|l|c|c|c|}
        \hline
        Farbe
        & \makebox[4em][c]{Blau}
        & \makebox[4em][c]{Rot}
        & \makebox[4em][c]{Grün}
        \\
        \hline
        Mittelpunktswinkel
        & $180^\circ$
        & $120^\circ$
        & $60^\circ$
        \\
        \hline
      \end{tabular}
    \end{center}
    Für einen Einsatz von 5 Euro darf ein Spieler das Glücksrad dreimal
    drehen. Erzielt der Spieler dreimal die gleiche Farbe, werden ihm 10
    Euro ausgezahlt. Erzielt er drei verschiedene Farben, wird ein anderer
    Betrag ausgezahlt. In allen anderen Fällen erfolgt keine Auszahlung.
    \begin{enumerate}[a)]
      \item Bestimme die Wahrscheinlichkeit der folgenden Ereignisse:
            \begin{equation*}
              \begin{split}
                E_1:\;\;&\text{\glqq drei gleiche Farben\grqq}
                \\
                E_2:\;\;&\text{\glqq drei verschiedene Farben\grqq}
              \end{split}
            \end{equation*}
      \item Bei dem Spiel ist zu erwarten, dass sich die Einsätze der Spieler
            und die Auszahlungen auf lange Sicht ausgleichen. Bestimme den
            Betrag, der ausgezahlt wird, wenn drei verschiedene Farben
            erscheinen.
    \end{enumerate}
    % </PROBLEM>
  \fi
  %\ifoutline\outline\par
    % <OUTLINE>
    % </OUTLINE>
  %\fi
  %\ifoutcome\outcome\par
    % <OUTCOME>
    % </OUTCOME>
  %\fi
\end{exercise}

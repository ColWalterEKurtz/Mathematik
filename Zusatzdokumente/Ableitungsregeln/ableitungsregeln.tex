% Vorlage und globale Optionen
\documentclass
[
  draft    = true,
  fontsize = 11pt,
  parskip  = half-,
  BCOR     = 0pt,
  DIV      = 11,
  ngerman
]
{scrartcl}

% Standardpakete
\usepackage{fixltx2e}
\usepackage[utf8]{inputenc}
\usepackage[T1]{fontenc}
\usepackage{lmodern}
\usepackage{babel}
% Zusatzpakete
\usepackage{amsmath}

\pagestyle{empty}

% ------------------------------------------------------------------------------
\begin{document}
% ------------------------------------------------------------------------------

% --------------------------
\paragraph{Ableitungsregeln}
% --------------------------
\begingroup
  \newcommand{\vstrut}{\rule[-3ex]{0pt}{7ex}}
  \begin{alignat*}{4}
    \text{Konstante Funktionen:\vstrut}     &\qquad & f(x)&=c                 & \quad&\quad & f'(x)&=0                                                          \\
    \text{Potenzregel:\vstrut}              &\qquad & f(x)&=x^{n}             & \quad&\quad & f'(x)&=n\cdot x^{n-1}                                             \\
    \text{Faktorregel:\vstrut}              &\qquad & f(x)&=a\cdot g(x)       & \quad&\quad & f'(x)&=a\cdot g'(x)                                               \\
    \text{Summenregel:\vstrut}              &\qquad & f(x)&=g(x)\pm h(x)      & \quad&\quad & f'(x)&=g'(x)\pm h'(x)                                             \\
    \text{Produktregel:\vstrut}             &\qquad & f(x)&=g(x)\cdot h(x)    & \quad&\quad & f'(x)&=g'(x)\cdot h(x)+g(x)\cdot h'(x)                            \\
    \text{Quotientenregel:\vstrut}          &\qquad & f(x)&=\frac{g(x)}{h(x)} & \quad&\quad & f'(x)&=\frac{g'(x)\cdot h(x)-g(x)\cdot h'(x)}{\big[h(x)\big]^{2}} \\
    \text{Kettenregel:\vstrut}              &\qquad & f(x)&=g(h(x))           & \quad&\quad & f'(x)&=g'(h(x))\cdot h'(x)
  \end{alignat*}
\endgroup

% ------------------------------
\paragraph{Spezielle Funktionen}
% ------------------------------
\begingroup
  \newcommand{\separator}{\qquad&\text{\rule[-3ex]{0pt}{7ex}}\qquad}
  \begin{alignat*}{3}
    f(x)&=\sqrt[n]{x} & \separator & f'(x)&=\frac{1}{n\cdot\sqrt[n]{x^{n-1}}} \\
    f(x)&=\sin(x)     & \separator & f'(x)&=\cos(x)                           \\
    f(x)&=\cos(x)     & \separator & f'(x)&=-\sin(x)                          \\
    f(x)&=\tan(x)     & \separator & f'(x)&=\frac{1}{\cos^{2}(x)}             \\
    f(x)&=e^{x}       & \separator & f'(x)&=e^{x}                             \\
    f(x)&=a^{x}       & \separator & f'(x)&=a^{x}\cdot\ln(a)                  \\
    f(x)&=\ln(x)      & \separator & f'(x)&=\frac{1}{x}                       \\
    f(x)&=\log_{a}(x) & \separator & f'(x)&=\frac{1}{x}\cdot\frac{1}{\ln(a)}
  \end{alignat*}
\endgroup

% ------------------------------------------------------------------------------
\end{document}
% ------------------------------------------------------------------------------

\clearpage
% ----------------------------------------------------
\paragraph{Beispiel 1} \textit{(Potenzfunktionen)}\par
% ----------------------------------------------------
Um die Ableitung einer Potenzfunktion zu bilden, benötigt man
die \emph{Faktorregel} und die \emph{Potenzregel}.

Als Erstes spaltet man von der abzuleitenden Potenzunktion
\begin{align*}
   f(x)&=-5x^{4}
\intertext{ihren Koeffizienten als konstanten Faktor ab.
           In diesem Beispiel also}
      a&=-5\quad\text{und} \\
   g(x)&=x^{4}
\intertext{Dann gilt nach \emph{Faktorregel}}
  f'(x)&=a\cdot g'(x)
\intertext{wobei sich die Ableitung von $g$ unmittelbar aus der
           \emph{Potenzregel} ergibt:}
  g(x)&=x^{4}\quad\Rightarrow\quad g'(x)=4x^{3}
\intertext{Zusammen mit dem konstanten Faktor $a=-5$ erhält man als
           Ableitung von $f$:}
  f'(x)&=-5\cdot4x^{3}=-20x^{3}
\end{align*}

% --------------------------------------------
\paragraph{Beispiel 2} \textit{(Polynome)}\par
% --------------------------------------------
Um die Ableitung eines Polynoms (einer ganzrationalen Funktion) zu bilden,
benötigt man die \emph{Summenregel}, die \emph{Potenzregel}, die
\emph{Faktorregel} und die \emph{Regel zum Ableiten konstanter Funktionen}.

Als Erstes zerlegt man das abzuleitende Polynom
\begin{align*}
   f(x)&=x^{3}-3x^{2}+4x-7
\intertext{in seine jeweiligen Potenzfunktionen. In diesem Beispiel also}
   f(x)&=g(x)-h(x)+i(x)-j(x)\quad\text{mit} \\[1ex]
   g(x)&=x^{3}                              \\
   h(x)&=3x^{2}                             \\
   i(x)&=4x\quad\text{und}                  \\
   j(x)&=7
\intertext{Dann gilt nach \emph{Summenregel}:}
  f'(x)&=g'(x)-h'(x)+i'(x)-j'(x)
\end{align*}

Die Ableitungen $g'(x)$ bis $j'(x)$ kann man nun einzeln bilden, indem man die
\emph{Potenzregel}, die \emph{Faktorregel} und die \emph{Regel zum Ableiten
konstanter Funktionen} anwendet:
\begin{alignat*}{4}
   g(x)&=x^{3}  & \quad&\Rightarrow\quad & g'(x)&=3x^{2} & \qquad&\text{\emph{Potenzregel}}                        \\
   h(x)&=3x^{2} & \quad&\Rightarrow\quad & h'(x)&=6x     & \qquad&\text{\emph{Faktorregel} und \emph{Potenzregel}} \\
   i(x)&=4x     & \quad&\Rightarrow\quad & i'(x)&=4      & \qquad&\text{\emph{Faktorregel} und \emph{Potenzregel}} \\
   j(x)&=7      & \quad&\Rightarrow\quad & j'(x)&=0      & \qquad&\text{\emph{\glqq Konstante Funktion\grqq}}
\end{alignat*}

Zum Schluss setzt man die einzelnen Ableitungen gemäß \emph{Summenregel}
zusammen und erhält als Ableitung von $f$:
\begin{equation*}
  f'(x)=3x^{2}-6x+4
\end{equation*}

% --------------------------------------------
\paragraph{Beispiel 3} \textit{(Produkte)}\par
% --------------------------------------------

% ------------------------------------------------------------------------------
\end{document}
% ------------------------------------------------------------------------------


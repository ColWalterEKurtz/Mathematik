\documentclass
[
  draft    = true,
  fontsize = 11pt,
  parskip  = half-,
  BCOR     = 0pt,
  DIV      = calc,
  dvipsnames
]
{scrartcl}

%\usepackage
%[
%  paperwidth  = 402.2965pt,
%  paperheight = 260.18562pt,
%  top         = 0pt,
%  left        = 0pt,
%  right       = 0pt,
%  bottom      = 0pt
%]
%{geometry}

% Standardpakete
\usepackage{fixltx2e}
\usepackage[utf8]{inputenc}
\usepackage[T1]{fontenc}
\usepackage{lmodern}
\usepackage[ngerman]{babel}
% Zusatzpakete
\usepackage{amsmath}
\usepackage{amssymb}
\usepackage{enumerate}
\usepackage{ifthen}
\usepackage{graphicx}
\usepackage{tikz}
\usepackage{xcolor}
% TikZ-Bibliotheken (alphabetisch)
\usetikzlibrary{arrows, calc, decorations.pathmorphing,
                decorations.pathreplacing, decorations.shapes,
                decorations.text, intersections, patterns, shapes}

% ------
% sizeof
% ------
%
% Calculates the width and height of an image:
%
% \sizeof{%
% \begin{tikzpicture}
%   ...
% \end{tikzpicture}}
%
\newcommand{\sizeof}[1]
{%
\begingroup
\setbox0=\hbox{#1}%
\setlength  {\dimen0}{\wd0}%
\setlength  {\dimen1}{\ht0}%
\addtolength{\dimen1}{\dp0}%
\usebox0%
\typeout{}%
\typeout{\string\sizeof{...} at the end of line \the\inputlineno:}%
\typeout{}%
\typeout{\string\usepackage}%
\typeout{[}%
\typeout{ paperwidth = \the\dimen0,}%
\typeout{ paperheight = \the\dimen1,}%
\typeout{ top = 0pt,}%
\typeout{ left = 0pt,}%
\typeout{ right = 0pt,}%
\typeout{ bottom = 0pt}%
\typeout{]}%
\typeout{{geometry}}%
\typeout{}%
\endgroup
}

\pagestyle{empty}

% widths of the related minipages
\newcommand{\mathwidth}{0.29}
\newcommand{\codewidth}{0.70}

% ------------------------------------------------------------------------------
\begin{document}
% ------------------------------------------------------------------------------

% -----------------------------
\section{Vektoren und Matrizen}
% -----------------------------
\begin{minipage}{\mathwidth\textwidth}
  $\displaystyle
  a=\begin{pmatrix}
      1 & 2 & 3
    \end{pmatrix}
  $
\end{minipage}%
\hfill
\begin{minipage}{\codewidth\textwidth}
\begin{verbatim}
a = [1 2 3]
\end{verbatim}
\end{minipage}\bigskip

\begin{minipage}{\mathwidth\textwidth}
  $\displaystyle
  a=\begin{pmatrix}
      1 \\
      2 \\
      3
    \end{pmatrix}
  $
\end{minipage}%
\hfill
\begin{minipage}{\codewidth\textwidth}
\begin{verbatim}
a = [1 2 3]'
\end{verbatim}
\end{minipage}\bigskip

\begin{minipage}{\mathwidth\textwidth}
  $\displaystyle
  a=\begin{pmatrix}
      0 & 0 & 0 \\
      0 & 0 & 0 \\
      0 & 0 & 0
    \end{pmatrix}
  $
\end{minipage}%
\hfill
\begin{minipage}{\codewidth\textwidth}
\begin{verbatim}
a = zeros(3, 3)
\end{verbatim}
\end{minipage}\bigskip

\begin{minipage}{\mathwidth\textwidth}
  $\displaystyle
  a=\begin{pmatrix}
      1 & 0 & 0 \\
      0 & 1 & 0 \\
      0 & 0 & 1
    \end{pmatrix}
  $
\end{minipage}%
\hfill
\begin{minipage}{\codewidth\textwidth}
\begin{verbatim}
a = diag(ones(3, 1))
\end{verbatim}
\end{minipage}\bigskip

\begin{minipage}{\mathwidth\textwidth}
  $\displaystyle
  a=\begin{pmatrix}
      1 & 2 & 3 \\
      4 & 5 & 6 \\
      7 & 8 & 9
    \end{pmatrix}
  $
\end{minipage}%
\hfill
\begin{minipage}{\codewidth\textwidth}
\begin{verbatim}
a = [1 2 3; 4 5 6; 7 8 9]
\end{verbatim}
\end{minipage}\bigskip

\begin{minipage}{\mathwidth\textwidth}
  $\displaystyle
  a=\begin{pmatrix}
      1 \\
      2 \\
      3
    \end{pmatrix}
  $
\end{minipage}%
\hfill
\begin{minipage}{\codewidth\textwidth}
\begin{verbatim}
size(a)    => [3 1]
size(a, 1) =>  3
size(a, 2) =>  1
\end{verbatim}
\end{minipage}\bigskip

\begin{minipage}{\mathwidth\textwidth}
  $\displaystyle
  a_{ij}
  $
\end{minipage}%
\hfill
\begin{minipage}{\codewidth\textwidth}
\begin{verbatim}
a(i, j)
\end{verbatim}
\end{minipage}\bigskip

\begin{minipage}{\mathwidth\textwidth}
  $\displaystyle
  a_{i,1\ldots n}
  $
\end{minipage}%
\hfill
\begin{minipage}{\codewidth\textwidth}
\begin{verbatim}
a(i, :)
\end{verbatim}
\end{minipage}\bigskip

\begin{minipage}{\mathwidth\textwidth}
  $\displaystyle
  a_{1\ldots m,j}
  $
\end{minipage}%
\hfill
\begin{minipage}{\codewidth\textwidth}
\begin{verbatim}
a(:, j)
\end{verbatim}
\end{minipage}\bigskip

% ------------------------------------------------------------------------------
\end{document}
% ------------------------------------------------------------------------------


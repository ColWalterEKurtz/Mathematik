% layout and global options
\documentclass
[
  fontsize = 11pt,
  parskip  = half-,
  BCOR     = 0pt,
  DIV      = 11,
  ngerman
]
{scrartcl}

% default packages
\usepackage[utf8]{inputenc}
\usepackage[T1]{fontenc}
\usepackage{lmodern}
\usepackage{babel}
% extra packages
\usepackage{amsmath}
\usepackage{amssymb}
\usepackage{siunitx}

%\usepackage{enumerate}
%\usepackage{ifthen}
%\usepackage{graphicx}
%\usepackage{tikz}
%\usepackage{xcolor}
% TikZ-Bibliotheken (alphabetisch)
%\usetikzlibrary{arrows, calc, decorations.pathmorphing,
%                decorations.pathreplacing, decorations.shapes,
%                decorations.text, intersections, patterns, shapes}

\pagestyle{empty}

% ------------------------------------------------------------------------------
\begin{document}
% ------------------------------------------------------------------------------

% math width
\newlength{\mw}%
\setlength{\mw}{0.40\textwidth}%
% code width
\newlength{\cw}%
\setlength{\cw}{0.59\textwidth}%

% ------------------------
\section{Grundrechenarten}
% ------------------------
\begin{minipage}{\mw}
  \begin{equation*}
    \vphantom{\bigg(}%
    5+3
  \end{equation*}
\end{minipage}%
\hfill
\begin{minipage}{\cw}
\begin{verbatim}
5 + 3
\end{verbatim}
\end{minipage}
\begin{minipage}{\mw}
  \begin{equation*}
    \vphantom{\bigg(}%
    5-3
  \end{equation*}
\end{minipage}%
\hfill
\begin{minipage}{\cw}
\begin{verbatim}
5 - 3
\end{verbatim}
\end{minipage}
\begin{minipage}{\mw}
  \begin{equation*}
    \vphantom{\bigg(}%
    5\cdot3
  \end{equation*}
\end{minipage}%
\hfill
\begin{minipage}{\cw}
\begin{verbatim}
5 * 3
\end{verbatim}
\end{minipage}
\begin{minipage}{\mw}
  \begin{equation*}
    \vphantom{\bigg(}%
    5:3
  \end{equation*}
\end{minipage}%
\hfill
\begin{minipage}{\cw}
\begin{verbatim}
5 / 3
\end{verbatim}
\end{minipage}

% ----------------
\section{Potenzen}
% ----------------
\begin{minipage}{\mw}
  \begin{equation*}
    \vphantom{\bigg(}%
    5^3
  \end{equation*}
\end{minipage}%
\hfill
\begin{minipage}{\cw}
\begin{verbatim}
5^3
power(5, 3)
\end{verbatim}
\end{minipage}
\begin{minipage}{\mw}
  \begin{equation*}
    \vphantom{\bigg(}%
    e^{3}
  \end{equation*}
\end{minipage}%
\hfill
\begin{minipage}{\cw}
\begin{verbatim}
exp(3)
\end{verbatim}
\end{minipage}

% ---------------
\section{Wurzeln}
% ---------------
\begin{minipage}{\mw}
  \begin{equation*}
    \vphantom{\bigg(}%
    \sqrt{9}
  \end{equation*}
\end{minipage}%
\hfill
\begin{minipage}{\cw}
\begin{verbatim}
sqrt(9)
\end{verbatim}
\end{minipage}
\begin{minipage}{\mw}
  \begin{equation*}
    \vphantom{\bigg(}%
    \sqrt[3]{8}
  \end{equation*}
\end{minipage}%
\hfill
\begin{minipage}{\cw}
\begin{verbatim}
nthroot(8, 3)
\end{verbatim}
\end{minipage}

% -------------------
\section{Logarithmen}
% -------------------
\begin{minipage}{\mw}
  \begin{equation*}
    \vphantom{\bigg(}%
    \ln(e)
  \end{equation*}
\end{minipage}%
\hfill
\begin{minipage}{\cw}
\begin{verbatim}
log(e)
\end{verbatim}
\end{minipage}
\begin{minipage}{\mw}
  \begin{equation*}
    \vphantom{\bigg(}%
    \operatorname{ld}(32)
  \end{equation*}
\end{minipage}%
\hfill
\begin{minipage}{\cw}
\begin{verbatim}
log2(32)
\end{verbatim}
\end{minipage}
\begin{minipage}{\mw}
  \begin{equation*}
    \vphantom{\bigg(}%
    \log_{10}(\num{100})
  \end{equation*}
\end{minipage}%
\hfill
\begin{minipage}{\cw}
\begin{verbatim}
log10(100)
\end{verbatim}
\end{minipage}

% ------------------------
\section{Winkelfunktionen}
% ------------------------
\begin{minipage}{\mw}
  \begin{equation*}
    \vphantom{\bigg(}%
    \sin(90^\circ)
  \end{equation*}
\end{minipage}%
\hfill
\begin{minipage}{\cw}
\begin{verbatim}
sind(90)
\end{verbatim}
\end{minipage}
\begin{minipage}{\mw}
  \begin{equation*}
    \vphantom{\bigg(}%
    \sin\left(\frac{\pi}{2}\right)
  \end{equation*}
\end{minipage}%
\hfill
\begin{minipage}{\cw}
\begin{verbatim}
sin(pi/2)
\end{verbatim}
\end{minipage}

\begin{minipage}{\mw}
  \begin{equation*}
    \vphantom{\bigg(}%
    \cos(60^\circ)
  \end{equation*}
\end{minipage}%
\hfill
\begin{minipage}{\cw}
\begin{verbatim}
cosd(60)
\end{verbatim}
\end{minipage}
\begin{minipage}{\mw}
  \begin{equation*}
    \vphantom{\bigg(}%
    \cos\left(\frac{\pi}{3}\right)
  \end{equation*}
\end{minipage}%
\hfill
\begin{minipage}{\cw}
\begin{verbatim}
cos(pi/3)
\end{verbatim}
\end{minipage}

\begin{minipage}{\mw}
  \begin{equation*}
    \vphantom{\bigg(}%
    \tan(45^\circ)
  \end{equation*}
\end{minipage}%
\hfill
\begin{minipage}{\cw}
\begin{verbatim}
tand(45)
\end{verbatim}
\end{minipage}
\begin{minipage}{\mw}
  \begin{equation*}
    \vphantom{\bigg(}%
    \tan\left(\frac{\pi}{4}\right)
  \end{equation*}
\end{minipage}%
\hfill
\begin{minipage}{\cw}
\begin{verbatim}
tan(pi/4)
\end{verbatim}
\end{minipage}

\begin{minipage}{\mw}
  \begin{equation*}
    \vphantom{\bigg(}%
    \arcsin(1)
  \end{equation*}
\end{minipage}%
\hfill
\begin{minipage}{\cw}
\begin{verbatim}
asind(1) => 90
asin(1)  => pi/2
\end{verbatim}
\end{minipage}

\begin{minipage}{\mw}
  \begin{equation*}
    \vphantom{\bigg(}%
    \arccos(\num{0.5})
  \end{equation*}
\end{minipage}%
\hfill
\begin{minipage}{\cw}
\begin{verbatim}
acosd(0.5) => 60
acos(0.5)  => pi/3
\end{verbatim}
\end{minipage}

\begin{minipage}{\mw}
  \begin{equation*}
    \vphantom{\bigg(}%
    \arctan(1)
  \end{equation*}
\end{minipage}%
\hfill
\begin{minipage}{\cw}
\begin{verbatim}
atand(1) => 45
atan(1)  => pi/4
\end{verbatim}
\end{minipage}

% --------------------------------------
\section{Division mit Rest, ggT und kgV}
% --------------------------------------
\begin{minipage}{\mw}
  \begin{equation*}
    \vphantom{\bigg(}%
    20\bmod7
  \end{equation*}
\end{minipage}%
\hfill
\begin{minipage}{\cw}
\begin{verbatim}
mod(20, 7) => 6
\end{verbatim}
\end{minipage}
\begin{minipage}{\mw}
  \begin{equation*}
    \vphantom{\bigg(}%
    \operatorname{ggT}(48, 60)
  \end{equation*}
\end{minipage}%
\hfill
\begin{minipage}{\cw}
\begin{verbatim}
gcd(48, 60) => 12
\end{verbatim}
\end{minipage}
\begin{minipage}{\mw}
  \begin{equation*}
    \vphantom{\bigg(}%
    \operatorname{kgV}(48, 60)
  \end{equation*}
\end{minipage}%
\hfill
\begin{minipage}{\cw}
\begin{verbatim}
lcm(48, 60) => 240
\end{verbatim}
\end{minipage}

% ---------------------------------
\section{Betrag und Approximierung}
% ---------------------------------
\begin{minipage}{\mw}
  \begin{equation*}
    \vphantom{\bigg(}%
    |\num{2.7}|
  \end{equation*}
\end{minipage}%
\hfill
\begin{minipage}{\cw}
\begin{verbatim}
abs(2.7) => 2.7
\end{verbatim}
\end{minipage}
\begin{minipage}{\mw}
  \begin{equation*}
    \vphantom{\bigg(}%
    |\num{-2.7}|
  \end{equation*}
\end{minipage}%
\hfill
\begin{minipage}{\cw}
\begin{verbatim}
abs(-2.7) => 2.7
\end{verbatim}
\end{minipage}
\begin{minipage}{\mw}
  \begin{equation*}
    \vphantom{\bigg(}%
    \lfloor\num{2.7}\rfloor
  \end{equation*}
\end{minipage}%
\hfill
\begin{minipage}{\cw}
\begin{verbatim}
floor(2.7) => 2
\end{verbatim}
\end{minipage}
\begin{minipage}{\mw}
  \begin{equation*}
    \vphantom{\bigg(}%
    \lfloor\num{-2.7}\rfloor
  \end{equation*}
\end{minipage}%
\hfill
\begin{minipage}{\cw}
\begin{verbatim}
floor(-2.7) => -3
\end{verbatim}
\end{minipage}
\begin{minipage}{\mw}
  \begin{equation*}
    \vphantom{\bigg(}%
    \lceil\num{2.7}\rceil
  \end{equation*}
\end{minipage}%
\hfill
\begin{minipage}{\cw}
\begin{verbatim}
ceil(2.7) => 3
\end{verbatim}
\end{minipage}
\begin{minipage}{\mw}
  \begin{equation*}
    \vphantom{\bigg(}%
    \lceil\num{-2.7}\rceil
  \end{equation*}
\end{minipage}%
\hfill
\begin{minipage}{\cw}
\begin{verbatim}
ceil(-2.7) => -2
\end{verbatim}
\end{minipage}

\begin{minipage}{\mw}
  \centering
  runden
\end{minipage}%
\hfill
\begin{minipage}{\cw}
\begin{verbatim}
round(2.7)  =>  3
round(-2.7) => -3
\end{verbatim}
\end{minipage}\bigskip

\begin{minipage}{\mw}
  \centering
  Nachkommastellen\\
  abschneiden
\end{minipage}%
\hfill
\begin{minipage}{\cw}
\begin{verbatim}
fix(2.7)  =>  2
fix(-2.7) => -2
\end{verbatim}
\end{minipage}\bigskip

\begin{minipage}{\mw}
  \begin{equation*}
    \vphantom{\bigg(}%
    \frac{z}{n}=\num{1.5}
  \end{equation*}
\end{minipage}%
\hfill
\begin{minipage}{\cw}
\begin{verbatim}
[z n] = rat(1.5)
=> z = 3
=> n = 2
\end{verbatim}
\end{minipage}

% ----------------
\section{Polynome}
% ----------------
\begin{minipage}{\mw}
  \begin{equation*}
    \vphantom{\bigg(}%
    f(x)=2x^4-x^3+5x
  \end{equation*}
\end{minipage}%
\hfill
\begin{minipage}{\cw}
\begin{verbatim}
f = [2 -1 0 5 0]
\end{verbatim}
\end{minipage}
\begin{minipage}{\mw}
  \centering
  $\displaystyle\vphantom{\bigg(}$%
  Term anzeigen
\end{minipage}%
\hfill
\begin{minipage}{\cw}
\begin{verbatim}
polyout(f, "x")
\end{verbatim}
\end{minipage}
\begin{minipage}{\mw}
  \begin{equation*}
    \vphantom{\bigg(}%
    y=f(-3)
  \end{equation*}
\end{minipage}%
\hfill
\begin{minipage}{\cw}
\begin{verbatim}
y = polyval(f, -3)
\end{verbatim}
\end{minipage}
\begin{minipage}{\mw}
  \begin{equation*}
    \vphantom{\bigg(}%
    0=f(x)
  \end{equation*}
\end{minipage}%
\hfill
\begin{minipage}{\cw}
\begin{verbatim}
roots(f)
\end{verbatim}
\end{minipage}
\begin{minipage}{\mw}
  \begin{equation*}
    \vphantom{\bigg(}%
    f(x)=(x^2-4)\cdot(x+1)
  \end{equation*}
\end{minipage}%
\hfill
\begin{minipage}{\cw}
\begin{verbatim}
f = conv([1 0 -4], [1 1])
\end{verbatim}
\end{minipage}

\begin{minipage}{\mw}
  \begin{equation*}
    \vphantom{\bigg(}%
    f'(x)=\frac{\mathrm{d}}{\mathrm{d}x}\,f(x)
  \end{equation*}
\end{minipage}%
\hfill
\begin{minipage}{\cw}
\begin{verbatim}
df = polyder(f)
\end{verbatim}
\end{minipage}

\begin{minipage}{\mw}
  \begin{equation*}
    \vphantom{\bigg(}%
    F(x)=\int f(x)\,\mathrm{d}x
  \end{equation*}
\end{minipage}%
\hfill
\begin{minipage}{\cw}
\begin{verbatim}
F = polyint(f)
\end{verbatim}
\end{minipage}

% ---------------------------
\section{Funktionen zeichnen}
% ---------------------------
\begin{verbatim}
f = [1/4 0 -4];
x = [-5:0.1:5];
y = polyval(f, x);
set(gcf(), "visible", "off");
plot(x, y, "linewidth", 2);
grid("on");
axis("on", "equal", "auto");
set(gca(), "box", "off");
set(gca(), "linewidth", 1);
set(gca(), "xaxislocation", "origin");
set(gca(), "yaxislocation", "origin");
set(gcf(), "visible", "on");
\end{verbatim}

% -----------------------------
\section{Vektoren und Matrizen}
% -----------------------------
\begin{minipage}{\mw}
  \begin{equation*}
    \vec{v}=\begin{pmatrix}
              1 \\ 2 \\ 3
            \end{pmatrix}
  \end{equation*}
\end{minipage}%
\hfill
\begin{minipage}{\cw}
\begin{verbatim}
v = [1 2 3]'
v = [1; 2; 3]
\end{verbatim}
\end{minipage}

\begin{minipage}{\mw}
  \begin{equation*}
    \vphantom{\bigg(}%
    \vec{v}=\begin{pmatrix}
              1 & 2 & 3
            \end{pmatrix}
  \end{equation*}
\end{minipage}%
\hfill
\begin{minipage}{\cw}
\begin{verbatim}
v = [1 2 3]
\end{verbatim}
\end{minipage}
\begin{minipage}{\mw}%
  \begin{equation*}
    \vphantom{\bigg(}%
    \vec{v}=\begin{pmatrix}
              1 & 2 & 3 & 4
            \end{pmatrix}%
  \end{equation*}
\end{minipage}%
\hfill
\begin{minipage}{\cw}%
\begin{verbatim}
v = [1:4]
\end{verbatim}
\end{minipage}
\begin{minipage}{\mw}
  \begin{equation*}
    \vphantom{\bigg(}%
    r=|\vec{v}\;\!|
  \end{equation*}
\end{minipage}%
\hfill
\begin{minipage}{\cw}
\begin{verbatim}
r = norm(v)
\end{verbatim}
\end{minipage}
\begin{minipage}{\mw}
  \begin{equation*}
    \vphantom{\bigg(}%
    r=\vec{u}\cdot\vec{v}
  \end{equation*}
\end{minipage}%
\hfill
\begin{minipage}{\cw}
\begin{verbatim}
r = dot(u, v)
\end{verbatim}
\end{minipage}
\begin{minipage}{\mw}
  \begin{equation*}
    \vphantom{\bigg(}%
    \vec{n}=\vec{u}\times\vec{v}
  \end{equation*}
\end{minipage}%
\hfill
\begin{minipage}{\cw}
\begin{verbatim}
n = cross(u, v)
\end{verbatim}
\end{minipage}
\begin{minipage}{\mw}
  \begin{equation*}
    \vphantom{\bigg(}%
    s=v_1+v_2+\cdots+v_n
  \end{equation*}
\end{minipage}%
\hfill
\begin{minipage}{\cw}
\begin{verbatim}
s = sum(v)
\end{verbatim}
\end{minipage}
\begin{minipage}{\mw}
  \begin{equation*}
    \vphantom{\bigg(}%
    p=v_1\cdot v_2\cdot\cdots\cdot v_n
  \end{equation*}
\end{minipage}%
\hfill
\begin{minipage}{\cw}
\begin{verbatim}
p = prod(v)
\end{verbatim}
\end{minipage}

\begin{minipage}{\mw}
  \begin{equation*}
    A=\begin{pmatrix}
        1 & 2 & 3 \\
        4 & 5 & 6 \\
        7 & 8 & 9
      \end{pmatrix}
  \end{equation*}
\end{minipage}%
\hfill
\begin{minipage}{\cw}
\begin{verbatim}
A = [1 2 3; 4 5 6; 7 8 9]
\end{verbatim}
\end{minipage}\bigskip

\begin{minipage}{\mw}
  \begin{equation*}
    A=\begin{pmatrix}
        1 & 0 & 0 \\
        0 & 1 & 0 \\
        0 & 0 & 1
    \end{pmatrix}
  \end{equation*}
\end{minipage}%
\hfill
\begin{minipage}{\cw}
\begin{verbatim}
A = eye(3)
\end{verbatim}
\end{minipage}\bigskip

\begin{minipage}{\mw}
  \begin{equation*}
    A=\begin{pmatrix}
        0 & 0 & 0 & 0 \\
        0 & 0 & 0 & 0 \\
        0 & 0 & 0 & 0
      \end{pmatrix}
  \end{equation*}
\end{minipage}%
\hfill
\begin{minipage}{\cw}
\begin{verbatim}
A = zeros(3, 4)
\end{verbatim}
\end{minipage}\bigskip

\begin{minipage}{\mw}
  \begin{equation*}
    A=\begin{pmatrix}
        1 & 1 & 1 & 1 \\
        1 & 1 & 1 & 1 \\
        1 & 1 & 1 & 1
      \end{pmatrix}
  \end{equation*}
\end{minipage}%
\hfill
\begin{minipage}{\cw}
\begin{verbatim}
A = ones(3, 4)
\end{verbatim}
\end{minipage}\bigskip

\begin{minipage}{\mw}
  \begin{equation*}
    A=\begin{pmatrix}
        1 & 0 & 0 & 0 \\
        0 & 2 & 0 & 0 \\
        0 & 0 & 3 & 0 \\
        0 & 0 & 0 & 4
      \end{pmatrix}
  \end{equation*}
\end{minipage}%
\hfill
\begin{minipage}{\cw}
\begin{verbatim}
A = diag([1:4])
\end{verbatim}
\end{minipage}\bigskip

\begin{minipage}{\mw}
  \begin{equation*}
    a=\begin{pmatrix}
        1 \\ 2 \\ 3
      \end{pmatrix}
  \end{equation*}
\end{minipage}%
\hfill
\begin{minipage}{\cw}
\begin{verbatim}
size(a)    => [3 1]
size(a, 1) =>  3
size(a, 2) =>  1
\end{verbatim}
\end{minipage}

\begin{minipage}{\mw}
  \begin{equation*}
    \vphantom{\bigg(}
    a_{ij}
  \end{equation*}
\end{minipage}%
\hfill
\begin{minipage}{\cw}
\begin{verbatim}
a(i, j)
\end{verbatim}
\end{minipage}
\begin{minipage}{\mw}
  \begin{equation*}
    \vphantom{\bigg(}
    a_{i,1\ldots n}
  \end{equation*}
\end{minipage}%
\hfill
\begin{minipage}{\cw}
\begin{verbatim}
a(i, :)
\end{verbatim}
\end{minipage}
\begin{minipage}{\mw}
  \begin{equation*}
    \vphantom{\bigg(}
    a_{1\ldots m,j}
  \end{equation*}
\end{minipage}%
\hfill
\begin{minipage}{\cw}
\begin{verbatim}
a(:, j)
\end{verbatim}
\end{minipage}

% ---------------------------------------
\begin{minipage}{\mw}
  \begin{equation*}
    \vphantom{\bigg(}%
    M=A^t
  \end{equation*}
\end{minipage}%
\hfill
\begin{minipage}{\cw}
\begin{verbatim}
M = A'
\end{verbatim}
\end{minipage}
\begin{minipage}{\mw}
  \begin{equation*}
    \vphantom{\bigg(}%
    M=A^{-1}
  \end{equation*}
\end{minipage}%
\hfill
\begin{minipage}{\cw}
\begin{verbatim}
M = inv(A)
\end{verbatim}
\end{minipage}
\begin{minipage}{\mw}
  \begin{equation*}
    \vphantom{\bigg(}%
    t=\operatorname{Spur}(A)
  \end{equation*}
\end{minipage}%
\hfill
\begin{minipage}{\cw}
\begin{verbatim}
t = trace(A)
\end{verbatim}
\end{minipage}
\begin{minipage}{\mw}
  \begin{equation*}
    \vphantom{\bigg(}%
    r=\operatorname{Rang}(A)
  \end{equation*}
\end{minipage}%
\hfill
\begin{minipage}{\cw}
\begin{verbatim}
r = rank(A)
\end{verbatim}
\end{minipage}
\begin{minipage}{\mw}
  \begin{equation*}
    \vphantom{\bigg(}%
    d=\det(A)
  \end{equation*}
\end{minipage}%
\hfill
\begin{minipage}{\cw}
\begin{verbatim}
d = det(A)
\end{verbatim}
\end{minipage}
\begin{minipage}{\mw}
  \begin{equation*}
    \vphantom{\bigg(}%
    A\cdot x=b
  \end{equation*}
\end{minipage}%
\hfill
\begin{minipage}{\cw}
\begin{verbatim}
x = A \ b
\end{verbatim}
\end{minipage}

\begin{minipage}{\mw}
  \begin{equation*}
    \vphantom{\bigg(}%
    A\cdot x=\lambda\cdot x
  \end{equation*}
\end{minipage}%
\hfill
\begin{minipage}{\cw}
\begin{verbatim}
[vec lam] = eig(A)
vec1 = vec(:, 1)
lam1 = lam(1, 1)
\end{verbatim}
\end{minipage}

% --------------------------
\section{Binomialverteilung}
% --------------------------
\begin{minipage}{\mw}
  \begin{equation*}
    \vphantom{\bigg(}%
    5!
  \end{equation*}
\end{minipage}%
\hfill
\begin{minipage}{\cw}
\begin{verbatim}
factorial(5) => 120
\end{verbatim}
\end{minipage}

\begin{minipage}{\mw}
  \begin{equation*}
    \vphantom{\bigg(}%
    \binom{8}{5}=\frac{8!}{5!\cdot(8-5)!}
  \end{equation*}
\end{minipage}%
\hfill
\begin{minipage}{\cw}
\begin{verbatim}
nchoosek(8, 5) => 56
\end{verbatim}
\end{minipage}

\begin{minipage}{\mw}
  \begin{equation*}
    \vphantom{\bigg(}%
    P_{100;\num{0.4}}(X=35)
  \end{equation*}
\end{minipage}%
\hfill
\begin{minipage}{\cw}
\begin{verbatim}
binopdf(35, 100, 0.4) => 0.049133
\end{verbatim}
\end{minipage}
\begin{minipage}{\mw}
  \begin{equation*}
    \vphantom{\bigg(}%
    P_{100;\num{0.4}}(X\leq35)
  \end{equation*}
\end{minipage}%
\hfill
\begin{minipage}{\cw}
\begin{verbatim}
binocdf(35, 100, 0.4) => 0.17947
\end{verbatim}
\end{minipage}

% ------------------------------------------------------------------------------
\end{document}
% ------------------------------------------------------------------------------


% Vorlage und globale Optionen
\documentclass
[
  draft      = true,
  fontsize   = 11pt,
  parskip    = half-,
  BCOR       = 0pt,
  DIV        = 10,
  dvipsnames % vermeidet 'option clash' mit xcolor
]
{scrartcl}

% Standardpakete
\usepackage{fixltx2e}
\usepackage[utf8]{inputenc}
\usepackage[T1]{fontenc}
\usepackage{lmodern}
\usepackage[ngerman]{babel}
% Zusatzpakete
\usepackage{fp}
\usepackage{graphicx}
\usepackage{ifthen}
\usepackage{tikz}
\usepackage{xcolor}
% TikZ-Bibliotheken (alphabetisch)
\usetikzlibrary{arrows, calc, decorations.pathmorphing,
                decorations.pathreplacing, decorations.shapes,
                decorations.text, intersections, patterns, shapes}

% ------
% sizeof
% ------
%
% Calculates the width and height of an image:
%
% \sizeof{%
% \begin{tikzpicture}
%   ...
% \end{tikzpicture}}
%
\newcommand{\sizeof}[1]
{%
  \begingroup
    \setbox0=\hbox{#1}%
    \setlength  {\dimen0}{\wd0}%
    \setlength  {\dimen1}{\ht0}%
    \addtolength{\dimen1}{\dp0}%
    \makebox[3em][r]{$w=$\,}\the\dimen0\\
    \makebox[3em][r]{$h=$\,}\the\dimen1\par
  \endgroup
}

% --------
% shapedst
% --------
%
% #1  name of first shape
% #2  name of second shape
% #3  macro
%
% Example:
%   \shapedst{A}{B}{\mydistance}
%
\newcommand{\shapedst}[3]
{%
  % define new macro if missing
  \ifthenelse{\isundefined{#3}}{\def#3{\relax}}{\relax}%
  % get x-coordinate from vector
  \pgfextractx{\dimen0}{\pgfpointdiff{\pgfpointanchor{#1}{center}}%
                                     {\pgfpointanchor{#2}{center}}}%
  % get y-coordinate from vector
  \pgfextracty{\dimen1}{\pgfpointdiff{\pgfpointanchor{#1}{center}}%
                                     {\pgfpointanchor{#2}{center}}}%
  % calculate length
  \pgfmathsetmacro{#3}{veclen(\the\dimen0,\the\dimen1)}%
  % add unit 'pt'
  \edef#3{#3pt}%
}

% --------
% shapedir
% --------
%
% #1  name of first shape
% #2  name of second shape
% #3  macro
%
% Example:
%   \shapedir{A}{B}{\mydirection}
%
\newcommand{\shapedir}[3]
{%
  % define new macro if missing
  \ifthenelse{\isundefined{#3}}{\def#3{\relax}}{\relax}%
  % get x-coordinate from vector
  \pgfextractx{\dimen0}{\pgfpointdiff{\pgfpointanchor{#1}{center}}%
                                     {\pgfpointanchor{#2}{center}}}%
  % get y-coordinate from vector
  \pgfextracty{\dimen1}{\pgfpointdiff{\pgfpointanchor{#1}{center}}%
                                     {\pgfpointanchor{#2}{center}}}%
  % arctangent of y/x in degrees
  % this also takes into account the quadrants
  \pgfmathsetmacro{#3}{atan2(\the\dimen1,\the\dimen0)}%
}


% ------------------------------------------------------------------------------
\begin{document}
% ------------------------------------------------------------------------------

% Titelseite
\begin{titlepage}
  \centering
  \Huge\bfseries
  \vspace*{\fill}
  Zeichnen mit \LaTeX\\
  \vspace{\fill}
  \begin{tikzpicture}
    % das Aussehen der Ringe
    \tikzstyle{R}=[fill=OrangeRed,   draw=Black, line width=2pt, even odd rule];
    \tikzstyle{G}=[fill=ForestGreen, draw=Black, line width=2pt, even odd rule];
    \tikzstyle{B}=[fill=RoyalBlue,   draw=Black, line width=2pt, even odd rule];
    % Ringe zeichnen
    \filldraw[G] (30:1)  circle[radius=18mm] (30:1)  circle[radius=22mm];
    \filldraw[R] (150:1) circle[radius=18mm] (150:1) circle[radius=22mm];
    \filldraw[B] (270:1) circle[radius=18mm] (270:1) circle[radius=22mm];
    \begin{scope}[even odd rule]
      % blauer Ring als Clipping-Bereich (etwas groesser)
      \clip (270:1) circle[radius=15mm] (270:1) circle[radius=25mm];
      % gruener Ring
      \filldraw[G] (30:1) circle[radius=18mm] (30:1) circle[radius=22mm];
    \end{scope}
  \end{tikzpicture}\\
  \vspace{\fill}
  \vspace{\fill}
  \normalsize\mdseries
  \today
\end{titlepage}

% Inhalt
\pagenumbering{roman}
\tableofcontents
\clearpage

\pagenumbering{arabic}
% -----------------------------------------
\section{Pakete, Optionen und Bibliotheken}
% -----------------------------------------
\begin{footnotesize}
% parcolor version 2011-09-30
\begingroup
\ttfamily
\definecolor{R}{named}{Red}
\definecolor{G}{named}{ForestGreen}
\definecolor{B}{named}{RoyalBlue}
\definecolor{C}{named}{Cyan}
\definecolor{M}{named}{Magenta}
\definecolor{Y}{named}{YellowOrange}
\definecolor{background}{rgb}{0.82, 0.82, 0.92}
\dimen255=\textwidth
\advance\dimen255 by -2\fboxsep
\noindent
\colorbox{background}
{%
\parbox{\dimen255}
{%
\rule[-0.5ex]{0pt}{2.5ex}\hspace*{0.0em}\textcolor{G}{\textbf{\%~Vorlage~und~globale~Optionen}}\\
\rule[-0.5ex]{0pt}{2.5ex}\hspace*{0.0em}\textbackslash{}documentclass\\
\rule[-0.5ex]{0pt}{2.5ex}\hspace*{0.0em}[\\
\rule[-0.5ex]{0pt}{2.5ex}\hspace*{1.0em}draft~~~~~~=~true,\\
\rule[-0.5ex]{0pt}{2.5ex}\hspace*{1.0em}fontsize~~~=~11pt,\\
\rule[-0.5ex]{0pt}{2.5ex}\hspace*{1.0em}parskip~~~~=~half{-},\\
\rule[-0.5ex]{0pt}{2.5ex}\hspace*{1.0em}BCOR~~~~~~~=~0pt,\\
\rule[-0.5ex]{0pt}{2.5ex}\hspace*{1.0em}DIV~~~~~~~~=~10,\\
\rule[-0.5ex]{0pt}{2.5ex}\hspace*{1.0em}\textcolor{R}{\textbf{dvipsnames}}~\textcolor{G}{\textbf{\%~vermeidet~'option~clash'~mit~xcolor}}\\
\rule[-0.5ex]{0pt}{2.5ex}\hspace*{0.0em}]\\
\rule[-0.5ex]{0pt}{2.5ex}\hspace*{0.0em}\{scrartcl\}\\
\rule[-0.5ex]{0pt}{2.5ex}\hspace*{0.0em}\\
\rule[-0.5ex]{0pt}{2.5ex}\hspace*{0.0em}\textcolor{G}{\textbf{\%~Standardpakete}}\\
\rule[-0.5ex]{0pt}{2.5ex}\hspace*{0.0em}\textbackslash{}usepackage\{fixltx2e\}\\
\rule[-0.5ex]{0pt}{2.5ex}\hspace*{0.0em}\textbackslash{}usepackage[utf8]\{inputenc\}\\
\rule[-0.5ex]{0pt}{2.5ex}\hspace*{0.0em}\textbackslash{}usepackage[T1]\{fontenc\}\\
\rule[-0.5ex]{0pt}{2.5ex}\hspace*{0.0em}\textbackslash{}usepackage\{lmodern\}\\
\rule[-0.5ex]{0pt}{2.5ex}\hspace*{0.0em}\textbackslash{}usepackage[ngerman]\{babel\}\\
\rule[-0.5ex]{0pt}{2.5ex}\hspace*{0.0em}\textcolor{G}{\textbf{\%~Zusatzpakete}}\\
\rule[-0.5ex]{0pt}{2.5ex}\hspace*{0.0em}\textcolor{R}{\textbf{\textbackslash{}usepackage}}\{\textcolor{R}{\textbf{fp}}\}\\
\rule[-0.5ex]{0pt}{2.5ex}\hspace*{0.0em}\textcolor{R}{\textbf{\textbackslash{}usepackage}}\{\textcolor{R}{\textbf{graphicx}}\}\\
\rule[-0.5ex]{0pt}{2.5ex}\hspace*{0.0em}\textcolor{R}{\textbf{\textbackslash{}usepackage}}\{\textcolor{R}{\textbf{ifthen}}\}\\
\rule[-0.5ex]{0pt}{2.5ex}\hspace*{0.0em}\textcolor{R}{\textbf{\textbackslash{}usepackage}}\{\textcolor{R}{\textbf{tikz}}\}\\
\rule[-0.5ex]{0pt}{2.5ex}\hspace*{0.0em}\textcolor{R}{\textbf{\textbackslash{}usepackage}}\{\textcolor{R}{\textbf{xcolor}}\}\\
\rule[-0.5ex]{0pt}{2.5ex}\hspace*{0.0em}\textcolor{G}{\textbf{\%~TikZ{-}Bibliotheken~(alphabetisch)}}\\
\rule[-0.5ex]{0pt}{2.5ex}\hspace*{0.0em}\textcolor{R}{\textbf{\textbackslash{}usetikzlibrary}}\{\textcolor{R}{\textbf{arrows}},~\textcolor{R}{\textbf{calc}},~\textcolor{R}{\textbf{decorations.pathmorphing}},\\
\rule[-0.5ex]{0pt}{2.5ex}\hspace*{8.0em}\textcolor{R}{\textbf{decorations.pathreplacing}},~\textcolor{R}{\textbf{decorations.shapes}},\\
\rule[-0.5ex]{0pt}{2.5ex}\hspace*{8.0em}\textcolor{R}{\textbf{decorations.text}},~\textcolor{R}{\textbf{intersections}},~\textcolor{R}{\textbf{patterns}},~\textcolor{R}{\textbf{shapes}}\}}%
}%
\endgroup

\end{footnotesize}

% -------------------------------------
\section{Umgebungen und Zeichenbefehle}
% -------------------------------------

% --------------------------------------------
\subsection{Die \texttt{tikzpicture}-Umgebung}
% --------------------------------------------
\begin{minipage}{0.7\textwidth}
  \footnotesize
  % parcolor version 2011-09-30
\begingroup
\ttfamily
\definecolor{R}{named}{Red}
\definecolor{G}{named}{ForestGreen}
\definecolor{B}{named}{RoyalBlue}
\definecolor{C}{named}{Cyan}
\definecolor{M}{named}{Magenta}
\definecolor{Y}{named}{YellowOrange}
\definecolor{background}{rgb}{0.82, 0.82, 0.92}
\dimen255=\textwidth
\advance\dimen255 by -2\fboxsep
\noindent
\colorbox{background}
{%
\parbox{\dimen255}
{%
\rule[-0.5ex]{0pt}{2.5ex}\hspace*{0.0em}\textcolor{G}{\textbf{\%~die~tikzpicture{-}Umgebung~enthaelt~die~Zeichenbefehle}}\\
\rule[-0.5ex]{0pt}{2.5ex}\hspace*{0.0em}\textcolor{R}{\textbf{\textbackslash{}begin}}\{\textcolor{R}{\textbf{tikzpicture}}\}\\
\rule[-0.5ex]{0pt}{2.5ex}\hspace*{1.0em}\textcolor{G}{\textbf{\%~hier~wird~der~Rand~eines~Rechtecks~gezeichnet:}}\\
\rule[-0.5ex]{0pt}{2.5ex}\hspace*{1.0em}\textcolor{R}{\textbf{\textbackslash{}draw}}[line~width=1pt]~(0,~0)~\textcolor{R}{\textbf{rectangle}}~(3,~2);\\
\rule[-0.5ex]{0pt}{2.5ex}\hspace*{0.0em}\textcolor{R}{\textbf{\textbackslash{}end}}\{\textcolor{R}{\textbf{tikzpicture}}\}}%
}%
\endgroup

\end{minipage}\hfill
\begin{minipage}{0.29\textwidth}
  \centering
  % die tikzpicture-Umgebung enthaelt die Zeichenbefehle
  \begin{tikzpicture}
    % hier wird der Rand eines Rechtecks gezeichnet:
    \draw[line width=1pt] (0, 0) rectangle (3, 2);
  \end{tikzpicture}
\end{minipage}

\begin{minipage}{0.7\textwidth}
  \footnotesize
  \input{fillrectangle.code}
\end{minipage}\hfill
\begin{minipage}{0.29\textwidth}
  \centering
  % die tikzpicture-Umgebung enthaelt die Zeichenbefehle
  \begin{tikzpicture}
    % hier wird die Flaeche eines Rechtecks ausgefuellt:
    \fill[fill=LimeGreen] (0, 0) rectangle (3, 2);
  \end{tikzpicture}
\end{minipage}

% --------------------------------------
\subsection{Die \texttt{scope}-Umgebung}
% --------------------------------------
\begin{minipage}{0.7\textwidth}
  \footnotesize
  % parcolor version 2011-09-30
\begingroup
\ttfamily
\definecolor{R}{named}{Red}
\definecolor{G}{named}{ForestGreen}
\definecolor{B}{named}{RoyalBlue}
\definecolor{C}{named}{Cyan}
\definecolor{M}{named}{Magenta}
\definecolor{Y}{named}{YellowOrange}
\definecolor{background}{rgb}{0.82, 0.82, 0.92}
\dimen255=\textwidth
\advance\dimen255 by -2\fboxsep
\noindent
\colorbox{background}
{%
\parbox{\dimen255}
{%
\rule[-0.5ex]{0pt}{2.5ex}\hspace*{0.0em}\textcolor{G}{\textbf{\%~die~tikzpicture{-}Umgebung~enthaelt~die~Zeichenbefehle}}\\
\rule[-0.5ex]{0pt}{2.5ex}\hspace*{0.0em}\textbackslash{}begin\{tikzpicture\}\\
\rule[-0.5ex]{0pt}{2.5ex}\hspace*{1.0em}\textcolor{G}{\textbf{\%~die~scope{-}Umgebung~kann~z.B.~fuer~Transformationen}}\\
\rule[-0.5ex]{0pt}{2.5ex}\hspace*{1.0em}\textcolor{G}{\textbf{\%~genutzt~werden}}\\
\rule[-0.5ex]{0pt}{2.5ex}\hspace*{1.0em}\textcolor{R}{\textbf{\textbackslash{}begin}}\{\textcolor{R}{\textbf{scope}}\}[\textcolor{R}{\textbf{rotate=30}}]\\
\rule[-0.5ex]{0pt}{2.5ex}\hspace*{2.0em}\textcolor{G}{\textbf{\%~hier~wird~gleichzeitig~gezeichnet~und~gefuellt:}}\\
\rule[-0.5ex]{0pt}{2.5ex}\hspace*{2.0em}\textcolor{R}{\textbf{\textbackslash{}filldraw}}[fill=Yellow,~draw=Cerulean,~line~width=2pt]\\
\rule[-0.5ex]{0pt}{2.5ex}\hspace*{6.5em}(0,~0)~\textcolor{R}{\textbf{rectangle}}~(3,~2);\\
\rule[-0.5ex]{0pt}{2.5ex}\hspace*{1.0em}\textcolor{R}{\textbf{\textbackslash{}end}}\{\textcolor{R}{\textbf{scope}}\}\\
\rule[-0.5ex]{0pt}{2.5ex}\hspace*{0.0em}\textbackslash{}end\{tikzpicture\}}%
}%
\endgroup

\end{minipage}\hfill
\begin{minipage}{0.29\textwidth}
  \centering
  % die tikzpicture-Umgebung enthaelt die Zeichenbefehle
  \begin{tikzpicture}
    % die scope-Umgebung kann z.B. fuer Transformationen
    % genutzt werden
    \begin{scope}[rotate=30]
      % hier wird gleichzeitig gezeichnet und gefuellt:
      \filldraw[fill=Yellow, draw=Cerulean, line width=2pt]
               (0, 0) rectangle (3, 2);
    \end{scope}
  \end{tikzpicture}
\end{minipage}

% ----------------------------------
\subsection{Das \texttt{tikz}-Makro}
% ----------------------------------
\begin{minipage}{0.7\textwidth}
  \footnotesize
  % parcolor version 2011-09-30
\begingroup
\ttfamily
\definecolor{R}{named}{Red}
\definecolor{G}{named}{ForestGreen}
\definecolor{B}{named}{RoyalBlue}
\definecolor{C}{named}{Cyan}
\definecolor{M}{named}{Magenta}
\definecolor{Y}{named}{YellowOrange}
\definecolor{background}{rgb}{0.82, 0.82, 0.92}
\dimen255=\textwidth
\advance\dimen255 by -2\fboxsep
\noindent
\colorbox{background}
{%
\parbox{\dimen255}
{%
\rule[-0.5ex]{0pt}{2.5ex}\hspace*{0.0em}\textcolor{G}{\textbf{\%~wenn~die~Zeichnung~nur~aus~einem~einzigen~Pfad~besteht,}}\\
\rule[-0.5ex]{0pt}{2.5ex}\hspace*{0.0em}\textcolor{G}{\textbf{\%~kann~man~sie~mit~\textbackslash{}tikz~auch~direkt~in~den~Text~einfuegen}}\\
\rule[-0.5ex]{0pt}{2.5ex}\hspace*{0.0em}direkt~\textcolor{R}{\textbf{\textbackslash{}tikz~\textbackslash{}fill~(0,~0)~rectangle~(1em,~1ex);}}~im~Text}%
}%
\endgroup

\end{minipage}\hfill
\begin{minipage}{0.29\textwidth}
  \centering
  direkt \tikz \fill (0, 0) rectangle (1em, 1ex); im Text
\end{minipage}

% -------------------
\section{Koordinaten}
% -------------------

% ---------------------------------------------------
\subsection{Zweidimensionale kartesische Koordinaten}
% ---------------------------------------------------
\begin{minipage}{0.7\textwidth}
  \footnotesize
  % parcolor version 2011-09-30
\begingroup
\ttfamily
\definecolor{R}{named}{Red}
\definecolor{G}{named}{ForestGreen}
\definecolor{B}{named}{RoyalBlue}
\definecolor{C}{named}{Cyan}
\definecolor{M}{named}{Magenta}
\definecolor{Y}{named}{YellowOrange}
\definecolor{background}{rgb}{0.82, 0.82, 0.92}
\dimen255=\textwidth
\advance\dimen255 by -2\fboxsep
\noindent
\colorbox{background}
{%
\parbox{\dimen255}
{%
\rule[-0.5ex]{0pt}{2.5ex}\hspace*{0.0em}\textbackslash{}begin\{tikzpicture\}\\
\rule[-0.5ex]{0pt}{2.5ex}\hspace*{1.0em}\textcolor{G}{\textbf{\%~Kartesische~Koordinaten~in~der~gewohnten~Form:~(x,~y)}}\\
\rule[-0.5ex]{0pt}{2.5ex}\hspace*{1.0em}\textbackslash{}draw~\textcolor{R}{\textbf{(0,~0)}}~{-}{-}~\textcolor{R}{\textbf{(3,~0)}}~{-}{-}~\textcolor{R}{\textbf{(3,~1.5)}}~{-}{-}~cycle;\\
\rule[-0.5ex]{0pt}{2.5ex}\hspace*{0.0em}\textbackslash{}end\{tikzpicture\}}%
}%
\endgroup

\end{minipage}\hfill
\begin{minipage}{0.29\textwidth}
  \centering
  \begin{tikzpicture}[line width=1pt]
    % Kartesische Koordinaten in der gewohnten Form: (x, y)
    \draw (0, 0) -- (3, 0) -- (3, 1.5) -- cycle;
  \end{tikzpicture}
\end{minipage}

% ---------------------------------------------------
\subsection{Dreidimensionale kartesische Koordinaten}
% ---------------------------------------------------
\begin{minipage}{0.7\textwidth}
  \footnotesize
  % parcolor version 2011-09-30
\begingroup
\ttfamily
\definecolor{R}{named}{Red}
\definecolor{G}{named}{ForestGreen}
\definecolor{B}{named}{RoyalBlue}
\definecolor{C}{named}{Cyan}
\definecolor{M}{named}{Magenta}
\definecolor{Y}{named}{YellowOrange}
\definecolor{background}{rgb}{0.82, 0.82, 0.92}
\dimen255=\textwidth
\advance\dimen255 by -2\fboxsep
\noindent
\colorbox{background}
{%
\parbox{\dimen255}
{%
\rule[-0.5ex]{0pt}{2.5ex}\hspace*{0.0em}\textbackslash{}begin\{tikzpicture\}\\
\rule[-0.5ex]{0pt}{2.5ex}\hspace*{1.0em}\textcolor{G}{\textbf{\%~Kartesische~Koordinaten~in~der~Form:~(x,~y,~z)}}\\
\rule[-0.5ex]{0pt}{2.5ex}\hspace*{1.0em}\textbackslash{}draw[{-}{>}]~\textcolor{R}{\textbf{(0,~0,~0)}}~{-}{-}~\textcolor{R}{\textbf{(1,~0,~0)}}~node[right]~~~~~\{\$x\$\};\\
\rule[-0.5ex]{0pt}{2.5ex}\hspace*{1.0em}\textbackslash{}draw[{-}{>}]~\textcolor{R}{\textbf{(0,~0,~0)}}~{-}{-}~\textcolor{R}{\textbf{(0,~1,~0)}}~node[above]~~~~~\{\$y\$\};\\
\rule[-0.5ex]{0pt}{2.5ex}\hspace*{1.0em}\textbackslash{}draw[{-}{>}]~\textcolor{R}{\textbf{(0,~0,~0)}}~{-}{-}~\textcolor{R}{\textbf{(0,~0,~1)}}~node[below~left]\{\$z\$\};\\
\rule[-0.5ex]{0pt}{2.5ex}\hspace*{0.0em}\textbackslash{}end\{tikzpicture\}}%
}%
\endgroup

\end{minipage}\hfill
\begin{minipage}{0.29\textwidth}
  \centering
  \begin{tikzpicture}[line width=1pt]
    % Kartesische Koordinaten in der Form: (x, y, z)
    \draw[->] (0, 0, 0) -- (1, 0, 0) node[right]     {$x$};
    \draw[->] (0, 0, 0) -- (0, 1, 0) node[above]     {$y$};
    \draw[->] (0, 0, 0) -- (0, 0, 1) node[below left]{$z$};
  \end{tikzpicture}
\end{minipage}

% ---------------------------
\subsection{Polarkoordinaten}
% ---------------------------
\begin{minipage}{0.7\textwidth}
  \footnotesize
  \input{polar.code}
\end{minipage}\hfill
\begin{minipage}{0.29\textwidth}
  \centering
  \begin{tikzpicture}[line width=1pt]
    % Polarkoordinaten in der Form: (Winkel:Radius)
    \draw   (0:1) --  (60:1) -- (120:1) --
          (180:1) -- (240:1) -- (300:1) -- cycle;
  \end{tikzpicture}
\end{minipage}

% --------------------------
\subsection{Benannte Punkte}
% --------------------------
\begin{minipage}{0.7\textwidth}
  \footnotesize
  % parcolor version 2011-09-30
\begingroup
\ttfamily
\definecolor{R}{named}{Red}
\definecolor{G}{named}{ForestGreen}
\definecolor{B}{named}{RoyalBlue}
\definecolor{C}{named}{Cyan}
\definecolor{M}{named}{Magenta}
\definecolor{Y}{named}{YellowOrange}
\definecolor{background}{rgb}{0.82, 0.82, 0.92}
\dimen255=\textwidth
\advance\dimen255 by -2\fboxsep
\noindent
\colorbox{background}
{%
\parbox{\dimen255}
{%
\rule[-0.5ex]{0pt}{2.5ex}\hspace*{0.0em}\textbackslash{}begin\{tikzpicture\}\\
\rule[-0.5ex]{0pt}{2.5ex}\hspace*{1.0em}\textcolor{G}{\textbf{\%~mit~'\textbackslash{}coordinate'~koennen~Punkte~benannt~werden}}\\
\rule[-0.5ex]{0pt}{2.5ex}\hspace*{1.0em}\textcolor{R}{\textbf{\textbackslash{}coordinate}}~(\textcolor{B}{\textbf{A}})~\textcolor{R}{\textbf{at}}~(0,~0);\\
\rule[-0.5ex]{0pt}{2.5ex}\hspace*{1.0em}\textcolor{R}{\textbf{\textbackslash{}coordinate}}~(\textcolor{B}{\textbf{B}})~\textcolor{R}{\textbf{at}}~(3,~0);\\
\rule[-0.5ex]{0pt}{2.5ex}\hspace*{1.0em}\textcolor{R}{\textbf{\textbackslash{}coordinate}}~(\textcolor{B}{\textbf{C}})~\textcolor{R}{\textbf{at}}~(3,~2);\\
\rule[-0.5ex]{0pt}{2.5ex}\hspace*{1.0em}\textcolor{G}{\textbf{\%~die~Namen~ersetzen~dann~die~Koordinaten}}\\
\rule[-0.5ex]{0pt}{2.5ex}\hspace*{1.0em}\textbackslash{}draw~(\textcolor{B}{\textbf{A}})~{-}{-}~(\textcolor{B}{\textbf{B}})~{-}{-}~(\textcolor{B}{\textbf{C}})~{-}{-}~cycle;\\
\rule[-0.5ex]{0pt}{2.5ex}\hspace*{0.0em}\textbackslash{}end\{tikzpicture\}}%
}%
\endgroup

\end{minipage}\hfill
\begin{minipage}{0.29\textwidth}
  \centering
  \begin{tikzpicture}[line width=1pt]
    % mit '\coordinate' koennen Punkte benannt werden
    \coordinate (A) at (0, 0);
    \coordinate (B) at (3, 0);
    \coordinate (C) at (3, 2);
    % die Namen ersetzen dann die Koordinaten
    \draw (A) -- (B) -- (C) -- cycle;
  \end{tikzpicture}
\end{minipage}

% -------------------------
\subsection{Verschiebungen}
% -------------------------
\begin{minipage}{0.7\textwidth}
  \footnotesize
  % parcolor version 2011-09-30
\begingroup
\ttfamily
\definecolor{R}{named}{Red}
\definecolor{G}{named}{ForestGreen}
\definecolor{B}{named}{RoyalBlue}
\definecolor{C}{named}{Cyan}
\definecolor{M}{named}{Magenta}
\definecolor{Y}{named}{YellowOrange}
\definecolor{background}{rgb}{0.82, 0.82, 0.92}
\dimen255=\textwidth
\advance\dimen255 by -2\fboxsep
\noindent
\colorbox{background}
{%
\parbox{\dimen255}
{%
\rule[-0.5ex]{0pt}{2.5ex}\hspace*{0.0em}\textbackslash{}begin\{tikzpicture\}\\
\rule[-0.5ex]{0pt}{2.5ex}\hspace*{1.0em}\textcolor{G}{\textbf{\%~mit~'\textbackslash{}coordinate'~koennen~Punkte~benannt~werden}}\\
\rule[-0.5ex]{0pt}{2.5ex}\hspace*{1.0em}\textbackslash{}coordinate~(\textcolor{B}{\textbf{A}})~at~(1,~2);\\
\rule[-0.5ex]{0pt}{2.5ex}\hspace*{1.0em}\textcolor{G}{\textbf{\%~Verschiebungen~koennen~mit~kartesischen~und~mit}}\\
\rule[-0.5ex]{0pt}{2.5ex}\hspace*{1.0em}\textcolor{G}{\textbf{\%~Polarkoordinaten~definiert~werden}}\\
\rule[-0.5ex]{0pt}{2.5ex}\hspace*{1.0em}\textbackslash{}coordinate~(\textcolor{B}{\textbf{B}})~at~(\textcolor{R}{\textbf{[shift=\{(2,~{-}1)\}]}}\textcolor{B}{\textbf{A}});\\
\rule[-0.5ex]{0pt}{2.5ex}\hspace*{1.0em}\textbackslash{}coordinate~(\textcolor{B}{\textbf{C}})~at~(\textcolor{R}{\textbf{[shift=\{(270:3)\}]}}\textcolor{B}{\textbf{A}});\\
\rule[-0.5ex]{0pt}{2.5ex}\hspace*{1.0em}\textcolor{G}{\textbf{\%~die~Namen~ersetzen~dann~die~Koordinaten}}\\
\rule[-0.5ex]{0pt}{2.5ex}\hspace*{1.0em}\textbackslash{}draw~(\textcolor{B}{\textbf{A}})~{-}{-}~(\textcolor{B}{\textbf{B}})~{-}{-}~(\textcolor{B}{\textbf{C}})~{-}{-}~cycle;\\
\rule[-0.5ex]{0pt}{2.5ex}\hspace*{0.0em}\textbackslash{}end\{tikzpicture\}}%
}%
\endgroup

\end{minipage}\hfill
\begin{minipage}{0.29\textwidth}
  \centering
  \begin{tikzpicture}[line width=1pt]
    % mit '\coordinate' koennen Punkte benannt werden
    \coordinate (A) at (1, 2);
    % Verschiebungen koennen mit kartesischen und mit
    % Polarkoordinaten definiert werden
    \coordinate (B) at ([shift={(2, -1)}]A);
    \coordinate (C) at ([shift={(270:3)}]A);
    % die Namen ersetzen dann die Koordinaten
    \draw (A) -- (B) -- (C) -- cycle;
  \end{tikzpicture}
\end{minipage}

% --------------------------------------------
\subsection{Absolute und relative Koordinaten}
% --------------------------------------------
\begin{center}
  \begin{tikzpicture}[line width=1pt]
    \fill (0, 0) circle[radius=1pt] node[below]{A};
    % absolute Koordinaten
    \draw (0, 0) -- (1, 1) -- (0, 1) -- (1, 0);
    \begin{scope}[xshift=3cm]
      \fill (0, 0) circle[radius=1pt] node[below]{B};
      % relative Koordinaten ohne Verschiebung des Bezugspunktes
      \draw (0, 0) -- (1, 1) -- +(0, 1) -- +(1, 0);
    \end{scope}
    \begin{scope}[xshift=6cm]
      \fill (0, 0) circle[radius=1pt] node[below]{C};
      % relative Koordinaten mit Verschiebung des Bezugspunktes
      \draw (0, 0) -- (1, 1) -- ++(0, 1) -- ++(1, 0);
    \end{scope}
  \end{tikzpicture}
\end{center}
\begin{footnotesize}
  % parcolor version 2011-09-30
\begingroup
\ttfamily
\definecolor{R}{named}{Red}
\definecolor{G}{named}{ForestGreen}
\definecolor{B}{named}{RoyalBlue}
\definecolor{C}{named}{Cyan}
\definecolor{M}{named}{Magenta}
\definecolor{Y}{named}{YellowOrange}
\definecolor{background}{rgb}{0.82, 0.82, 0.92}
\dimen255=\textwidth
\advance\dimen255 by -2\fboxsep
\noindent
\colorbox{background}
{%
\parbox{\dimen255}
{%
\rule[-0.5ex]{0pt}{2.5ex}\hspace*{0.0em}\textbackslash{}begin\{tikzpicture\}\\
\rule[-0.5ex]{0pt}{2.5ex}\hspace*{1.0em}\textbackslash{}fill~(0,~0)~circle[radius=1pt]~node[below]\{A\};\\
\rule[-0.5ex]{0pt}{2.5ex}\hspace*{1.0em}\textcolor{G}{\textbf{\%~absolute~Koordinaten}}\\
\rule[-0.5ex]{0pt}{2.5ex}\hspace*{1.0em}\textbackslash{}draw~\textcolor{R}{\textbf{(0,~0)}}~{-}{-}~\textcolor{R}{\textbf{(1,~1)}}~{-}{-}~\textcolor{R}{\textbf{(0,~1)}}~{-}{-}~\textcolor{R}{\textbf{(1,~0)}};\\
\rule[-0.5ex]{0pt}{2.5ex}\hspace*{1.0em}\textbackslash{}begin\{scope\}[xshift=3cm]\\
\rule[-0.5ex]{0pt}{2.5ex}\hspace*{2.0em}\textbackslash{}fill~(0,~0)~circle[radius=1pt]~node[below]\{B\};\\
\rule[-0.5ex]{0pt}{2.5ex}\hspace*{2.0em}\textcolor{G}{\textbf{\%~relative~Koordinaten~ohne~Verschiebung~des~Bezugspunktes}}\\
\rule[-0.5ex]{0pt}{2.5ex}\hspace*{2.0em}\textbackslash{}draw~(0,~0)~{-}{-}~(1,~1)~{-}{-}~\textcolor{R}{\textbf{+(0,~1)}}~{-}{-}~\textcolor{R}{\textbf{+(1,~0)}};\\
\rule[-0.5ex]{0pt}{2.5ex}\hspace*{1.0em}\textbackslash{}end\{scope\}\\
\rule[-0.5ex]{0pt}{2.5ex}\hspace*{1.0em}\textbackslash{}begin\{scope\}[xshift=6cm]\\
\rule[-0.5ex]{0pt}{2.5ex}\hspace*{2.0em}\textbackslash{}fill~(0,~0)~circle[radius=1pt]~node[below]\{C\};\\
\rule[-0.5ex]{0pt}{2.5ex}\hspace*{2.0em}\textcolor{G}{\textbf{\%~relative~Koordinaten~mit~Verschiebung~des~Bezugspunktes}}\\
\rule[-0.5ex]{0pt}{2.5ex}\hspace*{2.0em}\textbackslash{}draw~(0,~0)~{-}{-}~(1,~1)~{-}{-}~\textcolor{R}{\textbf{++(0,~1)}}~{-}{-}~\textcolor{R}{\textbf{++(1,~0)}};\\
\rule[-0.5ex]{0pt}{2.5ex}\hspace*{1.0em}\textbackslash{}end\{scope\}\\
\rule[-0.5ex]{0pt}{2.5ex}\hspace*{0.0em}\textbackslash{}end\{tikzpicture\}}%
}%
\endgroup

\end{footnotesize}

% -------------
\section{Pfade}
% -------------

% ----------------------
\subsection{Polygonzüge}
% ----------------------
\begin{minipage}{0.7\textwidth}
  \footnotesize
  \input{lines.code}
\end{minipage}\hfill
\begin{minipage}{0.29\textwidth}
  \centering
  \begin{tikzpicture}[line width=1pt]
    % gradlinige Verbindung der gegebenen Koordinaten
    \draw (0, 0) -- (1, 1) -- (2, 0) -- (1, 0) -- (2, 1);
  \end{tikzpicture}
\end{minipage}

\begin{minipage}{0.7\textwidth}
  \footnotesize
  \input{linesvh.code}
\end{minipage}\hfill
\begin{minipage}{0.29\textwidth}
  \centering
  \begin{tikzpicture}[line width=1pt]
    % gradlinige Verbindung der gegebenen Koordinaten, aber
    % parallel zu den Koordinatenachsen:
    % erst vertikal dann horizontal
    \draw (0, 0) |- (3, 2);
  \end{tikzpicture}
\end{minipage}

\begin{minipage}{0.7\textwidth}
  \footnotesize
  \input{lineshv.code}
\end{minipage}\hfill
\begin{minipage}{0.29\textwidth}
  \centering
  \begin{tikzpicture}[line width=1pt]
    % gradlinige Verbindung der gegebenen Koordinaten, aber
    % parallel zu den Koordinatenachsen:
    % erst horizontal dann vertikal
    \draw (0, 0) -| (3, 2);
  \end{tikzpicture}
\end{minipage}

% -----------------------------
\subsection{Geschlossene Pfade}
% -----------------------------
\begin{minipage}{0.7\textwidth}
  \footnotesize
  \input{circle.code}
\end{minipage}\hfill
\begin{minipage}{0.29\textwidth}
  \centering
  \begin{tikzpicture}[line width=1pt]
    % Kreis mit Radius 8mm um den Mittelpunkt (0, 0)
    \draw (0, 0) circle[radius=8mm];
  \end{tikzpicture}
\end{minipage}

\begin{minipage}{0.7\textwidth}
  \footnotesize
  \input{ellipse.code}
\end{minipage}\hfill
\begin{minipage}{0.29\textwidth}
  \centering
  \begin{tikzpicture}[line width=1pt]
    % Ellipse mit dem Mittelpunkt (0, 0)
    % grosse Halbachse: 16mm (in x-Richtung)
    % kleine Halbachse:  8mm (in y-Richtung)
    \draw (0, 0) circle[x radius=16mm, y radius=8mm];
  \end{tikzpicture}
\end{minipage}

\begin{minipage}{0.7\textwidth}
  \footnotesize
  % parcolor version 2011-09-30
\begingroup
\ttfamily
\definecolor{R}{named}{Red}
\definecolor{G}{named}{ForestGreen}
\definecolor{B}{named}{RoyalBlue}
\definecolor{C}{named}{Cyan}
\definecolor{M}{named}{Magenta}
\definecolor{Y}{named}{YellowOrange}
\definecolor{background}{rgb}{0.82, 0.82, 0.92}
\dimen255=\textwidth
\advance\dimen255 by -2\fboxsep
\noindent
\colorbox{background}
{%
\parbox{\dimen255}
{%
\rule[-0.5ex]{0pt}{2.5ex}\hspace*{0.0em}\textbackslash{}begin\{tikzpicture\}\\
\rule[-0.5ex]{0pt}{2.5ex}\hspace*{1.0em}\textcolor{G}{\textbf{\%~zwei~Eckpunkte~definieren~ein~Rechteck}}\\
\rule[-0.5ex]{0pt}{2.5ex}\hspace*{1.0em}\textcolor{G}{\textbf{\%~links~unten:~(0,~0)}}\\
\rule[-0.5ex]{0pt}{2.5ex}\hspace*{1.0em}\textcolor{G}{\textbf{\%~rechts~oben:~(3,~2)}}\\
\rule[-0.5ex]{0pt}{2.5ex}\hspace*{1.0em}\textbackslash{}draw~\textcolor{R}{\textbf{(0,~0)~rectangle~(3,~2)}};\\
\rule[-0.5ex]{0pt}{2.5ex}\hspace*{0.0em}\textbackslash{}end\{tikzpicture\}}%
}%
\endgroup

\end{minipage}\hfill
\begin{minipage}{0.29\textwidth}
  \centering
  \begin{tikzpicture}[line width=1pt]
    % zwei Eckpunkte definieren ein Rechteck
    % links unten: (0, 0)
    % rechts oben: (3, 2)
    \draw (0, 0) rectangle (3, 2);
  \end{tikzpicture}
\end{minipage}

\begin{minipage}{0.7\textwidth}
  \footnotesize
  \input{cycle.code}
\end{minipage}\hfill
\begin{minipage}{0.29\textwidth}
  \centering
  \begin{tikzpicture}[line width=1pt]
    % cycle erzeugt einen geschlossenen Pfad, indem
    % eine Verbindung mit dem Anfang hergestellt wird
    \draw (0, 0) -| (3, 2) -- cycle;
  \end{tikzpicture}
\end{minipage}

% ---------------------
\subsection{Kreisbögen}
% ---------------------
\begin{minipage}{0.7\textwidth}
  \footnotesize
  % parcolor version 2011-09-30
\begingroup
\ttfamily
\definecolor{R}{named}{Red}
\definecolor{G}{named}{ForestGreen}
\definecolor{B}{named}{RoyalBlue}
\definecolor{C}{named}{Cyan}
\definecolor{M}{named}{Magenta}
\definecolor{Y}{named}{YellowOrange}
\definecolor{background}{rgb}{0.82, 0.82, 0.92}
\dimen255=\textwidth
\advance\dimen255 by -2\fboxsep
\noindent
\colorbox{background}
{%
\parbox{\dimen255}
{%
\rule[-0.5ex]{0pt}{2.5ex}\hspace*{0.0em}\textbackslash{}begin\{tikzpicture\}\\
\rule[-0.5ex]{0pt}{2.5ex}\hspace*{1.0em}\textbackslash{}fill~(0,~0)~circle[radius=1pt];\\
\rule[-0.5ex]{0pt}{2.5ex}\hspace*{1.0em}\textcolor{G}{\textbf{\%~vier~aneinadergesetzte~Halbkreise}}\\
\rule[-0.5ex]{0pt}{2.5ex}\hspace*{1.0em}\textbackslash{}draw~\textcolor{R}{\textbf{(6mm,~6mm)}}\\
\rule[-0.5ex]{0pt}{2.5ex}\hspace*{4.0em}\textcolor{R}{\textbf{arc}}[\textcolor{R}{\textbf{start~angle=0}},~~~\textcolor{R}{\textbf{end~angle=180}},~\textcolor{R}{\textbf{radius=6mm}}]\\
\rule[-0.5ex]{0pt}{2.5ex}\hspace*{4.0em}\textcolor{R}{\textbf{arc}}[\textcolor{R}{\textbf{start~angle=90}},~~\textcolor{R}{\textbf{end~angle=270}},~\textcolor{R}{\textbf{radius=6mm}}]\\
\rule[-0.5ex]{0pt}{2.5ex}\hspace*{4.0em}\textcolor{R}{\textbf{arc}}[\textcolor{R}{\textbf{start~angle=180}},~\textcolor{R}{\textbf{end~angle=360}},~\textcolor{R}{\textbf{radius=6mm}}]\\
\rule[-0.5ex]{0pt}{2.5ex}\hspace*{4.0em}\textcolor{R}{\textbf{arc}}[\textcolor{R}{\textbf{start~angle=270}},~\textcolor{R}{\textbf{end~angle=450}},~\textcolor{R}{\textbf{radius=6mm}}]\\
\rule[-0.5ex]{0pt}{2.5ex}\hspace*{4.0em}{-}{-}~cycle;\\
\rule[-0.5ex]{0pt}{2.5ex}\hspace*{0.0em}\textbackslash{}end\{tikzpicture\}}%
}%
\endgroup

\end{minipage}\hfill
\begin{minipage}{0.29\textwidth}
  \centering
  \begin{tikzpicture}[line width=1pt]
    \fill (0, 0) circle[radius=1pt];
    % vier aneinadergesetzte Halbkreise
    \draw (6mm, 6mm)
          arc[start angle=0,   end angle=180, radius=6mm]
          arc[start angle=90,  end angle=270, radius=6mm]
          arc[start angle=180, end angle=360, radius=6mm]
          arc[start angle=270, end angle=450, radius=6mm]
          -- cycle;
  \end{tikzpicture}
\end{minipage}

\begin{minipage}{0.7\textwidth}
  \footnotesize
  % parcolor version 2011-09-30
\begingroup
\ttfamily
\definecolor{R}{named}{Red}
\definecolor{G}{named}{ForestGreen}
\definecolor{B}{named}{RoyalBlue}
\definecolor{C}{named}{Cyan}
\definecolor{M}{named}{Magenta}
\definecolor{Y}{named}{YellowOrange}
\definecolor{background}{rgb}{0.82, 0.82, 0.92}
\dimen255=\textwidth
\advance\dimen255 by -2\fboxsep
\noindent
\colorbox{background}
{%
\parbox{\dimen255}
{%
\rule[-0.5ex]{0pt}{2.5ex}\hspace*{0.0em}\textbackslash{}begin\{tikzpicture\}\\
\rule[-0.5ex]{0pt}{2.5ex}\hspace*{1.0em}\textcolor{G}{\textbf{\%~eine~Liste~von~Optionen}}\\
\rule[-0.5ex]{0pt}{2.5ex}\hspace*{1.0em}\textcolor{R}{\textbf{\textbackslash{}tikzstyle}}\{\textcolor{B}{\textbf{myopts}}\}=[\textcolor{R}{\textbf{start~angle=0}},~~~\textcolor{R}{\textbf{x~radius=4mm}},\\
\rule[-0.5ex]{0pt}{2.5ex}\hspace*{11.0em}\textcolor{R}{\textbf{delta~angle=180}},~\textcolor{R}{\textbf{y~radius=12mm}}];\\
\rule[-0.5ex]{0pt}{2.5ex}\hspace*{1.0em}\textcolor{G}{\textbf{\%~fuenf~identische~aneinadergesetzte~Ellipsen}}\\
\rule[-0.5ex]{0pt}{2.5ex}\hspace*{1.0em}\textbackslash{}draw~\textcolor{R}{\textbf{(0,~0)}}~\textcolor{R}{\textbf{arc}}[\textcolor{B}{\textbf{myopts}}]~[\textcolor{R}{\textbf{rotate=72}}]\\
\rule[-0.5ex]{0pt}{2.5ex}\hspace*{7.5em}\textcolor{R}{\textbf{arc}}[\textcolor{B}{\textbf{myopts}}]~[\textcolor{R}{\textbf{rotate=72}}]\\
\rule[-0.5ex]{0pt}{2.5ex}\hspace*{7.5em}\textcolor{R}{\textbf{arc}}[\textcolor{B}{\textbf{myopts}}]~[\textcolor{R}{\textbf{rotate=72}}]\\
\rule[-0.5ex]{0pt}{2.5ex}\hspace*{7.5em}\textcolor{R}{\textbf{arc}}[\textcolor{B}{\textbf{myopts}}]~[\textcolor{R}{\textbf{rotate=72}}]\\
\rule[-0.5ex]{0pt}{2.5ex}\hspace*{7.5em}\textcolor{R}{\textbf{arc}}[\textcolor{B}{\textbf{myopts}}]~{-}{-}~cycle;\\
\rule[-0.5ex]{0pt}{2.5ex}\hspace*{0.0em}\textbackslash{}end\{tikzpicture\}}%
}%
\endgroup

\end{minipage}\hfill
\begin{minipage}{0.29\textwidth}
  \centering
  \begin{tikzpicture}[line width=1pt]
    % eine Liste von Optionen
    \tikzstyle{myopts}=[start angle=0,   x radius=4mm, 
                        delta angle=180, y radius=12mm];
    % fuenf identische aneinadergesetzte Ellipsen
    \draw (0, 0) arc[myopts] [rotate=72]
                 arc[myopts] [rotate=72]
                 arc[myopts] [rotate=72]
                 arc[myopts] [rotate=72]
                 arc[myopts] -- cycle;
  \end{tikzpicture}
\end{minipage}

% -----------------------
\subsection{Bézierkurven}
% -----------------------
\begin{minipage}{0.7\textwidth}
  \footnotesize
  \input{bezier1.code}
\end{minipage}\hfill
\begin{minipage}{0.29\textwidth}
  \centering
  \begin{tikzpicture}[line width=1pt]
    % Bezierkurve mit einem Kontrollpunkt
    \draw (0, 0) .. controls (1, 3) .. (3, 0);
    % Hilfslinien
    \draw[line width=0.8pt, draw=Red, style=dotted]
         (0, 0) -- (1, 3) -- (3, 0);
    % der Kontrollpunkt
    \fill[fill=Red] (1, 3) circle[radius=1pt];
  \end{tikzpicture}
\end{minipage}

\begin{minipage}{0.7\textwidth}
  \footnotesize
  % parcolor version 2011-09-30
\begingroup
\ttfamily
\definecolor{R}{named}{Red}
\definecolor{G}{named}{ForestGreen}
\definecolor{B}{named}{RoyalBlue}
\definecolor{C}{named}{Cyan}
\definecolor{M}{named}{Magenta}
\definecolor{Y}{named}{YellowOrange}
\definecolor{background}{rgb}{0.82, 0.82, 0.92}
\dimen255=\textwidth
\advance\dimen255 by -2\fboxsep
\noindent
\colorbox{background}
{%
\parbox{\dimen255}
{%
\rule[-0.5ex]{0pt}{2.5ex}\hspace*{0.0em}\textbackslash{}begin\{tikzpicture\}[line~width=1pt]\\
\rule[-0.5ex]{0pt}{2.5ex}\hspace*{1.0em}\textcolor{G}{\textbf{\%~Bezierkurve~mit~zwei~Kontrollpunkten}}\\
\rule[-0.5ex]{0pt}{2.5ex}\hspace*{1.0em}\textbackslash{}draw~(0,~0)~\textcolor{R}{\textbf{..~controls~(1,~2)~and~(1,~{-}1)~..}}~(3,~0);\\
\rule[-0.5ex]{0pt}{2.5ex}\hspace*{1.0em}\textcolor{G}{\textbf{\%~Hilfslinien}}\\
\rule[-0.5ex]{0pt}{2.5ex}\hspace*{1.0em}\textbackslash{}draw[line~width=0.8pt,~draw=Red,~style=dotted]\\
\rule[-0.5ex]{0pt}{2.5ex}\hspace*{3.5em}(1,~~2)~{-}{-}~(0,~0)\\
\rule[-0.5ex]{0pt}{2.5ex}\hspace*{3.5em}(1,~{-}1)~{-}{-}~(3,~0);\\
\rule[-0.5ex]{0pt}{2.5ex}\hspace*{1.0em}\textcolor{G}{\textbf{\%~Kontrollpunkte}}\\
\rule[-0.5ex]{0pt}{2.5ex}\hspace*{1.0em}\textbackslash{}fill[fill=Red]~(1,~~2)~circle[radius=1pt];\\
\rule[-0.5ex]{0pt}{2.5ex}\hspace*{1.0em}\textbackslash{}fill[fill=Red]~(1,~{-}1)~circle[radius=1pt];\\
\rule[-0.5ex]{0pt}{2.5ex}\hspace*{0.0em}\textbackslash{}end\{tikzpicture\}}%
}%
\endgroup

\end{minipage}\hfill
\begin{minipage}{0.29\textwidth}
  \centering
  \begin{tikzpicture}[line width=1pt]
    % Bezierkurve mit zwei Kontrollpunkten
    \draw (0, 0) .. controls (1, 2) and (1, -1) .. (3, 0);
    % Hilfslinien
    \draw[line width=0.8pt, draw=Red, style=dotted]
         (1,  2) -- (0, 0)
         (1, -1) -- (3, 0);
    % Kontrollpunkte
    \fill[fill=Red] (1,  2) circle[radius=1pt];
    \fill[fill=Red] (1, -1) circle[radius=1pt];
  \end{tikzpicture}
\end{minipage}

% ------------------------------------------------
\subsection{Verbindungen mit \texttt{to[out, in]}}
% ------------------------------------------------
\begin{minipage}{0.7\textwidth}
  \footnotesize
  \input{tooutin.code}
\end{minipage}\hfill
\begin{minipage}{0.29\textwidth}
  \centering
  \begin{tikzpicture}[line width=1pt]
    % Verbindungen mit vorgegebener Richtung
    \draw (0, 0) to[out=45,  in=225] (1, 2)
                 to[out=270, in=90]  (3, 0);
  \end{tikzpicture}
\end{minipage}

% -----------------------------------------
\subsection{Verbindungen mit \texttt{plot}}
% -----------------------------------------
\begin{minipage}{0.7\textwidth}
  \footnotesize
  % parcolor version 2011-09-30
\begingroup
\ttfamily
\definecolor{R}{named}{Red}
\definecolor{G}{named}{ForestGreen}
\definecolor{B}{named}{RoyalBlue}
\definecolor{C}{named}{Cyan}
\definecolor{M}{named}{Magenta}
\definecolor{Y}{named}{YellowOrange}
\definecolor{background}{rgb}{0.82, 0.82, 0.92}
\dimen255=\textwidth
\advance\dimen255 by -2\fboxsep
\noindent
\colorbox{background}
{%
\parbox{\dimen255}
{%
\rule[-0.5ex]{0pt}{2.5ex}\hspace*{0.0em}\textcolor{G}{\textbf{\%~coordinates~\{...\}~legt~die~zu~verbindenden~Punkte~fest}}\\
\rule[-0.5ex]{0pt}{2.5ex}\hspace*{0.0em}\textbackslash{}draw~\textcolor{R}{\textbf{plot}}[\textcolor{R}{\textbf{smooth}}]~\textcolor{R}{\textbf{coordinates}}\\
\rule[-0.5ex]{0pt}{2.5ex}\hspace*{0.0em}\{\\
\rule[-0.5ex]{0pt}{2.5ex}\hspace*{1.0em}({-}1.5,~2.25)~~({-}1.4,~1.96)~~({-}1.3,~1.69)~~({-}1.2,~1.44)\\
\rule[-0.5ex]{0pt}{2.5ex}\hspace*{1.0em}({-}1.1,~1.21)~~({-}1.0,~1.00)~~({-}0.9,~0.81)~~({-}0.8,~0.64)\\
\rule[-0.5ex]{0pt}{2.5ex}\hspace*{1.0em}...\\
\rule[-0.5ex]{0pt}{2.5ex}\hspace*{1.0em}(~0.5,~0.25)~~(~0.6,~0.36)~~(~0.7,~0.49)~~(~0.8,~0.64)\\
\rule[-0.5ex]{0pt}{2.5ex}\hspace*{1.0em}(~0.9,~0.81)~~(~1.0,~1.00)~~(~1.1,~1.21)~~(~1.2,~1.44)\\
\rule[-0.5ex]{0pt}{2.5ex}\hspace*{1.0em}(~1.3,~1.69)~~(~1.4,~1.96)~~(~1.5,~2.25)\\
\rule[-0.5ex]{0pt}{2.5ex}\hspace*{0.0em}\};}%
}%
\endgroup

\end{minipage}\hfill
\begin{minipage}{0.29\textwidth}
  \centering
  \begin{tikzpicture}[line width=1pt]
    % coordinates {...} legt die zu verbindenden Punkte fest
    \draw plot[smooth] coordinates
    {
      (-1.5, 2.25)  (-1.4, 1.96)  (-1.3, 1.69)  (-1.2, 1.44)
      (-1.1, 1.21)  (-1.0, 1.00)  (-0.9, 0.81)  (-0.8, 0.64)
      (-0.7, 0.49)  (-0.6, 0.36)  (-0.5, 0.25)  (-0.4, 0.16)
      (-0.3, 0.09)  (-0.2, 0.04)  (-0.1, 0.01)  ( 0.0, 0.00)
      ( 0.1, 0.01)  ( 0.2, 0.04)  ( 0.3, 0.09)  ( 0.4, 0.16)
      ( 0.5, 0.25)  ( 0.6, 0.36)  ( 0.7, 0.49)  ( 0.8, 0.64)
      ( 0.9, 0.81)  ( 1.0, 1.00)  ( 1.1, 1.21)  ( 1.2, 1.44)
      ( 1.3, 1.69)  ( 1.4, 1.96)  ( 1.5, 2.25)
    };
  \end{tikzpicture}
\end{minipage}

\begin{minipage}{0.7\textwidth}
  \footnotesize
  \input{plotfile.code}
  % parcolor version 2011-09-30
\begingroup
\ttfamily
\definecolor{R}{named}{Red}
\definecolor{G}{named}{ForestGreen}
\definecolor{B}{named}{RoyalBlue}
\definecolor{C}{named}{Cyan}
\definecolor{M}{named}{Magenta}
\definecolor{Y}{named}{YellowOrange}
\definecolor{background}{rgb}{0.82, 0.82, 0.92}
\dimen255=\textwidth
\advance\dimen255 by -2\fboxsep
\noindent
\colorbox{background}
{%
\parbox{\dimen255}
{%
\underline{\makebox[\linewidth]{\hspace*{\fill}\textsf{\textit{spirograph.xy}}\hspace*{\fill}}}\\
\rule[-0.5ex]{0pt}{2.5ex}\hspace*{1.0em}0.250~~{-}0.000\\
\rule[-0.5ex]{0pt}{2.5ex}\hspace*{1.0em}0.250~~{-}0.022\\
\rule[-0.5ex]{0pt}{2.5ex}\hspace*{1.0em}0.250~~{-}0.044\\
\rule[-0.5ex]{0pt}{2.5ex}\hspace*{1.0em}\ldots}%
}%
\endgroup

\end{minipage}\hfill
\begin{minipage}{0.29\textwidth}
  \centering
  \begin{tikzpicture}[line width=1pt]
    % Koordinaten aus einer Datei laden
    \draw plot[smooth] file{spirograph.xy};
  \end{tikzpicture}
\end{minipage}

% ----------------------
\subsection{Gitternetze}
% ----------------------
\begin{minipage}{0.7\textwidth}
  \footnotesize
  \input{grid.code}
\end{minipage}\hfill
\begin{minipage}{0.29\textwidth}
  \centering
  \begin{tikzpicture}
    % ein Gitternetz mit verschiedenen Schrittweiten
    \draw (0, 0) grid[xstep=3mm, ystep=2mm] (3, 1);
  \end{tikzpicture}
\end{minipage}

% ---------------------------------------------
\subsection{Flächen mit \texttt{even odd rule}}
% ---------------------------------------------
\begin{minipage}{0.7\textwidth}
  \footnotesize
  % parcolor version 2011-09-30
\begingroup
\ttfamily
\definecolor{R}{named}{Red}
\definecolor{G}{named}{ForestGreen}
\definecolor{B}{named}{RoyalBlue}
\definecolor{C}{named}{Cyan}
\definecolor{M}{named}{Magenta}
\definecolor{Y}{named}{YellowOrange}
\definecolor{background}{rgb}{0.82, 0.82, 0.92}
\dimen255=\textwidth
\advance\dimen255 by -2\fboxsep
\noindent
\colorbox{background}
{%
\parbox{\dimen255}
{%
\rule[-0.5ex]{0pt}{2.5ex}\hspace*{0.0em}\textbackslash{}begin\{tikzpicture\}\\
\rule[-0.5ex]{0pt}{2.5ex}\hspace*{1.0em}\textcolor{G}{\textbf{\%~ein~(einziger)~Pfad,~der~aus~zwei~Kreisen~besteht}}\\
\rule[-0.5ex]{0pt}{2.5ex}\hspace*{1.0em}\textbackslash{}filldraw[fill=Cerulean,~draw=Black,~\textcolor{R}{\textbf{even~odd~rule}}]\\
\rule[-0.5ex]{0pt}{2.5ex}\hspace*{5.5em}(0,~0)~circle[radius=6mm]\\
\rule[-0.5ex]{0pt}{2.5ex}\hspace*{5.5em}(0,~0)~circle[radius=9mm];\\
\rule[-0.5ex]{0pt}{2.5ex}\hspace*{0.0em}\textbackslash{}end\{tikzpicture\}}%
}%
\endgroup

\end{minipage}\hfill
\begin{minipage}{0.29\textwidth}
  \centering
  \begin{tikzpicture}[line width=1pt]
    % ein (einziger) Pfad, der aus zwei Kreisen besteht
    \filldraw[fill=Cerulean, draw=Black, even odd rule]
             (0, 0) circle[radius=6mm]
             (0, 0) circle[radius=9mm];
  \end{tikzpicture}
\end{minipage}

% --------------------------
\section{Pfadmodifikationen}
% --------------------------

% ------------------------------
\subsection{\texttt{line width}}
% ------------------------------
\begin{minipage}{0.7\textwidth}
  \footnotesize
  \input{linewidth.code}
\end{minipage}\hfill
\begin{minipage}{0.29\textwidth}
  \centering
  \begin{tikzpicture}
    % ein ziemlich dicker Strich
    \draw[line width=5mm] (0, 0) -- (3, 0);
  \end{tikzpicture}
\end{minipage}

% ------------------------
\subsection{\texttt{draw}}
% ------------------------
\begin{minipage}{0.7\textwidth}
  \footnotesize
  \input{draw.code}
\end{minipage}\hfill
\begin{minipage}{0.29\textwidth}
  \centering
  \begin{tikzpicture}
    % ein dicker, roter Strich
    \draw[line width=5mm, draw=RubineRed] (0, 0) -- (3, 0);
  \end{tikzpicture}
\end{minipage}

% ----------------------------
\subsection{\texttt{line cap}}
% ----------------------------
\begin{minipage}{0.7\textwidth}
  \footnotesize
  \input{capround.code}
\end{minipage}\hfill
\begin{minipage}{0.29\textwidth}
  \centering
  \begin{tikzpicture}
    % abgerundete Enden
    \draw[line width=5mm, line cap=round] (0, 0) -- (3, 0);
    \fill[fill=White] (0, 0) circle[radius=1pt];
    \fill[fill=White] (3, 0) circle[radius=1pt];
  \end{tikzpicture}
\end{minipage}

\begin{minipage}{0.7\textwidth}
  \footnotesize
  \input{caprect.code}
\end{minipage}\hfill
\begin{minipage}{0.29\textwidth}
  \centering
  \begin{tikzpicture}
    % ueberstehende rechteckige Enden (default)
    \draw[line width=5mm, line cap=rect] (0, 0) -- (3, 0);
    \fill[fill=White] (0, 0) circle[radius=1pt];
    \fill[fill=White] (3, 0) circle[radius=1pt];
  \end{tikzpicture}
\end{minipage}

\begin{minipage}{0.7\textwidth}
  \footnotesize
  \input{capbutt.code}
\end{minipage}\hfill
\begin{minipage}{0.29\textwidth}
  \centering
  \begin{tikzpicture}
    % buendige rechteckige Enden
    \draw[line width=5mm, line cap=butt] (0, 0) -- (3, 0);
    \fill[fill=White] (0, 0) circle[radius=1pt];
    \fill[fill=White] (3, 0) circle[radius=1pt];
  \end{tikzpicture}
\end{minipage}

% -----------------------------
\subsection{\texttt{line join}}
% -----------------------------
\begin{minipage}{0.7\textwidth}
  \footnotesize
  \input{joinmiter.code}
\end{minipage}\hfill
\begin{minipage}{0.29\textwidth}
  \centering
  \begin{tikzpicture}[line width=3pt]
    % spitze Ecken (default)
    \draw[line join=miter] (0, 0) -- (2, 0.3) -- (0, 0.6);
  \end{tikzpicture}
\end{minipage}

\begin{minipage}{0.7\textwidth}
  \footnotesize
  \input{joinround.code}
\end{minipage}\hfill
\begin{minipage}{0.29\textwidth}
  \centering
  \begin{tikzpicture}[line width=3pt]
    % abgerundete Ecken
    \draw[line join=round] (0, 0) -- (2, 0.3) -- (0, 0.6);
  \end{tikzpicture}
\end{minipage}

\begin{minipage}{0.7\textwidth}
  \footnotesize
  \input{joinbevel.code}
\end{minipage}\hfill
\begin{minipage}{0.29\textwidth}
  \centering
  \begin{tikzpicture}[line width=3pt]
    % abgeflachte Ecken
    \draw[line join=bevel] (0, 0) -- (2, 0.3) -- (0, 0.6);
  \end{tikzpicture}
\end{minipage}

% -----------------------------------
\subsection{\texttt{rounded corners}}
% -----------------------------------
\begin{minipage}{0.7\textwidth}
  \footnotesize
  \input{roundedcorners.code}
\end{minipage}\hfill
\begin{minipage}{0.29\textwidth}
  \centering
  \begin{tikzpicture}[line width=1pt, scale=0.75]
    % abgerundete Ecken
    \draw[rounded corners=2mm] plot coordinates
    {
      (0, 0) (2, 0) (0, 2)
      (2, 2) (1, 4) (0, 2)
      (0, 0) (2, 2) (2, 0)
    };
  \end{tikzpicture}
\end{minipage}

% ---------------------------
\subsection{\texttt{shorten}}
% ---------------------------
\begin{minipage}{0.7\textwidth}
  \footnotesize
  \input{shorten.code}
\end{minipage}\hfill
\begin{minipage}{0.29\textwidth}
  \centering
  \begin{tikzpicture}[line width=1pt]
    \fill (0, 0) circle[radius=1pt];
    \fill (3, 0) circle[radius=1pt];
    % verkuerzte Strecken
    \draw[shorten <=1mm, shorten >=5mm] (0, 0) -- (3, 0);
  \end{tikzpicture}
\end{minipage}

% -----------------
% -------------------------
\subsection{\texttt{style}}
% -------------------------
\begin{minipage}{0.7\textwidth}
  \footnotesize
  \input{styledotted.code}
\end{minipage}\hfill
\begin{minipage}{0.29\textwidth}
  \centering
  \begin{tikzpicture}[line width=1pt]
    % eine gepunktete Linie
    \draw[style=dotted] (0, 0) -- (3, 0);
  \end{tikzpicture}
\end{minipage}

\begin{minipage}{0.7\textwidth}
  \footnotesize
  \input{stylelooselydotted.code}
\end{minipage}\hfill
\begin{minipage}{0.29\textwidth}
  \centering
  \begin{tikzpicture}[line width=1pt]
    % gepunktet mit groesseren Abstaenden
    \draw[style=loosely dotted] (0, 0) -- (3, 0);
  \end{tikzpicture}
\end{minipage}

\begin{minipage}{0.7\textwidth}
  \footnotesize
  \input{styledenselydotted.code}
\end{minipage}\hfill
\begin{minipage}{0.29\textwidth}
  \centering
  \begin{tikzpicture}[line width=1pt]
    % gepunktet mit kleineren Abstaenden
    \draw[style=densely dotted] (0, 0) -- (3, 0);
  \end{tikzpicture}
\end{minipage}

\begin{minipage}{0.7\textwidth}
  \footnotesize
  % parcolor version 2011-09-30
\begingroup
\ttfamily
\definecolor{R}{named}{Red}
\definecolor{G}{named}{ForestGreen}
\definecolor{B}{named}{RoyalBlue}
\definecolor{C}{named}{Cyan}
\definecolor{M}{named}{Magenta}
\definecolor{Y}{named}{YellowOrange}
\definecolor{background}{rgb}{0.82, 0.82, 0.92}
\dimen255=\textwidth
\advance\dimen255 by -2\fboxsep
\noindent
\colorbox{background}
{%
\parbox{\dimen255}
{%
\rule[-0.5ex]{0pt}{2.5ex}\hspace*{0.0em}\textcolor{G}{\textbf{\%~eine~gestrichelte~Linie}}\\
\rule[-0.5ex]{0pt}{2.5ex}\hspace*{0.0em}\textbackslash{}draw[\textcolor{R}{\textbf{style=dashed}}]~(0,~0)~{-}{-}~(3,~0);}%
}%
\endgroup

\end{minipage}\hfill
\begin{minipage}{0.29\textwidth}
  \centering
  \begin{tikzpicture}[line width=1pt]
    % eine gestrichelte Linie
    \draw[style=dashed] (0, 0) -- (3, 0);
  \end{tikzpicture}
\end{minipage}

\begin{minipage}{0.7\textwidth}
  \footnotesize
  \input{stylelooselydashed.code}
\end{minipage}\hfill
\begin{minipage}{0.29\textwidth}
  \centering
  \begin{tikzpicture}[line width=1pt]
    % gestrichelt mit groesseren Abstaenden
    \draw[style=loosely dashed] (0, 0) -- (3, 0);
  \end{tikzpicture}
\end{minipage}

\begin{minipage}{0.7\textwidth}
  \footnotesize
  \input{styledenselydashed.code}
\end{minipage}\hfill
\begin{minipage}{0.29\textwidth}
  \centering
  \begin{tikzpicture}[line width=1pt]
    % gestrichelt mit kleineren Abstaenden
    \draw[style=densely dashed] (0, 0) -- (3, 0);
  \end{tikzpicture}
\end{minipage}


% -----------------

% --------------------------------
\subsection{\texttt{dash pattern}}
% --------------------------------
\begin{minipage}{0.7\textwidth}
  \footnotesize
  \input{dashpattern.code}
\end{minipage}\hfill
\begin{minipage}{0.29\textwidth}
  \centering
  \begin{tikzpicture}[line width=1pt]
    % ein eigenes Muster
    \draw[dash pattern=on 3pt off 2pt on 1pt off 2pt]
         (0, 0) -- (3, 0);
  \end{tikzpicture}
\end{minipage}

% ------------------------
% -----------------------
\subsection{Pfeilspitzen}
% -----------------------
\begin{minipage}{0.7\textwidth}
  \footnotesize
  \input{arrowright.code}
\end{minipage}\hfill
\begin{minipage}{0.29\textwidth}
  \centering
  \begin{tikzpicture}[line width=1pt]
    % Pfeilspitze rechts
    \draw[->] (0, 0) -- (3, 0);
  \end{tikzpicture}
\end{minipage}

\begin{minipage}{0.7\textwidth}
  \footnotesize
  \input{arrowleft.code}
\end{minipage}\hfill
\begin{minipage}{0.29\textwidth}
  \centering
  \begin{tikzpicture}[line width=1pt]
    % Pfeilspitze links
    \draw[<-] (0, 0) -- (3, 0);
  \end{tikzpicture}
\end{minipage}

\begin{minipage}{0.7\textwidth}
  \footnotesize
  \input{arrowleftright.code}
\end{minipage}\hfill
\begin{minipage}{0.29\textwidth}
  \centering
  \begin{tikzpicture}[line width=1pt]
    % Pfeilspitzen links und rechts
    \draw[<->] (0, 0) -- (3, 0);
  \end{tikzpicture}
\end{minipage}

\begin{minipage}{0.7\textwidth}
  \footnotesize
  \input{arrowdouble.code}
\end{minipage}\hfill
\begin{minipage}{0.29\textwidth}
  \centering
  \begin{tikzpicture}[line width=1pt]
    % doppelte Spitze
    \draw[<<->>] (0, 0) -- (3, 0);
  \end{tikzpicture}
\end{minipage}

\begin{minipage}{0.7\textwidth}
  \footnotesize
  \input{arrowpipeonly.code}
\end{minipage}\hfill
\begin{minipage}{0.29\textwidth}
  \centering
  \begin{tikzpicture}[line width=1pt]
    % stumpfe Enden
    \draw[|-|] (0, 0) -- (3, 0);
  \end{tikzpicture}
\end{minipage}

\begin{minipage}{0.7\textwidth}
  \footnotesize
  \input{arrowpipe.code}
\end{minipage}\hfill
\begin{minipage}{0.29\textwidth}
  \centering
  \begin{tikzpicture}[line width=1pt]
    % Spitze mit Begrenzung
    \draw[|<->|, >=latex] (0, 0) -- (3, 0);
  \end{tikzpicture}
\end{minipage}

\begin{minipage}{0.7\textwidth}
  \footnotesize
  \input{arrowlatex.code}
\end{minipage}\hfill
\begin{minipage}{0.29\textwidth}
  \centering
  \begin{tikzpicture}[line width=1pt]
    % Form: latex
    \draw[<->, >=latex] (0, 0) -- (3, 0);
  \end{tikzpicture}
\end{minipage}

\begin{minipage}{0.7\textwidth}
  \footnotesize
  \input{arrowstealth.code}
\end{minipage}\hfill
\begin{minipage}{0.29\textwidth}
  \centering
  \begin{tikzpicture}[line width=1pt]
    % Form: stealth
    \draw[<->, >=stealth] (0, 0) -- (3, 0);
  \end{tikzpicture}
\end{minipage}

\begin{minipage}{0.7\textwidth}
  \footnotesize
  \input{arrowtoreversed.code}
\end{minipage}\hfill
\begin{minipage}{0.29\textwidth}
  \centering
  \begin{tikzpicture}[line width=1pt]
    % doppelte Spitze: to reversed
    \draw[<<->>, >=to reversed] (0, 0) -- (3, 0);
  \end{tikzpicture}
\end{minipage}

\begin{minipage}{0.7\textwidth}
  \footnotesize
  \input{arrowstealthreversed.code}
\end{minipage}\hfill
\begin{minipage}{0.29\textwidth}
  \centering
  \begin{tikzpicture}[line width=1pt]
    % doppelte Spitze: stealth reversed
    \draw[<<->>, >=stealth reversed] (0, 0) -- (3, 0);
  \end{tikzpicture}
\end{minipage}

\begin{minipage}{0.7\textwidth}
  \footnotesize
  \input{arrowcircle.code}
\end{minipage}\hfill
\begin{minipage}{0.29\textwidth}
  \centering
  \begin{tikzpicture}[line width=1pt]
    % runde Enden
    % benoetigt \usetikzlibrary{arrows}
    \draw[o-o] (0, 0) -- (3, 0);
  \end{tikzpicture}
\end{minipage}

\begin{minipage}{0.7\textwidth}
  \footnotesize
  \input{arrowparenthesis.code}
\end{minipage}\hfill
\begin{minipage}{0.29\textwidth}
  \centering
  \begin{tikzpicture}[line width=1pt]
    % abgerundete Spitzen
    % benoetigt \usetikzlibrary{arrows}
    \draw[(-)] (0, 0) -- (3, 0);
  \end{tikzpicture}
\end{minipage}

\begin{minipage}{0.7\textwidth}
  \footnotesize
  \input{arrowbrackets.code}
\end{minipage}\hfill
\begin{minipage}{0.29\textwidth}
  \centering
  \begin{tikzpicture}[line width=1pt]
    % eckige Spitzen
    % benoetigt \usetikzlibrary{arrows}
    \draw[[-{]}] (0, 0) -- (3, 0);
  \end{tikzpicture}
\end{minipage}

\begin{minipage}{0.7\textwidth}
  \footnotesize
  \input{arrowdiamond.code}
\end{minipage}\hfill
\begin{minipage}{0.29\textwidth}
  \centering
  \begin{tikzpicture}[line width=1pt]
    % Form: diamond
    % benoetigt \usetikzlibrary{arrows}
    \draw[<->, >=diamond] (0, 0) -- (3, 0);
  \end{tikzpicture}
\end{minipage}


% ------------------------

% ------------------------
% -----------------------
\subsection{Dekorationen}
% -----------------------
\begin{minipage}{0.7\textwidth}
  \footnotesize
  \input{decorationbent.code}
\end{minipage}\hfill
\begin{minipage}{0.29\textwidth}
  \centering
  \begin{tikzpicture}[line width=1pt]
    % benoetigt \usetikzlibrary{decorations.pathmorphing}
    \draw[decorate, decoration=bent] (0, 0) -- (3, 0);
  \end{tikzpicture}
\end{minipage}

\begin{minipage}{0.7\textwidth}
  \footnotesize
  \input{decorationbumps.code}
\end{minipage}\hfill
\begin{minipage}{0.29\textwidth}
  \centering
  \begin{tikzpicture}[line width=1pt]
    % benoetigt \usetikzlibrary{decorations.pathmorphing}
    \draw[decorate, decoration=bumps] (0, 0) -- (3, 0);
  \end{tikzpicture}
\end{minipage}

\begin{minipage}{0.7\textwidth}
  \footnotesize
  \input{decorationcoil.code}
\end{minipage}\hfill
\begin{minipage}{0.29\textwidth}
  \centering
  \begin{tikzpicture}[line width=1pt]
    % benoetigt \usetikzlibrary{decorations.pathmorphing}
    \draw[decorate, decoration=coil] (0, 0) -- (3, 0);
  \end{tikzpicture}
\end{minipage}

\begin{minipage}{0.7\textwidth}
  \footnotesize
  \input{decorationrandomsteps.code}
\end{minipage}\hfill
\begin{minipage}{0.29\textwidth}
  \centering
  \begin{tikzpicture}[line width=1pt]
    % benoetigt \usetikzlibrary{decorations.pathmorphing}
    \draw[decorate, decoration=random steps] (0, 0) -- (3, 0);
  \end{tikzpicture}
\end{minipage}

\begin{minipage}{0.7\textwidth}
  \footnotesize
  \input{decorationsaw.code}
\end{minipage}\hfill
\begin{minipage}{0.29\textwidth}
  \centering
  \begin{tikzpicture}[line width=1pt]
    % benoetigt \usetikzlibrary{decorations.pathmorphing}
    \draw[decorate, decoration=saw] (0, 0) -- (3, 0);
  \end{tikzpicture}
\end{minipage}

\begin{minipage}{0.7\textwidth}
  \footnotesize
  \input{decorationsnake.code}
\end{minipage}\hfill
\begin{minipage}{0.29\textwidth}
  \centering
  \begin{tikzpicture}[line width=1pt]
    % benoetigt \usetikzlibrary{decorations.pathmorphing}
    \draw[decorate, decoration=snake] (0, 0) -- (3, 0);
  \end{tikzpicture}
\end{minipage}

\begin{minipage}{0.7\textwidth}
  \footnotesize
  \input{decorationzigzag.code}
\end{minipage}\hfill
\begin{minipage}{0.29\textwidth}
  \centering
  \begin{tikzpicture}[line width=1pt]
    % benoetigt \usetikzlibrary{decorations.pathmorphing}
    \draw[decorate, decoration=zigzag] (0, 0) -- (3, 0);
  \end{tikzpicture}
\end{minipage}

\begin{minipage}{0.7\textwidth}
  \footnotesize
  \input{decorationborder.code}
\end{minipage}\hfill
\begin{minipage}{0.29\textwidth}
  \centering
  \begin{tikzpicture}[line width=1pt]
    % benoetigt \usetikzlibrary{decorations.pathreplacing}
    \draw[decorate, decoration=border] (0, 0) -- (3, 0);
  \end{tikzpicture}
\end{minipage}

\begin{minipage}{0.7\textwidth}
  \footnotesize
  \input{decorationbrace.code}
\end{minipage}\hfill
\begin{minipage}{0.29\textwidth}
  \centering
  \begin{tikzpicture}[line width=1pt]
    % benoetigt \usetikzlibrary{decorations.pathreplacing}
    \draw[decorate, decoration=brace] (0, 0) -- (3, 0);
  \end{tikzpicture}
\end{minipage}

\begin{minipage}{0.7\textwidth}
  \footnotesize
  \input{decorationexpandingwaves.code}
\end{minipage}\hfill
\begin{minipage}{0.29\textwidth}
  \centering
  \begin{tikzpicture}[line width=1pt]
    % benoetigt \usetikzlibrary{decorations.pathreplacing}
    \draw[decorate, decoration={expanding waves, angle=10}]
         (0, 0) -- (3, 0);
  \end{tikzpicture}
\end{minipage}

\begin{minipage}{0.7\textwidth}
  \footnotesize
  \input{decorationticks.code}
\end{minipage}\hfill
\begin{minipage}{0.29\textwidth}
  \centering
  \begin{tikzpicture}[line width=1pt]
    % benoetigt \usetikzlibrary{decorations.pathreplacing}
    \draw[decorate, decoration=ticks] (0, 0) -- (3, 0);
  \end{tikzpicture}
\end{minipage}

\begin{minipage}{0.7\textwidth}
  \footnotesize
  \input{decorationwaves.code}
\end{minipage}\hfill
\begin{minipage}{0.29\textwidth}
  \centering
  \begin{tikzpicture}[line width=1pt]
    % benoetigt \usetikzlibrary{decorations.pathreplacing}
    \draw[decorate, decoration=waves] (0, 0) -- (3, 0);
  \end{tikzpicture}
\end{minipage}

\begin{minipage}{0.7\textwidth}
  \footnotesize
  \input{decorationcrosses.code}
\end{minipage}\hfill
\begin{minipage}{0.29\textwidth}
  \centering
  \begin{tikzpicture}[line width=1pt]
    % benoetigt \usetikzlibrary{decorations.shapes}
    \draw[decorate, decoration=crosses] (0, 0) -- (3, 0);
  \end{tikzpicture}
\end{minipage}

\begin{minipage}{0.7\textwidth}
  \footnotesize
  \input{decorationshapebackgrounds.code}
\end{minipage}\hfill
\begin{minipage}{0.29\textwidth}
  \centering
  \begin{tikzpicture}[line width=1pt]
    % benoetigt \usetikzlibrary{decorations.shapes}
    \draw[decorate, decoration=shape backgrounds]
         (0, 0) -- (3, 0);
  \end{tikzpicture}
\end{minipage}

\begin{minipage}{0.7\textwidth}
  \footnotesize
  \input{decorationtriangles.code}
\end{minipage}\hfill
\begin{minipage}{0.29\textwidth}
  \centering
  \begin{tikzpicture}[line width=1pt]
    % benoetigt \usetikzlibrary{decorations.shapes}
    \draw[decorate, decoration=triangles] (0, 0) -- (3, 0);
  \end{tikzpicture}
\end{minipage}

\begin{minipage}{0.7\textwidth}
  \footnotesize
  \input{decorationtextalongpath.code}
\end{minipage}\hfill
\begin{minipage}{0.29\textwidth}
  \centering
  \begin{tikzpicture}
    \clip (-3mm, -1mm) rectangle (28mm, 13mm);
    % benoetigt \usetikzlibrary{decorations.text}
    \draw[decorate, decoration={text along path,
                                text={abcdefghijklmnopqr}}]
         (0, 0) .. controls (1, 2) and (1, 0) .. (3, 0);
  \end{tikzpicture}
\end{minipage}


% ------------------------

% --------------------
\section{Beschriftung}
% --------------------

% --------------------------------------
\subsection{Text innerhalb eines Pfades}
% --------------------------------------
\begin{minipage}{0.7\textwidth}
  \footnotesize
  \input{nodepath.code}
\end{minipage}\hfill
\begin{minipage}{0.29\textwidth}
  \centering
  \begin{tikzpicture}[line width=1pt]
    % 'node' fuegt einen Text in einen Pfad ein
    \draw (0, 0) node{A} -- node{B} (2, 0) node{C};
  \end{tikzpicture}
\end{minipage}

\begin{minipage}{0.7\textwidth}
  \footnotesize
  % parcolor version 2011-09-30
\begingroup
\ttfamily
\definecolor{R}{named}{Red}
\definecolor{G}{named}{ForestGreen}
\definecolor{B}{named}{RoyalBlue}
\definecolor{C}{named}{Cyan}
\definecolor{M}{named}{Magenta}
\definecolor{Y}{named}{YellowOrange}
\definecolor{background}{rgb}{0.82, 0.82, 0.92}
\dimen255=\textwidth
\advance\dimen255 by -2\fboxsep
\noindent
\colorbox{background}
{%
\parbox{\dimen255}
{%
\rule[-0.5ex]{0pt}{2.5ex}\hspace*{0.0em}\textbackslash{}begin\{tikzpicture\}\\
\rule[-0.5ex]{0pt}{2.5ex}\hspace*{1.0em}\textcolor{G}{\textbf{\%~der~Text~laesst~sich~orthogonal~verschieben}}\\
\rule[-0.5ex]{0pt}{2.5ex}\hspace*{1.0em}\textbackslash{}draw~\textcolor{B}{\textbf{(0,~0)}}~node[\textcolor{R}{\textbf{left}}]~\{A\}\\
\rule[-0.5ex]{0pt}{2.5ex}\hspace*{4.0em}\textcolor{B}{\textbf{{-}{-}}}~~~~~node[\textcolor{R}{\textbf{above}}]\{B\}\\
\rule[-0.5ex]{0pt}{2.5ex}\hspace*{7.5em}node[\textcolor{R}{\textbf{below}}]\{C\}\\
\rule[-0.5ex]{0pt}{2.5ex}\hspace*{4.0em}\textcolor{B}{\textbf{(2,~0)}}~node[\textcolor{R}{\textbf{right}}]\{D\};\\
\rule[-0.5ex]{0pt}{2.5ex}\hspace*{0.0em}\textbackslash{}end\{tikzpicture\}}%
}%
\endgroup

\end{minipage}\hfill
\begin{minipage}{0.29\textwidth}
  \centering
  \begin{tikzpicture}[line width=1pt]
    % der Text laesst sich orthogonal verschieben
    \draw (0, 0) node[left] {A}
          --     node[above]{B}
                 node[below]{C}
          (2, 0) node[right]{D};
  \end{tikzpicture}
\end{minipage}

\begin{minipage}{0.7\textwidth}
  \footnotesize
  % parcolor version 2011-09-30
\begingroup
\ttfamily
\definecolor{R}{named}{Red}
\definecolor{G}{named}{ForestGreen}
\definecolor{B}{named}{RoyalBlue}
\definecolor{C}{named}{Cyan}
\definecolor{M}{named}{Magenta}
\definecolor{Y}{named}{YellowOrange}
\definecolor{background}{rgb}{0.82, 0.82, 0.92}
\dimen255=\textwidth
\advance\dimen255 by -2\fboxsep
\noindent
\colorbox{background}
{%
\parbox{\dimen255}
{%
\rule[-0.5ex]{0pt}{2.5ex}\hspace*{0.0em}\textbackslash{}begin\{tikzpicture\}\\
\rule[-0.5ex]{0pt}{2.5ex}\hspace*{1.0em}\textcolor{G}{\textbf{\%~die~Kombination~verschiebt~diagonal}}\\
\rule[-0.5ex]{0pt}{2.5ex}\hspace*{1.0em}\textbackslash{}draw~\textcolor{B}{\textbf{(0,~0)}}~node[\textcolor{R}{\textbf{above~left}}]~~\{A\}\\
\rule[-0.5ex]{0pt}{2.5ex}\hspace*{7.5em}node[\textcolor{R}{\textbf{below~left}}]~~\{B\}\\
\rule[-0.5ex]{0pt}{2.5ex}\hspace*{4.0em}\textcolor{B}{\textbf{{-}{-}}}\\
\rule[-0.5ex]{0pt}{2.5ex}\hspace*{4.0em}\textcolor{B}{\textbf{(2,~0)}}~node[\textcolor{R}{\textbf{above~right}}]~\{C\}\\
\rule[-0.5ex]{0pt}{2.5ex}\hspace*{7.5em}node[\textcolor{R}{\textbf{below~right}}]~\{D\};\\
\rule[-0.5ex]{0pt}{2.5ex}\hspace*{0.0em}\textbackslash{}end\{tikzpicture\}}%
}%
\endgroup

\end{minipage}\hfill
\begin{minipage}{0.29\textwidth}
  \centering
  \begin{tikzpicture}[line width=1pt]
    % die Kombination verschiebt diagonal
    \draw (0, 0) node[above left]  {A}
                 node[below left]  {B}
          --
          (2, 0) node[above right] {C}
                 node[below right] {D};
  \end{tikzpicture}
\end{minipage}

\begin{minipage}{0.7\textwidth}
  \footnotesize
  \input{pathdistance.code}
\end{minipage}\hfill
\begin{minipage}{0.29\textwidth}
  \centering
  \begin{tikzpicture}[line width=1pt]
    % die Distanz kann auch angegeben werden
    \draw (0, 0) node[left]     {A}
          --     node[above=5mm]{B}
                 node[below=5mm]{C}
          (2, 0) node[right]    {D};
  \end{tikzpicture}
\end{minipage}

\begin{minipage}{0.7\textwidth}
  \footnotesize
  % parcolor version 2011-09-30
\begingroup
\ttfamily
\definecolor{R}{named}{Red}
\definecolor{G}{named}{ForestGreen}
\definecolor{B}{named}{RoyalBlue}
\definecolor{C}{named}{Cyan}
\definecolor{M}{named}{Magenta}
\definecolor{Y}{named}{YellowOrange}
\definecolor{background}{rgb}{0.82, 0.82, 0.92}
\dimen255=\textwidth
\advance\dimen255 by -2\fboxsep
\noindent
\colorbox{background}
{%
\parbox{\dimen255}
{%
\rule[-0.5ex]{0pt}{2.5ex}\hspace*{0.0em}\textbackslash{}begin\{tikzpicture\}\\
\rule[-0.5ex]{0pt}{2.5ex}\hspace*{1.0em}\textcolor{G}{\textbf{\%~die~Option~'shift'~mit~Polarkoordinaten}}\\
\rule[-0.5ex]{0pt}{2.5ex}\hspace*{1.0em}\textbackslash{}draw~\textcolor{B}{\textbf{(0,~0)~{-}{-}~(2,~0)}}~node[\textcolor{R}{\textbf{shift=\{(~80:5mm)\}}}]~\{A\}\\
\rule[-0.5ex]{0pt}{2.5ex}\hspace*{12.5em}node[\textcolor{R}{\textbf{shift=\{(~40:5mm)\}}}]~\{B\}\\
\rule[-0.5ex]{0pt}{2.5ex}\hspace*{12.5em}node[\textcolor{R}{\textbf{shift=\{(~~0:5mm)\}}}]~\{C\}\\
\rule[-0.5ex]{0pt}{2.5ex}\hspace*{12.5em}node[\textcolor{R}{\textbf{shift=\{(320:5mm)\}}}]~\{D\}\\
\rule[-0.5ex]{0pt}{2.5ex}\hspace*{12.5em}node[\textcolor{R}{\textbf{shift=\{(280:5mm)\}}}]~\{E\};\\
\rule[-0.5ex]{0pt}{2.5ex}\hspace*{0.0em}\textbackslash{}end\{tikzpicture\}}%
}%
\endgroup

\end{minipage}\hfill
\begin{minipage}{0.29\textwidth}
  \centering
  \begin{tikzpicture}[line width=1pt]
    % die Option 'shift' mit Polarkoordinaten
    \draw (0, 0) -- (2, 0) node[shift={( 80:5mm)}] {A}
                           node[shift={( 40:5mm)}] {B}
                           node[shift={(  0:5mm)}] {C}
                           node[shift={(320:5mm)}] {D}
                           node[shift={(280:5mm)}] {E};
  \end{tikzpicture}
\end{minipage}

\begin{minipage}{0.7\textwidth}
  \footnotesize
  % parcolor version 2011-09-30
\begingroup
\ttfamily
\definecolor{R}{named}{Red}
\definecolor{G}{named}{ForestGreen}
\definecolor{B}{named}{RoyalBlue}
\definecolor{C}{named}{Cyan}
\definecolor{M}{named}{Magenta}
\definecolor{Y}{named}{YellowOrange}
\definecolor{background}{rgb}{0.82, 0.82, 0.92}
\dimen255=\textwidth
\advance\dimen255 by -2\fboxsep
\noindent
\colorbox{background}
{%
\parbox{\dimen255}
{%
\rule[-0.5ex]{0pt}{2.5ex}\hspace*{0.0em}\textbackslash{}begin\{tikzpicture\}\\
\rule[-0.5ex]{0pt}{2.5ex}\hspace*{1.0em}\textcolor{G}{\textbf{\%~die~Option~'pos'~verschiebt~die~Position~der}}\\
\rule[-0.5ex]{0pt}{2.5ex}\hspace*{1.0em}\textcolor{G}{\textbf{\%~Beschriftung~entlang~des~gegebenen~Pfades}}\\
\rule[-0.5ex]{0pt}{2.5ex}\hspace*{1.0em}\textbackslash{}draw~\textcolor{B}{\textbf{(0,~0)~arc[start~angle=0,~end~angle=270,~radius=1]}}\\
\rule[-0.5ex]{0pt}{2.5ex}\hspace*{4.0em}node[\textcolor{R}{\textbf{pos=0.0}},~shift=\{(~~0:3mm)\}]~\{\$0\^{}\{\textbackslash{}circ\}\$\}\\
\rule[-0.5ex]{0pt}{2.5ex}\hspace*{4.0em}node[\textcolor{R}{\textbf{pos=0.5}},~shift=\{(135:3mm)\}]~\{\$135\^{}\{\textbackslash{}circ\}\$\}\\
\rule[-0.5ex]{0pt}{2.5ex}\hspace*{4.0em}node[\textcolor{R}{\textbf{pos=1.0}},~shift=\{(270:3mm)\}]~\{\$270\^{}\{\textbackslash{}circ\}\$\};\\
\rule[-0.5ex]{0pt}{2.5ex}\hspace*{0.0em}\textbackslash{}end\{tikzpicture\}}%
}%
\endgroup

\end{minipage}\hfill
\begin{minipage}{0.29\textwidth}
  \centering
  \begin{tikzpicture}[line width=1pt]
    % die Option 'pos' verschiebt die Position der
    % Beschriftung entlang des gegebenen Pfades
    \draw (0, 0) arc[start angle=0, end angle=270, radius=1]
          node[pos=0.0, shift={(  0:3mm)}] {$0^{\circ}$}
          node[pos=0.5, shift={(135:3mm)}] {$135^{\circ}$}
          node[pos=1.0, shift={(270:3mm)}] {$270^{\circ}$};
  \end{tikzpicture}
\end{minipage}

% -----------------------------------------
\subsection{Text als eigenständiges Objekt}
% -----------------------------------------
\begin{minipage}{0.7\textwidth}
  \footnotesize
  \input{nodeat.code}
\end{minipage}\hfill
\begin{minipage}{0.29\textwidth}
  \centering
  \begin{tikzpicture}[line width=1pt]
    % mit '\node' kann man Text an beliebige Stellen setzen
    \node at (0, 0) {$(0,0)$};
    \node at (2, 1) {$(2,1)$};
  \end{tikzpicture}
\end{minipage}

\begin{minipage}{0.7\textwidth}
  \footnotesize
  % parcolor version 2011-09-30
\begingroup
\ttfamily
\definecolor{R}{named}{Red}
\definecolor{G}{named}{ForestGreen}
\definecolor{B}{named}{RoyalBlue}
\definecolor{C}{named}{Cyan}
\definecolor{M}{named}{Magenta}
\definecolor{Y}{named}{YellowOrange}
\definecolor{background}{rgb}{0.82, 0.82, 0.92}
\dimen255=\textwidth
\advance\dimen255 by -2\fboxsep
\noindent
\colorbox{background}
{%
\parbox{\dimen255}
{%
\rule[-0.5ex]{0pt}{2.5ex}\hspace*{0.0em}\textbackslash{}begin\{tikzpicture\}\\
\rule[-0.5ex]{0pt}{2.5ex}\hspace*{1.0em}\textcolor{G}{\textbf{\%~mit~'shape'~kann~man~'Knoten'~grafisch~gestalten}}\\
\rule[-0.5ex]{0pt}{2.5ex}\hspace*{1.0em}\textbackslash{}node[\textcolor{R}{\textbf{shape=rectangle}},~\textcolor{R}{\textbf{fill=Green}}]~at~(0,~0)~\{\$(0,0)\$\};\\
\rule[-0.5ex]{0pt}{2.5ex}\hspace*{1.0em}\textbackslash{}node[\textcolor{R}{\textbf{shape=circle}},~~~~\textcolor{R}{\textbf{draw=Black}}]~at~(2,~1)~\{\$(2,1)\$\};\\
\rule[-0.5ex]{0pt}{2.5ex}\hspace*{0.0em}\textbackslash{}end\{tikzpicture\}}%
}%
\endgroup

\end{minipage}\hfill
\begin{minipage}{0.29\textwidth}
  \centering
  \begin{tikzpicture}[line width=1pt]
    % mit 'shape' kann man 'Knoten' grafisch gestalten
    \node[shape=rectangle, fill=Green] at (0, 0) {$(0,0)$};
    \node[shape=circle,    draw=Black] at (2, 1) {$(2,1)$};
  \end{tikzpicture}
\end{minipage}

\begin{minipage}{0.7\textwidth}
  \footnotesize
  % parcolor version 2011-09-30
\begingroup
\ttfamily
\definecolor{R}{named}{Red}
\definecolor{G}{named}{ForestGreen}
\definecolor{B}{named}{RoyalBlue}
\definecolor{C}{named}{Cyan}
\definecolor{M}{named}{Magenta}
\definecolor{Y}{named}{YellowOrange}
\definecolor{background}{rgb}{0.82, 0.82, 0.92}
\dimen255=\textwidth
\advance\dimen255 by -2\fboxsep
\noindent
\colorbox{background}
{%
\parbox{\dimen255}
{%
\rule[-0.5ex]{0pt}{2.5ex}\hspace*{0.0em}\textbackslash{}begin\{tikzpicture\}\\
\rule[-0.5ex]{0pt}{2.5ex}\hspace*{1.0em}\textcolor{G}{\textbf{\%~ellipse~benoetigt~\textbackslash{}usetikzlibrary\{shapes\}}}\\
\rule[-0.5ex]{0pt}{2.5ex}\hspace*{1.0em}\textbackslash{}node[\textcolor{R}{\textbf{shape=ellipse}},~fill=Cerulean,~draw=Black]\\
\rule[-0.5ex]{0pt}{2.5ex}\hspace*{3.5em}at~(0,~0)~\{\textbackslash{}LaTeX\};\\
\rule[-0.5ex]{0pt}{2.5ex}\hspace*{0.0em}\textbackslash{}end\{tikzpicture\}}%
}%
\endgroup

\end{minipage}\hfill
\begin{minipage}{0.29\textwidth}
  \centering
  \begin{tikzpicture}[line width=1pt]
    % ellipse benoetigt \usetikzlibrary{shapes}
    \node[shape=ellipse, fill=Cerulean, draw=Black]
         at (0, 0) {\LaTeX};
  \end{tikzpicture}
\end{minipage}

\begin{minipage}{0.7\textwidth}
  \footnotesize
  \input{nodetrapezium.code}
\end{minipage}\hfill
\begin{minipage}{0.29\textwidth}
  \centering
  \begin{tikzpicture}[line width=1pt]
    % trapezium benoetigt \usetikzlibrary{shapes}
    \node[shape=trapezium, fill=Cerulean, draw=Black]
         at (0, 0) {\LaTeX};
  \end{tikzpicture}
\end{minipage}

\begin{minipage}{0.7\textwidth}
  \footnotesize
  \input{nodediamond.code}
\end{minipage}\hfill
\begin{minipage}{0.29\textwidth}
  \centering
  \begin{tikzpicture}[line width=1pt]
    % diamond benoetigt \usetikzlibrary{shapes}
    \node[shape=diamond,  fill=Cerulean,
          shape aspect=2, draw=Black]
         at (0, 0) {\LaTeX};
  \end{tikzpicture}
\end{minipage}

\begin{minipage}{0.7\textwidth}
  \footnotesize
  \input{nodesemicircle.code}
\end{minipage}\hfill
\begin{minipage}{0.29\textwidth}
  \centering
  \begin{tikzpicture}[line width=1pt]
    % semicircle benoetigt \usetikzlibrary{shapes}
    \node[shape=semicircle, fill=Cerulean, draw=Black]
         at (0, 0) {\LaTeX};
  \end{tikzpicture}
\end{minipage}

\begin{minipage}{0.7\textwidth}
  \footnotesize
  \input{nodeisoscelestriangle.code}
\end{minipage}\hfill
\begin{minipage}{0.29\textwidth}
  \centering
  \begin{tikzpicture}[line width=1pt]
    % isosceles triangle benoetigt \usetikzlibrary{shapes}
    \node[shape=isosceles triangle, fill=Cerulean, draw=Black]
         at (0, 0) {\LaTeX};
  \end{tikzpicture}
\end{minipage}

\begin{minipage}{0.7\textwidth}
  \footnotesize
  \input{nodekite.code}
\end{minipage}\hfill
\begin{minipage}{0.29\textwidth}
  \centering
  \begin{tikzpicture}[line width=1pt]
    % kite benoetigt \usetikzlibrary{shapes}
    % \rotatebox benoetigt \usepackage{graphicx}
    \node[shape=kite, fill=Cerulean, draw=Black]
         at (0, 0) {\rotatebox{90}{\LaTeX}};
  \end{tikzpicture}
\end{minipage}

\begin{minipage}{0.7\textwidth}
  \footnotesize
  \input{nodedart.code}
\end{minipage}\hfill
\begin{minipage}{0.29\textwidth}
  \centering
  \begin{tikzpicture}[line width=1pt]
    % dart benoetigt \usetikzlibrary{shapes}
    \node[shape=dart, fill=Cerulean, draw=Black]
         at (0, 0) {\LaTeX};
  \end{tikzpicture}
\end{minipage}

\begin{minipage}{0.7\textwidth}
  \footnotesize
  \input{nodecircularsector.code}
\end{minipage}\hfill
\begin{minipage}{0.29\textwidth}
  \centering
  \begin{tikzpicture}[line width=1pt]
    % circular sector benoetigt \usetikzlibrary{shapes}
    \node[shape=circular sector, fill=Cerulean, draw=Black]
         at (0, 0) {\LaTeX};
  \end{tikzpicture}
\end{minipage}

\begin{minipage}{0.7\textwidth}
  \footnotesize
  \input{nodecylinder.code}
\end{minipage}\hfill
\begin{minipage}{0.29\textwidth}
  \centering
  \begin{tikzpicture}[line width=1pt]
    % cylinder benoetigt \usetikzlibrary{shapes}
    \node[shape=cylinder, fill=Cerulean, draw=Black]
         at (0, 0) {\LaTeX};
  \end{tikzpicture}
\end{minipage}

\begin{minipage}{0.7\textwidth}
  \footnotesize
  \input{nodecoords.code}
\end{minipage}\hfill
\begin{minipage}{0.29\textwidth}
  \centering
  \begin{tikzpicture}[line width=1pt]
    % Knoten koennen benannt werden
    \node[shape=ellipse, draw=Black] (A) at (0, 0) {$(0,0)$};
    \node[shape=ellipse, draw=Black] (B) at (2, 1) {$(2,1)$};
    % aus den Namen lassen sich Koordinaten bilden
    \draw (A.0) to[out=0, in=180] (B.180);
  \end{tikzpicture}
\end{minipage}

\begin{minipage}{0.7\textwidth}
  \footnotesize
  \input{nodeinnersep.code}
\end{minipage}\hfill
\begin{minipage}{0.29\textwidth}
  \centering
  \begin{tikzpicture}
    \tikzstyle{every node}=[shape=rectangle, fill=LimeGreen];
    % die Abstaende zwischen Text und Rechteck
    \node[inner sep=3mm] at (0, 0) {Donald E. Knuth};
  \end{tikzpicture}
\end{minipage}

\begin{minipage}{0.7\textwidth}
  \footnotesize
  \input{nodeinnerxsep.code}
\end{minipage}\hfill
\begin{minipage}{0.29\textwidth}
  \centering
  \begin{tikzpicture}
    \tikzstyle{every node}=[shape=rectangle, fill=LimeGreen];
    % die horizontalen Abstaende zwischen Text und Rechteck
    \node[inner xsep=3mm] at (0, 0) {Donald E. Knuth};
  \end{tikzpicture}
\end{minipage}

\begin{minipage}{0.7\textwidth}
  \footnotesize
  \input{nodeinnerysep.code}
\end{minipage}\hfill
\begin{minipage}{0.29\textwidth}
  \centering
  \begin{tikzpicture}
    \tikzstyle{every node}=[shape=rectangle, fill=LimeGreen];
    % die vertikalen Abstaende zwischen Text und Rechteck
    \node[inner ysep=3mm] at (0, 0) {Donald E. Knuth};
  \end{tikzpicture}
\end{minipage}

\begin{minipage}{0.7\textwidth}
  \footnotesize
  \input{nodeoutersep.code}
\end{minipage}\hfill
\begin{minipage}{0.29\textwidth}
  \centering
  \begin{tikzpicture}
    \tikzstyle{every node}=[shape=rectangle, fill=LimeGreen];
    % aeussere Abstaende beeinflussen die 'anchor'-Positionen
    \node[outer sep=3mm] (A) at (0, 0) {Donald E. Knuth};
    % Referenzpunkt markieren
    \fill (A.north west) circle[radius=1pt];
  \end{tikzpicture}
\end{minipage}

\begin{minipage}{0.7\textwidth}
  \footnotesize
  % parcolor version 2011-09-30
\begingroup
\ttfamily
\definecolor{R}{named}{Red}
\definecolor{G}{named}{ForestGreen}
\definecolor{B}{named}{RoyalBlue}
\definecolor{C}{named}{Cyan}
\definecolor{M}{named}{Magenta}
\definecolor{Y}{named}{YellowOrange}
\definecolor{background}{rgb}{0.82, 0.82, 0.92}
\dimen255=\textwidth
\advance\dimen255 by -2\fboxsep
\noindent
\colorbox{background}
{%
\parbox{\dimen255}
{%
\rule[-0.5ex]{0pt}{2.5ex}\hspace*{0.0em}\textcolor{G}{\textbf{\%~aeussere~Abstaende~beeinflussen~die~'anchor'{-}Positionen}}\\
\rule[-0.5ex]{0pt}{2.5ex}\hspace*{0.0em}\textbackslash{}node[\textcolor{R}{\textbf{outer~xsep=3mm}}]~(A)~at~(0,~0)~\{Donald~E.~Knuth\};\\
\rule[-0.5ex]{0pt}{2.5ex}\hspace*{0.0em}\textcolor{G}{\textbf{\%~Referenzpunkt~markieren}}\\
\rule[-0.5ex]{0pt}{2.5ex}\hspace*{0.0em}\textbackslash{}fill~(\textcolor{R}{\textbf{A.north~west}})~circle[radius=1pt];}%
}%
\endgroup

\end{minipage}\hfill
\begin{minipage}{0.29\textwidth}
  \centering
  \begin{tikzpicture}
    \tikzstyle{every node}=[shape=rectangle, fill=LimeGreen];
    % aeussere Abstaende beeinflussen die 'anchor'-Positionen
    \node[outer xsep=3mm] (A) at (0, 0) {Donald E. Knuth};
    % Referenzpunkt markieren
    \fill (A.north west) circle[radius=1pt];
  \end{tikzpicture}
\end{minipage}

\begin{minipage}{0.7\textwidth}
  \footnotesize
  \input{nodeouterysep.code}
\end{minipage}\hfill
\begin{minipage}{0.29\textwidth}
  \centering
  \begin{tikzpicture}
    \tikzstyle{every node}=[shape=rectangle, fill=LimeGreen];
    % aeussere Abstaende beeinflussen die 'anchor'-Positionen
    \node[outer ysep=3mm] (A) at (0, 0) {Donald E. Knuth};
    % Referenzpunkt markieren
    \fill (A.north west) circle[radius=1pt];
  \end{tikzpicture}
\end{minipage}

\begin{minipage}{0.7\textwidth}
  \footnotesize
  \input{nodeminimumwidth.code}
\end{minipage}\hfill
\begin{minipage}{0.29\textwidth}
  \centering
  \begin{tikzpicture}
    \tikzstyle{every node}=[shape=rectangle, fill=LimeGreen];
    % Mindestbreite der 'shape'
    \node[minimum width=4cm] at (0, 0) {Donald E. Knuth};
  \end{tikzpicture}
\end{minipage}

\begin{minipage}{0.7\textwidth}
  \footnotesize
  \input{nodeminimumheight.code}
\end{minipage}\hfill
\begin{minipage}{0.29\textwidth}
  \centering
  \begin{tikzpicture}
    \tikzstyle{every node}=[shape=rectangle, fill=LimeGreen];
    % Mindesthoehe der 'shape'
    \node[minimum height=1cm] at (0, 0) {Donald E. Knuth};
  \end{tikzpicture}
\end{minipage}

\begin{minipage}{0.7\textwidth}
  \footnotesize
  \input{nodealignleft.code}
\end{minipage}\hfill
\begin{minipage}{0.29\textwidth}
  \centering
  \begin{tikzpicture}
    \tikzstyle{every node}=[shape=rectangle, fill=LimeGreen];
    % Ausrichtung von mehrzeiligem Text
    \node[align=left] at (0, 0) {Donald E.\\Knuth};
  \end{tikzpicture}
\end{minipage}

\begin{minipage}{0.7\textwidth}
  \footnotesize
  % parcolor version 2011-09-30
\begingroup
\ttfamily
\definecolor{R}{named}{Red}
\definecolor{G}{named}{ForestGreen}
\definecolor{B}{named}{RoyalBlue}
\definecolor{C}{named}{Cyan}
\definecolor{M}{named}{Magenta}
\definecolor{Y}{named}{YellowOrange}
\definecolor{background}{rgb}{0.82, 0.82, 0.92}
\dimen255=\textwidth
\advance\dimen255 by -2\fboxsep
\noindent
\colorbox{background}
{%
\parbox{\dimen255}
{%
\rule[-0.5ex]{0pt}{2.5ex}\hspace*{0.0em}\textcolor{G}{\textbf{\%~Ausrichtung~von~mehrzeiligem~Text}}\\
\rule[-0.5ex]{0pt}{2.5ex}\hspace*{0.0em}\textbackslash{}node[\textcolor{R}{\textbf{align=center}}]~at~(0,~0)~\{Donald~E.\textcolor{R}{\textbf{\textbackslash{}\textbackslash{}}}Knuth\};}%
}%
\endgroup

\end{minipage}\hfill
\begin{minipage}{0.29\textwidth}
  \centering
  \begin{tikzpicture}
    \tikzstyle{every node}=[shape=rectangle, fill=LimeGreen];
    % Ausrichtung von mehrzeiligem Text
    \node[align=center] at (0, 0) {Donald E.\\Knuth};
  \end{tikzpicture}
\end{minipage}

\begin{minipage}{0.7\textwidth}
  \footnotesize
  \input{nodealignright.code}
\end{minipage}\hfill
\begin{minipage}{0.29\textwidth}
  \centering
  \begin{tikzpicture}
    \tikzstyle{every node}=[shape=rectangle, fill=LimeGreen];
    % Ausrichtung von mehrzeiligem Text
    \node[align=right] at (0, 0) {Donald E.\\Knuth};
  \end{tikzpicture}
\end{minipage}

\begin{minipage}{0.7\textwidth}
  \footnotesize
  \input{nodetextwidth.code}
\end{minipage}\hfill
\begin{minipage}{0.29\textwidth}
  \centering
  \begin{tikzpicture}
    \tikzstyle{every node}=[shape=rectangle, fill=LimeGreen];
    % Breite des Textbereichs
    \node[text width=25mm] at (0, 0) {Donald E. Knuth};
  \end{tikzpicture}
\end{minipage}

\begin{minipage}{0.7\textwidth}
  \footnotesize
  \input{nodetextheight.code}
\end{minipage}\hfill
\begin{minipage}{0.29\textwidth}
  \centering
  \begin{tikzpicture}
    \tikzstyle{every node}=[shape=rectangle, fill=LimeGreen];
    % Hoehe der Textzeile
    \node[text height=3ex] at (0, 0) {Donald E. Knuth};
  \end{tikzpicture}
\end{minipage}

\begin{minipage}{0.7\textwidth}
  \footnotesize
  \input{nodetextdepth.code}
\end{minipage}\hfill
\begin{minipage}{0.29\textwidth}
  \centering
  \begin{tikzpicture}
    \tikzstyle{every node}=[shape=rectangle, fill=LimeGreen];
    % Tiefe der Textzeile
    \node[text depth=3ex] at (0, 0) {Donald E. Knuth};
  \end{tikzpicture}
\end{minipage}

% ------------------
\section{Füllmuster}
% ------------------
\begin{minipage}{0.7\textwidth}
  \footnotesize
  \input{patternhorizontallines.code}
\end{minipage}\hfill
\begin{minipage}{0.29\textwidth}
  \centering
  \begin{tikzpicture}[line width=1pt]
    \draw[pattern=horizontal lines,
          pattern color=Black!50!White]
          (0, 0) rectangle (3, 1);
  \end{tikzpicture}
\end{minipage}

\begin{minipage}{0.7\textwidth}
  \footnotesize
  \input{patternverticallines.code}
\end{minipage}\hfill
\begin{minipage}{0.29\textwidth}
  \centering
  \begin{tikzpicture}[line width=1pt]
    \draw[pattern=vertical lines,
          pattern color=Black!50!White]
          (0, 0) rectangle (3, 1);
  \end{tikzpicture}
\end{minipage}

\begin{minipage}{0.7\textwidth}
  \footnotesize
  \input{patternnortheastlines.code}
\end{minipage}\hfill
\begin{minipage}{0.29\textwidth}
  \centering
  \begin{tikzpicture}[line width=1pt]
    \draw[pattern=north east lines,
          pattern color=Black!50!White]
          (0, 0) rectangle (3, 1);
  \end{tikzpicture}
\end{minipage}

\begin{minipage}{0.7\textwidth}
  \footnotesize
  \input{patternnorthwestlines.code}
\end{minipage}\hfill
\begin{minipage}{0.29\textwidth}
  \centering
  \begin{tikzpicture}[line width=1pt]
    \draw[pattern=north west lines,
          pattern color=Black!50!White]
          (0, 0) rectangle (3, 1);
  \end{tikzpicture}
\end{minipage}

\begin{minipage}{0.7\textwidth}
  \footnotesize
  \input{patterngrid.code}
\end{minipage}\hfill
\begin{minipage}{0.29\textwidth}
  \centering
  \begin{tikzpicture}[line width=1pt]
    \draw[pattern=grid,
          pattern color=Black!50!White]
          (0, 0) rectangle (3, 1);
  \end{tikzpicture}
\end{minipage}

\begin{minipage}{0.7\textwidth}
  \footnotesize
  \input{patterncrosshatch.code}
\end{minipage}\hfill
\begin{minipage}{0.29\textwidth}
  \centering
  \begin{tikzpicture}[line width=1pt]
    \draw[pattern=crosshatch,
          pattern color=Black!50!White]
          (0, 0) rectangle (3, 1);
  \end{tikzpicture}
\end{minipage}

\begin{minipage}{0.7\textwidth}
  \footnotesize
  \input{patterndots.code}
\end{minipage}\hfill
\begin{minipage}{0.29\textwidth}
  \centering
  \begin{tikzpicture}[line width=1pt]
    \draw[pattern=dots,
          pattern color=Black!50!White]
          (0, 0) rectangle (3, 1);
  \end{tikzpicture}
\end{minipage}

\begin{minipage}{0.7\textwidth}
  \footnotesize
  % parcolor version 2011-09-30
\begingroup
\ttfamily
\definecolor{R}{named}{Red}
\definecolor{G}{named}{ForestGreen}
\definecolor{B}{named}{RoyalBlue}
\definecolor{C}{named}{Cyan}
\definecolor{M}{named}{Magenta}
\definecolor{Y}{named}{YellowOrange}
\definecolor{background}{rgb}{0.82, 0.82, 0.92}
\dimen255=\textwidth
\advance\dimen255 by -2\fboxsep
\noindent
\colorbox{background}
{%
\parbox{\dimen255}
{%
\rule[-0.5ex]{0pt}{2.5ex}\hspace*{0.0em}\textbackslash{}draw[\textcolor{R}{\textbf{pattern}}=\textcolor{B}{\textbf{crosshatch~dots}},\\
\rule[-0.5ex]{0pt}{2.5ex}\hspace*{3.0em}\textcolor{R}{\textbf{pattern~color}}=\textcolor{B}{\textbf{Black!50!White}}]\\
\rule[-0.5ex]{0pt}{2.5ex}\hspace*{3.0em}(0,~0)~rectangle~(3,~1);}%
}%
\endgroup

\end{minipage}\hfill
\begin{minipage}{0.29\textwidth}
  \centering
  \begin{tikzpicture}[line width=1pt]
    \draw[pattern=crosshatch dots,
          pattern color=Black!50!White]
          (0, 0) rectangle (3, 1);
  \end{tikzpicture}
\end{minipage}

\begin{minipage}{0.7\textwidth}
  \footnotesize
  \input{patternbricks.code}
\end{minipage}\hfill
\begin{minipage}{0.29\textwidth}
  \centering
  \begin{tikzpicture}[line width=1pt]
    \draw[pattern=bricks,
          pattern color=Black!50!White]
          (0, 0) rectangle (3, 1);
  \end{tikzpicture}
\end{minipage}

\begin{minipage}{0.7\textwidth}
  \footnotesize
  \input{patternfivepointedstars.code}
\end{minipage}\hfill
\begin{minipage}{0.29\textwidth}
  \centering
  \begin{tikzpicture}[line width=1pt]
    \draw[pattern=fivepointed stars,
          pattern color=Black!50!White]
          (0, 0) rectangle (3, 1);
  \end{tikzpicture}
\end{minipage}

\begin{minipage}{0.7\textwidth}
  \footnotesize
  \input{patternsixpointedstars.code}
\end{minipage}\hfill
\begin{minipage}{0.29\textwidth}
  \centering
  \begin{tikzpicture}[line width=1pt]
    \draw[pattern=sixpointed stars,
          pattern color=Black!50!White]
          (0, 0) rectangle (3, 1);
  \end{tikzpicture}
\end{minipage}

\begin{minipage}{0.7\textwidth}
  \footnotesize
  \input{patterncheckerboard.code}
\end{minipage}\hfill
\begin{minipage}{0.29\textwidth}
  \centering
  \begin{tikzpicture}[line width=1pt]
    \draw[pattern=checkerboard,
          pattern color=Black!50!White]
          (0, 0) rectangle (3, 1);
  \end{tikzpicture}
\end{minipage}

% ----------------
\section{Clipping}
% ----------------

% ----------------------------------------
\subsection{Clipping mit einfachen Pfaden}
% ----------------------------------------
\begin{minipage}{0.7\textwidth}
  \footnotesize
  % parcolor version 2011-09-30
\begingroup
\ttfamily
\definecolor{R}{named}{Red}
\definecolor{G}{named}{ForestGreen}
\definecolor{B}{named}{RoyalBlue}
\definecolor{C}{named}{Cyan}
\definecolor{M}{named}{Magenta}
\definecolor{Y}{named}{YellowOrange}
\definecolor{background}{rgb}{0.82, 0.82, 0.92}
\dimen255=\textwidth
\advance\dimen255 by -2\fboxsep
\noindent
\colorbox{background}
{%
\parbox{\dimen255}
{%
\rule[-0.5ex]{0pt}{2.5ex}\hspace*{0.0em}\textbackslash{}begin\{tikzpicture\}\\
\rule[-0.5ex]{0pt}{2.5ex}\hspace*{1.0em}\textcolor{R}{\textbf{\textbackslash{}begin}}\{\textcolor{R}{\textbf{scope}}\}\\
\rule[-0.5ex]{0pt}{2.5ex}\hspace*{2.0em}\textcolor{G}{\textbf{\%~Rechteck~ueber~der~rechten~Haelfte~des~Kreises}}\\
\rule[-0.5ex]{0pt}{2.5ex}\hspace*{2.0em}\textcolor{R}{\textbf{\textbackslash{}clip}}~(0,~{-}1)~\textcolor{R}{\textbf{rectangle}}~(1,~1);\\
\rule[-0.5ex]{0pt}{2.5ex}\hspace*{2.0em}\textcolor{G}{\textbf{\%~den~Teil~des~Kreises~ausfuellen,~der~im}}\\
\rule[-0.5ex]{0pt}{2.5ex}\hspace*{2.0em}\textcolor{G}{\textbf{\%~Clipping{-}Bereich~liegt}}\\
\rule[-0.5ex]{0pt}{2.5ex}\hspace*{2.0em}\textbackslash{}fill[fill=LimeGreen]~(0,~0)~circle[radius=1];\\
\rule[-0.5ex]{0pt}{2.5ex}\hspace*{1.0em}\textcolor{R}{\textbf{\textbackslash{}end}}\{\textcolor{R}{\textbf{scope}}\}\\
\rule[-0.5ex]{0pt}{2.5ex}\hspace*{1.0em}\textcolor{G}{\textbf{\%~den~Rand~des~Kreises~zeichnen}}\\
\rule[-0.5ex]{0pt}{2.5ex}\hspace*{1.0em}\textbackslash{}draw~(0,~0)~circle[radius=1];\\
\rule[-0.5ex]{0pt}{2.5ex}\hspace*{0.0em}\textbackslash{}end\{tikzpicture\}}%
}%
\endgroup

\end{minipage}\hfill
\begin{minipage}{0.29\textwidth}
  \centering
  \begin{tikzpicture}[line width=1pt]
    \begin{scope}
      % Rechteck ueber der rechten Haelfte des Kreises
      \clip (0, -1) rectangle (1, 1);
      % den Teil des Kreises ausfuellen, der im
      % Clipping-Bereich liegt 
      \fill[fill=LimeGreen] (0, 0) circle[radius=1];
    \end{scope}
    % den Rand des Kreises zeichnen
    \draw (0, 0) circle[radius=1];
  \end{tikzpicture}
\end{minipage}

\begin{minipage}{0.7\textwidth}
  \footnotesize
  % parcolor version 2011-09-30
\begingroup
\ttfamily
\definecolor{R}{named}{Red}
\definecolor{G}{named}{ForestGreen}
\definecolor{B}{named}{RoyalBlue}
\definecolor{C}{named}{Cyan}
\definecolor{M}{named}{Magenta}
\definecolor{Y}{named}{YellowOrange}
\definecolor{background}{rgb}{0.82, 0.82, 0.92}
\dimen255=\textwidth
\advance\dimen255 by -2\fboxsep
\noindent
\colorbox{background}
{%
\parbox{\dimen255}
{%
\rule[-0.5ex]{0pt}{2.5ex}\hspace*{0.0em}\textbackslash{}begin\{tikzpicture\}\\
\rule[-0.5ex]{0pt}{2.5ex}\hspace*{1.0em}\textcolor{R}{\textbf{\textbackslash{}begin}}\{\textcolor{R}{\textbf{scope}}\}\\
\rule[-0.5ex]{0pt}{2.5ex}\hspace*{2.0em}\textcolor{G}{\textbf{\%~Durchschnitt~zweier~Clipping{-}Bereiche}}\\
\rule[-0.5ex]{0pt}{2.5ex}\hspace*{2.0em}\textcolor{R}{\textbf{\textbackslash{}clip}}~({-}5mm,~0)~\textcolor{R}{\textbf{circle}}[radius=1cm];\\
\rule[-0.5ex]{0pt}{2.5ex}\hspace*{2.0em}\textcolor{R}{\textbf{\textbackslash{}clip}}~(~5mm,~0)~\textcolor{R}{\textbf{circle}}[radius=1cm];\\
\rule[-0.5ex]{0pt}{2.5ex}\hspace*{2.0em}\textcolor{G}{\textbf{\%~Flaeche~im~Clipping{-}Bereich~ausfuellen}}\\
\rule[-0.5ex]{0pt}{2.5ex}\hspace*{2.0em}\textbackslash{}fill[fill=LimeGreen]~(0,~0)~circle[radius=1cm];\\
\rule[-0.5ex]{0pt}{2.5ex}\hspace*{1.0em}\textcolor{R}{\textbf{\textbackslash{}end}}\{\textcolor{R}{\textbf{scope}}\}\\
\rule[-0.5ex]{0pt}{2.5ex}\hspace*{1.0em}\textcolor{G}{\textbf{\%~Kreisraender~zeichnen}}\\
\rule[-0.5ex]{0pt}{2.5ex}\hspace*{1.0em}\textbackslash{}draw~({-}5mm,~0)~circle[radius=1cm];\\
\rule[-0.5ex]{0pt}{2.5ex}\hspace*{1.0em}\textbackslash{}draw~(~5mm,~0)~circle[radius=1cm];\\
\rule[-0.5ex]{0pt}{2.5ex}\hspace*{0.0em}\textbackslash{}end\{tikzpicture\}}%
}%
\endgroup

\end{minipage}\hfill
\begin{minipage}{0.29\textwidth}
  \centering
  \begin{tikzpicture}[line width=1pt]
    \begin{scope}
      % Durchschnitt zweier Clipping-Bereiche
      \clip (-5mm, 0) circle[radius=1cm];
      \clip ( 5mm, 0) circle[radius=1cm];
      % Flaeche ausfuellen
      \fill[fill=LimeGreen] (0, 0) circle[radius=1cm];
    \end{scope}
    % Kreisraender zeichnen
    \draw (-5mm, 0) circle[radius=1cm];
    \draw ( 5mm, 0) circle[radius=1cm];
  \end{tikzpicture}
\end{minipage}

% ----------------------------------------------
\subsection{Clipping mit \texttt{even odd rule}}
% ----------------------------------------------
\begin{minipage}{0.7\textwidth}
  \footnotesize
  % parcolor version 2011-09-30
\begingroup
\ttfamily
\definecolor{R}{named}{Red}
\definecolor{G}{named}{ForestGreen}
\definecolor{B}{named}{RoyalBlue}
\definecolor{C}{named}{Cyan}
\definecolor{M}{named}{Magenta}
\definecolor{Y}{named}{YellowOrange}
\definecolor{background}{rgb}{0.82, 0.82, 0.92}
\dimen255=\textwidth
\advance\dimen255 by -2\fboxsep
\noindent
\colorbox{background}
{%
\parbox{\dimen255}
{%
\rule[-0.5ex]{0pt}{2.5ex}\hspace*{0.0em}\textbackslash{}begin\{tikzpicture\}\\
\rule[-0.5ex]{0pt}{2.5ex}\hspace*{1.0em}\textcolor{G}{\textbf{\%~die~Optionen~'even~odd~rule'~kann~dem~clip{-}Befehl}}\\
\rule[-0.5ex]{0pt}{2.5ex}\hspace*{1.0em}\textcolor{G}{\textbf{\%~nicht~direkt~uebergeben~werden}}\\
\rule[-0.5ex]{0pt}{2.5ex}\hspace*{1.0em}\textcolor{R}{\textbf{\textbackslash{}begin}}\{\textcolor{R}{\textbf{scope}}\}[\textcolor{R}{\textbf{even~odd~rule}}]\\
\rule[-0.5ex]{0pt}{2.5ex}\hspace*{2.0em}\textcolor{G}{\textbf{\%~linker~Kreisring~als~Clipping{-}Bereich}}\\
\rule[-0.5ex]{0pt}{2.5ex}\hspace*{2.0em}\textcolor{R}{\textbf{\textbackslash{}clip}}~({-}5mm,~0)~circle[radius=8mm]\\
\rule[-0.5ex]{0pt}{2.5ex}\hspace*{5.0em}({-}5mm,~0)~circle[radius=12mm];\\
\rule[-0.5ex]{0pt}{2.5ex}\hspace*{2.0em}\textcolor{G}{\textbf{\%~rechten~Kreisring~ausfuellen}}\\
\rule[-0.5ex]{0pt}{2.5ex}\hspace*{2.0em}\textbackslash{}fill[fill=LimeGreen]\\
\rule[-0.5ex]{0pt}{2.5ex}\hspace*{4.5em}(5mm,~0)~circle[radius=8mm]\\
\rule[-0.5ex]{0pt}{2.5ex}\hspace*{4.5em}(5mm,~0)~circle[radius=12mm];\\
\rule[-0.5ex]{0pt}{2.5ex}\hspace*{1.0em}\textcolor{R}{\textbf{\textbackslash{}end}}\{\textcolor{R}{\textbf{scope}}\}\\
\rule[-0.5ex]{0pt}{2.5ex}\hspace*{1.0em}\textcolor{G}{\textbf{\%~Raender~der~Kreisringe~zeichnen}}\\
\rule[-0.5ex]{0pt}{2.5ex}\hspace*{1.0em}\textbackslash{}draw~({-}5mm,~0)~circle[radius=8mm]\\
\rule[-0.5ex]{0pt}{2.5ex}\hspace*{4.0em}({-}5mm,~0)~circle[radius=12mm];\\
\rule[-0.5ex]{0pt}{2.5ex}\hspace*{1.0em}\textbackslash{}draw~(~5mm,~0)~circle[radius=8mm]\\
\rule[-0.5ex]{0pt}{2.5ex}\hspace*{4.0em}(~5mm,~0)~circle[radius=12mm];\\
\rule[-0.5ex]{0pt}{2.5ex}\hspace*{0.0em}\textbackslash{}end\{tikzpicture\}}%
}%
\endgroup

\end{minipage}\hfill
\begin{minipage}{0.29\textwidth}
  \centering
  \begin{tikzpicture}[line width=1pt]
    % die Optionen 'even odd rule' kann dem clip-Befehl
    % nicht direkt uebergeben werden
    \begin{scope}[even odd rule]
      % linker Kreisring als Clipping-Bereich
      \clip (-5mm, 0) circle[radius=8mm]
            (-5mm, 0) circle[radius=12mm];
      % rechten Kreisring ausfuellen
      \fill[fill=LimeGreen]
           (5mm, 0) circle[radius=8mm]
           (5mm, 0) circle[radius=12mm];
    \end{scope}
    % Raender der Kreisringe zeichnen
    \draw (-5mm, 0) circle[radius=8mm]
          (-5mm, 0) circle[radius=12mm];
    \draw ( 5mm, 0) circle[radius=8mm]
          ( 5mm, 0) circle[radius=12mm];
  \end{tikzpicture}
\end{minipage}

% --------------------
\section{Rastergrafik}
% --------------------
\begin{minipage}{0.7\textwidth}
  \footnotesize
  % parcolor version 2011-09-30
\begingroup
\ttfamily
\definecolor{R}{named}{Red}
\definecolor{G}{named}{ForestGreen}
\definecolor{B}{named}{RoyalBlue}
\definecolor{C}{named}{Cyan}
\definecolor{M}{named}{Magenta}
\definecolor{Y}{named}{YellowOrange}
\definecolor{background}{rgb}{0.82, 0.82, 0.92}
\dimen255=\textwidth
\advance\dimen255 by -2\fboxsep
\noindent
\colorbox{background}
{%
\parbox{\dimen255}
{%
\rule[-0.5ex]{0pt}{2.5ex}\hspace*{0.0em}\textbackslash{}begin\{tikzpicture\}\\
\rule[-0.5ex]{0pt}{2.5ex}\hspace*{1.0em}\textcolor{G}{\textbf{\%~die~Datei~'rastergrafik.png'~laden~und~unter~dem~Namen}}\\
\rule[-0.5ex]{0pt}{2.5ex}\hspace*{1.0em}\textcolor{G}{\textbf{\%~'einbild'~fuer~spaetere~Verwendung~verfuegbar~machen}}\\
\rule[-0.5ex]{0pt}{2.5ex}\hspace*{1.0em}\textcolor{R}{\textbf{\textbackslash{}pgfdeclareimage}}[interpolate=true,~width=4cm]\\
\rule[-0.5ex]{0pt}{2.5ex}\hspace*{9.0em}\{\textcolor{B}{\textbf{einbild}}\}\{rastergrafik.png\};\\
\rule[-0.5ex]{0pt}{2.5ex}\hspace*{1.0em}\textcolor{G}{\textbf{\%~Bild~mit~der~unteren~linken~Ecke~an~der~Position}}\\
\rule[-0.5ex]{0pt}{2.5ex}\hspace*{1.0em}\textcolor{G}{\textbf{\%~(0,~0)~einfuegen}}\\
\rule[-0.5ex]{0pt}{2.5ex}\hspace*{1.0em}\textcolor{R}{\textbf{\textbackslash{}pgftext}}[bottom,~left,~at=\textbackslash{}pgfpoint\{0cm\}\{0cm\}]\\
\rule[-0.5ex]{0pt}{2.5ex}\hspace*{5.0em}\{\textcolor{R}{\textbf{\textbackslash{}pgfuseimage}}\{\textcolor{B}{\textbf{einbild}}\}\};\\
\rule[-0.5ex]{0pt}{2.5ex}\hspace*{1.0em}\textcolor{G}{\textbf{\%~Text~einfuegen}}\\
\rule[-0.5ex]{0pt}{2.5ex}\hspace*{1.0em}\textbackslash{}node[above~right,~text=White,~font=\textbackslash{}sffamily]\\
\rule[-0.5ex]{0pt}{2.5ex}\hspace*{3.5em}(A)~at~(0,~0)~\{Auge\};\\
\rule[-0.5ex]{0pt}{2.5ex}\hspace*{1.0em}\textcolor{G}{\textbf{\%~weissen~Pfeil~zeichnen}}\\
\rule[-0.5ex]{0pt}{2.5ex}\hspace*{1.0em}\textbackslash{}draw[line~width=0.8pt,~{-}{>},~{>}=stealth,~White]\\
\rule[-0.5ex]{0pt}{2.5ex}\hspace*{3.5em}(A)~to[out=90,~in=180]~(1.35,~3.25);\\
\rule[-0.5ex]{0pt}{2.5ex}\hspace*{0.0em}\textbackslash{}end\{tikzpicture\}}%
}%
\endgroup

\end{minipage}\hfill
\begin{minipage}{0.29\textwidth}
  \centering
  \begin{tikzpicture}
    % die Datei 'rastergrafik.png' laden und unter dem Namen
    % 'einbild' fuer spaetere Verwendung verfuegbar machen
    \pgfdeclareimage[interpolate=true, width=4cm]
                    {einbild}{rastergrafik.png};
    % Bild mit der unteren linken Ecke an der Position
    % (0, 0) einfuegen
    \pgftext[bottom, left, at=\pgfpoint{0cm}{0cm}]
            {\pgfuseimage{einbild}};
    % Text einfuegen
    \node[above right, text=White, font=\sffamily]
         (A) at (0, 0) {Auge};
    % weissen Pfeil zeichnen
    \draw[line width=0.8pt, ->, >=stealth, White]
         (A) to[out=90, in=180] (1.35, 3.25);
  \end{tikzpicture}
\end{minipage}

% -----------------
\section{Schleifen}
% -----------------
\begin{minipage}{0.7\textwidth}
  \footnotesize
  % parcolor version 2011-09-30
\begingroup
\ttfamily
\definecolor{R}{named}{Red}
\definecolor{G}{named}{ForestGreen}
\definecolor{B}{named}{RoyalBlue}
\definecolor{C}{named}{Cyan}
\definecolor{M}{named}{Magenta}
\definecolor{Y}{named}{YellowOrange}
\definecolor{background}{rgb}{0.82, 0.82, 0.92}
\dimen255=\textwidth
\advance\dimen255 by -2\fboxsep
\noindent
\colorbox{background}
{%
\parbox{\dimen255}
{%
\rule[-0.5ex]{0pt}{2.5ex}\hspace*{0.0em}\textbackslash{}begin\{tikzpicture\}\\
\rule[-0.5ex]{0pt}{2.5ex}\hspace*{1.0em}\textbackslash{}fill~(0,~0)~circle~(1pt);\\
\rule[-0.5ex]{0pt}{2.5ex}\hspace*{1.0em}\textbackslash{}draw~(0,~0)~circle~(2mm);\\
\rule[-0.5ex]{0pt}{2.5ex}\hspace*{1.0em}\textbackslash{}draw~(0,~0)~circle~(2cm);\\
\rule[-0.5ex]{0pt}{2.5ex}\hspace*{1.0em}\textcolor{G}{\textbf{\%~alle~Werte~von~0~bis~330~in~30er~Schritten}}\\
\rule[-0.5ex]{0pt}{2.5ex}\hspace*{1.0em}\textcolor{R}{\textbf{\textbackslash{}foreach}}~\textcolor{B}{\textbf{\textbackslash{}x}}~\textcolor{R}{\textbf{in~\{0,30,...,330\}}}\\
\rule[-0.5ex]{0pt}{2.5ex}\hspace*{1.0em}\textcolor{R}{\textbf{\{}}\\
\rule[-0.5ex]{0pt}{2.5ex}\hspace*{2.0em}\textbackslash{}draw~(\textcolor{B}{\textbf{\textbackslash{}x}}:2mm)~{-}{-}~(\textcolor{B}{\textbf{\textbackslash{}x}}:2cm);\\
\rule[-0.5ex]{0pt}{2.5ex}\hspace*{2.0em}\textbackslash{}node[rotate=\textcolor{B}{\textbf{\textbackslash{}x}}]~at~(\textcolor{B}{\textbf{\textbackslash{}x}}:26mm)\\
\rule[-0.5ex]{0pt}{2.5ex}\hspace*{4.5em}\{\{\textbackslash{}footnotesize\$\textcolor{B}{\textbf{\textbackslash{}x}}\^{}\{\textbackslash{}circ\}\$\}\};\\
\rule[-0.5ex]{0pt}{2.5ex}\hspace*{1.0em}\textcolor{R}{\textbf{\}}}\\
\rule[-0.5ex]{0pt}{2.5ex}\hspace*{0.0em}\textbackslash{}end\{tikzpicture\}}%
}%
\endgroup

\end{minipage}\hfill
\begin{minipage}{0.29\textwidth}
  \centering
  \begin{tikzpicture}[scale=0.66]
    \fill (0, 0) circle (1pt);
    \draw (0, 0) circle (2mm);
    \draw (0, 0) circle (2cm);
    % alle Werte von 0 bis 330 in 30er Schritten
    \foreach \x in {0,30,...,330}
    {
      \draw (\x:2mm) -- (\x:2cm);
      \node[rotate=\x] at (\x:26mm)
           {{\footnotesize$\x^{\circ}$}};
    }
  \end{tikzpicture}
\end{minipage}

\begin{minipage}{0.7\textwidth}
  \footnotesize
  % parcolor version 2011-09-30
\begingroup
\ttfamily
\definecolor{R}{named}{Red}
\definecolor{G}{named}{ForestGreen}
\definecolor{B}{named}{RoyalBlue}
\definecolor{C}{named}{Cyan}
\definecolor{M}{named}{Magenta}
\definecolor{Y}{named}{YellowOrange}
\definecolor{background}{rgb}{0.82, 0.82, 0.92}
\dimen255=\textwidth
\advance\dimen255 by -2\fboxsep
\noindent
\colorbox{background}
{%
\parbox{\dimen255}
{%
\rule[-0.5ex]{0pt}{2.5ex}\hspace*{0.0em}\textbackslash{}begin\{tikzpicture\}\\
\rule[-0.5ex]{0pt}{2.5ex}\hspace*{1.0em}\textbackslash{}coordinate~(A)~at~(0,~0);\\
\rule[-0.5ex]{0pt}{2.5ex}\hspace*{1.0em}\textbackslash{}coordinate~(B)~at~(3,~2);\\
\rule[-0.5ex]{0pt}{2.5ex}\hspace*{1.0em}\textcolor{G}{\textbf{\%~Punkte~verbinden~und~bezeichnen}}\\
\rule[-0.5ex]{0pt}{2.5ex}\hspace*{1.0em}\textbackslash{}draw~(A)~node[left]\{A\}~{-}{-}~(B)~node[right]\{B\};\\
\rule[-0.5ex]{0pt}{2.5ex}\hspace*{1.0em}\textcolor{G}{\textbf{\%~die~Schleifenvariable~durchlaeuft~nur~ganzzahlige~Werte}}\\
\rule[-0.5ex]{0pt}{2.5ex}\hspace*{1.0em}\textcolor{R}{\textbf{\textbackslash{}foreach}}~\textcolor{B}{\textbf{\textbackslash{}i}}~\textcolor{R}{\textbf{in~\{0,...,13\}}}\\
\rule[-0.5ex]{0pt}{2.5ex}\hspace*{1.0em}\textcolor{R}{\textbf{\{}}\\
\rule[-0.5ex]{0pt}{2.5ex}\hspace*{2.0em}\textcolor{G}{\textbf{\%~rationale~Werte~sollten~aus~der~Schleifenvariablen}}\\
\rule[-0.5ex]{0pt}{2.5ex}\hspace*{2.0em}\textcolor{G}{\textbf{\%~berechnet~werden:~hier~liegt~\textbackslash{}x~im~Intervall~[0,1]}}\\
\rule[-0.5ex]{0pt}{2.5ex}\hspace*{2.0em}\textcolor{R}{\textbf{\textbackslash{}pgfmathsetmacro}}\{\textcolor{B}{\textbf{\textbackslash{}x}}\}\{\textcolor{B}{\textbf{\textbackslash{}i}}\textcolor{R}{\textbf{/13}}\}\\
\rule[-0.5ex]{0pt}{2.5ex}\hspace*{2.0em}\textcolor{G}{\textbf{\%~die~Strecke~AB~wird~mit~14~Punkten~in~13~gleich~lange}}\\
\rule[-0.5ex]{0pt}{2.5ex}\hspace*{2.0em}\textcolor{G}{\textbf{\%~Teilstrecken~unterteilt}}\\
\rule[-0.5ex]{0pt}{2.5ex}\hspace*{2.0em}\textbackslash{}fill~(\$(A)!\textcolor{B}{\textbf{\textbackslash{}x}}!(B)\$)~circle[radius=1pt];\\
\rule[-0.5ex]{0pt}{2.5ex}\hspace*{1.0em}\textcolor{R}{\textbf{\}}}\\
\rule[-0.5ex]{0pt}{2.5ex}\hspace*{0.0em}\textbackslash{}end\{tikzpicture\}}%
}%
\endgroup

\end{minipage}\hfill
\begin{minipage}{0.29\textwidth}
  \centering
  \begin{tikzpicture}
    \coordinate (A) at (0, 0);
    \coordinate (B) at (3, 2);
    % Punkte verbinden und bezeichnen
    \draw (A) node[left]{A} -- (B) node[right]{B};
    % die Schleifenvariable durchlaeuft nur ganzzahlige Werte
    \foreach \i in {0,...,13}
    {
      % rationale Werte sollten aus der Schleifenvariablen
      % berechnet werden: hier liegt \x im Intervall [0,1]
      \pgfmathsetmacro{\x}{\i/13}
      % die Strecke AB wird mit 14 Punkten in 13 gleich lange
      % Teilstrecken unterteilt
      \fill ($(A)!\x!(B)$) circle[radius=1pt];
    }
  \end{tikzpicture}
\end{minipage}

% -------------------------------------------
\section{Einfache Berechnung von Koordinaten}
% -------------------------------------------

% --------------------------------------------
\subsection{Der partway und distance modifier}
% --------------------------------------------
\begin{minipage}{0.7\textwidth}
  \footnotesize
  % parcolor version 2011-09-30
\begingroup
\ttfamily
\definecolor{R}{named}{Red}
\definecolor{G}{named}{ForestGreen}
\definecolor{B}{named}{RoyalBlue}
\definecolor{C}{named}{Cyan}
\definecolor{M}{named}{Magenta}
\definecolor{Y}{named}{YellowOrange}
\definecolor{background}{rgb}{0.82, 0.82, 0.92}
\dimen255=\textwidth
\advance\dimen255 by -2\fboxsep
\noindent
\colorbox{background}
{%
\parbox{\dimen255}
{%
\rule[-0.5ex]{0pt}{2.5ex}\hspace*{0.0em}\textbackslash{}begin\{tikzpicture\}\\
\rule[-0.5ex]{0pt}{2.5ex}\hspace*{1.0em}\textcolor{G}{\textbf{\%~zwei~Punkte~definieren}}\\
\rule[-0.5ex]{0pt}{2.5ex}\hspace*{1.0em}\textbackslash{}coordinate~(A)~at~(1,~1);\\
\rule[-0.5ex]{0pt}{2.5ex}\hspace*{1.0em}\textbackslash{}coordinate~(B)~at~(3,~2);\\
\rule[-0.5ex]{0pt}{2.5ex}\hspace*{1.0em}\textcolor{G}{\textbf{\%~Punkte~markieren}}\\
\rule[-0.5ex]{0pt}{2.5ex}\hspace*{1.0em}\textbackslash{}fill~(A)~circle[radius=1pt];\\
\rule[-0.5ex]{0pt}{2.5ex}\hspace*{1.0em}\textbackslash{}fill~(B)~circle[radius=1pt];\\
\rule[-0.5ex]{0pt}{2.5ex}\hspace*{1.0em}\textcolor{G}{\textbf{\%~Punkte~beschriften~(distance~modifier)}}\\
\rule[-0.5ex]{0pt}{2.5ex}\hspace*{1.0em}\textbackslash{}node~at~(\textcolor{R}{\textbf{\$(A)!3mm!180:(B)\$}})~\{\{\textbackslash{}footnotesize\$A\$\}\};\\
\rule[-0.5ex]{0pt}{2.5ex}\hspace*{1.0em}\textbackslash{}node~at~(\textcolor{R}{\textbf{\$(B)!3mm!180:(A)\$}})~\{\{\textbackslash{}footnotesize\$B\$\}\};\\
\rule[-0.5ex]{0pt}{2.5ex}\hspace*{1.0em}\textcolor{G}{\textbf{\%~Punkte~verbinden}}\\
\rule[-0.5ex]{0pt}{2.5ex}\hspace*{1.0em}\textbackslash{}draw~(A)~{-}{-}~(B);\\
\rule[-0.5ex]{0pt}{2.5ex}\hspace*{1.0em}\textcolor{G}{\textbf{\%~Kreis~um~A~zeichnen~mit~der~halben~Strecke~AB~als~Radius}}\\
\rule[-0.5ex]{0pt}{2.5ex}\hspace*{1.0em}\textbackslash{}draw~(A)~circle[radius=\textcolor{R}{\textbf{\{sqrt(5)/2\}}}];\\
\rule[-0.5ex]{0pt}{2.5ex}\hspace*{1.0em}\textcolor{G}{\textbf{\%~alle~Winkel~aus~[30,330]~in~30er~Schritten}}\\
\rule[-0.5ex]{0pt}{2.5ex}\hspace*{1.0em}\textbackslash{}foreach~\textbackslash{}angle~in~\{30,60,...,330\}\\
\rule[-0.5ex]{0pt}{2.5ex}\hspace*{1.0em}\{\\
\rule[-0.5ex]{0pt}{2.5ex}\hspace*{2.0em}\textcolor{G}{\textbf{\%~Punkt~auf~dem~Kreis~berechnen~(partway~modifier)}}\\
\rule[-0.5ex]{0pt}{2.5ex}\hspace*{2.0em}\textbackslash{}coordinate~(X)~at~(\textcolor{R}{\textbf{\$(A)!0.5!\textbackslash{}angle:(B)\$}});\\
\rule[-0.5ex]{0pt}{2.5ex}\hspace*{2.0em}\textcolor{G}{\textbf{\%~Punkt~X~markieren}}\\
\rule[-0.5ex]{0pt}{2.5ex}\hspace*{2.0em}\textbackslash{}fill~(X)~circle[radius=1pt];\\
\rule[-0.5ex]{0pt}{2.5ex}\hspace*{2.0em}\textcolor{G}{\textbf{\%~Punkt~X~beschriften~(distance~modifier)}}\\
\rule[-0.5ex]{0pt}{2.5ex}\hspace*{2.0em}\textbackslash{}node~at~(\textcolor{R}{\textbf{\$(X)!5mm!180:(A)\$}})\\
\rule[-0.5ex]{0pt}{2.5ex}\hspace*{6.5em}\{\{\textbackslash{}footnotesize\$\textbackslash{}angle\^{}\{\textbackslash{}circ\}\$\}\};\\
\rule[-0.5ex]{0pt}{2.5ex}\hspace*{1.0em}\}\\
\rule[-0.5ex]{0pt}{2.5ex}\hspace*{0.0em}\textbackslash{}end\{tikzpicture\}}%
}%
\endgroup

\end{minipage}\hfill
\begin{minipage}{0.29\textwidth}
  \centering
  \begin{tikzpicture}[scale=0.9]
    % zwei Punkte definieren
    \coordinate (A) at (1, 1);
    \coordinate (B) at (3, 2);
    % Punkte markieren
    \fill (A) circle[radius=1pt];
    \fill (B) circle[radius=1pt];
    % Punkte beschriften (distance modifier)
    \node at ($(A)!3mm!180:(B)$) {{\footnotesize$A$}};
    \node at ($(B)!3mm!180:(A)$) {{\footnotesize$B$}};
    % Punkte verbinden
    \draw (A) -- (B);
    % Kreis um A zeichnen mit der halben Strecke AB als Radius
    \draw (A) circle[radius={sqrt(5)/2}];
    % alle Winkel aus [30,330] in 30er Schritten
    \foreach \angle in {30,60,...,330}
    {
      % Punkt auf dem Kreis berechnen (partway modifier)
      \coordinate (X) at ($(A)!0.5!\angle:(B)$);
      % Punkt X markieren
      \fill (X) circle[radius=1pt];
      % Punkt X beschriften (distance modifier)
      \node at ($(X)!5mm!180:(A)$)
               {{\footnotesize$\angle^{\circ}$}};
    }
  \end{tikzpicture}
\end{minipage}

% -----------------------
\subsection{Projektionen}
% -----------------------
\begin{minipage}{0.7\textwidth}
  \footnotesize
  % parcolor version 2011-09-30
\begingroup
\ttfamily
\definecolor{R}{named}{Red}
\definecolor{G}{named}{ForestGreen}
\definecolor{B}{named}{RoyalBlue}
\definecolor{C}{named}{Cyan}
\definecolor{M}{named}{Magenta}
\definecolor{Y}{named}{YellowOrange}
\definecolor{background}{rgb}{0.82, 0.82, 0.92}
\dimen255=\textwidth
\advance\dimen255 by -2\fboxsep
\noindent
\colorbox{background}
{%
\parbox{\dimen255}
{%
\rule[-0.5ex]{0pt}{2.5ex}\hspace*{0.0em}\textbackslash{}begin\{tikzpicture\}\\
\rule[-0.5ex]{0pt}{2.5ex}\hspace*{1.0em}\textcolor{G}{\textbf{\%~Punkte~definieren}}\\
\rule[-0.5ex]{0pt}{2.5ex}\hspace*{1.0em}\textbackslash{}coordinate~(A)~at~(1,~1);\\
\rule[-0.5ex]{0pt}{2.5ex}\hspace*{1.0em}\textbackslash{}coordinate~(B)~at~(3,~1);\\
\rule[-0.5ex]{0pt}{2.5ex}\hspace*{1.0em}\textbackslash{}coordinate~(C)~at~(2,~2);\\
\rule[-0.5ex]{0pt}{2.5ex}\hspace*{1.0em}\textcolor{G}{\textbf{\%~Punkt~C~auf~AB~projizieren}}\\
\rule[-0.5ex]{0pt}{2.5ex}\hspace*{1.0em}\textbackslash{}coordinate~(D)~at~(\textcolor{R}{\textbf{\$(A)!(C)!(B)\$}});\\
\rule[-0.5ex]{0pt}{2.5ex}\hspace*{1.0em}\textcolor{G}{\textbf{\%~AB~erst~90~Grad~um~A~drehen~und~dann~C~projizieren}}\\
\rule[-0.5ex]{0pt}{2.5ex}\hspace*{1.0em}\textbackslash{}coordinate~(E)~at~(\textcolor{R}{\textbf{\$(A)!(C)!90:(B)\$}});\\
\rule[-0.5ex]{0pt}{2.5ex}\hspace*{1.0em}\textcolor{G}{\textbf{\%~Punkte~markieren~und~beschriften}}\\
\rule[-0.5ex]{0pt}{2.5ex}\hspace*{1.0em}\textbackslash{}fill~(A)~circle[radius=1pt]~node[below]~\{A\}\\
\rule[-0.5ex]{0pt}{2.5ex}\hspace*{4.0em}(B)~circle[radius=1pt]~node[below]~\{B\}\\
\rule[-0.5ex]{0pt}{2.5ex}\hspace*{4.0em}(C)~circle[radius=1pt]~node[above]~\{C\}\\
\rule[-0.5ex]{0pt}{2.5ex}\hspace*{4.0em}(D)~circle[radius=1pt]~node[below]~\{D\}\\
\rule[-0.5ex]{0pt}{2.5ex}\hspace*{4.0em}(E)~circle[radius=1pt]~node[above]~\{E\};\\
\rule[-0.5ex]{0pt}{2.5ex}\hspace*{1.0em}\textcolor{G}{\textbf{\%~Punkte~verbinden}}\\
\rule[-0.5ex]{0pt}{2.5ex}\hspace*{1.0em}\textbackslash{}draw~(A)~{-}{-}~(B)~(C)~{-}{-}~(D)~(C)~{-}{-}~(E);\\
\rule[-0.5ex]{0pt}{2.5ex}\hspace*{0.0em}\textbackslash{}end\{tikzpicture\}}%
}%
\endgroup

\end{minipage}\hfill
\begin{minipage}{0.29\textwidth}
  \centering
  \begin{tikzpicture}
    % Punkte definieren
    \coordinate (A) at (1, 1);
    \coordinate (B) at (3, 1);
    \coordinate (C) at (2, 2);
    % Punkt C auf AB pro­ji­zie­ren
    \coordinate (D) at ($(A)!(C)!(B)$);
    % AB erst 90 Grad um A drehen und dann C pro­ji­zie­ren
    \coordinate (E) at ($(A)!(C)!90:(B)$);
    % Punkte markieren und beschriften
    \fill (A) circle[radius=1pt] node[below] {A}
          (B) circle[radius=1pt] node[below] {B}
          (C) circle[radius=1pt] node[above] {C}
          (D) circle[radius=1pt] node[below] {D}
          (E) circle[radius=1pt] node[above] {E};
    % Punkte verbinden
    \draw (A) -- (B) (C) -- (D) (C) -- (E);
  \end{tikzpicture}
\end{minipage}

% -------------------------------
\section{Schnittpunkte bestimmen}
% -------------------------------

% ------------------------------------------------
\subsection{Schnittpunkt zweier Geraden bestimmen}
% ------------------------------------------------
\begin{minipage}{0.7\textwidth}
  \footnotesize
  % parcolor version 2011-09-30
\begingroup
\ttfamily
\definecolor{R}{named}{Red}
\definecolor{G}{named}{ForestGreen}
\definecolor{B}{named}{RoyalBlue}
\definecolor{C}{named}{Cyan}
\definecolor{M}{named}{Magenta}
\definecolor{Y}{named}{YellowOrange}
\definecolor{background}{rgb}{0.82, 0.82, 0.92}
\dimen255=\textwidth
\advance\dimen255 by -2\fboxsep
\noindent
\colorbox{background}
{%
\parbox{\dimen255}
{%
\rule[-0.5ex]{0pt}{2.5ex}\hspace*{0.0em}\textbackslash{}begin\{tikzpicture\}\\
\rule[-0.5ex]{0pt}{2.5ex}\hspace*{1.0em}\textbackslash{}coordinate~(A)~at~(0,~0);\\
\rule[-0.5ex]{0pt}{2.5ex}\hspace*{1.0em}\textbackslash{}coordinate~(B)~at~(2,~1);\\
\rule[-0.5ex]{0pt}{2.5ex}\hspace*{1.0em}\textbackslash{}coordinate~(C)~at~([shift=\{(270:1cm)\}]A);\\
\rule[-0.5ex]{0pt}{2.5ex}\hspace*{1.0em}\textbackslash{}coordinate~(D)~at~([shift=\{(135:1cm)\}]B);\\
\rule[-0.5ex]{0pt}{2.5ex}\hspace*{1.0em}\textcolor{G}{\textbf{\%~Schnittpunkt~berechnen}}\\
\rule[-0.5ex]{0pt}{2.5ex}\hspace*{1.0em}\textbackslash{}coordinate~(S)~at~(\textcolor{R}{\textbf{intersection~of~A{-}{-}B~and~C{-}{-}D}});\\
\rule[-0.5ex]{0pt}{2.5ex}\hspace*{1.0em}\textcolor{G}{\textbf{\%~Punkte~verbinden}}\\
\rule[-0.5ex]{0pt}{2.5ex}\hspace*{1.0em}\textbackslash{}draw~(A)~node[left]\{\$A\$\}~{-}{-}~(B)~node[right]\{\$B\$\}\\
\rule[-0.5ex]{0pt}{2.5ex}\hspace*{4.0em}(C)~node[left]\{\$C\$\}~{-}{-}~(D)~node[right]\{\$D\$\};\\
\rule[-0.5ex]{0pt}{2.5ex}\hspace*{1.0em}\textcolor{G}{\textbf{\%~Punkte~markieren}}\\
\rule[-0.5ex]{0pt}{2.5ex}\hspace*{1.0em}\textbackslash{}fill~(A)~circle[radius=1pt]~(B)~circle[radius=1pt]\\
\rule[-0.5ex]{0pt}{2.5ex}\hspace*{4.0em}(C)~circle[radius=1pt]~(D)~circle[radius=1pt]\\
\rule[-0.5ex]{0pt}{2.5ex}\hspace*{4.0em}(S)~circle[radius=1pt]~node[below~right]~\{\$S\$\};\\
\rule[-0.5ex]{0pt}{2.5ex}\hspace*{0.0em}\textbackslash{}end\{tikzpicture\}}%
}%
\endgroup

\end{minipage}\hfill
\begin{minipage}{0.29\textwidth}
  \centering
  \begin{tikzpicture}
    \coordinate (A) at (0, 0);
    \coordinate (B) at (2, 1);
    \coordinate (C) at ([shift={(270:1cm)}]A);
    \coordinate (D) at ([shift={(135:1cm)}]B);
    % Schnittpunkt berechnen
    \coordinate (S) at (intersection of A--B and C--D);
    % Punkte verbinden
    \draw (A) node[left]{$A$} -- (B) node[right]{$B$}
          (C) node[left]{$C$} -- (D) node[right]{$D$};
    % Punkte markieren
    \fill (A) circle[radius=1pt] (B) circle[radius=1pt]
          (C) circle[radius=1pt] (D) circle[radius=1pt]
          (S) circle[radius=1pt] node[below right] {$S$};
  \end{tikzpicture}
\end{minipage}

% ---------------------------------------------------
\subsection{Schnittpunkte beliebiger Pfade bestimmen}
% ---------------------------------------------------
\begin{minipage}{0.7\textwidth}
  \footnotesize
  % parcolor version 2011-09-30
\begingroup
\ttfamily
\definecolor{R}{named}{Red}
\definecolor{G}{named}{ForestGreen}
\definecolor{B}{named}{RoyalBlue}
\definecolor{C}{named}{Cyan}
\definecolor{M}{named}{Magenta}
\definecolor{Y}{named}{YellowOrange}
\definecolor{background}{rgb}{0.82, 0.82, 0.92}
\dimen255=\textwidth
\advance\dimen255 by -2\fboxsep
\noindent
\colorbox{background}
{%
\parbox{\dimen255}
{%
\rule[-0.5ex]{0pt}{2.5ex}\hspace*{0.0em}\textbackslash{}begin\{tikzpicture\}\\
\rule[-0.5ex]{0pt}{2.5ex}\hspace*{1.0em}\textcolor{G}{\textbf{\%~zuerst~muessen~den~zu~schneidenden~Pfaden}}\\
\rule[-0.5ex]{0pt}{2.5ex}\hspace*{1.0em}\textcolor{G}{\textbf{\%~Namen~zugewiesen~werden}}\\
\rule[-0.5ex]{0pt}{2.5ex}\hspace*{1.0em}\textbackslash{}draw[\textcolor{R}{\textbf{name~path}}=\textcolor{B}{\textbf{kreisA}}]~(0,~0)~circle[radius=1];\\
\rule[-0.5ex]{0pt}{2.5ex}\hspace*{1.0em}\textbackslash{}draw[\textcolor{R}{\textbf{name~path}}=\textcolor{B}{\textbf{kreisB}}]~(1,~0)~circle[radius=1];\\
\rule[-0.5ex]{0pt}{2.5ex}\hspace*{1.0em}\textcolor{G}{\textbf{\%~mit~\textbackslash{}path~findet~die~Berechnung~statt,~aber}}\\
\rule[-0.5ex]{0pt}{2.5ex}\hspace*{1.0em}\textcolor{G}{\textbf{\%~es~wird~nichts~gezeichnet}}\\
\rule[-0.5ex]{0pt}{2.5ex}\hspace*{1.0em}\textbackslash{}path[\textcolor{R}{\textbf{name~intersections}}=\{\textcolor{R}{\textbf{of}}=\textcolor{B}{\textbf{kreisA}}~\textcolor{R}{\textbf{and}}~\textcolor{B}{\textbf{kreisB}}\}];\\
\rule[-0.5ex]{0pt}{2.5ex}\hspace*{1.0em}\textcolor{G}{\textbf{\%~'name~intersections'~benennt~die~Schnittpunkte~alle}}\\
\rule[-0.5ex]{0pt}{2.5ex}\hspace*{1.0em}\textcolor{G}{\textbf{\%~nach~dem~Schema~'intersection{-}i'}}\\
\rule[-0.5ex]{0pt}{2.5ex}\hspace*{1.0em}\textbackslash{}fill~(\textcolor{B}{\textbf{intersection{-}1}})~circle[radius=1pt]~node[above]\{1\};\\
\rule[-0.5ex]{0pt}{2.5ex}\hspace*{1.0em}\textbackslash{}fill~(\textcolor{B}{\textbf{intersection{-}2}})~circle[radius=1pt]~node[below]\{2\};\\
\rule[-0.5ex]{0pt}{2.5ex}\hspace*{0.0em}\textbackslash{}end\{tikzpicture\}}%
}%
\endgroup

\end{minipage}\hfill
\begin{minipage}{0.29\textwidth}
  \centering
  \begin{tikzpicture}
    % zuerst muessen den zu schneidenden Pfaden
    % Namen zugewiesen werden
    \draw[name path=kreisA] (0, 0) circle[radius=1];
    \draw[name path=kreisB] (1, 0) circle[radius=1];
    % mit \path findet die Berechnung statt, aber
    % es wird nichts gezeichnet
    \path[name intersections={of=kreisA and kreisB}];
    % 'name intersections' benennt die Schnittpunkte alle
    % nach dem Schema 'intersection-i'
    \fill (intersection-1) circle[radius=1pt] node[above]{1};
    \fill (intersection-2) circle[radius=1pt] node[below]{2};
  \end{tikzpicture}
\end{minipage}

\begin{minipage}{0.7\textwidth}
  \footnotesize
  % parcolor version 2011-09-30
\begingroup
\ttfamily
\definecolor{R}{named}{Red}
\definecolor{G}{named}{ForestGreen}
\definecolor{B}{named}{RoyalBlue}
\definecolor{C}{named}{Cyan}
\definecolor{M}{named}{Magenta}
\definecolor{Y}{named}{YellowOrange}
\definecolor{background}{rgb}{0.82, 0.82, 0.92}
\dimen255=\textwidth
\advance\dimen255 by -2\fboxsep
\noindent
\colorbox{background}
{%
\parbox{\dimen255}
{%
\rule[-0.5ex]{0pt}{2.5ex}\hspace*{0.0em}\textbackslash{}begin\{tikzpicture\}\\
\rule[-0.5ex]{0pt}{2.5ex}\hspace*{1.0em}\textcolor{G}{\textbf{\%~zwei~Ellipsen~mit~vier~Schnittpunkten}}\\
\rule[-0.5ex]{0pt}{2.5ex}\hspace*{1.0em}\textbackslash{}draw[\textcolor{R}{\textbf{name~path}}=\textcolor{B}{\textbf{A}}]~(0,~0)~circle[x~radius=1,~y~radius=2];\\
\rule[-0.5ex]{0pt}{2.5ex}\hspace*{1.0em}\textbackslash{}draw[\textcolor{R}{\textbf{name~path}}=\textcolor{B}{\textbf{B}}]~(0,~0)~circle[x~radius=2,~y~radius=1];\\
\rule[-0.5ex]{0pt}{2.5ex}\hspace*{1.0em}\textcolor{G}{\textbf{\%~'total'~liefert~die~Anzahl~der~Schnittpunkte,~wobei}}\\
\rule[-0.5ex]{0pt}{2.5ex}\hspace*{1.0em}\textcolor{G}{\textbf{\%~das~Makro~\textbackslash{}n~nur~innerhalb~des~Pfades~definiert~ist}}\\
\rule[-0.5ex]{0pt}{2.5ex}\hspace*{1.0em}\textbackslash{}fill[\textcolor{R}{\textbf{name~intersections}}=\{\textcolor{R}{\textbf{of}}=\textcolor{B}{\textbf{A}}~\textcolor{R}{\textbf{and}}~\textcolor{B}{\textbf{B}},~\textcolor{R}{\textbf{total}}=\textcolor{B}{\textbf{\textbackslash{}n}}\}]\\
\rule[-0.5ex]{0pt}{2.5ex}\hspace*{3.5em}\textbackslash{}foreach~\textcolor{B}{\textbf{\textbackslash{}i}}~in~\{1,...,\textcolor{B}{\textbf{\textbackslash{}n}}\}\\
\rule[-0.5ex]{0pt}{2.5ex}\hspace*{3.5em}\{\\
\rule[-0.5ex]{0pt}{2.5ex}\hspace*{4.5em}(intersection{-}\textcolor{B}{\textbf{\textbackslash{}i}})~circle[radius=1pt]\\
\rule[-0.5ex]{0pt}{2.5ex}\hspace*{4.5em}(intersection{-}\textcolor{B}{\textbf{\textbackslash{}i}})~[scale=1.2]~node~\{\textcolor{B}{\textbf{\textbackslash{}i}}\}\\
\rule[-0.5ex]{0pt}{2.5ex}\hspace*{3.5em}\};\\
\rule[-0.5ex]{0pt}{2.5ex}\hspace*{0.0em}\textbackslash{}end\{tikzpicture\}}%
}%
\endgroup

\end{minipage}\hfill
\begin{minipage}{0.29\textwidth}
  \centering
  \begin{tikzpicture}
    % zwei Ellipsen mit vier Schnittpunkten
    \draw[name path=A] (0, 0) circle[x radius=1, y radius=2];
    \draw[name path=B] (0, 0) circle[x radius=2, y radius=1];
    % 'total' liefert die Anzahl der Schnittpunkte, wobei
    % das Makro \n nur innerhalb des Pfades definiert ist
    \fill[name intersections={of=A and B, total=\n}]
         \foreach \i in {1,...,\n}
         {
           (intersection-\i) circle[radius=1pt]
           (intersection-\i) [scale=1.2] node {\i}
         };
  \end{tikzpicture}
\end{minipage}

% -----------------------------------------
\section{Rechnen mit dem Paket \texttt{fp}}
% -----------------------------------------
% parcolor version 2011-09-30
\begingroup
\ttfamily
\definecolor{R}{named}{Red}
\definecolor{G}{named}{ForestGreen}
\definecolor{B}{named}{RoyalBlue}
\definecolor{C}{named}{Cyan}
\definecolor{M}{named}{Magenta}
\definecolor{Y}{named}{YellowOrange}
\definecolor{background}{rgb}{0.82, 0.82, 0.92}
\dimen255=\textwidth
\advance\dimen255 by -2\fboxsep
\noindent
\colorbox{background}
{%
\parbox{\dimen255}
{%
\rule[-0.5ex]{0pt}{2.5ex}\hspace*{0.0em}\textcolor{G}{\textbf{\%~Konstanten}}\\
\rule[-0.5ex]{0pt}{2.5ex}\hspace*{0.0em}\textcolor{R}{\textbf{\textbackslash{}FPe}}~~~~~~~~~~~~~~~~~~~\textcolor{G}{\textbf{\%~2.718281828459045235}}\\
\rule[-0.5ex]{0pt}{2.5ex}\hspace*{0.0em}\textcolor{R}{\textbf{\textbackslash{}FPpi}}~~~~~~~~~~~~~~~~~~\textcolor{G}{\textbf{\%~3.141592653589793238}}\\
\rule[-0.5ex]{0pt}{2.5ex}\hspace*{0.0em}\textcolor{G}{\textbf{\%~Zuweisungen}}\\
\rule[-0.5ex]{0pt}{2.5ex}\hspace*{0.0em}\textcolor{R}{\textbf{\textbackslash{}FPset}}~~~\{\textcolor{B}{\textbf{\textbackslash{}x}}\}\{2\}~~~~~~~\textcolor{G}{\textbf{\%~x~:=~2}}\\
\rule[-0.5ex]{0pt}{2.5ex}\hspace*{0.0em}\textcolor{R}{\textbf{\textbackslash{}FPset}}~~~\{\textcolor{B}{\textbf{\textbackslash{}y}}\}\{2.5\}~~~~~\textcolor{G}{\textbf{\%~y~:=~2.5}}\\
\rule[-0.5ex]{0pt}{2.5ex}\hspace*{0.0em}\textcolor{G}{\textbf{\%~unaere~Operationen}}\\
\rule[-0.5ex]{0pt}{2.5ex}\hspace*{0.0em}\textcolor{R}{\textbf{\textbackslash{}FPabs}}~~~\{\textcolor{B}{\textbf{\textbackslash{}a}}\}\{\textbackslash{}x\}~~~~~~\textcolor{G}{\textbf{\%~a~:=~abs(x)}}\\
\rule[-0.5ex]{0pt}{2.5ex}\hspace*{0.0em}\textcolor{R}{\textbf{\textbackslash{}FPneg}}~~~\{\textcolor{B}{\textbf{\textbackslash{}a}}\}\{\textbackslash{}x\}~~~~~~\textcolor{G}{\textbf{\%~a~:=~{-}x}}\\
\rule[-0.5ex]{0pt}{2.5ex}\hspace*{0.0em}\textcolor{G}{\textbf{\%~binaere~Operationen}}\\
\rule[-0.5ex]{0pt}{2.5ex}\hspace*{0.0em}\textcolor{R}{\textbf{\textbackslash{}FPadd}}~~~\{\textcolor{B}{\textbf{\textbackslash{}a}}\}\{\textbackslash{}x\}\{\textbackslash{}y\}~~\textcolor{G}{\textbf{\%~a~:=~x~+~y}}\\
\rule[-0.5ex]{0pt}{2.5ex}\hspace*{0.0em}\textcolor{R}{\textbf{\textbackslash{}FPsub}}~~~\{\textcolor{B}{\textbf{\textbackslash{}a}}\}\{\textbackslash{}x\}\{\textbackslash{}y\}~~\textcolor{G}{\textbf{\%~a~:=~x~{-}~y}}\\
\rule[-0.5ex]{0pt}{2.5ex}\hspace*{0.0em}\textcolor{R}{\textbf{\textbackslash{}FPmul}}~~~\{\textcolor{B}{\textbf{\textbackslash{}a}}\}\{\textbackslash{}x\}\{\textbackslash{}y\}~~\textcolor{G}{\textbf{\%~a~:=~x~*~y}}\\
\rule[-0.5ex]{0pt}{2.5ex}\hspace*{0.0em}\textcolor{R}{\textbf{\textbackslash{}FPdiv}}~~~\{\textcolor{B}{\textbf{\textbackslash{}a}}\}\{\textbackslash{}x\}\{\textbackslash{}y\}~~\textcolor{G}{\textbf{\%~a~:=~x~/~y}}\\
\rule[-0.5ex]{0pt}{2.5ex}\hspace*{0.0em}\textcolor{R}{\textbf{\textbackslash{}FPmin}}~~~\{\textcolor{B}{\textbf{\textbackslash{}a}}\}\{\textbackslash{}x\}\{\textbackslash{}y\}~~\textcolor{G}{\textbf{\%~a~:=~min(x,y)}}\\
\rule[-0.5ex]{0pt}{2.5ex}\hspace*{0.0em}\textcolor{R}{\textbf{\textbackslash{}FPmax}}~~~\{\textcolor{B}{\textbf{\textbackslash{}a}}\}\{\textbackslash{}x\}\{\textbackslash{}y\}~~\textcolor{G}{\textbf{\%~a~:=~max(x,y)}}\\
\rule[-0.5ex]{0pt}{2.5ex}\hspace*{0.0em}\textcolor{G}{\textbf{\%~Nachkommastellen}}\\
\rule[-0.5ex]{0pt}{2.5ex}\hspace*{0.0em}\textcolor{R}{\textbf{\textbackslash{}FPround}}~\{\textcolor{B}{\textbf{\textbackslash{}a}}\}\{\textbackslash{}x\}\{\textbackslash{}y\}~~\textcolor{G}{\textbf{\%~a~:=~x~auf~y~Nachkommastellen~gerundet}}\\
\rule[-0.5ex]{0pt}{2.5ex}\hspace*{0.0em}\textcolor{R}{\textbf{\textbackslash{}FPtrunc}}~\{\textcolor{B}{\textbf{\textbackslash{}a}}\}\{\textbackslash{}x\}\{\textbackslash{}y\}~~\textcolor{G}{\textbf{\%~a~:=~x~nach~y~Nachkommastellen~abgeschnitten}}\\
\rule[-0.5ex]{0pt}{2.5ex}\hspace*{0.0em}\textcolor{R}{\textbf{\textbackslash{}FPclip}}~~\{\textcolor{B}{\textbf{\textbackslash{}a}}\}\{\textbackslash{}x\}~~~~~~\textcolor{G}{\textbf{\%~a~:=~x~nur~mit~signifikanten~Nachkommastellen}}\\
\rule[-0.5ex]{0pt}{2.5ex}\hspace*{0.0em}\textcolor{G}{\textbf{\%~Potenzen~und~Wurzeln}}\\
\rule[-0.5ex]{0pt}{2.5ex}\hspace*{0.0em}\textcolor{R}{\textbf{\textbackslash{}FPpow}}~~~\{\textcolor{B}{\textbf{\textbackslash{}a}}\}\{\textbackslash{}x\}\{\textbackslash{}y\}~~\textcolor{G}{\textbf{\%~a~:=~x\^{}y}}\\
\rule[-0.5ex]{0pt}{2.5ex}\hspace*{0.0em}\textcolor{R}{\textbf{\textbackslash{}FProot}}~~\{\textcolor{B}{\textbf{\textbackslash{}a}}\}\{\textbackslash{}x\}\{\textbackslash{}y\}~~\textcolor{G}{\textbf{\%~a~:=~x\^{}(1/y)}}\\
\rule[-0.5ex]{0pt}{2.5ex}\hspace*{0.0em}\textcolor{G}{\textbf{\%~Trigonometrische~Funktionen}}\\
\rule[-0.5ex]{0pt}{2.5ex}\hspace*{0.0em}\textcolor{R}{\textbf{\textbackslash{}FPsin}}~~~\{\textcolor{B}{\textbf{\textbackslash{}a}}\}\{\textbackslash{}x\}~~~~~~\textcolor{G}{\textbf{\%~a~:=~sin(x)}}\\
\rule[-0.5ex]{0pt}{2.5ex}\hspace*{0.0em}\textcolor{R}{\textbf{\textbackslash{}FPcos}}~~~\{\textcolor{B}{\textbf{\textbackslash{}a}}\}\{\textbackslash{}x\}~~~~~~\textcolor{G}{\textbf{\%~a~:=~cos(x)}}\\
\rule[-0.5ex]{0pt}{2.5ex}\hspace*{0.0em}\textcolor{R}{\textbf{\textbackslash{}FPtan}}~~~\{\textcolor{B}{\textbf{\textbackslash{}a}}\}\{\textbackslash{}x\}~~~~~~\textcolor{G}{\textbf{\%~a~:=~tan(x)}}\\
\rule[-0.5ex]{0pt}{2.5ex}\hspace*{0.0em}\textcolor{R}{\textbf{\textbackslash{}FPcot}}~~~\{\textcolor{B}{\textbf{\textbackslash{}a}}\}\{\textbackslash{}x\}~~~~~~\textcolor{G}{\textbf{\%~a~:=~cot(x)}}\\
\rule[-0.5ex]{0pt}{2.5ex}\hspace*{0.0em}\textcolor{R}{\textbf{\textbackslash{}FParcsin}}\{\textcolor{B}{\textbf{\textbackslash{}a}}\}\{\textbackslash{}x\}~~~~~~\textcolor{G}{\textbf{\%~a~:=~arcsin(x)}}\\
\rule[-0.5ex]{0pt}{2.5ex}\hspace*{0.0em}\textcolor{R}{\textbf{\textbackslash{}FParccos}}\{\textcolor{B}{\textbf{\textbackslash{}a}}\}\{\textbackslash{}x\}~~~~~~\textcolor{G}{\textbf{\%~a~:=~arccos(x)}}\\
\rule[-0.5ex]{0pt}{2.5ex}\hspace*{0.0em}\textcolor{R}{\textbf{\textbackslash{}FParctan}}\{\textcolor{B}{\textbf{\textbackslash{}a}}\}\{\textbackslash{}x\}~~~~~~\textcolor{G}{\textbf{\%~a~:=~arctan(x)}}\\
\rule[-0.5ex]{0pt}{2.5ex}\hspace*{0.0em}\textcolor{R}{\textbf{\textbackslash{}FParccot}}\{\textcolor{B}{\textbf{\textbackslash{}a}}\}\{\textbackslash{}x\}~~~~~~\textcolor{G}{\textbf{\%~a~:=~arccot(x)}}\\
\rule[-0.5ex]{0pt}{2.5ex}\hspace*{0.0em}\textcolor{G}{\textbf{\%~Exponential{-}~und~Logarithmusfunktion}}\\
\rule[-0.5ex]{0pt}{2.5ex}\hspace*{0.0em}\textcolor{R}{\textbf{\textbackslash{}FPexp}}~~~\{\textcolor{B}{\textbf{\textbackslash{}a}}\}\{\textbackslash{}x\}~~~~~~\textcolor{G}{\textbf{\%~a~:=~exp(x)}}\\
\rule[-0.5ex]{0pt}{2.5ex}\hspace*{0.0em}\textcolor{R}{\textbf{\textbackslash{}FPln}}~~~~\{\textcolor{B}{\textbf{\textbackslash{}a}}\}\{\textbackslash{}x\}~~~~~~\textcolor{G}{\textbf{\%~a~:=~ln(x)}}\\
\rule[-0.5ex]{0pt}{2.5ex}\hspace*{0.0em}\textcolor{G}{\textbf{\%~Fallunterscheidungen}}\\
\rule[-0.5ex]{0pt}{2.5ex}\hspace*{0.0em}\textcolor{R}{\textbf{\textbackslash{}FPiflt}}~\{\textbackslash{}x\}\{\textbackslash{}y\}~...~\textcolor{R}{\textbf{\textbackslash{}else}}~...~\textcolor{R}{\textbf{\textbackslash{}fi}}~~~\textcolor{G}{\textbf{\%~ist~(x~{<}~y)~?}}\\
\rule[-0.5ex]{0pt}{2.5ex}\hspace*{0.0em}\textcolor{R}{\textbf{\textbackslash{}FPifeq}}~\{\textbackslash{}x\}\{\textbackslash{}y\}~...~\textcolor{R}{\textbf{\textbackslash{}else}}~...~\textcolor{R}{\textbf{\textbackslash{}fi}}~~~\textcolor{G}{\textbf{\%~ist~(x~=~y)~?}}\\
\rule[-0.5ex]{0pt}{2.5ex}\hspace*{0.0em}\textcolor{R}{\textbf{\textbackslash{}FPifgt}}~\{\textbackslash{}x\}\{\textbackslash{}y\}~...~\textcolor{R}{\textbf{\textbackslash{}else}}~...~\textcolor{R}{\textbf{\textbackslash{}fi}}~~~\textcolor{G}{\textbf{\%~ist~(x~{>}~y)~?}}\\
\rule[-0.5ex]{0pt}{2.5ex}\hspace*{0.0em}\textcolor{R}{\textbf{\textbackslash{}FPifint}}\{\textbackslash{}x\}~~~~~...~\textcolor{R}{\textbf{\textbackslash{}else}}~...~\textcolor{R}{\textbf{\textbackslash{}fi}}~~~\textcolor{G}{\textbf{\%~ist~x~eine~ganze~Zahl?}}}%
}%
\endgroup


% --------------------------------------------
\subsection{Beispiel: Höhen- und Kathetensatz}
% --------------------------------------------
\begin{center}
  \begin{tikzpicture}
    % Seitenlaengen
    \FPset{\a}{3}           % a = 3
    \FPset{\b}{4}           % b = 4
    \FPset{\c}{5}           % c = 5
    % Kathetensatz: p
    \FPmul{\p}{\a}{\a}      % p = a * a
    \FPdiv{\p}{\p}{\c}      % p = p / c
    % Kathetensatz: q
    \FPmul{\q}{\b}{\b}      % q = b * b
    \FPdiv{\q}{\q}{\c}      % q = q / c
    % Hoehensatz: h
    \FPmul{\h}{\q}{\p}      % h = p * q
    \FProot{\h}{\h}{2}      % h = 2-te wurzel aus h
    % Koordinaten
    \coordinate (A) at ( 0,  0);
    \coordinate (B) at (\c,  0);
    \coordinate (C) at (\q, \h);
    % Dreieck zeichnen
    \draw[line width=1pt]
         (A) -- node[below]       {$c$}
         (B) -- node[above right] {$a$}
         (C) -- node[above left]  {$b$}
         (A);
    % Hoehe zeichnen
    \draw[line width=0.75pt]
         (C) -- node[below right] {$h$} (\q, 0);
    % Punkte zeichnen
    \fill (A) circle[radius=1pt] node[below left]  {$A$};
    \fill (B) circle[radius=1pt] node[below right] {$B$};
    \fill (C) circle[radius=1pt] node[above=3pt]   {$C$};
  \end{tikzpicture}
\end{center}
% parcolor version 2011-09-30
\begingroup
\ttfamily
\definecolor{R}{named}{Red}
\definecolor{G}{named}{ForestGreen}
\definecolor{B}{named}{RoyalBlue}
\definecolor{C}{named}{Cyan}
\definecolor{M}{named}{Magenta}
\definecolor{Y}{named}{YellowOrange}
\definecolor{background}{rgb}{0.82, 0.82, 0.92}
\dimen255=\textwidth
\advance\dimen255 by -2\fboxsep
\noindent
\colorbox{background}
{%
\parbox{\dimen255}
{%
\rule[-0.5ex]{0pt}{2.5ex}\hspace*{0.0em}\textbackslash{}begin\{tikzpicture\}\\
\rule[-0.5ex]{0pt}{2.5ex}\hspace*{1.0em}\textcolor{G}{\textbf{\%~Seitenlaengen}}\\
\rule[-0.5ex]{0pt}{2.5ex}\hspace*{1.0em}\textcolor{R}{\textbf{\textbackslash{}FPset}}\{\textcolor{B}{\textbf{\textbackslash{}a}}\}\{3\}~~~~~~~~~~~\textcolor{G}{\textbf{\%~a~=~3}}\\
\rule[-0.5ex]{0pt}{2.5ex}\hspace*{1.0em}\textcolor{R}{\textbf{\textbackslash{}FPset}}\{\textcolor{B}{\textbf{\textbackslash{}b}}\}\{4\}~~~~~~~~~~~\textcolor{G}{\textbf{\%~b~=~4}}\\
\rule[-0.5ex]{0pt}{2.5ex}\hspace*{1.0em}\textcolor{R}{\textbf{\textbackslash{}FPset}}\{\textcolor{B}{\textbf{\textbackslash{}c}}\}\{5\}~~~~~~~~~~~\textcolor{G}{\textbf{\%~c~=~5}}\\
\rule[-0.5ex]{0pt}{2.5ex}\hspace*{1.0em}\textcolor{G}{\textbf{\%~Kathetensatz:~p}}\\
\rule[-0.5ex]{0pt}{2.5ex}\hspace*{1.0em}\textcolor{R}{\textbf{\textbackslash{}FPmul}}\{\textcolor{B}{\textbf{\textbackslash{}p}}\}\{\textbackslash{}a\}\{\textbackslash{}a\}~~~~~~\textcolor{G}{\textbf{\%~p~=~a~*~a}}\\
\rule[-0.5ex]{0pt}{2.5ex}\hspace*{1.0em}\textcolor{R}{\textbf{\textbackslash{}FPdiv}}\{\textcolor{B}{\textbf{\textbackslash{}p}}\}\{\textbackslash{}p\}\{\textbackslash{}c\}~~~~~~\textcolor{G}{\textbf{\%~p~=~p~/~c}}\\
\rule[-0.5ex]{0pt}{2.5ex}\hspace*{1.0em}\textcolor{G}{\textbf{\%~Kathetensatz:~q}}\\
\rule[-0.5ex]{0pt}{2.5ex}\hspace*{1.0em}\textcolor{R}{\textbf{\textbackslash{}FPmul}}\{\textcolor{B}{\textbf{\textbackslash{}q}}\}\{\textbackslash{}b\}\{\textbackslash{}b\}~~~~~~\textcolor{G}{\textbf{\%~q~=~b~*~b}}\\
\rule[-0.5ex]{0pt}{2.5ex}\hspace*{1.0em}\textcolor{R}{\textbf{\textbackslash{}FPdiv}}\{\textcolor{B}{\textbf{\textbackslash{}q}}\}\{\textbackslash{}q\}\{\textbackslash{}c\}~~~~~~\textcolor{G}{\textbf{\%~q~=~q~/~c}}\\
\rule[-0.5ex]{0pt}{2.5ex}\hspace*{1.0em}\textcolor{G}{\textbf{\%~Hoehensatz:~h}}\\
\rule[-0.5ex]{0pt}{2.5ex}\hspace*{1.0em}\textcolor{R}{\textbf{\textbackslash{}FPmul}}\{\textcolor{B}{\textbf{\textbackslash{}h}}\}\{\textbackslash{}q\}\{\textbackslash{}p\}~~~~~~\textcolor{G}{\textbf{\%~h~=~p~*~q}}\\
\rule[-0.5ex]{0pt}{2.5ex}\hspace*{1.0em}\textcolor{R}{\textbf{\textbackslash{}FProot}}\{\textcolor{B}{\textbf{\textbackslash{}h}}\}\{\textbackslash{}h\}\{2\}~~~~~~\textcolor{G}{\textbf{\%~h~=~2{-}te~wurzel~aus~h}}\\
\rule[-0.5ex]{0pt}{2.5ex}\hspace*{1.0em}\textcolor{G}{\textbf{\%~Koordinaten}}\\
\rule[-0.5ex]{0pt}{2.5ex}\hspace*{1.0em}\textbackslash{}coordinate~(A)~at~(~0,~~0);\\
\rule[-0.5ex]{0pt}{2.5ex}\hspace*{1.0em}\textbackslash{}coordinate~(B)~at~(\textcolor{B}{\textbf{\textbackslash{}c}},~~0);\\
\rule[-0.5ex]{0pt}{2.5ex}\hspace*{1.0em}\textbackslash{}coordinate~(C)~at~(\textcolor{B}{\textbf{\textbackslash{}q}},~\textcolor{B}{\textbf{\textbackslash{}h}});\\
\rule[-0.5ex]{0pt}{2.5ex}\hspace*{1.0em}\textcolor{G}{\textbf{\%~Dreieck~zeichnen}}\\
\rule[-0.5ex]{0pt}{2.5ex}\hspace*{1.0em}\textbackslash{}draw[line~width=1pt]\\
\rule[-0.5ex]{0pt}{2.5ex}\hspace*{3.5em}(A)~{-}{-}~node[below]~~~~~~~\{\$c\$\}\\
\rule[-0.5ex]{0pt}{2.5ex}\hspace*{3.5em}(B)~{-}{-}~node[above~right]~\{\$a\$\}\\
\rule[-0.5ex]{0pt}{2.5ex}\hspace*{3.5em}(C)~{-}{-}~node[above~left]~~\{\$b\$\}\\
\rule[-0.5ex]{0pt}{2.5ex}\hspace*{3.5em}(A);\\
\rule[-0.5ex]{0pt}{2.5ex}\hspace*{1.0em}\textcolor{G}{\textbf{\%~Hoehe~zeichnen}}\\
\rule[-0.5ex]{0pt}{2.5ex}\hspace*{1.0em}\textbackslash{}draw[line~width=0.75pt]\\
\rule[-0.5ex]{0pt}{2.5ex}\hspace*{3.5em}(C)~{-}{-}~node[below~right]~\{\$h\$\}~(\textcolor{B}{\textbf{\textbackslash{}q}},~0);\\
\rule[-0.5ex]{0pt}{2.5ex}\hspace*{1.0em}\textcolor{G}{\textbf{\%~Punkte~zeichnen}}\\
\rule[-0.5ex]{0pt}{2.5ex}\hspace*{1.0em}\textbackslash{}fill~(A)~circle[radius=1pt]~node[below~left]~~\{\$A\$\};\\
\rule[-0.5ex]{0pt}{2.5ex}\hspace*{1.0em}\textbackslash{}fill~(B)~circle[radius=1pt]~node[below~right]~\{\$B\$\};\\
\rule[-0.5ex]{0pt}{2.5ex}\hspace*{1.0em}\textbackslash{}fill~(C)~circle[radius=1pt]~node[above=3pt]~~~\{\$C\$\};\\
\rule[-0.5ex]{0pt}{2.5ex}\hspace*{0.0em}\textbackslash{}end\{tikzpicture\}}%
}%
\endgroup


% ---------------------
\section{Eigene Makros}
% ---------------------

% --------------------------------------------------
\subsection{Abstand zwischen zwei Punkten berechnen}
% --------------------------------------------------
\begin{footnotesize}
% parcolor version 2011-09-30
\begingroup
\ttfamily
\definecolor{R}{named}{Red}
\definecolor{G}{named}{ForestGreen}
\definecolor{B}{named}{RoyalBlue}
\definecolor{C}{named}{Cyan}
\definecolor{M}{named}{Magenta}
\definecolor{Y}{named}{YellowOrange}
\definecolor{background}{rgb}{0.82, 0.82, 0.92}
\dimen255=\textwidth
\advance\dimen255 by -2\fboxsep
\noindent
\colorbox{background}
{%
\parbox{\dimen255}
{%
\rule[-0.5ex]{0pt}{2.5ex}\hspace*{0.0em}\textcolor{G}{\textbf{\%~{-}{-}{-}{-}{-}{-}{-}{-}}}\\
\rule[-0.5ex]{0pt}{2.5ex}\hspace*{0.0em}\textcolor{G}{\textbf{\%~shapedst}}\\
\rule[-0.5ex]{0pt}{2.5ex}\hspace*{0.0em}\textcolor{G}{\textbf{\%~{-}{-}{-}{-}{-}{-}{-}{-}}}\\
\rule[-0.5ex]{0pt}{2.5ex}\hspace*{0.0em}\textcolor{G}{\textbf{\%}}\\
\rule[-0.5ex]{0pt}{2.5ex}\hspace*{0.0em}\textcolor{G}{\textbf{\%~~\textbackslash{}shapedst\{A\}\{B\}\{\textbackslash{}mydistance\}}}\\
\rule[-0.5ex]{0pt}{2.5ex}\hspace*{0.0em}\textcolor{G}{\textbf{\%}}\\
\rule[-0.5ex]{0pt}{2.5ex}\hspace*{0.0em}\textbackslash{}newcommand\{\textcolor{R}{\textbf{\textbackslash{}shapedst}}\}[3]\\
\rule[-0.5ex]{0pt}{2.5ex}\hspace*{0.0em}\{\textcolor{G}{\textbf{\%}}\\
\rule[-0.5ex]{0pt}{2.5ex}\hspace*{1.0em}\textcolor{G}{\textbf{\%~define~new~macro~if~missing}}\\
\rule[-0.5ex]{0pt}{2.5ex}\hspace*{1.0em}\textbackslash{}ifthenelse\{\textbackslash{}isundefined\{\#3\}\}\{\textbackslash{}def\#3\{\textbackslash{}relax\}\}\{\textbackslash{}relax\}\textcolor{G}{\textbf{\%}}\\
\rule[-0.5ex]{0pt}{2.5ex}\hspace*{1.0em}\textcolor{G}{\textbf{\%~get~x{-}coordinate~from~vector}}\\
\rule[-0.5ex]{0pt}{2.5ex}\hspace*{1.0em}\textbackslash{}pgfextractx\{\textbackslash{}dimen0\}\{\textbackslash{}pgfpointdiff\{\textbackslash{}pgfpointanchor\{\#1\}\{center\}\}\textcolor{G}{\textbf{\%}}\\
\rule[-0.5ex]{0pt}{2.5ex}\hspace*{18.5em}\{\textbackslash{}pgfpointanchor\{\#2\}\{center\}\}\}\textcolor{G}{\textbf{\%}}\\
\rule[-0.5ex]{0pt}{2.5ex}\hspace*{1.0em}\textcolor{G}{\textbf{\%~get~y{-}coordinate~from~vector}}\\
\rule[-0.5ex]{0pt}{2.5ex}\hspace*{1.0em}\textbackslash{}pgfextracty\{\textbackslash{}dimen1\}\{\textbackslash{}pgfpointdiff\{\textbackslash{}pgfpointanchor\{\#1\}\{center\}\}\textcolor{G}{\textbf{\%}}\\
\rule[-0.5ex]{0pt}{2.5ex}\hspace*{18.5em}\{\textbackslash{}pgfpointanchor\{\#2\}\{center\}\}\}\textcolor{G}{\textbf{\%}}\\
\rule[-0.5ex]{0pt}{2.5ex}\hspace*{1.0em}\textcolor{G}{\textbf{\%~calculate~length}}\\
\rule[-0.5ex]{0pt}{2.5ex}\hspace*{1.0em}\textbackslash{}pgfmathsetmacro\{\#3\}\{veclen(\textbackslash{}the\textbackslash{}dimen0,\textbackslash{}the\textbackslash{}dimen1)\}\textcolor{G}{\textbf{\%}}\\
\rule[-0.5ex]{0pt}{2.5ex}\hspace*{1.0em}\textcolor{G}{\textbf{\%~add~unit~'pt'}}\\
\rule[-0.5ex]{0pt}{2.5ex}\hspace*{1.0em}\textbackslash{}edef\#3\{\#3pt\}\textcolor{G}{\textbf{\%}}\\
\rule[-0.5ex]{0pt}{2.5ex}\hspace*{0.0em}\}}%
}%
\endgroup

\end{footnotesize}

% --------------------------------------------------------------
\subsection{Richtung von einem zu einem anderen Punkt berechnen}
% --------------------------------------------------------------
\begin{footnotesize}
% parcolor version 2011-09-30
\begingroup
\ttfamily
\definecolor{R}{named}{Red}
\definecolor{G}{named}{ForestGreen}
\definecolor{B}{named}{RoyalBlue}
\definecolor{C}{named}{Cyan}
\definecolor{M}{named}{Magenta}
\definecolor{Y}{named}{YellowOrange}
\definecolor{background}{rgb}{0.82, 0.82, 0.92}
\dimen255=\textwidth
\advance\dimen255 by -2\fboxsep
\noindent
\colorbox{background}
{%
\parbox{\dimen255}
{%
\rule[-0.5ex]{0pt}{2.5ex}\hspace*{0.0em}\textcolor{G}{\textbf{\%~{-}{-}{-}{-}{-}{-}{-}{-}}}\\
\rule[-0.5ex]{0pt}{2.5ex}\hspace*{0.0em}\textcolor{G}{\textbf{\%~shapedir}}\\
\rule[-0.5ex]{0pt}{2.5ex}\hspace*{0.0em}\textcolor{G}{\textbf{\%~{-}{-}{-}{-}{-}{-}{-}{-}}}\\
\rule[-0.5ex]{0pt}{2.5ex}\hspace*{0.0em}\textcolor{G}{\textbf{\%}}\\
\rule[-0.5ex]{0pt}{2.5ex}\hspace*{0.0em}\textcolor{G}{\textbf{\%~\textbackslash{}shapedir\{A\}\{B\}\{\textbackslash{}mydirection\}}}\\
\rule[-0.5ex]{0pt}{2.5ex}\hspace*{0.0em}\textcolor{G}{\textbf{\%}}\\
\rule[-0.5ex]{0pt}{2.5ex}\hspace*{0.0em}\textbackslash{}newcommand\{\textcolor{R}{\textbf{\textbackslash{}shapedir}}\}[3]\\
\rule[-0.5ex]{0pt}{2.5ex}\hspace*{0.0em}\{\textcolor{G}{\textbf{\%}}\\
\rule[-0.5ex]{0pt}{2.5ex}\hspace*{1.0em}\textcolor{G}{\textbf{\%~define~new~macro~if~missing}}\\
\rule[-0.5ex]{0pt}{2.5ex}\hspace*{1.0em}\textbackslash{}ifthenelse\{\textbackslash{}isundefined\{\#3\}\}\{\textbackslash{}def\#3\{\textbackslash{}relax\}\}\{\textbackslash{}relax\}\textcolor{G}{\textbf{\%}}\\
\rule[-0.5ex]{0pt}{2.5ex}\hspace*{1.0em}\textcolor{G}{\textbf{\%~get~x{-}coordinate~from~vector}}\\
\rule[-0.5ex]{0pt}{2.5ex}\hspace*{1.0em}\textbackslash{}pgfextractx\{\textbackslash{}dimen0\}\{\textbackslash{}pgfpointdiff\{\textbackslash{}pgfpointanchor\{\#1\}\{center\}\}\textcolor{G}{\textbf{\%}}\\
\rule[-0.5ex]{0pt}{2.5ex}\hspace*{18.5em}\{\textbackslash{}pgfpointanchor\{\#2\}\{center\}\}\}\textcolor{G}{\textbf{\%}}\\
\rule[-0.5ex]{0pt}{2.5ex}\hspace*{1.0em}\textcolor{G}{\textbf{\%~get~y{-}coordinate~from~vector}}\\
\rule[-0.5ex]{0pt}{2.5ex}\hspace*{1.0em}\textbackslash{}pgfextracty\{\textbackslash{}dimen1\}\{\textbackslash{}pgfpointdiff\{\textbackslash{}pgfpointanchor\{\#1\}\{center\}\}\textcolor{G}{\textbf{\%}}\\
\rule[-0.5ex]{0pt}{2.5ex}\hspace*{18.5em}\{\textbackslash{}pgfpointanchor\{\#2\}\{center\}\}\}\textcolor{G}{\textbf{\%}}\\
\rule[-0.5ex]{0pt}{2.5ex}\hspace*{1.0em}\textcolor{G}{\textbf{\%~arctangent~of~y/x~in~degrees}}\\
\rule[-0.5ex]{0pt}{2.5ex}\hspace*{1.0em}\textcolor{G}{\textbf{\%~this~also~takes~into~account~the~quadrants}}\\
\rule[-0.5ex]{0pt}{2.5ex}\hspace*{1.0em}\textbackslash{}pgfmathsetmacro\{\#3\}\{atan2(\textbackslash{}the\textbackslash{}dimen1,\textbackslash{}the\textbackslash{}dimen0)\}\textcolor{G}{\textbf{\%}}\\
\rule[-0.5ex]{0pt}{2.5ex}\hspace*{0.0em}\}}%
}%
\endgroup

\end{footnotesize}

% -------------------------------------------------------------
\subsection{Beispiel: Die Möndchen des Hippokrates (von Chios)}
% -------------------------------------------------------------
\begin{center}
  \newcommand{\moendchen}[2]
  {
    \begin{scope}
      % Koordinaten der Eckpunkte
      \coordinate (A) at (-#1, 0);
      \coordinate (B) at ( #1, 0);
      \coordinate (C) at ($(0, 0)!#1!#2:(B)$);
      % Abstand und Richtung von B nach C
      \shapedir{B}{C}{\dirBC}
      \shapedst{B}{C}{\dstBC}
      % Abstand und Richtung von C nach A
      \shapedir{C}{A}{\dirCA}
      \shapedst{C}{A}{\dstCA}
      % Dreieck
      \filldraw[fill=OliveGreen] (A) -- (B) -- (C) -- cycle;
      % Moendchen ueber a
      \filldraw[fill=LimeGreen]
               (B) arc[start angle=0, end angle=#2, radius=#1]
                   arc[start angle=\dirBC, delta angle=-180, radius=\dstBC/2];
      % Moendchen ueber b
      \filldraw[fill=LimeGreen]
               (C) arc[start angle=#2, end angle=180, radius=#1]
                   arc[start angle=\dirCA, delta angle=-180, radius=\dstCA/2];
      % Eckpunkte
      \fill (A) circle[radius=1pt];
      \fill (B) circle[radius=1pt];
      \fill (C) circle[radius=1pt];
    \end{scope}
  }
  \begin{tikzpicture}
    \moendchen{20mm}{120}
    \begin{scope}[xshift=7cm]
      \moendchen{25mm}{80}
    \end{scope}
  \end{tikzpicture}
\end{center}
\begin{footnotesize}
% parcolor version 2011-09-30
\begingroup
\ttfamily
\definecolor{R}{named}{Red}
\definecolor{G}{named}{ForestGreen}
\definecolor{B}{named}{RoyalBlue}
\definecolor{C}{named}{Cyan}
\definecolor{M}{named}{Magenta}
\definecolor{Y}{named}{YellowOrange}
\definecolor{background}{rgb}{0.82, 0.82, 0.92}
\dimen255=\textwidth
\advance\dimen255 by -2\fboxsep
\noindent
\colorbox{background}
{%
\parbox{\dimen255}
{%
\rule[-0.5ex]{0pt}{2.5ex}\hspace*{0.0em}\textbackslash{}newcommand\{\textcolor{R}{\textbf{\textbackslash{}moendchen}}\}[2]\\
\rule[-0.5ex]{0pt}{2.5ex}\hspace*{0.0em}\{\\
\rule[-0.5ex]{0pt}{2.5ex}\hspace*{1.0em}\textbackslash{}begin\{scope\}\\
\rule[-0.5ex]{0pt}{2.5ex}\hspace*{2.0em}\textcolor{G}{\textbf{\%~Koordinaten~der~Eckpunkte}}\\
\rule[-0.5ex]{0pt}{2.5ex}\hspace*{2.0em}\textbackslash{}coordinate~(A)~at~({-}\#1,~0);\\
\rule[-0.5ex]{0pt}{2.5ex}\hspace*{2.0em}\textbackslash{}coordinate~(B)~at~(~\#1,~0);\\
\rule[-0.5ex]{0pt}{2.5ex}\hspace*{2.0em}\textbackslash{}coordinate~(C)~at~(\$(0,~0)!\#1!\#2:(B)\$);\\
\rule[-0.5ex]{0pt}{2.5ex}\hspace*{2.0em}\textcolor{G}{\textbf{\%~Abstand~und~Richtung~von~B~nach~C}}\\
\rule[-0.5ex]{0pt}{2.5ex}\hspace*{2.0em}\textcolor{R}{\textbf{\textbackslash{}shapedir}}\{B\}\{C\}\{\textbackslash{}dirBC\}\\
\rule[-0.5ex]{0pt}{2.5ex}\hspace*{2.0em}\textcolor{R}{\textbf{\textbackslash{}shapedst}}\{B\}\{C\}\{\textbackslash{}dstBC\}\\
\rule[-0.5ex]{0pt}{2.5ex}\hspace*{2.0em}\textcolor{G}{\textbf{\%~Abstand~und~Richtung~von~C~nach~A}}\\
\rule[-0.5ex]{0pt}{2.5ex}\hspace*{2.0em}\textcolor{R}{\textbf{\textbackslash{}shapedir}}\{C\}\{A\}\{\textbackslash{}dirCA\}\\
\rule[-0.5ex]{0pt}{2.5ex}\hspace*{2.0em}\textcolor{R}{\textbf{\textbackslash{}shapedst}}\{C\}\{A\}\{\textbackslash{}dstCA\}\\
\rule[-0.5ex]{0pt}{2.5ex}\hspace*{2.0em}\textcolor{G}{\textbf{\%~Dreieck}}\\
\rule[-0.5ex]{0pt}{2.5ex}\hspace*{2.0em}\textbackslash{}filldraw[fill=OliveGreen]~(A)~{-}{-}~(B)~{-}{-}~(C)~{-}{-}~cycle;\\
\rule[-0.5ex]{0pt}{2.5ex}\hspace*{2.0em}\textcolor{G}{\textbf{\%~Moendchen~ueber~a}}\\
\rule[-0.5ex]{0pt}{2.5ex}\hspace*{2.0em}\textbackslash{}filldraw[fill=LimeGreen]\\
\rule[-0.5ex]{0pt}{2.5ex}\hspace*{6.5em}(B)~arc[start~angle=0,~end~angle=\#2,~radius=\#1]\\
\rule[-0.5ex]{0pt}{2.5ex}\hspace*{8.5em}arc[start~angle=\textbackslash{}dirBC,~delta~angle={-}180,~radius=\textbackslash{}dstBC/2];\\
\rule[-0.5ex]{0pt}{2.5ex}\hspace*{2.0em}\textcolor{G}{\textbf{\%~Moendchen~ueber~b}}\\
\rule[-0.5ex]{0pt}{2.5ex}\hspace*{2.0em}\textbackslash{}filldraw[fill=LimeGreen]\\
\rule[-0.5ex]{0pt}{2.5ex}\hspace*{6.5em}(C)~arc[start~angle=\#2,~end~angle=180,~radius=\#1]\\
\rule[-0.5ex]{0pt}{2.5ex}\hspace*{8.5em}arc[start~angle=\textbackslash{}dirCA,~delta~angle={-}180,~radius=\textbackslash{}dstCA/2];\\
\rule[-0.5ex]{0pt}{2.5ex}\hspace*{2.0em}\textcolor{G}{\textbf{\%~Eckpunkte}}\\
\rule[-0.5ex]{0pt}{2.5ex}\hspace*{2.0em}\textbackslash{}fill~(A)~circle[radius=1pt];\\
\rule[-0.5ex]{0pt}{2.5ex}\hspace*{2.0em}\textbackslash{}fill~(B)~circle[radius=1pt];\\
\rule[-0.5ex]{0pt}{2.5ex}\hspace*{2.0em}\textbackslash{}fill~(C)~circle[radius=1pt];\\
\rule[-0.5ex]{0pt}{2.5ex}\hspace*{1.0em}\textbackslash{}end\{scope\}\\
\rule[-0.5ex]{0pt}{2.5ex}\hspace*{0.0em}\}\\
\rule[-0.5ex]{0pt}{2.5ex}\hspace*{0.0em}\\
\rule[-0.5ex]{0pt}{2.5ex}\hspace*{0.0em}\textbackslash{}begin\{tikzpicture\}\\
\rule[-0.5ex]{0pt}{2.5ex}\hspace*{1.0em}\textcolor{R}{\textbf{\textbackslash{}moendchen}}\{20mm\}\{120\}\\
\rule[-0.5ex]{0pt}{2.5ex}\hspace*{1.0em}\textbackslash{}begin\{scope\}[xshift=7cm]\\
\rule[-0.5ex]{0pt}{2.5ex}\hspace*{2.0em}\textcolor{R}{\textbf{\textbackslash{}moendchen}}\{25mm\}\{80\}\\
\rule[-0.5ex]{0pt}{2.5ex}\hspace*{1.0em}\textbackslash{}end\{scope\}\\
\rule[-0.5ex]{0pt}{2.5ex}\hspace*{0.0em}\textbackslash{}end\{tikzpicture\}}%
}%
\endgroup

\end{footnotesize}

% ---------------------------------------------------------------
\subsection{Beschriftung einer Strecke auf der Mittelsenkrechten}
% ---------------------------------------------------------------
\begin{footnotesize}
% parcolor version 2011-09-30
\begingroup
\ttfamily
\definecolor{R}{named}{Red}
\definecolor{G}{named}{ForestGreen}
\definecolor{B}{named}{RoyalBlue}
\definecolor{C}{named}{Cyan}
\definecolor{M}{named}{Magenta}
\definecolor{Y}{named}{YellowOrange}
\definecolor{background}{rgb}{0.82, 0.82, 0.92}
\dimen255=\textwidth
\advance\dimen255 by -2\fboxsep
\noindent
\colorbox{background}
{%
\parbox{\dimen255}
{%
\rule[-0.5ex]{0pt}{2.5ex}\hspace*{0.0em}\textcolor{G}{\textbf{\%~{-}{-}{-}{-}{-}{-}}}\\
\rule[-0.5ex]{0pt}{2.5ex}\hspace*{0.0em}\textcolor{G}{\textbf{\%~clnode}}\\
\rule[-0.5ex]{0pt}{2.5ex}\hspace*{0.0em}\textcolor{G}{\textbf{\%~{-}{-}{-}{-}{-}{-}}}\\
\rule[-0.5ex]{0pt}{2.5ex}\hspace*{0.0em}\textcolor{G}{\textbf{\%}}\\
\rule[-0.5ex]{0pt}{2.5ex}\hspace*{0.0em}\textcolor{G}{\textbf{\%~\textbackslash{}clnode\{B\}\{C\}\{2mm\}\{\$a\$\}}}\\
\rule[-0.5ex]{0pt}{2.5ex}\hspace*{0.0em}\textcolor{G}{\textbf{\%}}\\
\rule[-0.5ex]{0pt}{2.5ex}\hspace*{0.0em}\textbackslash{}newcommand\{\textcolor{R}{\textbf{\textbackslash{}clnode}}\}[4]\\
\rule[-0.5ex]{0pt}{2.5ex}\hspace*{0.0em}\{\\
\rule[-0.5ex]{0pt}{2.5ex}\hspace*{1.0em}\textbackslash{}node~at~(\$(\#1)!0.5!(\#2)!\#3!270:(\#2)\$)~\{\#4\};\\
\rule[-0.5ex]{0pt}{2.5ex}\hspace*{0.0em}\}}%
}%
\endgroup

\end{footnotesize}

% ----------------------------------------------------------------
\subsection{Beschriftung eines Winkels auf der Winkelhalbierenden}
% ----------------------------------------------------------------
\begin{footnotesize}
% parcolor version 2011-09-30
\begingroup
\ttfamily
\definecolor{R}{named}{Red}
\definecolor{G}{named}{ForestGreen}
\definecolor{B}{named}{RoyalBlue}
\definecolor{C}{named}{Cyan}
\definecolor{M}{named}{Magenta}
\definecolor{Y}{named}{YellowOrange}
\definecolor{background}{rgb}{0.82, 0.82, 0.92}
\dimen255=\textwidth
\advance\dimen255 by -2\fboxsep
\noindent
\colorbox{background}
{%
\parbox{\dimen255}
{%
\rule[-0.5ex]{0pt}{2.5ex}\hspace*{0.0em}\textcolor{G}{\textbf{\%~{-}{-}{-}{-}{-}{-}}}\\
\rule[-0.5ex]{0pt}{2.5ex}\hspace*{0.0em}\textcolor{G}{\textbf{\%~canode}}\\
\rule[-0.5ex]{0pt}{2.5ex}\hspace*{0.0em}\textcolor{G}{\textbf{\%~{-}{-}{-}{-}{-}{-}}}\\
\rule[-0.5ex]{0pt}{2.5ex}\hspace*{0.0em}\textcolor{G}{\textbf{\%}}\\
\rule[-0.5ex]{0pt}{2.5ex}\hspace*{0.0em}\textcolor{G}{\textbf{\%~\textbackslash{}canode\{B\}\{A\}\{C\}\{4mm\}\{7mm\}\{\$\textbackslash{}alpha\$\}}}\\
\rule[-0.5ex]{0pt}{2.5ex}\hspace*{0.0em}\textcolor{G}{\textbf{\%}}\\
\rule[-0.5ex]{0pt}{2.5ex}\hspace*{0.0em}\textbackslash{}newcommand\{\textcolor{R}{\textbf{\textbackslash{}canode}}\}[6]\\
\rule[-0.5ex]{0pt}{2.5ex}\hspace*{0.0em}\{\\
\rule[-0.5ex]{0pt}{2.5ex}\hspace*{1.0em}\textbackslash{}begin\{scope\}\\
\rule[-0.5ex]{0pt}{2.5ex}\hspace*{2.0em}\textbackslash{}begin\{scope\}\\
\rule[-0.5ex]{0pt}{2.5ex}\hspace*{3.0em}\textcolor{G}{\textbf{\%~der~Kreisbogen}}\\
\rule[-0.5ex]{0pt}{2.5ex}\hspace*{3.0em}\textbackslash{}clip~(\#1)~{-}{-}~(\#2)~{-}{-}~(\#3)~{-}{-}~cycle;\\
\rule[-0.5ex]{0pt}{2.5ex}\hspace*{3.0em}\textbackslash{}draw~(\#2)~circle[radius=\#5];\\
\rule[-0.5ex]{0pt}{2.5ex}\hspace*{2.0em}\textbackslash{}end\{scope\}\\
\rule[-0.5ex]{0pt}{2.5ex}\hspace*{2.0em}\textbackslash{}coordinate~(tempnodeA)~at~(\$(\#2)!\#4!(\#1)\$);\\
\rule[-0.5ex]{0pt}{2.5ex}\hspace*{2.0em}\textbackslash{}coordinate~(tempnodeB)~at~(\$(\#2)!\#4!(\#3)\$);\\
\rule[-0.5ex]{0pt}{2.5ex}\hspace*{2.0em}\textbackslash{}coordinate~(tempnodeC)~at~(\$(tempnodeA)!0.5!(tempnodeB)\$);\\
\rule[-0.5ex]{0pt}{2.5ex}\hspace*{2.0em}\textbackslash{}node~at~(\$(\#2)!\#4!(tempnodeC)\$)~\{\#6\};\\
\rule[-0.5ex]{0pt}{2.5ex}\hspace*{1.0em}\textbackslash{}end\{scope\}\\
\rule[-0.5ex]{0pt}{2.5ex}\hspace*{0.0em}\}}%
}%
\endgroup

\end{footnotesize}

% --------------------------------------
\subsection{Das Paket \texttt{geometry}}
% --------------------------------------
\begin{footnotesize}
% parcolor version 2011-09-30
\begingroup
\ttfamily
\definecolor{R}{named}{Red}
\definecolor{G}{named}{ForestGreen}
\definecolor{B}{named}{RoyalBlue}
\definecolor{C}{named}{Cyan}
\definecolor{M}{named}{Magenta}
\definecolor{Y}{named}{YellowOrange}
\definecolor{background}{rgb}{0.82, 0.82, 0.92}
\dimen255=\textwidth
\advance\dimen255 by -2\fboxsep
\noindent
\colorbox{background}
{%
\parbox{\dimen255}
{%
\rule[-0.5ex]{0pt}{2.5ex}\hspace*{0.0em}\textcolor{G}{\textbf{\%~genaue~Kontrolle~ueber~die~Groesse~der~Seite~und~die~Breite~der~Raender}}\\
\rule[-0.5ex]{0pt}{2.5ex}\hspace*{0.0em}\textcolor{R}{\textbf{\textbackslash{}usepackage}}\\
\rule[-0.5ex]{0pt}{2.5ex}\hspace*{0.0em}[\\
\rule[-0.5ex]{0pt}{2.5ex}\hspace*{1.0em}\textcolor{R}{\textbf{paperwidth}}~~=~76.3pt,~\textcolor{G}{\textbf{\%~Hoehe~der~Seite}}\\
\rule[-0.5ex]{0pt}{2.5ex}\hspace*{1.0em}\textcolor{R}{\textbf{paperheight}}~=~87.2pt,~\textcolor{G}{\textbf{\%~Breite~der~Seite}}\\
\rule[-0.5ex]{0pt}{2.5ex}\hspace*{1.0em}\textcolor{R}{\textbf{top}}~~~~~~~~~=~0pt,~~~~\textcolor{G}{\textbf{\%~Rand~oben}}\\
\rule[-0.5ex]{0pt}{2.5ex}\hspace*{1.0em}\textcolor{R}{\textbf{left}}~~~~~~~~=~0pt,~~~~\textcolor{G}{\textbf{\%~Rand~links}}\\
\rule[-0.5ex]{0pt}{2.5ex}\hspace*{1.0em}\textcolor{R}{\textbf{right}}~~~~~~~=~0pt,~~~~\textcolor{G}{\textbf{\%~Rand~rechts}}\\
\rule[-0.5ex]{0pt}{2.5ex}\hspace*{1.0em}\textcolor{R}{\textbf{bottom}}~~~~~~=~0pt~~~~~\textcolor{G}{\textbf{\%~Rand~unten}}\\
\rule[-0.5ex]{0pt}{2.5ex}\hspace*{0.0em}]\\
\rule[-0.5ex]{0pt}{2.5ex}\hspace*{0.0em}\{\textcolor{R}{\textbf{geometry}}\}}%
}%
\endgroup

\end{footnotesize}

% ----------------------------------------------------
\subsection{Breite und Höhe einer Zeichnung ermitteln}
% ----------------------------------------------------
\begin{footnotesize}
% parcolor version 2011-09-30
\begingroup
\ttfamily
\definecolor{R}{named}{Red}
\definecolor{G}{named}{ForestGreen}
\definecolor{B}{named}{RoyalBlue}
\definecolor{C}{named}{Cyan}
\definecolor{M}{named}{Magenta}
\definecolor{Y}{named}{YellowOrange}
\definecolor{background}{rgb}{0.82, 0.82, 0.92}
\dimen255=\textwidth
\advance\dimen255 by -2\fboxsep
\noindent
\colorbox{background}
{%
\parbox{\dimen255}
{%
\rule[-0.5ex]{0pt}{2.5ex}\hspace*{0.0em}\textcolor{G}{\textbf{\%~{-}{-}{-}{-}{-}{-}\\
\rule[-0.5ex]{0pt}{2.5ex}\hspace*{0.0em}\%~sizeof\\
\rule[-0.5ex]{0pt}{2.5ex}\hspace*{0.0em}\%~{-}{-}{-}{-}{-}{-}\\
\rule[-0.5ex]{0pt}{2.5ex}\hspace*{0.0em}\%\\
\rule[-0.5ex]{0pt}{2.5ex}\hspace*{0.0em}\%~\textbackslash{}sizeof\{\%\\
\rule[-0.5ex]{0pt}{2.5ex}\hspace*{0.0em}\%~\textbackslash{}begin\{tikzpicture\}\\
\rule[-0.5ex]{0pt}{2.5ex}\hspace*{0.0em}\%~~~...\\
\rule[-0.5ex]{0pt}{2.5ex}\hspace*{0.0em}\%~\textbackslash{}end\{tikzpicture\}\}\\
\rule[-0.5ex]{0pt}{2.5ex}\hspace*{0.0em}\%}}\\
\rule[-0.5ex]{0pt}{2.5ex}\hspace*{0.0em}\textbackslash{}newcommand\{\textcolor{R}{\textbf{\textbackslash{}sizeof}}\}[1]\\
\rule[-0.5ex]{0pt}{2.5ex}\hspace*{0.0em}\{\textcolor{G}{\textbf{\%}}\\
\rule[-0.5ex]{0pt}{2.5ex}\hspace*{1.0em}\textbackslash{}begingroup\\
\rule[-0.5ex]{0pt}{2.5ex}\hspace*{2.0em}\textbackslash{}setbox0=\textbackslash{}hbox\{\#1\}\textcolor{G}{\textbf{\%}}\\
\rule[-0.5ex]{0pt}{2.5ex}\hspace*{2.0em}\textbackslash{}setlength~~\{\textbackslash{}dimen0\}\{\textbackslash{}wd0\}\textcolor{G}{\textbf{\%}}\\
\rule[-0.5ex]{0pt}{2.5ex}\hspace*{2.0em}\textbackslash{}setlength~~\{\textbackslash{}dimen1\}\{\textbackslash{}ht0\}\textcolor{G}{\textbf{\%}}\\
\rule[-0.5ex]{0pt}{2.5ex}\hspace*{2.0em}\textbackslash{}addtolength\{\textbackslash{}dimen1\}\{\textbackslash{}dp0\}\textcolor{G}{\textbf{\%}}\\
\rule[-0.5ex]{0pt}{2.5ex}\hspace*{2.0em}\textbackslash{}makebox[3em][r]\{\$w=\$\textbackslash{},\}\textbackslash{}the\textbackslash{}dimen0\textbackslash{}\textbackslash{}\\
\rule[-0.5ex]{0pt}{2.5ex}\hspace*{2.0em}\textbackslash{}makebox[3em][r]\{\$h=\$\textbackslash{},\}\textbackslash{}the\textbackslash{}dimen1\textbackslash{}par\\
\rule[-0.5ex]{0pt}{2.5ex}\hspace*{1.0em}\textbackslash{}endgroup\\
\rule[-0.5ex]{0pt}{2.5ex}\hspace*{0.0em}\}}%
}%
\endgroup

\end{footnotesize}

% --------------------
\section{Maßeinheiten}
% --------------------
Alle Koordinaten müssen im Intervall $[-16\,383, 16\,383]$\,pt
liegen ($\approx\pm5,\!73$\,m).
\begin{center}
  \renewcommand{\arraystretch}{1.2}
  \newcommand{\unitbox}[3][4em]{\makebox[#1][r]{#2\,\texttt{#3}}}%
  \begin{tabular}{|c|l|c|}
    \hline
    \texttt{sp}  & scaled point  & \unitbox{65\,536}{sp} $=$       \unitbox[4em]  {1}    {pt} \\
    \texttt{pt}  & point         & \unitbox      {1}{pt} $\approx$ \unitbox[4em]  {0,35} {mm} \\
    \texttt{bp}  & big point     & \unitbox      {1}{bp} $\approx$ \unitbox[4em]  {1,004}{pt} \\
    \texttt{dd}  & didot point   & \unitbox      {1}{dd} $\approx$ \unitbox[4em]  {1,07} {pt} \\
    \texttt{mm}  & millimeter    & \unitbox      {1}{mm} $\approx$ \unitbox[4em]  {2,85} {pt} \\
    \texttt{pc}  & pica          & \unitbox      {1}{pc} $=$       \unitbox[4em] {12}    {pt} \\
    \texttt{cc}  & cicero        & \unitbox      {1}{cc} $\approx$ \unitbox[4em] {12,84} {pt} \\
    \texttt{cm}  & centimeter    & \unitbox      {1}{cm} $\approx$ \unitbox[4em] {28,45} {pt} \\
    \texttt{in}  & inch          & \unitbox      {1}{in} $\approx$ \unitbox[4em] {72,27} {pt} \\
    \hline
    \texttt{em}  & \multicolumn{2}{|l|}{Breite eines großen M}                                \\
    \texttt{ex}  & \multicolumn{2}{|l|}{Höhe eines kleinen x}                                 \\
    \hline
  \end{tabular}
\end{center}

% --------------
\section{Farben}
% --------------

% --------------------------------------------------------
\subsection{Die \texttt{dvipsnames} vordefinierter Farben}
% --------------------------------------------------------
\begin{center}
  \footnotesize
  \setlength{\tabcolsep}{1mm}
  \newcommand{\showcolor}[1]{\textcolor{#1}{\rule{7mm}{7mm}}&\raisebox{2.4mm}{#1}}
  \begin{tabular}{clclclcl}
    \showcolor{Apricot}        &
    \showcolor{Aquamarine}     &
    \showcolor{Bittersweet}    &
    \showcolor{Black}          \\
    \showcolor{Blue}           &
    \showcolor{BlueGreen}      &
    \showcolor{BlueViolet}     &
    \showcolor{BrickRed}       \\
    \showcolor{Brown}          &
    \showcolor{BurntOrange}    &
    \showcolor{CadetBlue}      &
    \showcolor{CarnationPink}  \\
    \showcolor{Cerulean}       &
    \showcolor{CornflowerBlue} &
    \showcolor{Cyan}           &
    \showcolor{Dandelion}      \\
    \showcolor{DarkOrchid}     &
    \showcolor{Emerald}        &
    \showcolor{ForestGreen}    &
    \showcolor{Fuchsia}        \\
    \showcolor{Goldenrod}      &
    \showcolor{Gray}           &
    \showcolor{Green}          &
    \showcolor{GreenYellow}    \\
    \showcolor{JungleGreen}    &
    \showcolor{Lavender}       &
    \showcolor{LimeGreen}      &
    \showcolor{Magenta}        \\
    \showcolor{Mahogany}       &
    \showcolor{Maroon}         &
    \showcolor{Melon}          &
    \showcolor{MidnightBlue}   \\
    \showcolor{Mulberry}       &
    \showcolor{NavyBlue}       &
    \showcolor{OliveGreen}     &
    \showcolor{Orange}         \\
    \showcolor{OrangeRed}      &
    \showcolor{Orchid}         &
    \showcolor{Peach}          &
    \showcolor{Periwinkle}     \\
    \showcolor{PineGreen}      &
    \showcolor{Plum}           &
    \showcolor{ProcessBlue}    &
    \showcolor{Purple}         \\
    \showcolor{RawSienna}      &
    \showcolor{Red}            &
    \showcolor{RedOrange}      &
    \showcolor{RedViolet}      \\
    \showcolor{Rhodamine}      &
    \showcolor{RoyalBlue}      &
    \showcolor{RoyalPurple}    &
    \showcolor{RubineRed}      \\
    \showcolor{Salmon}         &
    \showcolor{SeaGreen}       &
    \showcolor{Sepia}          &
    \showcolor{SkyBlue}        \\
    \showcolor{SpringGreen}    &
    \showcolor{Tan}            &
    \showcolor{TealBlue}       &
    \showcolor{Thistle}        \\
    \showcolor{Turquoise}      &
    \showcolor{Violet}         &
    \showcolor{VioletRed}      &
    \showcolor{White}          \\
    \showcolor{WildStrawberry} &
    \showcolor{Yellow}         &
    \showcolor{YellowGreen}    &
    \showcolor{YellowOrange}   \\
  \end{tabular}
\end{center}

% -----------------------------------
\subsection{Eigene Farben definieren}
% -----------------------------------
% parcolor version 2011-09-30
\begingroup
\ttfamily
\definecolor{R}{named}{Red}
\definecolor{G}{named}{ForestGreen}
\definecolor{B}{named}{RoyalBlue}
\definecolor{C}{named}{Cyan}
\definecolor{M}{named}{Magenta}
\definecolor{Y}{named}{YellowOrange}
\definecolor{background}{rgb}{0.82, 0.82, 0.92}
\dimen255=\textwidth
\advance\dimen255 by -2\fboxsep
\noindent
\colorbox{background}
{%
\parbox{\dimen255}
{%
\rule[-0.5ex]{0pt}{2.5ex}\hspace*{0.0em}\textcolor{G}{\textbf{\%~Farbe~im~RGB{-}System~definieren~[0,~255]}}\\
\rule[-0.5ex]{0pt}{2.5ex}\hspace*{0.0em}\textcolor{R}{\textbf{\textbackslash{}definecolor}}\{UniBlau\}\{\textcolor{R}{\textbf{RGB}}\}\{3,~3,~133\}\\
\rule[-0.5ex]{0pt}{2.5ex}\hspace*{0.0em}\\
\rule[-0.5ex]{0pt}{2.5ex}\hspace*{0.0em}\textcolor{G}{\textbf{\%~Mischung~aus~80\%~LimeGreen~und~20\%~Cyan}}\\
\rule[-0.5ex]{0pt}{2.5ex}\hspace*{0.0em}\textcolor{R}{\textbf{\textbackslash{}colorlet}}\{LimeGreenCyan\}\{LimeGreen!80!Cyan\}}%
}%
\endgroup


% --------------
\section{Octave}
% --------------
Octave dient zur numerischen Berechnung von Funktionswerten. Sobald man
den Octave-Code in \emph{<Dateiname.m>} gespeichert hat, kann man ihn
wie folgt ausführen:\par
\begin{footnotesize}
\input{octavecall.code}
\end{footnotesize}

% -----------------------------------------------------------
\subsection{Die Wertepaare für \texttt{plot} direkt ausgeben}
% -----------------------------------------------------------
\begin{footnotesize}
% parcolor version 2011-09-30
\begingroup
\ttfamily
\definecolor{R}{named}{Red}
\definecolor{G}{named}{ForestGreen}
\definecolor{B}{named}{RoyalBlue}
\definecolor{C}{named}{Cyan}
\definecolor{M}{named}{Magenta}
\definecolor{Y}{named}{YellowOrange}
\definecolor{background}{rgb}{0.82, 0.82, 0.92}
\dimen255=\textwidth
\advance\dimen255 by -2\fboxsep
\noindent
\colorbox{background}
{%
\parbox{\dimen255}
{%
\rule[-0.5ex]{0pt}{2.5ex}\hspace*{0.0em}\textcolor{G}{\textbf{\%~LaTeX{-}Zeichenbefehl~beginnen}}\\
\rule[-0.5ex]{0pt}{2.5ex}\hspace*{0.0em}printf(\grqq{}\textbackslash{}\textbackslash{}draw~plot[smooth]~coordinates~\{\grqq{});\\
\rule[-0.5ex]{0pt}{2.5ex}\hspace*{0.0em}\textcolor{G}{\textbf{\%~Anzahl~der~berechneten~Punkte}}\\
\rule[-0.5ex]{0pt}{2.5ex}\hspace*{0.0em}n~=~0;\\
\rule[-0.5ex]{0pt}{2.5ex}\hspace*{0.0em}\textcolor{G}{\textbf{\%~Intervall~und~Schrittweite~der~x{-}Werte}}\\
\rule[-0.5ex]{0pt}{2.5ex}\hspace*{0.0em}for~\textcolor{R}{\textbf{x}}~=~\textcolor{R}{\textbf{{-}1.5:0.1:1.5}}\\
\rule[-0.5ex]{0pt}{2.5ex}\hspace*{1.0em}\textcolor{G}{\textbf{\%~Zeile~nach~4~Punkten~umbrechen~und~einruecken}}\\
\rule[-0.5ex]{0pt}{2.5ex}\hspace*{1.0em}if~(mod(n++,~4)~==~0)\\
\rule[-0.5ex]{0pt}{2.5ex}\hspace*{2.0em}printf(\grqq{}\textbackslash{}n~~~~~\grqq{});\\
\rule[-0.5ex]{0pt}{2.5ex}\hspace*{1.0em}endif\\
\rule[-0.5ex]{0pt}{2.5ex}\hspace*{1.0em}\textcolor{G}{\textbf{\%~Funktionswert~berechnen:~y~=~{-}x\^{}2~+~2x~+~1}}\\
\rule[-0.5ex]{0pt}{2.5ex}\hspace*{1.0em}\textcolor{R}{\textbf{y}}~=~\textcolor{R}{\textbf{{-}x**2~+~2*x~+~1}};\\
\rule[-0.5ex]{0pt}{2.5ex}\hspace*{1.0em}\textcolor{G}{\textbf{\%~Wertepaar~'~(x,~y)'~ausgeben}}\\
\rule[-0.5ex]{0pt}{2.5ex}\hspace*{1.0em}printf(\grqq{}~(\%4.1f,~\%5.2f)\grqq{},~x,~y);\\
\rule[-0.5ex]{0pt}{2.5ex}\hspace*{0.0em}end\\
\rule[-0.5ex]{0pt}{2.5ex}\hspace*{0.0em}\textcolor{G}{\textbf{\%~LaTeX{-}Zeichenbefehl~beenden}}\\
\rule[-0.5ex]{0pt}{2.5ex}\hspace*{0.0em}printf(\grqq{}~\};\textbackslash{}n\grqq{});}%
}%
\endgroup

\end{footnotesize}

% -------------------------------------------------------------------
\subsection{Die Wertepaare für \texttt{plot} in eine Datei schreiben}
% -------------------------------------------------------------------
\begin{footnotesize}
% parcolor version 2011-09-30
\begingroup
\ttfamily
\definecolor{R}{named}{Red}
\definecolor{G}{named}{ForestGreen}
\definecolor{B}{named}{RoyalBlue}
\definecolor{C}{named}{Cyan}
\definecolor{M}{named}{Magenta}
\definecolor{Y}{named}{YellowOrange}
\definecolor{background}{rgb}{0.82, 0.82, 0.92}
\dimen255=\textwidth
\advance\dimen255 by -2\fboxsep
\noindent
\colorbox{background}
{%
\parbox{\dimen255}
{%
\rule[-0.5ex]{0pt}{2.5ex}\hspace*{0.0em}\textcolor{G}{\textbf{\%~Datei~'f.xy'~zum~Schreiben~oeffnen}}\\
\rule[-0.5ex]{0pt}{2.5ex}\hspace*{0.0em}FID~=~fopen(\grqq{}\textcolor{B}{\textbf{f.xy}}\grqq{},~\grqq{}w\grqq{});\\
\rule[-0.5ex]{0pt}{2.5ex}\hspace*{0.0em}\textcolor{G}{\textbf{\%~Intervall~und~Schrittweite~der~x{-}Werte}}\\
\rule[-0.5ex]{0pt}{2.5ex}\hspace*{0.0em}for~\textcolor{R}{\textbf{x}}~=~\textcolor{R}{\textbf{{-}5:0.1:5}}\\
\rule[-0.5ex]{0pt}{2.5ex}\hspace*{1.0em}\textcolor{G}{\textbf{\%~Funktionswert~berechnen:~y~=~{-}x\^{}2~+~2x~+~1}}\\
\rule[-0.5ex]{0pt}{2.5ex}\hspace*{1.0em}\textcolor{R}{\textbf{y}}~=~\textcolor{R}{\textbf{{-}x**2~+~2*x~+~1}};\\
\rule[-0.5ex]{0pt}{2.5ex}\hspace*{1.0em}\textcolor{G}{\textbf{\%~Wertepaar~'x~y'~ausgeben}}\\
\rule[-0.5ex]{0pt}{2.5ex}\hspace*{1.0em}fprintf(FID,~\grqq{}\%6.2f\textbackslash{}t\%6.2f\textbackslash{}n\grqq{},~x,~y);\\
\rule[-0.5ex]{0pt}{2.5ex}\hspace*{0.0em}end\\
\rule[-0.5ex]{0pt}{2.5ex}\hspace*{0.0em}\textcolor{G}{\textbf{\%~Datei~schliessen}}\\
\rule[-0.5ex]{0pt}{2.5ex}\hspace*{0.0em}fclose(FID);}%
}%
\endgroup

\end{footnotesize}

\newcommand{\pythagoras}[2]
{%
  \begin{scope}
    % Koordinaten der Eckpunkte
    \coordinate (A) at (-#1, 0);
    \coordinate (B) at ( #1, 0);
    \coordinate (C) at ($(0, 0)!#1!#2:(B)$);
    % Eckpunkte des Quadrats ueber a berechnen
    \coordinate (aB) at ($(B)!1!270:(C)$);
    \coordinate (aC) at ($(C)!1!90:(B)$);
    % Eckpunkte des Quadrats ueber b berechnen
    \coordinate (bA) at ($(A)!1!90:(C)$);
    \coordinate (bC) at ($(C)!1!270:(A)$);
    % Eckpunkte des Quadrats ueber c berechnen
    \coordinate (cA) at ($(A)!1!270:(B)$);
    \coordinate (cB) at ($(B)!1!90:(A)$);
    % Mittelpunkte der Quadrate berechnen
    \coordinate (ma) at ($(B)!0.5!(aC)$);
    \coordinate (mb) at ($(A)!0.5!(bC)$);
    \coordinate (mc) at ($(A)!0.5!(cB)$);
    % Quadrate
    \filldraw[fill=LimeGreen]  (B) -- (aB) -- (aC) -- (C);
    \filldraw[fill=LimeGreen]  (A) -- (bA) -- (bC) -- (C);
    \filldraw[fill=OliveGreen] (A) -- (cA) -- (cB) -- (B);
    % Beschriftung
    \node at (ma) {{\footnotesize$a^{2}$}};
    \node at (mb) {{\footnotesize$b^{2}$}};
    \node at (mc) {{\footnotesize$c^{2}$}};
    % Dreieck
    \draw (A) -- (B) -- (C) -- cycle;
    % Eckpunkte markieren
    \fill (A) circle[radius=1pt] (B) circle[radius=1pt]
          (C) circle[radius=1pt];
 \end{scope}
}

% ------------------------------------------------------------------------------
\end{document}
% ------------------------------------------------------------------------------

% parcolor version 2011-09-30
\begingroup
\ttfamily
\definecolor{R}{named}{Red}
\definecolor{G}{named}{ForestGreen}
\definecolor{B}{named}{RoyalBlue}
\definecolor{C}{named}{Cyan}
\definecolor{M}{named}{Magenta}
\definecolor{Y}{named}{YellowOrange}
\definecolor{background}{rgb}{0.82, 0.82, 0.92}
\dimen255=\textwidth
\advance\dimen255 by -2\fboxsep
\noindent
\colorbox{background}
{%
\parbox{\dimen255}
{%
\rule[-0.5ex]{0pt}{2.5ex}\hspace*{0.0em}\textbackslash{}begin\{tikzpicture\}\\
\rule[-0.5ex]{0pt}{2.5ex}\hspace*{1.0em}\textcolor{R}{\textbf{\textbackslash{}begin}}\{\textcolor{R}{\textbf{scope}}\}\\
\rule[-0.5ex]{0pt}{2.5ex}\hspace*{2.0em}\textcolor{G}{\textbf{\%~Rechteck~ueber~der~rechten~Haelfte~des~Kreises}}\\
\rule[-0.5ex]{0pt}{2.5ex}\hspace*{2.0em}\textcolor{R}{\textbf{\textbackslash{}clip}}~(0,~{-}1)~\textcolor{R}{\textbf{rectangle}}~(1,~1);\\
\rule[-0.5ex]{0pt}{2.5ex}\hspace*{2.0em}\textcolor{G}{\textbf{\%~den~Teil~des~Kreises~ausfuellen,~der~im}}\\
\rule[-0.5ex]{0pt}{2.5ex}\hspace*{2.0em}\textcolor{G}{\textbf{\%~Clipping{-}Bereich~liegt}}\\
\rule[-0.5ex]{0pt}{2.5ex}\hspace*{2.0em}\textbackslash{}fill[fill=LimeGreen]~(0,~0)~circle[radius=1];\\
\rule[-0.5ex]{0pt}{2.5ex}\hspace*{1.0em}\textcolor{R}{\textbf{\textbackslash{}end}}\{\textcolor{R}{\textbf{scope}}\}\\
\rule[-0.5ex]{0pt}{2.5ex}\hspace*{1.0em}\textcolor{G}{\textbf{\%~den~Rand~des~Kreises~zeichnen}}\\
\rule[-0.5ex]{0pt}{2.5ex}\hspace*{1.0em}\textbackslash{}draw~(0,~0)~circle[radius=1];\\
\rule[-0.5ex]{0pt}{2.5ex}\hspace*{0.0em}\textbackslash{}end\{tikzpicture\}}%
}%
\endgroup

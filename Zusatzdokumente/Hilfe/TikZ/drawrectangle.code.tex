% parcolor version 2011-09-30
\begingroup
\ttfamily
\definecolor{R}{named}{Red}
\definecolor{G}{named}{ForestGreen}
\definecolor{B}{named}{RoyalBlue}
\definecolor{C}{named}{Cyan}
\definecolor{M}{named}{Magenta}
\definecolor{Y}{named}{YellowOrange}
\definecolor{background}{rgb}{0.82, 0.82, 0.92}
\dimen255=\textwidth
\advance\dimen255 by -2\fboxsep
\noindent
\colorbox{background}
{%
\parbox{\dimen255}
{%
\rule[-0.5ex]{0pt}{2.5ex}\hspace*{0.0em}\textcolor{G}{\textbf{\%~die~tikzpicture{-}Umgebung~enthaelt~die~Zeichenbefehle}}\\
\rule[-0.5ex]{0pt}{2.5ex}\hspace*{0.0em}\textcolor{R}{\textbf{\textbackslash{}begin}}\{\textcolor{R}{\textbf{tikzpicture}}\}\\
\rule[-0.5ex]{0pt}{2.5ex}\hspace*{1.0em}\textcolor{G}{\textbf{\%~hier~wird~der~Rand~eines~Rechtecks~gezeichnet:}}\\
\rule[-0.5ex]{0pt}{2.5ex}\hspace*{1.0em}\textcolor{R}{\textbf{\textbackslash{}draw}}[line~width=1pt]~(0,~0)~\textcolor{R}{\textbf{rectangle}}~(3,~2);\\
\rule[-0.5ex]{0pt}{2.5ex}\hspace*{0.0em}\textcolor{R}{\textbf{\textbackslash{}end}}\{\textcolor{R}{\textbf{tikzpicture}}\}}%
}%
\endgroup

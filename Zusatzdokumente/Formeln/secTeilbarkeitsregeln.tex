% ------------------------
\mysec{Teilbarkeitsregeln}
% ------------------------
\begingroup
\newcommand{\numbox}[1]{\par\hangindent=1.75em\makebox[1.75em][r]{\textbf{#1},\enskip}\ignorespaces}%
Eine Zahl $n$ ist genau dann teilbar durch\ldots\medskip
\numbox{2} wenn ihre letzte Ziffer gerade ist.
\numbox{3} wenn ihre Quersumme durch 3 teilbar ist:
           \begin{equation*}
             \operatorname{Q}(\num{32046})=6+4+0+2+3=15
           \end{equation*}
\numbox{5} wenn ihre letzte Ziffer 0 oder 5 ist.
\numbox{7} wenn ihre alternierende 3er-Quer\-sum\-me durch 7 teilbar ist:
           \begin{equation*}
             \tilde{\operatorname{Q}}_3(\num{32046})=046-32=14
           \end{equation*}
\numbox{11} \raggedright
            wenn ihre alternierende Quersumme durch 11 teilbar ist:
            \begin{equation*}
              \tilde{\operatorname{Q}}(\num{71357})=7-5+3-1+7=11
            \end{equation*}
\numbox{13} wenn ihre alternierende 3er-Quer\-sum\-me durch 13 teilbar
            ist (vgl. $n=7$).\par
\endgroup


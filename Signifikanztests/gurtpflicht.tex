\begin{exercise}
      {ID-be7d3130161c0d449576c55eec885c2f616fda3a}
      {Gurtpflicht}
  \ifproblem\problem\par
    In einem Zeitungsbericht wird behauptet, dass sich nur \pc{70} der
    Autofahrer angurten. Ein Autoklub behauptet, dass der Anteil in Wirklichkeit
    höher ist. Die Polizei meint dagegen, dass der Anteil in Wirklichkeit
    kleiner ist. Es wird ein Test der Nullhypothese $H_0:p=\num{0.7}$
    (Stichprobenumfang 100; Signifikanzniveau \pc{5}) durchgeführt.
    \begin{enumerate}[a)]
      \item Welche Alternative $H_1$ und welchen Annahmebereich geben der
            Autoklub bzw. die Polizei an?
      \item Die Stichprobe ergibt, dass 79 Fahrer angegurtet sind. Wie fällt
            die Entscheidung des Autoklubs bzw. der Polizei aus?
    \end{enumerate}
  \fi
  %\ifoutline\outline\par
  %\fi
  %\ifoutcome\outcome\par
  %\fi
\end{exercise}

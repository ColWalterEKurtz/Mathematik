\begin{exercise}
      {ID-a55f71c9b978c2ee23cc0b98f8eaef3bce016a7f}
      {Tennis}
  \ifproblem\problem
    Die Aussage: \glqq Heiko und Sandro spielen gleich gut Tennis\grqq{}
    bedeutet: Heiko gewinnt gegen Sandro mit der Wahrscheinlichkeit $p=\pc{50}$
    (Nullhypothese $H_0$). Muss man die Hypothese auf dem \pc{5}-Signifikanzniveau
    verwerfen, wenn Heiko folgenden Anteil an Spielen gewinnt?
    \begin{center}
      a) 10 von 16
      \qquad\qquad\qquad
      b) 40 von 64
    \end{center}
    Nutzen Sie überschlagsmäßig die $1,\!96\sigma$-Regel.
  \fi
  %\ifoutline\outline
  %\fi
  %\ifoutcome\outcome
  %\fi
\end{exercise}

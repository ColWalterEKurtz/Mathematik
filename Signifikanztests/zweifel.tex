% 2021-11-11
\begin{exercise}
      {ID-694850175e5d27550fcac2047853a8ad829b4e19}
      {Zweifel}
  \ifproblem\problem\par
    % <PROBLEM>
    Eine angenommene Wahrscheinlichkeit $p$ soll nicht
    bezweifelt werden, wenn eine beobachtete Häufigkeit
    innerhalb des zugehörigen $2\sigma$-Intervalls liegt.
    Man vermutet $p=\num{0.75}$.
    \begin{enumerate}[a)]
      \item Muss man diese Vermutung anzweifeln, wenn
            \begin{enumerate}[1.]
              \setlength{\itemsep}{-1ex}%
              \item bei \num{100} Versuchen \num{70} Treffer
              \item bei \num{400} Versuchen \num{280} Treffer
            \end{enumerate}
            beobachtet werden?
      \item In welchem Bereich darf die relative
            Trefferhäufigkeit bei $n=\num{1000}$
            Versuchen liegen, damit man die
            Wahrscheinlichkeit $p=\num{0.75}$
            nicht zu bezweifeln braucht?
    \end{enumerate}
    % </PROBLEM>
  \fi
  %\ifoutline\outline\par
    % <OUTLINE>
    % </OUTLINE>
  %\fi
  %\ifoutcome\outcome\par
    % <OUTCOME>
    % </OUTCOME>
  %\fi
\end{exercise}

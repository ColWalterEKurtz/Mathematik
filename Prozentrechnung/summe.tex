\begin{exercise}
      {ID-812972e6c6b1b8e11b6ad6d09d94a0a8430c6d93}
      {Summe}
  \ifproblem\problem
    In einer Summe ist der erste Summand $\frac{5}{12}$ des zweiten Summanden.
    Wie viel Prozent der Summe entfällt auf den kleineren Summanden?
  \fi
  %\ifoutline\outline
  %\fi
  \ifoutcome\outcome
  \begin{equation*}
    p\,\text{\%}
    =\frac{a}{a+b}
    =\frac{\frac{5}{12}b}{\frac{5}{12}b+b}
    =\frac{\frac{5}{12}b}{\left(\frac{5}{12}+1\right)\cdot b}
    =\frac{5}{17}
    \approx\num{29.4}\,\text{\%}
  \end{equation*}
  \fi
\end{exercise}

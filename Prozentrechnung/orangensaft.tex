\begin{exercise}
      {ID-b79a8f57a767f9f382519ad269100099fe589a43}
      {Orangensaft}
  \ifproblem\problem\par
    Ein geschäftstüchtiger Getränkehändler erhöht den Preis für pfandfreien
    Orangensaft \glqq ganz unauffällig\grqq{} auf folgende Weise: Er verkauft
    den Saft jetzt in \num{0.75}-Liter-Flaschen, und zwar zu dem Preis, den er
    früher für die 1-Liter-Flaschen verlangt hat. Um wie viel Prozent
    hat sich der Orangensaft für den Kunden verteuert?
  \fi
  %\ifoutline\outline\par
  %\fi
  \ifoutcome\outcome\par
    \begin{equation*}
      \text{\siml{1000}}\cdot x_{\text{alt}}=\text{\siml{750}}\cdot x_{\text{neu}}
      \quad\Rightarrow\quad
      x_{\text{neu}}=\frac{4}{3}\cdot x_{\text{alt}}
    \end{equation*}
    Der Orangensaft hat sich also um etwa \pc{33.3} verteuert.
  \fi
\end{exercise}

\begin{exercise}
      {ID-a9ea670367d660d7498d0581bfe978ba3e8fa481}
      {Stundenlohn}
  \ifproblem\problem\par
    Bei der Verkürzung der Arbeitszeit von 40 auf 38 Stunden pro Woche
    soll der Lohn gleich bleiben. Um wie viel Prozent wird folglich der
    Stundenlohn erhöht?
  \fi
  %\ifoutline\outline\par
  %\fi
  \ifoutcome\outcome\par
    \begin{equation*}
      40\cdot x_{\text{alt}}=38\cdot x_{\text{neu}}
      \quad\Rightarrow\quad
      x_{\text{neu}}\approx(1+0,\!053)\cdot x_{\text{alt}}
    \end{equation*}
    Der Stundenlohn hat sich also um etwa \pc{5.3} erhöht.
  \fi
\end{exercise}

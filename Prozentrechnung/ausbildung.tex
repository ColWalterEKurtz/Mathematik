\begin{exercise}
      {ID-8499e1bdae41204067ee26a95cb47f690702e956}
      {Ausbildung}
  \ifproblem\problem
    Für die Ausbildung einer Schülerin oder eines Schülers an öffentlichen Schulen
    gaben die öffentlichen Haushalte im Jahr 2010 durchschnittlich \eur{5\,800} aus.
    Die höchsten Ausgaben je Schülerin bzw. Schüler wurden für Thüringen (\eur{7\,700})
    ermittelt, die niedrigsten für Nordrhein-Westfalen (\eur{5\,000}).
    Um wie viel Prozent liegen die Ausgaben von Nordrhein-Westfalen unter dem
    bundesweiten Durchschnitt bzw. unter den Ausgaben von Thüringen?
  \fi
  %\ifoutline\outline
  %\fi
  %\ifoutcome\outcome
  %\fi
\end{exercise}

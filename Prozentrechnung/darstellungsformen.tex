\begin{exercise}
      {ID-0d22bc540ce1c9f607acc5c6d917375bb5a91d19}
      {Darstellungsformen}
  \ifproblem\problem\par
    Schreibe die Prozentangabe als Bruch und als Dezimalzahl:
    \begin{alignat*}{6}
      \text{a)}&\quad & \text{\pc{1}}&\qquad\qquad  & \text{e)}&\quad & \text{\pc{75}}&\qquad\qquad  & \text{i)}&\quad & \text{\pc{1.5}}&    \\[1ex]
      \text{b)}&\quad & \text{\pc{10}}&\qquad\qquad & \text{f)}&\quad & \text{\pc{99}}&\qquad\qquad  & \text{j)}&\quad & \text{\pc{17.125}}& \\[1ex]
      \text{c)}&\quad & \text{\pc{25}}&\qquad\qquad & \text{g)}&\quad & \text{\pc{100}}&\qquad\qquad & \text{k)}&\quad & \text{\pc{60.25}}&  \\[1ex]
      \text{d)}&\quad & \text{\pc{50}}&\qquad\qquad & \text{h)}&\quad & \text{\pc{119}}&\qquad\qquad & \text{l)}&\quad & \text{\pc{0.1}}&
    \end{alignat*}
  \fi
  %\ifoutline\outline\par
  %\fi
  \ifoutcome\outcome\par
    \begin{alignat*}{4}
      \text{a)}&\quad & \text{\pc{1}}&=\frac{1}{100}=\num{0.01}               & \qquad\quad\text{g)}&\quad & \text{\pc{100}}&=\frac{100}{100}=\num{1}                           \\[1ex]
      \text{b)}&\quad & \text{\pc{10}}&=\frac{10}{100}=\num{0.1}              & \qquad\quad\text{h)}&\quad & \text{\pc{119}}&=\frac{119}{100}=\num{1.19}                        \\[1ex]
      \text{c)}&\quad & \text{\pc{25}}&=\frac{25}{100}=\frac{1}{4}=\num{0.25} & \qquad\quad\text{i)}&\quad & \text{\pc{1.5}}&=\frac{15}{10\cdot100}=\num{0.015}                 \\[1ex]
      \text{d)}&\quad & \text{\pc{50}}&=\frac{50}{100}=\frac{1}{2}=\num{0.5}  & \qquad\quad\text{j)}&\quad & \text{\pc{17.125}}&=\frac{\num{17125}}{1000\cdot100}=\num{0.17125} \\[1ex]
      \text{e)}&\quad & \text{\pc{75}}&=\frac{75}{100}=\frac{3}{4}=\num{0.75} & \qquad\quad\text{k)}&\quad & \text{\pc{60.25}}&=\frac{6025}{100\cdot100}=\num{0.6025}           \\[1ex]
      \text{f)}&\quad & \text{\pc{99}}&=\frac{99}{100}=\num{0.99}             & \qquad\quad\text{l)}&\quad & \text{\pc{0.1}}&=\frac{1}{10\cdot100}=\num{0.001}
    \end{alignat*}
  \fi
\end{exercise}

\begin{exercise}
      {ID-fb4004ab3bdc3572643baf75d0273385b53147b8}
      {Reaktor}
  \ifproblem\problem\par
    Die Produktion einer bestimmten Menge eines chemischen Stoffes dauert 60 Stunden.
    Innerhalb dieser Zeit ist es einmal möglich den Reaktor so zu überlasten, dass
    die Produktion 12 Stunden lang um \pc{200} gesteigert wird.
    \begin{enumerate}[a)]
      \item Um wie viel Prozent kann die Gesamtmenge des hergestellten Stoffes
            durch die zwölfstündige Maßnahme gesteigert werden?
      \item Wie lange dauert die Produktion der ursprünglichen Menge im
            beschleunigten Verfahren?
    \end{enumerate}
  \fi
  %\ifoutline\outline\par
  %\fi
  %\ifoutcome\outcome\par
  %\fi
\end{exercise}

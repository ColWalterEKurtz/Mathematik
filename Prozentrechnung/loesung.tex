\begin{exercise}
      {ID-aad90c4416cee91998d3d1bded42a8c0fcfb3628}
      {Lösung}
  \ifproblem\problem\par
    In einem Behälter befinden sich genau \sikg{25} einer \pc{4}-igen wässrigen
    Lösung, d.\,h., \pc{4} dieser Lösung bestehen aus der gelösten Substanz,
    der Rest besteht aus Wasser.\par
    Wieviel Prozent des Wassers sind dieser Lösung zu entziehen, damit eine
    neue Lösung entsteht, deren Wasseranteil nur noch \pc{90} beträgt?
  \fi
  %\ifoutline\outline\par
  %\fi
  %\ifoutcome\outcome\par
  %\fi
\end{exercise}

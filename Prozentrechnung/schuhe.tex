\begin{exercise}
      {ID-945a7819b366ed62a2ecb07dd6801d19eb8051e7}
      {Schuhe}
  \ifproblem\problem
    \xxa{} bummelt an einem Schuhgeschäft vorbei, entdeckt im Schaufenster ein
    Plakat mit der Aufschrift {\itshape\glqq Nur heute: Mindestens \pc{25} Rabatt
    auf alle Schuhe!\grqq} und beschließt, sich die Schuhe einmal genauer anzusehen.
    \begin{enumerate}[a)]
      \item Auf ein Paar Schuhe, das gestern noch \eur{76} gekostet hat, gibt es
            die angekündigten \pc{25} Rabatt. Wie viel Euro spart \xxa, wenn sie
            sich die Schuhe heute kauft?
      \item Auf ein Paar Schuhe, das gestern noch \eur{84} gekostet hat, gibt es
            die angekündigten \pc{25} Rabatt. Wie viel Euro muss \xxa{} heute
            für die Schuhe zahlen?
      \item \xxa{} findet im Geschäft ein Paar Schuhe, das von \eur{74} auf
            \eur{51.80} reduziert wurde. Wie viel Prozent Rabatt gibt es auf
            diese Schuhe?
      \item \xxa{} findet im Geschäft ein Paar Schuhe, das von \eur{135} auf
            \eur{97.20} reduziert wurde. Zu wie viel Prozent des alten
            Preises werden die Schuhe heute verkauft?
      \item Vor einem Paar Schuhe steht auf einem großen roten Schild:
            {\itshape\glqq \pc{45} Rabatt!!! Jetzt nur noch \eur{27.50}!\grqq}
            Wie teuer waren diese Schuhe gestern?
      \item \xxa{} kauft sich drei Paar neue Schuhe und muss an der Kasse
            \eur{152.32} zahlen. Die Verkäuferin sagt: \glqq Herzlichen
            Glückwunsch, das sind nur \pc{64} von dem Preis, den sie gestern
            hätten zahlen müssen.\grqq{}
            Wie teuer wäre der Einkauf gestern gewesen?
      \item \xxa{} weiß, dass auf Schuhe eine Mehrwertsteuer von \pc{19} erhoben
            wird. Wie teuer wäre der Einkauf geworden, wenn für Schuhe nur der
            reduzierte Mehrwertsteuersatz von \pc{7} gelten würde?
    \end{enumerate}
  \fi
  %\ifoutline\outline
  %\fi
  \ifoutcome\outcome
    \begingroup
      \newcommand{\noeq}{\quad&\quad}
      % a)
      \newcommand{\eqAA}{0,\!25\cdot\text{\eur{76}}&=\text{\eur{19}}}
      \newcommand{\eqAB}{\noeq}
      \newcommand{\eqAC}{\noeq}
      % b)
      \newcommand{\eqBA}{0,\!75\cdot\text{\eur{84}}&=\text{\eur{63}}}
      \newcommand{\eqBB}{\noeq}
      \newcommand{\eqBC}{\noeq}
      % c)
      \newcommand{\eqCA}{(1-x)\cdot\text{\eur{74}}&=\text{\eur{51,80}}}
      \newcommand{\eqCB}{\quad&\Rightarrow\quad}
      \newcommand{\eqCC}{x&=0,\!3=\text{\pc{30}}}
      % d)
      \newcommand{\eqDA}{x\cdot\text{\eur{135}}&=\text{\eur{97,20}}}
      \newcommand{\eqDB}{\quad&\Rightarrow\quad}
      \newcommand{\eqDC}{x&=0,\!72=\text{\pc{72}}}
      % e)
      \newcommand{\eqEA}{(1-0,\!45)\cdot x&=\text{\eur{27,50}}}
      \newcommand{\eqEB}{\quad&\Rightarrow\quad}
      \newcommand{\eqEC}{x&=\text{\eur{50}}}
      % f)
      \newcommand{\eqFA}{0,\!64\cdot x&=\text{\eur{152,32}}}
      \newcommand{\eqFB}{\quad&\Rightarrow\quad}
      \newcommand{\eqFC}{x&=\text{\eur{238}}}
      % g)
      \newcommand{\eqGA}{\text{\eur{152,32}}:1,\!19\cdot1,\!07&=\text{\eur{136,96}}}
      \newcommand{\eqGB}{\noeq}
      \newcommand{\eqGC}{\noeq}
      % Aufgaben
      \setlength{\abovedisplayskip}{0pt}%
      \begin{alignat*}{4}
        \text{a)}&\quad & \eqAA & \eqAB & \eqAC \\[1ex]
        \text{b)}&\quad & \eqBA & \eqBB & \eqBC \\[1ex]
        \text{c)}&\quad & \eqCA & \eqCB & \eqCC \\[1ex]
        \text{d)}&\quad & \eqDA & \eqDB & \eqDC \\[1ex]
        \text{e)}&\quad & \eqEA & \eqEB & \eqEC \\[1ex]
        \text{f)}&\quad & \eqFA & \eqFB & \eqFC \\[1ex]
        \text{g)}&\quad & \eqGA & \eqGB & \eqGC
      \end{alignat*}
    \endgroup
  \fi
\end{exercise}

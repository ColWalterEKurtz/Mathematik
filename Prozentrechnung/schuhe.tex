\begin{exercise}
      {ID-945a7819b366ed62a2ecb07dd6801d19eb8051e7}
      {Schuhe}
  \ifproblem\problem
    \xxa{} bummelt an einem Schuhgeschäft vorbei, entdeckt im Schaufenster ein
    Plakat mit der Aufschrift {\itshape\glqq Nur heute: Mindestens \pc{25} Rabatt
    auf alle Schuhe!\grqq} und beschließt, sich die Schuhe einmal genauer anzusehen.
    \begin{enumerate}[a)]
      \item Auf ein Paar Schuhe, das gestern noch \eur{76} gekostet hat, gibt es
            die angekündigten \pc{25} Rabatt. Wie viel Euro spart \xxa, wenn sie
            sich die Schuhe heute kauft?
      \item Auf ein Paar Schuhe, das gestern noch \eur{84} gekostet hat, gibt es
            die angekündigten \pc{25} Rabatt. Wie viel Euro muss \xxa{} heute
            für die Schuhe zahlen?
      \item \xxa{} findet im Geschäft ein Paar Schuhe, das von \eur{74} auf
            \eur{51.80} reduziert wurde. Wie viel Prozent Rabatt gibt es auf
            diese Schuhe?
      \item \xxa{} findet im Geschäft ein Paar Schuhe, das von \eur{135} auf
            \eur{97.20} reduziert wurde. Zu wie viel Prozent des alten
            Preises werden die Schuhe heute verkauft?
      \item Vor einem Paar Schuhe steht auf einem großen roten Schild:
            {\itshape\glqq \pc{45} Rabatt!!! Jetzt nur noch \eur{27.50}!\grqq}
            Wie teuer waren diese Schuhe gestern?
      \item \xxa{} kauft sich drei Paar neue Schuhe und muss an der Kasse
            \eur{152.32} zahlen. Die Verkäuferin sagt: \glqq Herzlichen
            Glückwunsch, das sind nur \pc{64} von dem Preis, den sie gestern
            hätten zahlen müssen.\grqq{}
            Wie teuer wäre der Einkauf gestern gewesen?
      \item \xxa{} weiß, dass auf Schuhe eine Mehrwertsteuer von \pc{19} erhoben
            wird. Wie teuer wäre der Einkauf geworden, wenn für Schuhe nur der
            reduzierte Mehrwertsteuersatz von \pc{7} gelten würde?
    \end{enumerate}
  \fi
  %\ifoutline\outline
  %\fi
  %\ifoutcome\outcome
  %\fi
\end{exercise}

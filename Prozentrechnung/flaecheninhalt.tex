\begin{exercise}
      {ID-b70b7b77d7098a0e8fa2ff2820e6e3dfa29b2b4c}
      {Flächeninhalt}
  \ifproblem\problem\par
    Gegeben ist ein Rechteck. Wie ändert sich sein Flächeninhalt (in Prozent),
    wenn seine Länge um \pc{30} größer und seine Breite um \pc{30} kleiner wird?
  \fi
  %\ifoutline\outline\par
  %\fi
  \ifoutcome\outcome\par
    \begin{equation*}
      \frac{\frac{13}{10}a\cdot\frac{7}{10}b}{ab}
      =\frac{91}{100}=91\,\text{\%}
    \end{equation*}
    Das neue Rechteck ist also um \pc{9} kleiner.
  \fi
\end{exercise}

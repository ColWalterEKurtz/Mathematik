\begin{exercise}
      {ID-6c8c546cb9648cbc30d8e86ed3e67592bae4782b}
      {Pfahl im See}
  \ifproblem\problem
    Ein Pfahl steckt mit \pc{30} seiner Länge im Grund eines Sees.
    \pc{40} seiner Länge werden von Wasser umspült. \sicm{45} schauen aus
    dem Wasser heraus. Wie lang ist der Pfahl?
  \fi
  \ifoutline\outline
    \begin{center}
      \begin{tikzpicture}
        % Grund des Sees
        \draw[draw=White,
              pattern=north east lines,
              pattern color=Black] (-15mm, -8mm) rectangle (15mm, 0mm);
        \draw (-15mm, 0mm) -- (15mm, 0mm);
        % Pfahl
        \filldraw[draw= Black, fill=White] (-1mm, -4.5mm) rectangle (1mm, 10.5mm);
        % Wasseroberfläche
        \draw[decorate, decoration=bumps] (-15mm, 6mm) -- (15mm, 6mm);
        % Bechriftung
        \node[right] at (15mm, 10mm) {{\small\sicm{45}}};
        \node[right] at (15mm,  3mm) {{\small\pc{40}}};
        \node[right] at (15mm, -3mm) {{\small\pc{30}}};
      \end{tikzpicture}
    \end{center}
  \fi
  %\ifoutcome\outcome
  %\fi
\end{exercise}

\begin{exercise}
      {ID-51caed6448753cf4de9bf791661089e31e3c6b27}
      {Augensumme}
  \ifproblem\problem\par
    % <PROBLEM>
    Wie groß ist die Wahrscheinlichkeit, mit zwei Würfeln die Augensumme 5
    zu würfeln?
    % </PROBLEM>
  \fi
  %\ifoutline\outline\par
    % <OUTLINE>
    % </OUTLINE>
  %\fi
  \ifoutcome\outcome\par
    % <OUTCOME>
    Um Platz beim Zeichnen des Baumdiagramms zu sparen,
    werden irrelevente Resultate unter dem Buchstaben
    \glqq R\grqq{} (\glqq Rest\grqq) zusammengefasst.
    \begin{center}
      \begin{tikzpicture}
        % tree
        \begin{scope}[line width=0.6pt]
          % size settings
          \newcommand{\radius}{3mm}%
          \newcommand{\xscale}{3}%
          \newcommand{\yscale}{4}%
          % geometry
          \coordinate (Z)  at ( 4.500*\xscale*\radius,  2.800*\yscale*\radius);
          \coordinate (A)  at ( 0.500*\xscale*\radius,  1.000*\yscale*\radius);
          \coordinate (B)  at ( 2.500*\xscale*\radius,  1.000*\yscale*\radius);
          \coordinate (C)  at ( 4.500*\xscale*\radius,  1.000*\yscale*\radius);
          \coordinate (D)  at ( 6.500*\xscale*\radius,  1.000*\yscale*\radius);
          \coordinate (E)  at ( 8.500*\xscale*\radius,  1.000*\yscale*\radius);
          \coordinate (AA) at ( 0.000*\xscale*\radius,  0.000*\yscale*\radius);
          \coordinate (AB) at ( 1.000*\xscale*\radius,  0.000*\yscale*\radius);
          \coordinate (BA) at ( 2.000*\xscale*\radius,  0.000*\yscale*\radius);
          \coordinate (BB) at ( 3.000*\xscale*\radius,  0.000*\yscale*\radius);
          \coordinate (CA) at ( 4.000*\xscale*\radius,  0.000*\yscale*\radius);
          \coordinate (CB) at ( 5.000*\xscale*\radius,  0.000*\yscale*\radius);
          \coordinate (DA) at ( 6.000*\xscale*\radius,  0.000*\yscale*\radius);
          \coordinate (DB) at ( 7.000*\xscale*\radius,  0.000*\yscale*\radius);
          \coordinate (EA) at ( 8.500*\xscale*\radius,  0.000*\yscale*\radius);
          % edges
          \draw (Z) -- (A);
          \draw (Z) -- (B);
          \draw (Z) -- (C);
          \draw (Z) -- (D);
          \draw (Z) -- (E);
          \draw (A) -- (AA);
          \draw (A) -- (AB);
          \draw (B) -- (BA);
          \draw (B) -- (BB);
          \draw (C) -- (CA);
          \draw (C) -- (CB);
          \draw (D) -- (DA);
          \draw (D) -- (DB);
          \draw (E) -- (EA);
          % root
          \fill[fill=black] (Z) circle[radius=2pt];
          % nodes
          \begin{scope}[fill=white, draw=black]
            \filldraw (A)  circle[radius=\radius] node{1};
            \filldraw (B)  circle[radius=\radius] node{2};
            \filldraw (C)  circle[radius=\radius] node{3};
            \filldraw (D)  circle[radius=\radius] node{4};
            \filldraw (E)  circle[radius=\radius] node{R};
            \filldraw (AA) circle[radius=\radius] node{4};
            \filldraw (AB) circle[radius=\radius] node{R};
            \filldraw (BA) circle[radius=\radius] node{3};
            \filldraw (BB) circle[radius=\radius] node{R};
            \filldraw (CA) circle[radius=\radius] node{2};
            \filldraw (CB) circle[radius=\radius] node{R};
            \filldraw (DA) circle[radius=\radius] node{1};
            \filldraw (DB) circle[radius=\radius] node{R};
            \filldraw (EA) circle[radius=\radius] node{R};
          \end{scope}
          % label macros
          \newcommand{\rlabel}[5][0.5]%
          {%
            \coordinate (TEMP) at ($(#2)!#1!(#3)$);
            \coordinate (TEMP) at ($(TEMP)!#4!270:(#3)$);
            \node at (TEMP) {#5};
          }%
          % edge labels
          \rlabel     {Z}{A}{ 3mm}{$\frac{1}{6}$};
          \rlabel[0.7]{Z}{B}{ 3mm}{$\frac{1}{6}$};
          \rlabel[0.6]{Z}{C}{-2mm}{$\frac{1}{6}$};
          \rlabel[0.7]{Z}{D}{-3mm}{$\frac{1}{6}$};
          \rlabel     {Z}{E}{-3mm}{$\frac{2}{6}$};
          \rlabel{A}{AA}{ 3mm}{$\frac{1}{6}$};
          \rlabel{A}{AB}{-3mm}{$\frac{5}{6}$};
          \rlabel{B}{BA}{ 3mm}{$\frac{1}{6}$};
          \rlabel{B}{BB}{-3mm}{$\frac{5}{6}$};
          \rlabel{C}{CA}{ 3mm}{$\frac{1}{6}$};
          \rlabel{C}{CB}{-3mm}{$\frac{5}{6}$};
          \rlabel{D}{DA}{ 3mm}{$\frac{1}{6}$};
          \rlabel{D}{DB}{-3mm}{$\frac{5}{6}$};
          \rlabel{E}{EA}{-2mm}{$\frac{6}{6}$};
        \end{scope}
      \end{tikzpicture}
    \end{center}
    Es müssen also vier Pfadwahrscheinlichkeiten
    mit der Pfadregel (Produktregel) berechnet
    und anschließend mit der Summenregel
    zusammengefasst werden:
    \begin{equation*}
      P(\text{\glqq Augensumme 5\grqq})
      =
      \frac{1}{6}\cdot\frac{1}{6}
      +
      \frac{1}{6}\cdot\frac{1}{6}
      +
      \frac{1}{6}\cdot\frac{1}{6}
      +
      \frac{1}{6}\cdot\frac{1}{6}
      =
      \frac{4}{36}
      =
      \frac{1}{9}
      \approx
      \pc{11.1}
    \end{equation*}
    % </OUTCOME>
  \fi
\end{exercise}

\begin{exercise}
      {ID-635ab33cfc36b9e3d9bdd5592ce591522c736bb5}
      {Farbenblind}
  \ifproblem\problem\par
    % <PROBLEM>
    In einer Bevölkerung sind \pc{4} aller Männer farbenblind. Wie groß muss
    eine Gruppe von Männern mindestens sein, damit man mit einer Wahrscheinlichkeit von
    mindestens \pc{50} davon ausgehen kann, dass mindestens ein farbenblinder Mann dabei ist?
    % </PROBLEM>
  \fi
  %\ifoutline\outline\par
    % <OUTLINE>
    % </OUTLINE>
  %\fi
  \ifoutcome\outcome\par
    % <OUTCOME>
    In dieser Aufgabe ist die Höhe des Baumdiagramms gesucht.
    Die folgende Abbildung zeigt den Anfang eines möglichen
    Baumdiagramms. Farbenblinde Männer sind mit einem
    \glqq f\grqq{} gekennzeichnet, normalsichtige mit
    einem \glqq n\grqq.
    \begin{center}
      %<OCTAVE>
      \begin{tikzpicture}[line width=0.6pt]
        % tree
        \begin{scope}
          % some default colors
          \newcommand{\colr}{Red};%
          \newcommand{\colg}{ForestGreen};%
          \newcommand{\colb}{Cerulean};%
          \newcommand{\coly}{YellowOrange};%
          \newcommand{\cola}{Black!35!White};%
          \newcommand{\cole}{Black!55!White};%
          % size settings
          \newcommand{\radius}{3mm}%
          \newcommand{\xscale}{3}%
          \newcommand{\yscale}{4}%
          % background color of nodes
          \newcommand{\colora}{white}%
          \newcommand{\colorb}{white}%
          % default node text
          \newcommand{\ntexta}{f}%
          \newcommand{\ntextb}{n}%
          % default edge text
          \newcommand{\etexta}{$\frac{1}{2}$}%
          \newcommand{\etextb}{$\frac{1}{2}$}%
          % geometry
          \coordinate (Z)    at ( 7.500*\xscale*\radius,  4.000*\yscale*\radius);
          \coordinate (A)    at ( 3.500*\xscale*\radius,  3.000*\yscale*\radius);
          \coordinate (B)    at (11.500*\xscale*\radius,  3.000*\yscale*\radius);
          \coordinate (AA)   at ( 1.500*\xscale*\radius,  2.000*\yscale*\radius);
          \coordinate (AB)   at ( 5.500*\xscale*\radius,  2.000*\yscale*\radius);
          \coordinate (BA)   at ( 9.500*\xscale*\radius,  2.000*\yscale*\radius);
          \coordinate (BB)   at (13.500*\xscale*\radius,  2.000*\yscale*\radius);
          \coordinate (AAA)  at ( 0.500*\xscale*\radius,  1.000*\yscale*\radius);
          \coordinate (AAB)  at ( 2.500*\xscale*\radius,  1.000*\yscale*\radius);
          \coordinate (ABA)  at ( 4.500*\xscale*\radius,  1.000*\yscale*\radius);
          \coordinate (ABB)  at ( 6.500*\xscale*\radius,  1.000*\yscale*\radius);
          \coordinate (BAA)  at ( 8.500*\xscale*\radius,  1.000*\yscale*\radius);
          \coordinate (BAB)  at (10.500*\xscale*\radius,  1.000*\yscale*\radius);
          \coordinate (BBA)  at (12.500*\xscale*\radius,  1.000*\yscale*\radius);
          \coordinate (BBB)  at (14.500*\xscale*\radius,  1.000*\yscale*\radius);
          \coordinate (AAAA) at ( 0.000*\xscale*\radius,  0.000*\yscale*\radius);
          \coordinate (AAAB) at ( 1.000*\xscale*\radius,  0.000*\yscale*\radius);
          \coordinate (AABA) at ( 2.000*\xscale*\radius,  0.000*\yscale*\radius);
          \coordinate (AABB) at ( 3.000*\xscale*\radius,  0.000*\yscale*\radius);
          \coordinate (ABAA) at ( 4.000*\xscale*\radius,  0.000*\yscale*\radius);
          \coordinate (ABAB) at ( 5.000*\xscale*\radius,  0.000*\yscale*\radius);
          \coordinate (ABBA) at ( 6.000*\xscale*\radius,  0.000*\yscale*\radius);
          \coordinate (ABBB) at ( 7.000*\xscale*\radius,  0.000*\yscale*\radius);
          \coordinate (BAAA) at ( 8.000*\xscale*\radius,  0.000*\yscale*\radius);
          \coordinate (BAAB) at ( 9.000*\xscale*\radius,  0.000*\yscale*\radius);
          \coordinate (BABA) at (10.000*\xscale*\radius,  0.000*\yscale*\radius);
          \coordinate (BABB) at (11.000*\xscale*\radius,  0.000*\yscale*\radius);
          \coordinate (BBAA) at (12.000*\xscale*\radius,  0.000*\yscale*\radius);
          \coordinate (BBAB) at (13.000*\xscale*\radius,  0.000*\yscale*\radius);
          \coordinate (BBBA) at (14.000*\xscale*\radius,  0.000*\yscale*\radius);
          \coordinate (BBBB) at (15.000*\xscale*\radius,  0.000*\yscale*\radius);
          % edges
          \draw   (Z) -- (A);
          \draw   (Z) -- (B);
          \draw   (A) -- (AA);
          \draw   (A) -- (AB);
          \draw   (B) -- (BA);
          \draw   (B) -- (BB);
          \draw  (AA) -- (AAA);
          \draw  (AA) -- (AAB);
          \draw  (AB) -- (ABA);
          \draw  (AB) -- (ABB);
          \draw  (BA) -- (BAA);
          \draw  (BA) -- (BAB);
          \draw  (BB) -- (BBA);
          \draw  (BB) -- (BBB);
          \draw (AAA) -- (AAAA);
          \draw (AAA) -- (AAAB);
          \draw (AAB) -- (AABA);
          \draw (AAB) -- (AABB);
          \draw (ABA) -- (ABAA);
          \draw (ABA) -- (ABAB);
          \draw (ABB) -- (ABBA);
          \draw (ABB) -- (ABBB);
          \draw (BAA) -- (BAAA);
          \draw (BAA) -- (BAAB);
          \draw (BAB) -- (BABA);
          \draw (BAB) -- (BABB);
          \draw (BBA) -- (BBAA);
          \draw (BBA) -- (BBAB);
          \draw (BBB) -- (BBBA);
          \draw (BBB) -- (BBBB);
          % root
          \fill[fill=black] (Z) circle[radius=2pt];
          % nodes
          \filldraw[fill=\colora, draw=black] (A)    circle[radius=\radius] node{\ntexta};
          \filldraw[fill=\colorb, draw=black] (B)    circle[radius=\radius] node{\ntextb};
          \filldraw[fill=\colora, draw=black] (AA)   circle[radius=\radius] node{\ntexta};
          \filldraw[fill=\colorb, draw=black] (AB)   circle[radius=\radius] node{\ntextb};
          \filldraw[fill=\colora, draw=black] (BA)   circle[radius=\radius] node{\ntexta};
          \filldraw[fill=\colorb, draw=black] (BB)   circle[radius=\radius] node{\ntextb};
          \filldraw[fill=\colora, draw=black] (AAA)  circle[radius=\radius] node{\ntexta};
          \filldraw[fill=\colorb, draw=black] (AAB)  circle[radius=\radius] node{\ntextb};
          \filldraw[fill=\colora, draw=black] (ABA)  circle[radius=\radius] node{\ntexta};
          \filldraw[fill=\colorb, draw=black] (ABB)  circle[radius=\radius] node{\ntextb};
          \filldraw[fill=\colora, draw=black] (BAA)  circle[radius=\radius] node{\ntexta};
          \filldraw[fill=\colorb, draw=black] (BAB)  circle[radius=\radius] node{\ntextb};
          \filldraw[fill=\colora, draw=black] (BBA)  circle[radius=\radius] node{\ntexta};
          \filldraw[fill=\colorb, draw=black] (BBB)  circle[radius=\radius] node{\ntextb};
          \filldraw[fill=\colora, draw=black] (AAAA) circle[radius=\radius] node{\ntexta};
          \filldraw[fill=\colorb, draw=black] (AAAB) circle[radius=\radius] node{\ntextb};
          \filldraw[fill=\colora, draw=black] (AABA) circle[radius=\radius] node{\ntexta};
          \filldraw[fill=\colorb, draw=black] (AABB) circle[radius=\radius] node{\ntextb};
          \filldraw[fill=\colora, draw=black] (ABAA) circle[radius=\radius] node{\ntexta};
          \filldraw[fill=\colorb, draw=black] (ABAB) circle[radius=\radius] node{\ntextb};
          \filldraw[fill=\colora, draw=black] (ABBA) circle[radius=\radius] node{\ntexta};
          \filldraw[fill=\colorb, draw=black] (ABBB) circle[radius=\radius] node{\ntextb};
          \filldraw[fill=\colora, draw=black] (BAAA) circle[radius=\radius] node{\ntexta};
          \filldraw[fill=\colorb, draw=black] (BAAB) circle[radius=\radius] node{\ntextb};
          \filldraw[fill=\colora, draw=black] (BABA) circle[radius=\radius] node{\ntexta};
          \filldraw[fill=\colorb, draw=black] (BABB) circle[radius=\radius] node{\ntextb};
          \filldraw[fill=\colora, draw=black] (BBAA) circle[radius=\radius] node{\ntexta};
          \filldraw[fill=\colorb, draw=black] (BBAB) circle[radius=\radius] node{\ntextb};
          \filldraw[fill=\colora, draw=black] (BBBA) circle[radius=\radius] node{\ntexta};
          \filldraw[fill=\colorb, draw=black] (BBBB) circle[radius=\radius] node{\ntextb};
          % label macros
          \newcommand{\rlabel}[4]%
          {%
            \coordinate (TEMP) at ($(#1)!0.5!(#2)$);
            \coordinate (TEMP) at ($(TEMP)!#3!270:(#2)$);
            \node at (TEMP) {#4};
          }%
          \newcommand{\llabel}[4]{\rlabel{#2}{#1}{#3}{#4}};
          % edge labels
          \rlabel{Z}{A}{3mm}{\pc{4}};
          \llabel{Z}{B}{3mm}{\pc{96}};
          \rlabel{A}{AA}{4.5mm}{\rule{0pt}{2.5ex}\pc{4}};
          \llabel{A}{AB}{4.5mm}{\rule{0pt}{2.5ex}\pc{96}};
          \rlabel{B}{BA}{4.5mm}{\rule{0pt}{2.5ex}\pc{4}};
          \llabel{B}{BB}{4.5mm}{\rule{0pt}{2.5ex}\pc{96}};
          \rlabel{AA}{AAA}{3mm}{};
          \llabel{AA}{AAB}{3mm}{};
          \rlabel{AB}{ABA}{3mm}{};
          \llabel{AB}{ABB}{3mm}{};
          \rlabel{BA}{BAA}{3mm}{};
          \llabel{BA}{BAB}{3mm}{};
          \rlabel{BB}{BBA}{3mm}{};
          \llabel{BB}{BBB}{3mm}{};
          \rlabel{AAA}{AAAA}{3mm}{};
          \llabel{AAA}{AAAB}{3mm}{};
          \rlabel{AAB}{AABA}{3mm}{};
          \llabel{AAB}{AABB}{3mm}{};
          \rlabel{ABA}{ABAA}{3mm}{};
          \llabel{ABA}{ABAB}{3mm}{};
          \rlabel{ABB}{ABBA}{3mm}{};
          \llabel{ABB}{ABBB}{3mm}{};
          \rlabel{BAA}{BAAA}{3mm}{};
          \llabel{BAA}{BAAB}{3mm}{};
          \rlabel{BAB}{BABA}{3mm}{};
          \llabel{BAB}{BABB}{3mm}{};
          \rlabel{BBA}{BBAA}{3mm}{};
          \llabel{BBA}{BBAB}{3mm}{};
          \rlabel{BBB}{BBBA}{3mm}{};
          \llabel{BBB}{BBBB}{3mm}{};
          % path numbers
          \node[below=\radius] at (AAAA) {$\vdots$};
          \node[below=\radius] at (AAAB) {$\vdots$};
          \node[below=\radius] at (AABA) {$\vdots$};
          \node[below=\radius] at (AABB) {$\vdots$};
          \node[below=\radius] at (ABAA) {$\vdots$};
          \node[below=\radius] at (ABAB) {$\vdots$};
          \node[below=\radius] at (ABBA) {$\vdots$};
          \node[below=\radius] at (ABBB) {$\vdots$};
          \node[below=\radius] at (BAAA) {$\vdots$};
          \node[below=\radius] at (BAAB) {$\vdots$};
          \node[below=\radius] at (BABA) {$\vdots$};
          \node[below=\radius] at (BABB) {$\vdots$};
          \node[below=\radius] at (BBAA) {$\vdots$};
          \node[below=\radius] at (BBAB) {$\vdots$};
          \node[below=\radius] at (BBBA) {$\vdots$};
          \node[below=\radius] at (BBBB) {$\vdots$};
          \begin{scope}%[line width=0.6pt]
            \draw[decorate, decoration=brace]
            ([xshift=\radius, yshift=-4.0*\radius]BBBA)
            -- node[below=3pt] {diese Pfade enthalten alle mindestens ein \glqq f\grqq}
            ([xshift=-\radius, yshift=-4.0*\radius]AAAA);
          \end{scope}
        \end{scope}
      \end{tikzpicture}
      %</OCTAVE>
      %mytree(2,4)
    \end{center}
    Wenn in der Gruppe mindestens ein farbenblinder
    Mann dabei sein soll, dann führt nur der Pfad,
    der ausschließlich aus normalsichtigen Mannern
    besteht, nicht zum Ziel.
    Sobald also die Pfadwahrscheinlichkeit dieses Pfades
    auf \pc{50} oder etwas darunter gesunken ist,
    befindet sich in der Gruppe mit mindestens
    \pc{50}-iger Wahrscheinlichkeit mindestens ein
    farbenblinder Mann.
    \begin{equation*}
      \begin{split}
        P(\text{nnn}\ldots)&=\frac{96}{100}\cdot\frac{96}{100}\cdot\frac{96}{100}\cdot\;\cdots
        \\[1ex]
        \left(\frac{96}{100}\right)^{16}&\approx\num{0.520403}
        \\[1ex]
        \left(\frac{96}{100}\right)^{17}&\approx\num{0.499587}
        %for i = 15:18
          %printf("i=%d, %8.6f\n", i, (96/100)^i)
        %endfor
        %log(0.5)/log(0.96)
      \end{split}
    \end{equation*}
    Wenn man 17 Männer aus der Bevölkerung auswählt,
    dann ist die Wahrscheinlichkeit nur normalsichtige
    Männer ausgewählt zu haben auf ca. \pc{50} gesunken.
    Also ist mit \pc{50}-iger Wahrscheinlichkeit
    mindestens ein farbenblinder Mann dabei.
    % --------------------
    \paragraph{Bemerkung:}
    % --------------------
    Eigentlich handelt es sich bei diesem
    Auswahlprozess um ein Zufallsexperiment vom Typ
    \emph{Ziehen ohne Zurücklegen}, also um ein
    Expriment, bei dem sich in jedem Schritt die
    Wahrscheinlichkeiten ändern.
    Man geht aber davon aus, dass die Bevölkerung
    so groß ist, dass die Auswahl von 17 Männern
    noch keinen maßgeblichen Einfluss auf die
    bedingten Wahrscheinlichkeiten hat.
    % </OUTCOME>
  \fi
\end{exercise}

\begin{exercise}
      {ID-e0e8898a4b0b9b417ebeac7215d75df7773c6163}
      {Ballett}
  \ifproblem\problem\par
    Von den 500 Schülerinnen und Schülern einer Schule sind 350 Mädchen.
    Für Ballett interessieren sich \pc{80} der Mädchen, aber nur \pc{10}
    der Jungen. Eine Freikarte für eine Ballettaufführung wird ausgelost.
    Mit welcher Wahrscheinlichkeit erhält sie
    \begin{enumerate}[a)]
      \item ein ballettinteressiertes Mädchen?
      \item ein an Ballett uninteressiertes Mädchen?
      \item ein ballettinteressierter Junge?
      \item ein an Ballett uninteressierter Junge?
    \end{enumerate}
  \fi
  %\ifoutline\outline\par
  %\fi
  \ifoutcome\outcome\par
    \begin{center}
      \begin{tikzpicture}
        \newcommand{\result}[3]
        {%
          \begin{scope}[xshift=#1, yshift=#2]
            \filldraw[fill=white, draw=black, line width=0.8pt] (0, 0) circle[radius=5mm];
            \node at (0, 0) {#3};
          \end{scope}
        }%
        \node[right] at (5,  2.25) {$\frac{150}{500}\cdot\frac{10}{100}=\frac{3}{100}$};
        \node[right] at (5,  0.75) {$\frac{150}{500}\cdot\frac{90}{100}=\frac{27}{100}$};
        \node[right] at (5, -0.75) {$\frac{350}{500}\cdot\frac{80}{100}=\frac{56}{100}$};
        \node[right] at (5, -2.25) {$\frac{350}{500}\cdot\frac{20}{100}=\frac{14}{100}$};
        % Kanten
          \draw[line width=0.8pt] (2.0,  1.5) -- node[pos=0.45, above=1mm] {{\small\pc{10}}}   (4.00,  2.25);
          \draw[line width=0.8pt] (2.0,  1.5) -- node[pos=0.45, below=1mm] {{\small\pc{90}}}   (4.00,  0.75);
        \draw[line width=0.8pt]   (0.0,  0.0) -- node[above left]          {$\frac{150}{500}$} (2.00,  1.50);
        \draw[line width=0.8pt]   (0.0,  0.0) -- node[below left]          {$\frac{350}{500}$} (2.00, -1.50);
          \draw[line width=0.8pt] (2.0, -1.5) -- node[pos=0.45, above=1mm] {{\small\pc{80}}}   (4.00, -0.75);
          \draw[line width=0.8pt] (2.0, -1.5) -- node[pos=0.45, below=1mm] {{\small\pc{20}}}   (4.00, -2.25);
        % Start
        \fill (0, 0) circle[radius=0.75mm];
        % 1. Ebene
        \result{2cm}{ 1.5cm}{J}
        \result{2cm}{-1.5cm}{M}
        % 2. Ebene
        \result{4cm}{ 2.25cm}{I}
        \result{4cm}{ 0.75cm}{U}
        \result{4cm}{-0.75cm}{I}
        \result{4cm}{-2.25cm}{U}
      \end{tikzpicture}
    \end{center}

    \begin{equation*}
      \begin{split}
        P(\text{\glqq ballettinteressiertes Mädchen\grqq})   &= \frac{56}{100} = \pc{56} \\[2ex]
        P(\text{\glqq ballettuninteressiertes Mädchen\grqq}) &= \frac{14}{100} = \pc{14} \\[2ex]
        P(\text{\glqq ballettinteressierter Junge\grqq})     &= \frac{3}{100}  = \pc{3}  \\[2ex]
        P(\text{\glqq ballettuninteressierter Junge\grqq})   &= \frac{27}{100} = \pc{27}
      \end{split}
    \end{equation*}
  \fi
\end{exercise}

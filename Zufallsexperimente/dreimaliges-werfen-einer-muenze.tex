\begin{exercise}
      {ID-750209decdb850a379dd956a4ed0766b567921ed}
      {Dreimaliges Werfen einer Münze}
  \ifproblem\problem\par
    % <PROBLEM>
    Eine Münze wird dreimal geworfen. Zeichne das Baumdiagramm und bestimme
    die Wahrscheinlichkeit für folgende Ereignisse:
    \begin{equation*}
      \begin{split}
        E_1&:\;\text{\glqq mehr als zweimal Wappen\grqq} \\
        E_2&:\;\text{\glqq höchstens zweimal Wappen\grqq} \\
        E_3&:\;\text{\glqq mindestens einmal Zahl\grqq} \\
        E_4&:\;\text{\glqq genau einmal Wappen\grqq}
      \end{split}
    \end{equation*}
    % </PROBLEM>
  \fi
  %\ifoutline\outline\par
    % <OUTLINE>
    % </OUTLINE>
  %\fi
  \ifoutcome\outcome
    % <OUTCOME>
    \begin{center}
      %<OCTAVE>
      \begin{tikzpicture}[line width=0.6pt]
        % tree
        \begin{scope}
          % some default colors
          \newcommand{\colr}{Red};%
          \newcommand{\colg}{ForestGreen};%
          \newcommand{\colb}{Cerulean};%
          \newcommand{\coly}{YellowOrange};%
          \newcommand{\cola}{Black!35!White};%
          \newcommand{\cole}{Black!55!White};%
          % size settings
          \newcommand{\radius}{3mm}%
          \newcommand{\xscale}{4}%
          \newcommand{\yscale}{4}%
          % background color of nodes
          \newcommand{\colora}{white}%
          \newcommand{\colorb}{white}%
          % default node text
          \newcommand{\ntexta}{W}%
          \newcommand{\ntextb}{Z}%
          % default edge text
          \newcommand{\etexta}{$\frac{1}{2}$}%
          \newcommand{\etextb}{$\frac{1}{2}$}%
          % geometry
          \coordinate (Z)   at ( 3.500*\xscale*\radius,  3.000*\yscale*\radius);
          \coordinate (A)   at ( 1.500*\xscale*\radius,  2.000*\yscale*\radius);
          \coordinate (B)   at ( 5.500*\xscale*\radius,  2.000*\yscale*\radius);
          \coordinate (AA)  at ( 0.500*\xscale*\radius,  1.000*\yscale*\radius);
          \coordinate (AB)  at ( 2.500*\xscale*\radius,  1.000*\yscale*\radius);
          \coordinate (BA)  at ( 4.500*\xscale*\radius,  1.000*\yscale*\radius);
          \coordinate (BB)  at ( 6.500*\xscale*\radius,  1.000*\yscale*\radius);
          \coordinate (AAA) at ( 0.000*\xscale*\radius,  0.000*\yscale*\radius);
          \coordinate (AAB) at ( 1.000*\xscale*\radius,  0.000*\yscale*\radius);
          \coordinate (ABA) at ( 2.000*\xscale*\radius,  0.000*\yscale*\radius);
          \coordinate (ABB) at ( 3.000*\xscale*\radius,  0.000*\yscale*\radius);
          \coordinate (BAA) at ( 4.000*\xscale*\radius,  0.000*\yscale*\radius);
          \coordinate (BAB) at ( 5.000*\xscale*\radius,  0.000*\yscale*\radius);
          \coordinate (BBA) at ( 6.000*\xscale*\radius,  0.000*\yscale*\radius);
          \coordinate (BBB) at ( 7.000*\xscale*\radius,  0.000*\yscale*\radius);
          % edges
          \draw  (Z) -- (A);
          \draw  (Z) -- (B);
          \draw  (A) -- (AA);
          \draw  (A) -- (AB);
          \draw  (B) -- (BA);
          \draw  (B) -- (BB);
          \draw (AA) -- (AAA);
          \draw (AA) -- (AAB);
          \draw (AB) -- (ABA);
          \draw (AB) -- (ABB);
          \draw (BA) -- (BAA);
          \draw (BA) -- (BAB);
          \draw (BB) -- (BBA);
          \draw (BB) -- (BBB);
          % root
          \fill[fill=black] (Z) circle[radius=2pt];
          % nodes
          \filldraw[fill=\colora, draw=black] (A)   circle[radius=\radius] node{\ntexta};
          \filldraw[fill=\colorb, draw=black] (B)   circle[radius=\radius] node{\ntextb};
          \filldraw[fill=\colora, draw=black] (AA)  circle[radius=\radius] node{\ntexta};
          \filldraw[fill=\colorb, draw=black] (AB)  circle[radius=\radius] node{\ntextb};
          \filldraw[fill=\colora, draw=black] (BA)  circle[radius=\radius] node{\ntexta};
          \filldraw[fill=\colorb, draw=black] (BB)  circle[radius=\radius] node{\ntextb};
          \filldraw[fill=\colora, draw=black] (AAA) circle[radius=\radius] node{\ntexta};
          \filldraw[fill=\colorb, draw=black] (AAB) circle[radius=\radius] node{\ntextb};
          \filldraw[fill=\colora, draw=black] (ABA) circle[radius=\radius] node{\ntexta};
          \filldraw[fill=\colorb, draw=black] (ABB) circle[radius=\radius] node{\ntextb};
          \filldraw[fill=\colora, draw=black] (BAA) circle[radius=\radius] node{\ntexta};
          \filldraw[fill=\colorb, draw=black] (BAB) circle[radius=\radius] node{\ntextb};
          \filldraw[fill=\colora, draw=black] (BBA) circle[radius=\radius] node{\ntexta};
          \filldraw[fill=\colorb, draw=black] (BBB) circle[radius=\radius] node{\ntextb};
          % label macros
          \newcommand{\rlabel}[4]%
          {%
            \coordinate (TEMP) at ($(#1)!0.5!(#2)$);
            \coordinate (TEMP) at ($(TEMP)!#3!270:(#2)$);
            \node at (TEMP) {#4};
          }%
          \newcommand{\llabel}[4]{\rlabel{#2}{#1}{#3}{#4}};
          % edge labels
          \rlabel{Z}{A}{3mm}{\etexta};
          \llabel{Z}{B}{3mm}{\etextb};
          \rlabel{A}{AA}{3mm}{\etexta};
          \llabel{A}{AB}{3mm}{\etextb};
          \rlabel{B}{BA}{3mm}{\etexta};
          \llabel{B}{BB}{3mm}{\etextb};
          \rlabel{AA}{AAA}{3mm}{\etexta};
          \llabel{AA}{AAB}{3mm}{\etextb};
          \rlabel{AB}{ABA}{3mm}{\etexta};
          \llabel{AB}{ABB}{3mm}{\etextb};
          \rlabel{BA}{BAA}{3mm}{\etexta};
          \llabel{BA}{BAB}{3mm}{\etextb};
          \rlabel{BB}{BBA}{3mm}{\etexta};
          \llabel{BB}{BBB}{3mm}{\etextb};
        \end{scope}
      \end{tikzpicture}
      %</OCTAVE>
      %mytree(2, 3)
    \end{center}
    \begin{equation*}
      \begin{split}
        E_1&=\text{\glqq mehr als zweimal Wappen\grqq}
        =\{(\text{WWW})\}
        \\[1ex]
        P(E_1)&=\frac{1}{2}\cdot\frac{1}{2}\cdot\frac{1}{2}
        =\frac{1}{8}
        =\pc{12.5}
        \\[3ex]
        E_2&=\text{\glqq höchstens zweimal Wappen\grqq}
        \\
        &=\{
          (\text{WWZ});
          (\text{WZW});
          (\text{WZZ});
          (\text{ZWW});
          (\text{ZWZ});
          (\text{ZZW});
          (\text{ZZZ})
         \}
        \\
        &=\Omega\setminus\{(\text{WWW})\}
        \\[1ex]
        P(E_2)&=\frac{1}{2}\cdot\frac{1}{2}\cdot\frac{1}{2}\cdot7
        =\frac{7}{8}
        =\pc{87.5}
        \\[1ex]
        &=1-\frac{1}{2}\cdot\frac{1}{2}\cdot\frac{1}{2}
        =1-\frac{1}{8}
        =\frac{7}{8}
        =\pc{87.5}
        \\[3ex]
        E_3&=\text{\glqq mindestens einmal Zahl\grqq}
        \\
        &=\{
          (\text{WWZ});
          (\text{WZW});
          (\text{WZZ});
          (\text{ZWW});
          (\text{ZWZ});
          (\text{ZZW});
          (\text{ZZZ})
         \}
        \\
        &=\Omega\setminus\{(\text{WWW})\}
        \\[1ex]
        P(E_3)&=\frac{1}{2}\cdot\frac{1}{2}\cdot\frac{1}{2}\cdot7
        =\frac{7}{8}
        =\pc{87.5}
        \\[1ex]
        &=1-\frac{1}{2}\cdot\frac{1}{2}\cdot\frac{1}{2}
        =1-\frac{1}{8}
        =\frac{7}{8}
        =\pc{87.5}
        \\[3ex]
        E_4&=\text{\glqq genau einmal Wappen\grqq}
        =\{(\text{WZZ});(\text{ZWZ});(\text{ZZW}))\}
        \\[1ex]
        P(E_4)&=\frac{1}{2}\cdot\frac{1}{2}\cdot\frac{1}{2}\cdot3
        =\frac{3}{8}
        =\pc{37.5}
      \end{split}
    \end{equation*}
    % </OUTCOME>
  \fi
\end{exercise}

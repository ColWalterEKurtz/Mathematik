\begin{exercise}
      {ID-aeb06ea89571cbb9ea16bcfab3b10f98b697785e}
      {Zwei Kneipen}
  \ifproblem\problem\par
    % <PROBLEM>
    In einem Dorf gibt es zwei Kneipen.
    Zu Beginn gehen \pc{30} regelmäßig in die
    Kneipe A, \pc{25} in die Kneipe B und
    \pc{45} gehen in keine der beiden Kneipen.
    Die folgende Übergangstabelle zeigt das
    Wechselverhalten der Kneipenbesucher pro
    Monat.
    \begin{center}
      \begingroup
        \setlength{\tabcolsep}{0.75em}%
        \renewcommand{\arraystretch}{1.2}%
        \begin{tabular}{|c|c|c|c|}
          \hline
                 & von A      & von B     & von K     \\
          \hline
          nach A & \num{0.4}  & \num{0.2} & \num{0.1} \\
          \hline
          nach B & \num{0.25} & \num{0.6} & \num{0.2} \\
          \hline
          nach K & \num{0.35} & \num{0.2} & \num{0.7} \\
          \hline
        \end{tabular}
      \endgroup
    \end{center}
    \begin{enumerate}[a)]
      %\setlength{\itemsep}{-1ex}%
      \item Erstellen Sie ein Übergangsdiagramm, das das
            Wechselverhalten der Kneipenbesucher beschreibt.
      \item Bestimmen Sie die Übergangsmatrix für das
            Wechselverhelten der Kneipenbesucher.
      \item Geben Sie die Anfangsverteilung als Zustandsvektor an.
      \item Berechnen Sie die Verteilung der Kneipenbesucher
            nach einem Monat und nach zwei Monaten.
    \end{enumerate}
    % </PROBLEM>
  \fi
  %\ifoutline\outline\par
    % <OUTLINE>
    % </OUTLINE>
  %\fi
  %\ifoutcome\outcome\par
    % <OUTCOME>
    % </OUTCOME>
  %\fi
\end{exercise}

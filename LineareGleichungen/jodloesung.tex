\begin{exercise}
      {ID-d7970272ce1b80adf244acbf13201b476eef14b7}
      {Jodlösung}
  \ifproblem\problem\par
    Eine Lösung von \pc{16} Jod in Alkohol wiegt \sig{735}.
    Es soll eine \pc{10}-ige Jodlösung hergestellt werden.
    Wie viel Gramm Alkohol muss man hinzufügen?
  \fi
  \ifoutline\outline\par
    Die absolute Menge des Jod bleibt in beiden Lösungen gleich\ldots
  \fi
  \ifoutcome\outcome\par
    Die Variable $x$ steht für die Menge Alkohol in Gramm, die man hinzufügen
    muss, um den Jodanteil der Lösung auf \pc{10} zu senken.
    \begin{equation*}
      \num{0.16}\cdot 735=\num{0.1}\cdot(735+x)
      \quad\Rightarrow\quad
      x=441
    \end{equation*}
  \fi
\end{exercise}

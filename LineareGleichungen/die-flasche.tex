\begin{exercise}
      {ID-f6c038f6558edd5ad9fb7e0aa8052eb9f5650393}
      {Die Flasche}
  \ifproblem\problem\par
    Ein Sportler schwimmt gegen den Strom in einem Fluss und verliert unter
    einer Brücke eine Flasche. Ohne den Verlust zu bemerken schwimmt er
    zunächst noch eine viertel Stunde weiter. Als er bemerkt, dass er seine
    Flasche verloren hat, schwimmt er zurück und holt sie \sikm{2} von der
    Brücke entfernt wieder ein. Ermittle die Strömungsgeschwindigkeit des
    Flusses.
  \fi
  \ifoutline\outline\par
    Die (abhängige) Variable $d$ steht für die (orientierte) Entfernung
    zur Brücke. Die Zeit in Minuten wird mit $t$ bezeichnet.
    \begin{center}
      \begin{tikzpicture}
        \coordinate (A) at (3.0,  2.5);
        \coordinate (B) at (6.0, -3.0);
        \draw[->, >=latex] (-1, 0) -- (8, 0) node[below]{$t$};
        \draw[->, >=latex] (0, -3) -- (0, 3) node[left=1mm]{$d$};
        \draw (3.0, 1mm) -- (3.0, -1mm) node[below]{{\footnotesize$15$}};
        \fill (A) circle[radius=1pt];
        \fill (B) circle[radius=1pt];
        \draw (0, 0) -- node[above left]{$g_{2}$} (A) -- node[right=3pt]{$g_{3}$} (B);
        \draw (0, 0) -- node[below left]{$g_{1}$} (B);
      \end{tikzpicture}
    \end{center}
  \fi
  \ifoutcome\outcome\par
    Die (abhängige) Variable $d$ steht für die (orientierte) Entfernung
    zur Brücke. Die Zeit in Minuten wird mit $t$ bezeichnet.
    \begin{center}
      \begin{tikzpicture}
        \coordinate (A) at (3.0,  2.5);
        \coordinate (B) at (6.0, -3.0);
        \draw[->, >=latex] (-1, 0) -- (8, 0) node[below]{$t$};
        \draw[->, >=latex] (0, -3) -- (0, 3) node[left=1mm]{$d$};
        \draw (3.0, 1mm) -- (3.0, -1mm) node[below]{{\footnotesize$15$}};
        \fill (A) circle[radius=1pt];
        \fill (B) circle[radius=1pt];
        \draw (0, 0) -- node[above left]{$g_{2}$} (A) -- node[right=3pt]{$g_{3}$} (B);
        \draw (0, 0) -- node[below left]{$g_{1}$} (B);
      \end{tikzpicture}
    \end{center}
    Die Gerade $g_{1}$ beschreibt, wie sich die Flasche von der Brücke
    entfernt. Wenn man die Strömungsgeschwindigkeit des Flusses mit $s$
    bezeichnet, dann ist $g_{1}(t)=-st$. Die Gerade $g_{2}$ beschreibt,
    wie sich der Sportler von der Brücke entfernt, ohne den Verlust der
    Flasche bemerkt zu haben. Wenn man seine Geschwindigkeit (relativ
    zur Brücke) mit $v$ bezeichnet, dann ist $g_{2}(t)=(v-s)\cdot t$.
    Die Gerade $g_{3}$ beschreibt die Entfernung des Sportlers zur Brücke,
    nachdem er auf der Suche nach der Flasche umgekehrt ist. Mit den
    entsprechenden Verschiebungen ergibt sich:
    \begin{equation*}
      g_{3}(t)=-(v+s)\cdot(t-15)+15\cdot(v-s)
    \end{equation*}
    Gesucht wird jetzt der Schnittpunkt von $g_{1}$ und $g_{3}$.
    \newcommand{\aq}{\Leftrightarrow}%
    \begin{alignat*}{2}
         &\quad & -st&=-(v+s)\cdot(t-15)+15\cdot(v-s) \\
      \aq&\quad & -st&=-(vt-15v+st-15s)+15v-15s \\
      \aq&\quad & -st&=-vt+15v-st+15s+15v-15s \\
      \aq&\quad &   0&=-vt+30v=(30-t)\cdot v
    \end{alignat*}
    Wenn die Geschwindigkeit $v$ des Sportlers nicht Null ist,
    dann ergibt sich die einzige Lösung mit $t=30$. Also holt
    er die Flasche 30 Minuten nach dem Verlust unter der Brücke
    wieder ein.
    In diesen 30 Minuten hat die Flasche eine Strecke von \sikm{2}
    zurückgelegt, also fließt der Fluss mit \sikmh{4}.\par
    Beachtenswert ist die Tatsache, dass in der Lösung werder die Strömungsgeschwindigkeit
    des Flusses noch die Geschwindigkeit des Sportlers (sofern $v\neq0$) eine Rolle spielen.
    Entscheidend ist lediglich die Zeit, die bis zum Umkehren im Fluss vergeht.
  \fi
\end{exercise}

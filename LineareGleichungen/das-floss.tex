\begin{exercise}
      {ID-50c3c2e404df3294991c0c6e283d7445e34ee7e1}
      {Das Floß}
  \ifproblem\problem\par
    Ein Motorboot, das auf einem Fluss zwischen den Städten $A$ und $B$ pendelt,
    benötigt für die Fahrt flussabwärts 32 Stunden. Die Rückfahrt von $B$ nach
    $A$ dauert 48 Stunden. In wie vielen Stunden bewegt sich ein Floß von $A$
    nach $B$?
  \fi
  \ifoutline\outline\par
    Eine Gleichung ergibt sich aus den Wegen gleicher Länge -- egal, ob man
    flussaufwärts, oder flussabwärts fährt\ldots
  \fi
  \ifoutcome\outcome\par
    Mit der Variablen $v$ für die Geschwindigkeit des Motorbootes und
    der Variablen $s$ für die Strömungsgeschwindigkeit des Flusses kann
    man folgende Gleichung aufstellen:
    \begin{alignat*}{2}
                     &\quad & 32\cdot(v+s)&=48\cdot(v-s) \\
      \Leftrightarrow&\quad &      32v+32s&=48v-48s      \\
      \Leftrightarrow&\quad &          80s&=16v          \\
      \Leftrightarrow&\quad &           5s&=v
    \end{alignat*}
    Daraus ergibt sich, dass das Motorboot auf der Fahrt flussabwärts mit
    sechsfacher Strömungsgeschwindigkeit fährt. Da ein Floß nur mit
    einfacher Strömungsgeschwindigkeit \glqq fahren\grqq{} würde, bräuchte
    es die sechsfache Zeit. Es würde also $6\cdot32=192$ Stunden benötigen.
  \fi
\end{exercise}

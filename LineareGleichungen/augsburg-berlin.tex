\begin{exercise}
      {ID-0a1ac88fc6de5828c0f1babdd50ac203b2ee3761}
      {Augsburg -- Berlin}
  \ifproblem\problem\par
    Augsburg liegt etwa \sikm{600} weit von Berlin entfernt. Herr $A$ startet
    in Augsburg und fährt mit einer Durchschnittsgeschwindigkeit von \sikmh{140}
    nach Berlin. Gleichzeitig startet Herr $B$ in Berlin und fährt mit einer
    Durchschnittsgeschwindigkeit von \sikmh{110} nach Augsburg.
    Nach welcher Zeit begegnen sie sich und wie weit ist der Treffpunkt dann
    von Augsburg entfernt?
  \fi
  \ifoutline\outline\par
    Gesucht wird der Schnittpunkt zweier Geraden. Auf der $x$-Achse des Koordinatensystems
    ist die Zeit (in Stunden) und auf der $y$-Achse die Entfernung (in Kilometern) eingetragen.
    Da sich die beiden Fahrzeuge aufeinander zubewegen, muss eine der beiden
    Durchschnittsgeschwindigkeiten als negative Steigung interpretiert werden.
    \begin{center}
      \begin{tikzpicture}[scale=0.5]
        % x-Achse
        \draw[->, >=stealth] (-1, 0) -- (5, 0);
        % y-Achse
        \draw[->, >=stealth] (0, -1) -- (0, 5);
        % Start- und Endpunkte der Geraden
        \coordinate (A1) at ( -0.389,  -0.500);
        \coordinate (A2) at (  3.500,   4.500);
        \coordinate (B1) at ( -0.500,   4.000);
        \coordinate (B2) at (  4.000,  -0.500);
        \draw (A1) -- (A2);
        \draw (B1) -- (B2);
        \node[right] at (A2) {{\small$m_1=\sikmh{140}$}};
        \node[right] at (B2) {{\small$m_2=\sikmh{-110}$}};
        \draw (0.2, 3.5) -- (-0.2, 3.5) node[left] {{\small\sikm{600}}};
        \node[right] at (12, 2)
        {%
          \begin{minipage}{12em}%
            \setlength{\abovedisplayskip}{0pt}%
            \begin{equation*}
              \begin{split}
                g_1(x)&=\num{140}\cdot x \\[1ex]
                g_2(x)&=\num{-110}\cdot x+\num{600}
              \end{split}
            \end{equation*}
          \end{minipage}%
        };
      \end{tikzpicture}
    \end{center}
  \fi
  \ifoutcome\outcome\par
    \begin{equation*}
      \num{140}\cdot x=\num{-110}\cdot x+600
      \quad\Rightarrow\quad
      x=\num{2.4}
    \end{equation*}
    Die beiden Fahrzeuge begegnen sich also nach einer Fahrzeit von \num{2.4}
    Stunden und sind dann $g_1(\num{2.4})=\sikm{336}$ weit von Augsburg entfernt.
  \fi
\end{exercise}

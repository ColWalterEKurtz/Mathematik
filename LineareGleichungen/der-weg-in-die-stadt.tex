\begin{exercise}
      {ID-0a39950ace00a7c5ec3c0db4d6873aae39deb51a}
      {Der Weg in die Stadt}
  \ifproblem\problem\par
    Ein Dorf liegt \sikm{48} von der nächsten Stadt entfernt. Im Dorf starten zeitgleich
    ein Reiter auf einem Pferd und ein Briefträger auf einem Fahrrad ihren Weg in die Stadt.
    Das Pferd läuft mit einer Durchschnittsgeschwindigkeit von \sikmh{7}, der Briefträger
    schafft auf seinem Fahrrad duchschnittlich \sikmh{13}. Nach wie vielen Stunden beträgt
    der verbleibende Weg bis zur Stadt für den Briefträger nur noch ein Drittel des für
    den Reiter verbleibenden Weges?
  \fi
  \ifoutline\outline\par
    Die Variable $x$ steht für die Zeit in Stunden, die seit dem Aufbruch
    vergangen ist.
    \begin{equation*}
      48-13x=\frac{1}{3}(48-7x)
    \end{equation*}
  \fi
  \ifoutcome\outcome\par
    Die Variable $x$ steht für die Zeit in Stunden, die seit dem Aufbruch
    vergangen ist.
    \newcommand{\aq}{\Leftrightarrow}%
    \begin{alignat*}{2}
         &\quad & 48-13x&=\frac{1}{3}(48-7x) \\
      \aq&\quad & 48-13x&=16-\frac{7}{3}x \\
      \aq&\quad &     32&=\frac{32}{3}x \\
      \aq&\quad &      3&=x
    \end{alignat*}
    Nach 3 Stunden ist also der verbleibende Weg in die Stadt für den Briefträger
    nur noch ein Drittel des Weges, den der Reiter noch vor sich hat.
  \fi
\end{exercise}

\begin{exercise}
      {ID-0993bc1925a2ffbaf3da35dd3cdd103b1c84913c}
      {Vögel}
  \ifproblem\problem
    Auf drei Bäumen sitzen insgesamt \num{56} Vögel. Nachdem vom ersten Baum \num{7} auf
    den zweiten und vom zweiten \num{5} Vögel auf den dritten Baum geflogen waren,
    saßen nun auf dem zweiten Baum doppelt so viele Vögel wie auf dem ersten
    und auf dem dritten doppelt so viele Vögel wie auf dem zweiten Baum.
    Berechne, wie viele Vögel ursprünglich auf jedem der Bäume saßen.
  \fi
  \ifoutline\outline
    Nach den \glqq Flugmanövern\grqq{} der Vögel gilt folgende Gleichung:
    \begin{equation*}
      x+2x+4x=\num{56}
    \end{equation*}
    Die Variable $x$ steht hierbei für die Anzahl der Vögel auf dem ersten Baum.
  \fi
  \ifoutcome\outcome
    Ursprünglich saßen auf dem ersten Baum \num{15},
    auf dem zweiten Baum \num{14} und
    auf dem dritten Baum \num{27} Vögel.
  \fi
\end{exercise}

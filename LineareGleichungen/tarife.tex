\begin{exercise}
      {ID-f8353f7585ffc044034200db75b5d65e61dc1c9e}
      {Tarife}
  \ifproblem\problem
    Ein Elektrizitätswerk bietet folgende Tarife an:
    \begin{itemize}
      \item Tarif A: Monatlicher Grundpreis \eur{150} -- Arbeitspreis 20 Cent pro kWh.
      \item Tarif B: Monatlicher Grundpreis \eur{180} -- Arbeitspreis 15 Cent pro kWh.
    \end{itemize}
    Wie viele Kilowattstunden müssen abgenommen werden, damit es sich lohnt,
    den Tarif B zu wählen? Wie hoch ist dann die Rechnung vom Elektrizitätswerk?
  \fi
  %\ifoutline\outline
  %\fi
  %\ifoutcome\outcome
  %\fi
\end{exercise}

\begin{exercise}
      {ID-46a2e7d9897f119c30eea2bc87171079a7e5647f}
      {Altersunterschied}
  \ifproblem\problem
    Ein Vater ist \num{38} Jahre alt, sein Sohn \num{11} Jahre. Nach wie vielen Jahren ist
    der Vater genau doppelt so alt wie der Sohn?
  \fi
  \ifoutline\outline
    In folgender Gleichung steht der Parameter $v$ für das aktuelle Alter des
    Vaters und der Parameter $s$ für das aktuelle Alter des Sohnes. Die Variable
    $x$ steht für die Anzahl der Jahre, die vergehen müssen, bis der Vater genau
    doppelt so alt ist wie sein Sohn:
    \begin{equation*}
      v+x=2\cdot(s+x)
    \end{equation*}
    Die Aufgabe besteht nun darin, die Gleichung nach $x$ aufzulösen.
  \fi
  \ifoutcome\outcome
    In folgender Gleichung ist $x$ die gesuchte Größe, also muss die Gleichung
    nach $x$ aufgelöst werden:
    \begin{alignat*}{3}
                     &\quad &      v+x&=2\cdot(s+x) & \quad&                     \\
      \Leftrightarrow&\quad &      v+x&=2s+2x       & \quad&\vert-x\quad\vert-2s \\
      \Leftrightarrow&\quad &     v-2s&=x           & \quad&
    \intertext{Laut Aufgabenstellung gilt $v=\num{38}$ und $s=\num{11}$. Draus folgt:}
                     &\quad & \num{16}&=x           & \quad&
    \end{alignat*}
    Es müssen also \num{16} Jahre vergehen, bis der Vater genau doppelt so alt ist
    wie sein Sohn.
  \fi
\end{exercise}

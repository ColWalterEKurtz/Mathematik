\begin{exercise}
      {ID-2a346e833f905e4c76d6b3abe3f5fc15c0579545}
      {Bruchschreibweise}
  \ifproblem\problem\par
    % <PROBLEM>
    Schreibe folgende Dezimalzahlen als Bruch:
    % number
    \newcommand{\no}[1]
    {%
      \vphantom{\big(}%
      \makebox[3em][r]%
      {%
        #1)%
      }%
      \;\;&%
    }%
    \begin{align*}
      \no{a} 0,\!3=  & \no{e} 0,\!2=  & \no{i} 1,\!03=  \\
      \no{b} 2,\!3=  & \no{f} 2,\!1=  & \no{j} 4,\!567= \\
      \no{c} 0,\!03= & \no{g} 1,\!23= & \no{k} 0,\!007= \\
      \no{d} 0,\!25= & \no{h} 0,\!8=  & \no{l} 0,\!0101=
    \end{align*}
    % </PROBLEM>
  \fi
  %\ifoutline\outline\par
    % <OUTLINE>
    % </OUTLINE>
  %\fi
  %\ifoutcome\outcome\par
    % <OUTCOME>
    % </OUTCOME>
  %\fi
\end{exercise}

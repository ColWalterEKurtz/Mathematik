\begin{exercise}
      {ID-eabffb9a66bdbfe74fe53df681d2a50a0f2f4b12}
      {Quadrate}
  \ifproblem\problem
    \begin{center}
      \begin{tikzpicture}[scale=0.9]
        \begin{scope}
          \coordinate (A) at (0, 0);
          \coordinate (B) at (4, 0);
          \coordinate (C) at (4, 4);
          \coordinate (D) at (0, 4);
          \coordinate (E) at ($(A)!0.5!(B)$);
          \coordinate (F) at ($(B)!0.5!(C)$);
          \coordinate (G) at ($(C)!0.5!(D)$);
          \coordinate (H) at ($(D)!0.5!(A)$);
          \fill (A) circle (1pt) node[below left]  {{\small$A$}};
          \fill (B) circle (1pt) node[below right] {{\small$B$}};
          \fill (C) circle (1pt) node[above right] {{\small$C$}};
          \fill (D) circle (1pt) node[above left]  {{\small$D$}};
          \fill (E) circle (1pt) node[below]       {{\small$E$}};
          \fill (F) circle (1pt) node[right]       {{\small$F$}};
          \fill (G) circle (1pt) node[above]       {{\small$G$}};
          \fill (H) circle (1pt) node[left]        {{\small$H$}};
          \draw (A) -- (B) -- (C) -- (D) -- cycle;
          \draw (E) -- (F) -- (G) -- (H) -- cycle;
        \end{scope}
        \begin{scope}[xshift=8cm]
          \coordinate (P) at (0, 0);
          \coordinate (Q) at (4, 0);
          \coordinate (R) at (4, 4);
          \coordinate (S) at (0, 4);
          \coordinate (T) at ($(P)!0.66!(Q)$);
          \coordinate (U) at ($(Q)!0.66!(R)$);
          \coordinate (V) at ($(R)!0.66!(S)$);
          \coordinate (W) at ($(S)!0.66!(P)$);
          \fill (P) circle (1pt) node[below left]  {{\small$P$}};
          \fill (Q) circle (1pt) node[below right] {{\small$Q$}};
          \fill (R) circle (1pt) node[above right] {{\small$R$}};
          \fill (S) circle (1pt) node[above left]  {{\small$S$}};
          \fill (T) circle (1pt) node[below]       {{\small$T$}};
          \fill (U) circle (1pt) node[right]       {{\small$U$}};
          \fill (V) circle (1pt) node[above]       {{\small$V$}};
          \fill (W) circle (1pt) node[left]        {{\small$W$}};
          \draw (P) -- (Q) -- (R) -- (S) -- cycle;
          \draw (T) -- (U) -- (V) -- (W) -- cycle;
        \end{scope}
      \end{tikzpicture}
    \end{center}
    \begin{enumerate}[a)]
      \item Die Abbildung zeigt ein Quadrat $ABCD$, dem ein Quadrat $EFGH$ so
            einbeschrieben ist, dass die Eckpunkte des kleineren Quadrates $EFGH$
            genau auf den Seitenmitten des Quadrates $ABCD$ liegen. Der Flächeninhalt
            des Quadrates $ABCD$ beträgt \sicmm{6}. Ermittle den Flächeninhalt des
            Quadrates $EFGH$.
      \item Die Abbildung zeigt ein Quadrat $PQRS$, dem ein kleineres Quadrat $TUVW$
            einbeschrieben ist. Das größere Quadrat hat einen Flächeninhalt von
            \sicmm{36} und das kleinere Quadrat hat einen Flächeninhalt von \sicmm{25}.
            Ermittle den Flächeninhalt und den Umfang des Dreiecks $PTW$.
      \item Ein Quadrat habe die Seitenlänge \sicm{32}. Es werden nun die Mittelpunkte
            der vier Seiten so verbunden, dass ein einbeschriebenes Quadrat entsteht.
            Dieses Verfahren wird fortgesetzt: Es entstehen weitere, immer kleinere
            einbeschriebene Quadrate. Ermittle die Seitenlänge des zehnten, so
            einbeschriebenen Quadrates.
    \end{enumerate}
  \fi
  %\ifoutline\outline
  %\fi
  %\ifoutcome\outcome
  %\fi
\end{exercise}

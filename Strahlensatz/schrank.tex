\begin{exercise}
      {ID-98ba4f2c820b144b100e87e64003f62fcb5a79cf}
      {Schrank}
  \ifproblem\problem
    Die Wand eines Dachzimmers ist \simeter{4} breit. Sie ist auf der einen Seite
    \simeter{1.40} und auf der anderen Seite \simeter{3.50} hoch.
    Kann man an der Wand einen Schrank aufstellen, der \simeter{2.25} hoch und
    \simeter{2.40} breit ist?
  \fi
  %\ifoutline\outline
  %\fi
  \ifoutcome\outcome
    \begin{center}
      \begin{tikzpicture}
        \coordinate (A) at (0.0, 0.0);
        \coordinate (B) at (4.0, 0.0);
        \coordinate (C) at (4.0, 1.4);
        \coordinate (D) at (0.0, 3.5);
        \coordinate (E) at (0.0, 1.4);
        \coordinate (F) at (2.4, 1.4);
        \coordinate (G) at (2.4, 2.25);
        % Schrank
        \fill[fill=white!75!black] (A) rectangle node{{\footnotesize Schrank}} (G);
        % Umriss
        \draw[line width=0.6pt]
             (A) -- node[below]{\simeter{4}}
             (B) -- node[right]{\simeter{1.40}}
             (C) --
             (D) -- node[left]{\simeter{3.50}}
             cycle;
        % Strahlen
        \draw[line width=0.6pt] (C) -- ($(E)!5mm!180:(C)$);
        \draw[line width=0.6pt] (C) -- ($(D)!5mm!180:(C)$);
        % Strecke
        \draw[line width=0.6pt] (F) -- (G);
        % Punkte
        \fill[fill=black] (C) circle[radius=1.5pt] node[right]{$S$};
        \fill[fill=black] (G) circle[radius=1.5pt] node[above right]{$A$};
        \fill[fill=black] (D) circle[radius=1.5pt] node[above right]{$B$};
        \fill[fill=black] (F) circle[radius=1.5pt] node[below right]{$C$};
        \fill[fill=black] (E) circle[radius=1.5pt] node[below left]{$D$};
      \end{tikzpicture}
    \end{center}

    Wenn die Strecke $\overline{AC}+\simeter{1.40}$ mindestens \simeter{2.25}
    ergibt, dann passt der Schrank vor die Wand.

    Die Strecke $\overline{AC}$ lässt sich mit dem 2. Strahlensatz berechnen:
    \begin{equation*}
      \frac{\;\overline{AC}\;}{\overline{SC}}=\frac{\;\overline{BD}\;}{\overline{SD}}
      \quad\Rightarrow\quad
      \overline{AC}=\overline{SC}\cdot\frac{\;\overline{BD}\;}{\overline{SD}}
    \end{equation*}

    Mit den Angaben aus der Aufgabenstellung ergeben sich folgende Werte:
    \begin{equation*}
      \begin{split}
        \overline{SC}&=\simeter{4}-\simeter{2.4}=\simeter{1.6} \\
        \overline{BD}&=\simeter{3.5}-\simeter{1.4}=\simeter{2.1} \\
        \overline{SD}&=\simeter{4}
      \end{split}
    \end{equation*}

    Damit erhält man für $\overline{AC}$:
    \begin{equation*}
      \overline{AC}=\simeter{1.6}\cdot\frac{\simeter{2.1}}{\simeter{4}}
                   =\simeter{0.84}
    \end{equation*}

    An der Stelle, an der der Schrank endet, ist das Zimmer also nur
    noch \simeter{2.24} hoch. Der Schrank passt also nicht ins Zimmer.
  \fi
\end{exercise}

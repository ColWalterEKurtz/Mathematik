\begin{exercise}{trigonometrie.steigungswinkel}{Steigungswinkel}
  \ifproblem\problem\par
    \begin{enumerate}[a)]
      \item Die Steigung einer Straße wird mit \pc{18} angegeben. Berechne die Größe
            des zugehörigen Steigungswinkels $\alpha$.
      \item Ein Fahrzeug bewältigt im 1. Gang eine Steigung von bis zu \pc{44} und
            im 4. Gang eine Steigung von bis zu \pc{10}. Berechne jeweils die Größe
            des zugehörigen Steigungswinkels.
      \item Berechne für die Steigungswinkel
            $\alpha=\SI{8}{\degree}$,
            $\alpha=\SI{24}{\degree}$ und
            $\alpha=\SI{31}{\degree}$
            die Steigung in Prozent.
    \end{enumerate}
  \fi
  %\ifoutline\outline\par
  %\fi
  %\ifoutcome\outcome\par
  %\fi
\end{exercise}

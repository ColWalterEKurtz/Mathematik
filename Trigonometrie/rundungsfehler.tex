\begin{exercise}{trigonometrie.rundungsfehler}{Rundungsfehler}
  \ifproblem\problem
    Um die Höhe $h$ einer Wand zu bestimmen, wurde im Abstand
    $a=\SI{3.64}{\metre}$ der Winkel $\alpha=\SI{32}{\degree}$
    gemessen. Abstand und Winkel wurden auf die angegebenen
    Werte gerundet. Bestimme den kleinsten und den größten
    Wert für die Höhe der Wand, die sich aus den gerundeten
    Messgrößen ergeben können.
    \begin{center}
      \begin{tikzpicture}
        % Info
        % a       =     2.2745
        % b       =     4.2922
        % c       =     3.6400
        % alpha   =    32.0000
        % beta    =    90.0000
        % gamma   =    58.0000
        % ha      =     3.6400
        % hb      =     1.9289
        % hc      =     2.2745
        % flaeche =     4.1396
        % Koordinaten der Eckpunkte
        \coordinate (A) at (0.0000, 0.0000);
        \coordinate (B) at (3.6400, 0.0000);
        \coordinate (C) at (3.6400, 2.2745);
        % Koordinaten der Seitenmittelpunkte
        \coordinate (MAB) at (1.8200, 0.0000);
        \coordinate (MAC) at (1.8200, 1.1373);
        \coordinate (MBC) at (3.6400, 1.1373);
        % Koordinaten des Inkreismittelpunktes (r = 0.8112)
        \coordinate (I) at (2.8288, 0.8112);
        % Seiten des Dreiecks
        \draw[line width=0.6pt] (A) -- (B) -- (C) -- cycle;
        % Eckpunkte
        %\fill (A) circle[radius=1pt];
        %\fill (B) circle[radius=1pt];
        %\fill (C) circle[radius=1pt];
        % Beschriftung der Seiten
        \node at ($(MAB)!3mm!270:(B)$) {$a$};
        \node at ($(MBC)!3mm!270:(C)$) {$h$};
        % Beschriftung der Winkel
        \begin{scope}
          \clip (A) -- (B) -- (C) -- cycle;
          \draw (A) circle[radius=1];
          \node at ($(A)!6mm!(I)$) {{\small$\alpha$}};
          \draw (B) circle[radius=0.5];
          \fill ($(B)!2.5mm!(I)$) circle[radius=1pt];
        \end{scope}
      \end{tikzpicture}
    \end{center}
  \fi
  %\ifoutline\outline
  %\fi
  %\ifoutcome\outcome
  %\fi
\end{exercise}

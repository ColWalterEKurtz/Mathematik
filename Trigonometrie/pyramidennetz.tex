\begin{exercise}{trigonometrie.pyramidennetz}{Pyramidennetz}
  \ifproblem\problem\par
    Das sternförmige Netz einer vierseitigen Pyramide besteht
    aus einem Quadrat mit der Seite $a$ und vier angesetzten
    gleichschenkligen Dreiecken. Wie groß müssen die Basiswinkel
    dieser gleichschenkligen Dreiecke gewählt werden, damit
    nach dem Auffalten des Netzes eine Pyramide entsteht, in
    deren Spitze die gegenüberliegenden Seitenflächen einen
    Winkel von \SI{60}{\degree} einschließen?
  \fi
  \ifoutline\outline\par
    \begin{minipage}{0.4\textwidth}
      \centering
      \begin{tikzpicture}
        \coordinate (A) at (  0.0000,   0.0000);
        \coordinate (B) at ( -1.6667,  -0.7274);
        \coordinate (C) at (  1.6667,  -1.0911);
        \coordinate (D) at (  3.3333,  -0.3637);
        \coordinate (E) at (  0.8333,   2.6042);
        \coordinate (h1) at ( -0.8333,  -0.3637);
        \coordinate (h2) at (  0.0000,  -0.9092);
        \coordinate (h3) at (  2.5000,  -0.7274);
        \coordinate (h4) at (  1.6667,  -0.1818);
        \coordinate (h5) at (  0.8333,  -0.5455);
        \draw[line width=0.6pt, style=dashed] (A) -- (E);
        \draw[line width=0.6pt, style=dashed] (B) -- (A) -- (D);
        \draw[line width=0.6pt, style=dotted] (E) -- (h1) -- (h3) -- cycle;
        \draw[line width=0.6pt, style=dotted] (h2) -- (h4);
        \draw[line width=0.6pt, style=dotted] (E) -- (h5);
        \begin{scope}
          \clip (D) -- (C) -- (E) -- cycle;
          \filldraw[fill=black!15!white, draw=black, line width=0.4pt] (C) circle[radius=0.6000];
          \node at ([shift={(63.1433:0.3300)}]C) {{\small\rule[-0.5ex]{0pt}{2.2ex}$\beta$}};
        \end{scope}
        \begin{scope}
          \clip (h1) -- (E) -- (h3) -- cycle;
          \filldraw[fill=black!15!white, draw=black, line width=0.4pt] (E) circle[radius=1.0000];
          \node at ([shift={(268.6300:0.7000)}]E) {{\small\rule[-0.5ex]{0pt}{2.2ex}$\varphi$}};
        \end{scope}
        \begin{scope}
          \clip (D) -- (h3) -- (E) -- cycle;
          \filldraw[fill=black!15!white, draw=black, line width=0.4pt] (h3) circle[radius=0.4000];
          \fill ([shift={(70.0777:0.2000)}]h3) circle[radius=1.25pt];
        \end{scope}
        \begin{scope}
          \clip (h3) -- (h5) -- (E) -- cycle;
          \filldraw[fill=black!15!white, draw=black, line width=0.4pt] (h5) circle[radius=0.4000];
          \fill ([shift={(401.8866:0.2000)}]h5) circle[radius=1.25pt];
        \end{scope}
        \draw[line width=0.6pt, style=solid, join=bevel] (E) -- (B) -- (C) -- (E) -- (D) -- (C);
        \node at ($(h2)!4mm!180:(h4)$) {{\small$a$}};
        \node at ($(h3)!4mm!180:(h1)$) {{\small$a$}};
        \path (h3) -- node[pos=0.35, right] {{\small$h_{s}$}} (E);
      \end{tikzpicture}
    \end{minipage}%
    \hfill
    \begin{minipage}{0.55\textwidth}
      \begin{equation*}
        \begin{split}
          \varphi&=\SI{60}{\degree}\\[2ex]
          \sin\left(\frac{\varphi}{2}\right)&=\frac{\frac{a}{2}}{\;h_{s}\;}\\[2ex]
          \tan(\beta)&=\frac{\;h_{s}\;}{\frac{a}{2}}
        \end{split}
      \end{equation*}
    \end{minipage}
  \fi
  \ifoutcome\outcome\par
    Die Basiswinkel $\beta$ müssen eine Größe von etwa \SI{63.435}{\degree} besitzen.
  \fi
\end{exercise}

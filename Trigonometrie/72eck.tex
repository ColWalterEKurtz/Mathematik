\begin{exercise}{trigonometrie.72eck}{Reguläres 72-Eck}
  \ifproblem\problem
    \begin{minipage}{0.29\textwidth}
      \begin{tikzpicture}
        \newcommand{\radius}{1.4cm}
        \coordinate (M) at (0, 0);
        \coordinate (A) at (20:\radius);
        \coordinate (B) at (70:\radius);
        % Punkte
        \fill (M) circle[radius=1.25pt];
        \fill (A) circle[radius=1.25pt];
        \fill (B) circle[radius=1.25pt];
        % Kreis
        \draw[line width=0.6pt] (M) circle[radius=\radius];
        % Dreieck
        \draw[line width=0.6pt] (M) -- (A) -- (B) -- cycle;
        % Beschriftung
        \node[shift=(225:3mm)] at (M) {{\small$M$}};
        \node[shift=( 20:3mm)] at (A) {{\small$A$}};
        \node[shift=( 70:3mm)] at (B) {{\small$B$}};
        \node[shift=( 45:5mm)] at (M) {{\small$\alpha$}};
        \path (M) -- node[below right]{{\small$r$}} (A);
        \begin{scope}
          \clip (M) -- (A) -- (B) -- cycle;
          \draw[line width=0.4pt] (M) circle[radius=8mm];
        \end{scope}
      \end{tikzpicture}
    \end{minipage}%
    \hfill
    \begin{minipage}{0.70\textwidth}
      Bekannt sind der Mittelpunktswinkel $\alpha$ und der Radius $r$. Stelle
      eine Formel zur Berechnung des Flächeninhalts des Dreiecks $\triangle M\!AB$
      auf. Berechne mit der erhaltenen Formel den Flächeninhalt eines regulären
      72-Ecks mit beliebigem Radius.
    \end{minipage}
  \fi
  %\ifoutline\outline
  %\fi
  %\ifoutcome\outcome
  %\fi
\end{exercise}

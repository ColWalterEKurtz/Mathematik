\begin{exercise}
      {ID-25ad4aaf3aeaedcfea673f965c51157c659343f2}
      {Flächeninhalt}
  \ifproblem\problem\par
    Wie groß ist die graue Fläche?
    \begin{center}
      \begin{tikzpicture}
        % erste Reihe: links
        \begin{scope}
          \clip (-3.5, -2.2) rectangle (3.5, 2.2);
          \node[below right] at (-3.5, 2.2) {a)};
          \newcommand{\radius}{1cm}%
          \filldraw[fill=white!75!black] (-\radius, -\radius) rectangle (\radius, \radius);
          \filldraw[fill=white] (-\radius, -\radius) circle[radius=\radius];
          \filldraw[fill=white] ( \radius, -\radius) circle[radius=\radius];
          \filldraw[fill=white] ( \radius,  \radius) circle[radius=\radius];
          \filldraw[fill=white] (-\radius,  \radius) circle[radius=\radius];
          \draw (-\radius, -\radius) rectangle (\radius, \radius);
          \draw[style=dotted] (\radius, -\radius) -- ++(0:1.6*\radius) coordinate (U);
          \draw[style=dotted] (\radius,  \radius) -- ++(0:1.6*\radius) coordinate (O);
          \draw[|<->|, >=latex] (U) -- node[rotate=90, below]{{\small$a=\sicm{8}$}} (O);
        \end{scope}
        % erste Reihe: rechts
        \begin{scope}[xshift=8cm]
          \clip (-3.5, -2.2) rectangle (3.5, 2.2);
          \node[below right] at (-3.5, 2.2) {b)};
          \newcommand{\radius}{2.4cm}%
          \filldraw[fill=white!75!black] (\radius, -0.5*\radius) arc[start angle=0, end angle=180, radius=\radius] -- cycle;
          \filldraw[fill=white] (0, 0) circle[radius=0.5*\radius];
          \fill (0, 0) circle (0.8pt);
          \draw[style=dotted] (0,       -0.5*\radius) -- ++(270:5mm) coordinate (L);
          \draw[style=dotted] (\radius, -0.5*\radius) -- ++(270:5mm) coordinate (R);
          \draw[|<->|, >=latex] (L) -- node[below]{{\small$r=\sicm{6}$}} (R);
        \end{scope}
      \end{tikzpicture}
    \end{center}
    \begin{center}
      \begin{tikzpicture}
        % zweite Reihe: links
        \begin{scope}
          \clip (-3.5, -2.2) rectangle (3.5, 2.2);
          \node[below right] at (-3.5, 2.2) {c)};
          \newcommand{\radius}{1.75cm}%
          \begin{scope}
            \clip (-\radius, 0) circle[radius=\radius]; % links
            \clip (0, -\radius) circle[radius=\radius]; % unten
            \fill[fill=white!75!black] (-\radius, -\radius) rectangle (\radius, \radius);
          \end{scope}
          \begin{scope}
            \clip (0, -\radius) circle[radius=\radius]; % unten
            \clip (\radius, 0) circle[radius=\radius]; % rechts
            \fill[fill=white!75!black] (-\radius, -\radius) rectangle (\radius, \radius);
          \end{scope}
          \begin{scope}
            \clip (\radius, 0) circle[radius=\radius]; % rechts
            \clip (0, \radius) circle[radius=\radius]; % oben
            \fill[fill=white!75!black] (-\radius, -\radius) rectangle (\radius, \radius);
          \end{scope}
          \begin{scope}
            \clip (-\radius, 0) circle[radius=\radius]; % links
            \clip (0, \radius) circle[radius=\radius]; % oben
            \fill[fill=white!75!black] (-\radius, -\radius) rectangle (\radius, \radius);
          \end{scope}
          \begin{scope}
            \clip (-\radius, -\radius) rectangle (\radius, \radius);
            \draw (-\radius,  0)       circle[radius=\radius]; % links
            \draw ( 0,       -\radius) circle[radius=\radius]; % unten
            \draw ( \radius,  0)       circle[radius=\radius]; % rechts
            \draw ( 0,        \radius) circle[radius=\radius]; % oben
          \end{scope}
          \draw (-\radius, -\radius) rectangle (\radius, \radius);
          \draw[style=dotted] (\radius,  \radius) -- ++(0:8mm) coordinate (O);
          \draw[style=dotted] (\radius, -\radius) -- ++(0:8mm) coordinate (U);
          \draw[|<->|, >=latex] (U) -- node[rotate=90, below]{{\small$a=\sicm{3}$}} (O);
        \end{scope}
        % zweite Reihe: rechts
        \begin{scope}[xshift=8cm]
          \clip (-3.5, -2.2) rectangle (3.5, 2.2);
          \node[below right] at (-3.5, 2.2) {d)};
          \newcommand{\radius}{2cm}%
          \begin{scope}
            \clip (0, 0) circle[radius=\radius];
            \clip (225:\radius) -- ++(315:\radius) -- ++(45:\radius) -- ++(135:\radius) -- cycle;
            \fill[fill=white!75!black] (0, 0) circle[radius=\radius];
          \end{scope}
          \draw (0, 0) circle[radius=\radius];
          \filldraw[fill=white, join=bevel] (0, 0) -- (225:\radius) -- (315:\radius) -- cycle;
          \fill (0, 0) circle (0.9pt);
          \begin{scope}
            \clip (0, 0) -- (225:\radius) -- (315:\radius) -- cycle;
            \draw (0, 0) circle[radius=0.5cm];
            \fill (270:0.28) circle[radius=0.8pt];
          \end{scope}
          \node[rotate=45, above=-6pt] at (215:0.5*\radius) {{\small$r=\sicm{32}$}};
        \end{scope}
      \end{tikzpicture}
    \end{center}
    \begin{center}
      \begin{tikzpicture}
        % dritte Reihe: links
        \begin{scope}
          \clip (-3.5, -2.7) rectangle (3.5, 3.6);
          \node[below right] at (-3.5, 3.6) {e)};
          \begin{scope}
            \clip (-1, 0) circle[radius=2cm];
            \clip ( 1, 0) circle[radius=2cm];
            \fill[fill=white!75!black] (0, 0) circle[radius=2cm];
          \end{scope}
          \draw (-1, 0) circle[radius=2cm];
          \draw ( 1, 0) circle[radius=2cm];
          \fill ( 1, 0) circle[radius=0.9pt];
          \fill (-1, 0) circle[radius=0.9pt];
          \draw[<->, >=latex] (1, 0) -- node[below] {{\small$r=\sicm{6}$}} (3, 0);
        \end{scope}
        % dritte Reihe: rechts
        \begin{scope}[xshift=8cm]
          \clip (-3.5, -2.7) rectangle (3.5, 3.6);
          \node[below right] at (-3.5, 3.6) {f)};
          \newcommand{\radius}{1.75cm}%
          \begin{scope}
            \clip ( 90:\radius) circle[radius=\radius];
            \clip (210:\radius) circle[radius=\radius];
            \fill[fill=white!75!black] (0, 0) circle[radius=\radius];
          \end{scope}
          \begin{scope}
            \clip (210:\radius) circle[radius=\radius];
            \clip (330:\radius) circle[radius=\radius];
            \fill[fill=white!75!black] (0, 0) circle[radius=\radius];
          \end{scope}
          \begin{scope}
            \clip ( 90:\radius) circle[radius=\radius];
            \clip (330:\radius) circle[radius=\radius];
            \fill[fill=white!75!black] (0, 0) circle[radius=\radius];
          \end{scope}
          \draw (0, 0)        circle[radius=\radius];
          \draw ( 90:\radius) circle[radius=\radius];
          \draw (210:\radius) circle[radius=\radius];
          \draw (330:\radius) circle[radius=\radius];
          \fill (0, 0)        circle[radius=0.8pt];
          \fill ( 90:\radius) circle[radius=0.8pt];
          \fill (210:\radius) circle[radius=0.8pt];
          \fill (330:\radius) circle[radius=0.8pt];
          \draw[<->, >=latex] (90:\radius) -- node[rotate=45, above] {{\small$r=\sicm{6}$}} ++(45:\radius);
        \end{scope}
      \end{tikzpicture}
    \end{center}
  \fi
  %\ifoutline\outline\par
  %\fi
  %\ifoutcome\outcome\par
  %\fi
\end{exercise}

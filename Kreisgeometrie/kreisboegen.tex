\begin{exercise}
      {ID-f50b91e0c0a8af8ab764f9f28a82f3924149874e}
      {Kreisbögen}
  \ifproblem\problem
    Die beiden Punkte $A$ und $B$ sind \simeter{8} weit voneinander entfernt
    und sollen durch einen Weg aus halbkreisförmigen Kreisbögen miteinander
    verbunden werden. Welche der drei abgebildeten Varianten stellt die
    kürzeste Verbindung dar?
    \begin{center}
      \begin{tikzpicture}
        \coordinate (A) at (0, 0);
        \coordinate (B) at (8, 0);
        \fill (A) circle[radius=1.25pt] node[left]{$A$};
        \fill (B) circle[radius=1.25pt] node[right]{$B$};
        \draw (A) -- (B);
        % Variante 1
        \draw[line width=1.25pt, style=dotted]
              (B) arc[start angle=0, end angle=180, radius=4];
        % Variante 2
        \draw[line width=1pt, style=dashed]
              (B) arc[start angle=0,   end angle=180, radius=2]
                  arc[start angle=360, end angle=180, radius=2];
        % Variante 3
        \draw (B) arc[start angle=0,   end angle=180, radius=1]
                  arc[start angle=360, end angle=180, radius=1]
                  arc[start angle=0,   end angle=180, radius=1]
                  arc[start angle=360, end angle=180, radius=1];
      \end{tikzpicture}
    \end{center}
  \fi
  %\ifoutline\outline
  %\fi
  \ifoutcome\outcome
    Alle drei Wege besitzen dieselbe Länge:
    \begin{equation*}
      2\cdot\pi\cdot\simeter{4}\cdot\frac{1}{2}=
      2\cdot\pi\cdot\simeter{2}=
      2\cdot\pi\cdot\simeter{1}\cdot2\approx
      \simeter{12.57}
    \end{equation*}
  \fi
\end{exercise}

\begin{exercise}
      {ID-0d2e0e5ae3e52519d63218746ad2efb89e1d0ded}
      {Umlaufbahn}
  \ifproblem\problem\par
    Die Erde benötigt ca. \num{365} Tage, um die Sonne einmal zu umkreisen.
    Der mittlere Radius der Umlaufbahn beträgt etwa \sikm{150000000}.
    \begin{enumerate}[a)]
      \item Wie lang ist die Umlaufbahn der Erde?
      \item Mit welcher Geschwindigkeit bewegt sich die Erde durchs All?
    \end{enumerate}
  \fi
  %\ifoutline\outline\par
  %\fi
  \ifoutcome\outcome\par
    \begin{enumerate}[a)]
      \item Die Umlaufbahn der Erde um die Sonne besitzt eine
            Länge von etwa \sikm{942477796}.
      \item Die Erde dreht sich mit einer Geschwindigkeit von
            etwa \sikmh{107589} um die Sonne.
    \end{enumerate}
  \fi
\end{exercise}

\begin{exercise}
      {ID-a9fa4285bfefe98b00f46b022447e562b790752b}
      {Streifenzug}
  \ifproblem\problem
    Ein Quadrat $ABCD$ mit \sicm{14} Seitenlänge ist in 7 mal 7 gleichgroße
    Teilquadrate zerlegt. Aus einigen dieser Teilquadrate ist ein
    Streifenzug so zusammengestellt, wie es die Abbildung zeigt.
    Der Streifenzug ist durch die Schraffierung hervorgehoben.
    \begin{center}
      \begin{tikzpicture}[scale=0.5]
        \draw (0, 0) grid (7, 7);
        \begin{scope}[pattern color=black!50!white]
          \path[pattern=crosshatch] (0, 0) rectangle (7, 1);
          \path[pattern=crosshatch] (0, 1) rectangle (1, 7);
          \path[pattern=crosshatch] (1, 6) rectangle (6, 7);
          \path[pattern=crosshatch] (5, 2) rectangle (6, 6);
          \path[pattern=crosshatch] (2, 2) rectangle (5, 3);
          \path[pattern=crosshatch] (2, 3) rectangle (3, 5);
          \path[pattern=crosshatch] (3, 4) rectangle (4, 5);
        \end{scope}
      \fill (0, 0) circle[radius=2pt] node[below left]{$A$};
      \fill (7, 0) circle[radius=2pt] node[below right]{$B$};
      \fill (7, 7) circle[radius=2pt] node[above right]{$C$};
      \fill (0, 7) circle[radius=2pt] node[above left]{$D$};
      \end{tikzpicture}
    \end{center}
    Berechne den Umfang und den Flächeninhalt dieses Streifenzuges.
  \fi
  %\ifoutline\outline
  %\fi
  \ifoutcome\outcome
    Der Streifenzug besitzt eine Fläche von \sicmm{112} und
    einen Umfang von \sicm{116}.
  \fi
\end{exercise}

\begin{exercise}
      {ID-8fac9384aca3fa016083c326209411bf8cd1b962}
      {Gemeinsame Punkte}
  \ifproblem\problem
    Gegeben seien zwei unterschiedlich große Quadrate, wie sie hier dargestellt
    sind:
    \begin{center}
      \begin{tikzpicture}
        \begin{scope}
          \draw (0, 0) rectangle (1.5, 1.5);
          \draw (2, 0) rectangle (3, 1);
        \end{scope}
        \begin{scope}[xshift=5cm]
          \draw (0, 0) rectangle (1.5, 1.5);
          \draw[shift={(-0.9, -0.6)}] (2, 0) rectangle (3, 1);
          \fill (1.1, 0.0) circle[radius=0.8pt] node[below left]{$P$};
          \fill (1.5, 0.4) circle[radius=0.8pt] node[above right]{$Q$};
        \end{scope}
      \end{tikzpicture}
    \end{center}
    Bei der linken Abbildung haben sie keinen gemeinsamen Punkt, bei der
    rechten genau zwei, nämlich $P$ und $Q$. Wie können die Quadrate liegen,
    wenn sie genau a) einen Punkt, b) drei, c) vier, d) fünf, e) sechs f) sieben
    Punkte gemeinsam haben sollen? Zeichne die Quadrate in diesen verschiedenen Lagen!
  \fi
  %\ifoutline\outline
  %\fi
  %\ifoutcome\outcome
  %\fi
\end{exercise}

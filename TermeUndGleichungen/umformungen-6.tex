\begin{exercise}
      {ID-5865f69dad63d1e6b9b6913382f6f23eaec96dc0}
      {Umformungen 6}
  \ifproblem\problem\par
    \newcommand{\gap}{\;\;}%
    Löse die Klammern auf, und fasse so weit wie möglich zusammen:
    \begin{align*}
      \text{a)}\gap & (p+q)^2+(p-q)^2 &
      \text{e)}\gap & (2a+1)^2-(a-3)^2
      \\
      \text{b)}\gap & (3p+2q)^2-(2p-3q)^2 &
      \text{f)}\gap & (c+2d)(c-2d)+(c-d)(2c+d)
      \\
      \text{c)}\gap & (a+3b)^2+(3a+b)(3a-b) &
      \text{g)}\gap & (3x+2)(1-x)-(x-4)^2
      \\
      \text{d)}\gap & (5x+z)(5x-z)-(2x-5z)^2 &
      \text{h)}\gap & 5(y-2)^2-3(y+2)^2
    \end{align*}
  \fi
  %\ifoutline\outline\par
  %\fi
  \ifoutcome\outcome\par
    % first line with a fix width
    \newcommand{\toprow}[2]
    {%
      \makebox[1.5em][r]{\ensuremath{\displaystyle#1}}%
      &\phantom{\,=\:\,}%
      \makebox[18.5em][l]{\ensuremath{\displaystyle#2}}%
    }%
    % solition of exercise a)
    \newcommand{\solutionA}
    {%
      \begin{equation*}
        \begin{split}
          \toprow{\text{a)}}{(p+q)^2+(p-q)^2} \\
          &=p^2+2pq+q^2+p^2-2pq+q^2           \\
          &=p^2+p^2+2pq-2pq+q^2+q^2           \\
          &=2p^2+2q^2                         \\
          &=2\left(p^2+q^2\right)
        \end{split}
      \end{equation*}
    }%
    % solition of exercise b)
    \newcommand{\solutionB}
    {%
      \begin{equation*}
        \begin{split}
          \toprow{\text{b)}}{(3p+2q)^2-(2p-3q)^2}      \\
          &=9p^2+12pq+4q^2-\left(4p^2-12pq+9q^2\right) \\
          &=9p^2+12pq+4q^2-4p^2+12pq-9q^2              \\
          &=9p^2-4p^2+12pq+12pq+4q^2-9q^2              \\
          &=5p^2+24pq-5q^2
        \end{split}
      \end{equation*}
    }%
    % solition of exercise c)
    \newcommand{\solutionC}
    {%
      \begin{equation*}
        \begin{split}
          \toprow{\text{c)}}{(a+3b)^2+(3a+b)(3a-b)} \\
          &=a^2+6ab+36b^2+9a^2-b^2                  \\
          &=a^2+9a^2+6ab+36b^2-b^2                  \\
          &=10a^2+6ab+35b^2
        \end{split}
      \end{equation*}
    }%
    % solition of exercise d)
    \newcommand{\solutionD}
    {%
      \begin{equation*}
        \begin{split}
          \toprow{\text{d)}}{(5x+z)(5x-z)-(2x-5z)^2} \\
          &=25x^2-z^2-\left(4x^2-20xz+25z^2\right)   \\
          &=25x^2-z^2-4x^2+20xz-25z^2                \\
          &=25x^2-4x^2+20xz-z^2-25z^2                \\
          &=21x^2+20xz-26z^2
        \end{split}
      \end{equation*}
    }%
    % solition of exercise e)
    \newcommand{\solutionE}
    {%
      \begin{equation*}
        \begin{split}
          \toprow{\text{e)}}{(2a+1)^2-(a-3)^2} \\
          &=4a^2+4a+1-\left(a^2-6a+9\right)    \\
          &=4a^2+4a+1-a^2+6a-9                 \\
          &=4a^2-a^2+4a+6a+1-9                 \\
          &=3a^2+10a-8
        \end{split}
      \end{equation*}
    }%
    % solition of exercise f)
    \newcommand{\solutionF}
    {%
      \begin{equation*}
        \begin{split}
          \toprow{\text{f)}}{(c+2d)(c-2d)+(c-d)(2c+d)} \\
          &=c^2-4d^2+2c^2+cd-2cd-d^2                   \\
          &=c^2+2c^2+cd-2cd-4d^2-d^2                   \\
          &=3c^2-cd-5d^2
        \end{split}
      \end{equation*}
    }%
    % solition of exercise g)
    \newcommand{\solutionG}
    {%
      \begin{equation*}
        \begin{split}
          \toprow{\text{g)}}{(3x+2)(1-x)-(x-4)^2} \\
          &=3x-3x^2+2-2x-\left(x^2-8x+16\right)   \\
          &=3x-3x^2+2-2x-x^2+8x-16                \\
          &=-3x^2-x^2+3x-2x+8x+2-16               \\
          &=-4x^2+9x-14
        \end{split}
      \end{equation*}
    }%
    % solition of exercise h)
    \newcommand{\solutionH}
    {%
      \begin{equation*}
        \begin{split}
          \toprow{\text{h)}}{5(y-2)^2-3(y+2)^2}           \\
          &=5\left(y^2-4y+4\right)-3\left(y^2+4y+4\right) \\
          &=5y^2-20y+20-3y^2-12y-12                       \\
          &=5y^2-3y^2-20y-12y+20-12                       \\
          &=2y^2-32y+8
        \end{split}
      \end{equation*}
    }%

    \begin{minipage}[t]{0.49\textwidth}
      \small
      \solutionA
    \end{minipage}%
    \hfill
    \begin{minipage}[t]{0.49\textwidth}
      \small
      \solutionB
    \end{minipage}%

    \begin{minipage}[t]{0.49\textwidth}
      \small
      \solutionC
    \end{minipage}%
    \hfill
    \begin{minipage}[t]{0.49\textwidth}
      \small
      \solutionD
    \end{minipage}%

    \begin{minipage}[t]{0.49\textwidth}
      \small
      \solutionE
    \end{minipage}%
    \hfill
    \begin{minipage}[t]{0.49\textwidth}
      \small
      \solutionF
    \end{minipage}%

    \begin{minipage}[t]{0.49\textwidth}
      \small
      \solutionG
    \end{minipage}%
    \hfill
    \begin{minipage}[t]{0.49\textwidth}
      \small
      \solutionH
    \end{minipage}%

  \fi
\end{exercise}

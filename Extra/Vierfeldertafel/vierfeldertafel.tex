\documentclass
[
  draft    = true,
  fontsize = 11pt,
  parskip  = half-,
  BCOR     = 0pt,
  DIV      = 11
]
{scrartcl}

% Standardpakete
\usepackage[utf8]{inputenc}
\usepackage[T1]{fontenc}
\usepackage{lmodern}
\usepackage[ngerman]{babel}
% Zusatzpakete
\usepackage{amsmath}
\usepackage{amssymb}
\usepackage{tikz}

% ------------------------------------------------------------------------------
\begin{document}
% ------------------------------------------------------------------------------

% ----------------------------
\section*{Die Vierfeldertafel}
% ----------------------------
Um zwei Ereignisse $A$ und $B$ auf deren stochastische Unabhängigkeit zu
testen, bietet es sich an, das gegebene Baumdiagramm in eine Vierfeldertafel
zu übersetzen:\par
\begin{minipage}{0.36\textwidth}
  \begin{tikzpicture}[scale=0.9]
    \coordinate (O) at (0,  0);
    \coordinate (A) at (2,  2);
    \coordinate (B) at (2, -2);
    \coordinate (C) at (5,  3);
    \coordinate (D) at (5,  1);
    \coordinate (E) at (5, -1);
    \coordinate (F) at (5, -3);
    % Kanten
    \draw (O) -- (A);
    \draw (O) -- (B);
    \draw (A) -- (C);
    \draw (A) -- (D);
    \draw (B) -- (E);
    \draw (B) -- (F);
    % Start
    \fill (O) circle (1pt);
    % Knoten
    \fill[fill=white] (A) circle (5.5mm);
    \fill[fill=white] (B) circle (5.5mm);
    \fill[fill=white] (C) circle (5.5mm);
    \fill[fill=white] (D) circle (5.5mm);
    \fill[fill=white] (E) circle (5.5mm);
    \fill[fill=white] (F) circle (5.5mm);    
    \draw (A) circle (5mm) node{$A$};
    \draw (B) circle (5mm) node{$\bar{A}$};
    \draw (C) circle (5mm) node{$B$};
    \draw (D) circle (5mm) node{$\bar{B}$};
    \draw (E) circle (5mm) node{$B$};
    \draw (F) circle (5mm) node{$\bar{B}$};
    % Beschriftung
    \path (O) -- node[above left] {{\small$P(A)$}}                 (A);
    \path (O) -- node[below left] {{\small$P(\bar{A})$}}           (B);
    \path (A) -- node[above=2mm]  {{\small$P_{A}(B)$}}             (C);
    \path (A) -- node[below=2mm]  {{\small$P_{A}(\bar{B})$}}       (D);
    \path (B) -- node[above=2mm]  {{\small$P_{\bar{A}}(B)$}}       (E);
    \path (B) -- node[below=2mm]  {{\small$P_{\bar{A}}(\bar{B})$}} (F);
  \end{tikzpicture}
\end{minipage}%
\hfill
\begin{minipage}{0.62\textwidth}
  \newcommand{\bx}[1]{\makebox[4.9em][c]{\rule[-2.2ex]{0pt}{6ex}#1}}%
  \hfill
  \begin{tabular}{|c||c|c||c|}
    \hline
                   & \bx{$A$}                & \bx{$\bar{A}$}               &                   \\
    \hline
    \hline
    \bx{$B$}       & \bx{$P(A\cap B)$}       & \bx{$P(\bar{A}\cap B)$}      & \bx{$P(B)$}       \\
    \hline
    \bx{$\bar{B}$} & \bx{$P(A\cap\bar{B})$}  & \bx{$P(\bar{A}\cap\bar{B})$} & \bx{$P(\bar{B})$} \\
    \hline
    \hline
                   & \bx{$P(A)$}             & \bx{$P(\bar{A})$}            & \bx{1}            \\
    \hline
  \end{tabular}
\end{minipage}\medskip

Die Pfadwahrscheinlichkeiten in der Mitte der Tabelle berechnet man wie gewohnt
nach der Pfadregel:
\begin{alignat*}{3}
       P(A\cap B)&=P(A)\cdot P_{A}(B)       & \quad&\quad &      P(\bar{A}\cap B)&=P(\bar{A})\cdot P_{\bar{A}}(B) \\
  P(A\cap\bar{B})&=P(A)\cdot P_{A}(\bar{B}) & \quad&\quad & P(\bar{A}\cap\bar{B})&=P(\bar{A})\cdot P_{\bar{A}}(\bar{B})
\end{alignat*}

Die Randwahrscheinlichkeiten ergeben sich aus der Summe der zugehörigen
Pfadwahrscheinlichkeiten und entsprechen jeweils der Einzelwahrscheinlichkeit
des in der Tabelle gegenüberliegenden Ereignisses:
\begin{equation*}
  \begin{split}
                P(A\cap B)+P(A\cap\bar{B})&=P(A)\cdot P_{A}(B)+P(A)\cdot P_{A}(\bar{B})   \\
                                          &=P(A)\cdot\left(P_{A}(B)+P_{A}(\bar{B})\right) \\
                                          &=P(A)\cdot1=P(A)                               \\[2ex]
    P(\bar{A}\cap B)+P(\bar{A}\cap\bar{B})&=P(\bar{A})                                    \\[4ex]
               P(A\cap B)+P(\bar{A}\cap B)&=P(B\cap A)+P(B\cap\bar{A})                    \\
                                          &=P(B)\cdot P_{B}(A)+P(B)\cdot P_{B}(\bar{A})   \\
                                          &=P(B)\cdot\left(P_{B}(A)+P_{B}(\bar{A})\right) \\
                                          &=P(B)\cdot1=P(B)                               \\[2ex]
     P(A\cap\bar{B})+P(\bar{A}\cap\bar{B})&=P(\bar{B})
  \end{split}
\end{equation*}

Zur Kontrolle muss im Feld unten rechts die Summe der Randwahrscheinlichkeiten
immer 1 ergeben, denn $P(A\cup\bar{A})=P(B\cup\bar{B})=P(\Omega)=1$.

Stochastisch unabhängig sind zwei Ereignisse $A$ und $B$ genau dann, wenn
die Wahrscheinlichkeit für das Eintreten des einen nicht von der Realisierung
des anderen abhängt. Mit anderen Worten:
\begin{equation*}
  P(B)=P_{A}(B)\quad\text{und}\quad P(A)=P_{B}(A)
\end{equation*}
Nach der Pfadregel lässt sich in einem Baumdiagramm die Wahrscheinlichkeit für
das gemeinsame Auftreten zweier Ereignisse $A$ und $B$ stets wie folgt berechnen:
\begin{equation*}
  P(A\cap B)=P(A)\cdot P_{A}(B)
\end{equation*}
Wenn sie stochastisch unabhängig sind, ist die Wahrscheinlichkeit für das
gemeinsame Auftreten wegen $P_{A}(B)=P(B)$ gerade gleich dem Produkt der
Einzelwahrscheinlichkeiten:
\begin{equation*}
  P(A\cap B)=P(A)\cdot P(B).
\end{equation*}
Das heißt, um zu testen, ob zwei Ereignisse stochastisch unabhängig sind,
vergleicht man die Pfadwahrscheinlichkeit in der mitte der Tabelle mit dem
Produkt der beiden zugehörigen Randwahrscheinlichkeiten.

% ------------------------------------------------------------------------------
\end{document}
% ------------------------------------------------------------------------------

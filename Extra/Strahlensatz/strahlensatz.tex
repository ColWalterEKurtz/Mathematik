\documentclass
[
  draft    = true,
  fontsize = 11pt,
  parskip  = half-,
  BCOR     = 0pt,
  DIV      = 11
]
{scrartcl}

% Standardpakete
\usepackage[utf8]{inputenc}
\usepackage[T1]{fontenc}
\usepackage{lmodern}
\usepackage[ngerman]{babel}
% Zusatzpakete
\usepackage{amsmath}
\usepackage{amssymb}
\usepackage{enumerate}
\usepackage{tikz}

\usetikzlibrary{patterns}

% ------------------------------------------------------------------------------
\begin{document}
% ------------------------------------------------------------------------------

% ----------------
% punktkoordinaten
% ----------------
%
% Koordinaten der beteiligten Punkte im Strahlensatz
%
\newcommand{\punktkoordinaten}
{
  \coordinate (S)   at (0cm, 0cm);
  \coordinate (A)   at ( 15:1.8cm);
  \coordinate (B)   at (-15:1.8cm);
  \coordinate (C)   at ( 15:3.0cm);
  \coordinate (D)   at (-15:3.0cm);
  \coordinate (S1E) at ([shift={(15:5mm)}]  C);
  \coordinate (S2E) at ([shift={(-15:5mm)}] D);
  \coordinate (P1A) at ([shift={(270:5mm)}] B);
  \coordinate (P1E) at ([shift={(90:5mm)}]  A);
  \coordinate (P2A) at ([shift={(270:5mm)}] D);
  \coordinate (P2E) at ([shift={(90:5mm)}]  C);
  \coordinate (HA)  at (intersection of S--S2E and A--[shift={(255:10cm)}]A);
  \coordinate (HB)  at (intersection of S--S1E and B--[shift={(105:10cm)}]B);
  \coordinate (HC)  at (intersection of P1A--P1E and C--[shift={(180:10cm)}]C);
  \coordinate (HD)  at (intersection of P1A--P1E and D--[shift={(180:10cm)}]D);
  \coordinate (G)   at (intersection of C--D and A--[shift={(345:10cm)}]A);
  \coordinate (H)   at (intersection of C--D and B--[shift={(15:10cm)}]B);
}

% ------
% Punkte
% ------
\newcommand{\pktS}[1][left]                {\fill (S)  circle (0.7pt);\node[#1] at (S)  {{\footnotesize$S$}};}
\newcommand{\pktA}[1][above left]          {\fill (A)  circle (0.7pt);\node[#1] at (A)  {{\footnotesize$A$}};}
\newcommand{\pktB}[1][below left]          {\fill (B)  circle (0.7pt);\node[#1] at (B)  {{\footnotesize$B$}};}
\newcommand{\pktC}[1][shift={(52.5:4mm)}]  {\fill (C)  circle (0.7pt);\node[#1] at (C)  {{\footnotesize$C$}};}
\newcommand{\pktD}[1][shift={(307.5:4mm)}] {\fill (D)  circle (0.7pt);\node[#1] at (D)  {{\footnotesize$D$}};}
\newcommand{\pktE}[1][above left]          {\fill (HB) circle (0.7pt);\node[#1] at (HB) {{\footnotesize$E$}};}
\newcommand{\pktF}[1][below left]          {\fill (HA) circle (0.7pt);\node[#1] at (HA) {{\footnotesize$F$}};}
\newcommand{\pktG}[1][shift={(37.5:4mm)}]  {\fill (G)  circle (0.7pt);\node[#1] at (G)  {{\footnotesize$G$}};}
\newcommand{\pktH}[1][shift={(322.5:4mm)}] {\fill (H)  circle (0.7pt);\node[#1] at (H)  {{\footnotesize$H$}};}

% --------
% Strecken
% --------
\newcommand{\lineSA}{\ensuremath{\overline{S\!A}}}
\newcommand{\lineSC}{\ensuremath{\overline{SC}}}
\newcommand{\lineSB}{\ensuremath{\overline{S\!B}}}
\newcommand{\lineSD}{\ensuremath{\overline{S\!D}}}
\newcommand{\lineAC}{\ensuremath{\overline{AC}}}
\newcommand{\lineAB}{\ensuremath{\overline{AB}}}
\newcommand{\lineAF}{\ensuremath{\overline{AF}}}
\newcommand{\lineBD}{\ensuremath{\overline{B\!D}}}
\newcommand{\lineBE}{\ensuremath{\overline{B\!E}}}
\newcommand{\lineCD}{\ensuremath{\overline{C\!D}}}
\newcommand{\lineCH}{\ensuremath{\overline{C\!H}}}
\newcommand{\lineGD}{\ensuremath{\overline{G\!D}}}

% ------------
% strahlensatz
% ------------
%
% Linien fuer einen Strahlensatz
%
\newcommand{\strahlensatz}
{
  % Strahlen
  \draw (S1E) -- (S) -- (S2E);
  % Parallelen
  \draw (P1A) -- (P1E);
  \draw (P2A) -- (P2E);
}

% ---------------------
\section*{Strahlensatz}
% ---------------------

% -------------------------------------------------
\subsection*{Streckenverhältnisse auf den Strahlen}
% -------------------------------------------------
Die beiden Dreiecke $\triangle ABC$ und $\triangle ABD$ teilen sich die
gemeinsame Grundseite $\overline{AB}$. Außerdem sind ihre Höhen gleich lang,
denn die Gerade durch $A$ und $B$ verläuft parallel zur Geraden durch $C$
und $D$. Damit sind die Dreiecke $\triangle ABC$ und $\triangle ABD$
-- unabhängig von der in der Skizze vorhandenen Symmetrie -- gleich groß.
\begin{center}
  \begin{tikzpicture}[scale=1.5]
    \begin{scope}
      \punktkoordinaten
      \fill[fill=black!15!white] (A) -- (B) -- (C) -- cycle;
      \fill[fill=black!15!white] (A) -- (B) -- (D) -- cycle;
      \pktS
      \pktA
      \pktB
      \pktC
      \pktD
      \draw[thin] (A) -- (D);
      \draw[thin] (B) -- (C);
      \draw (C) -- (HC);
      \draw (D) -- (HD);
      \strahlensatz
      \begin{scope}
        \clip (HD) rectangle (C);
        \draw (HC) circle (2mm);
        \draw (HD) circle (2mm);
        \fill ([shift={(315:1mm)}]HC) circle (0.5pt);
        \fill ([shift={( 45:1mm)}]HD) circle (0.5pt);
      \end{scope}
    \end{scope}
    \begin{scope}[xshift=5cm]
      \punktkoordinaten
      \fill[fill=black!15!white] (S) -- (B) -- (C) -- cycle;
      \fill[fill=black!15!white] (S) -- (A) -- (D) -- cycle;
      \pktS
      \pktA
      \pktB
      \pktC
      \pktD
      \draw[thin] (A) -- (D);
      \draw[thin] (B) -- (C);
      \strahlensatz
    \end{scope}
  \end{tikzpicture}
\end{center}
Addiert man zu den beiden Dreiecken $\triangle ABC$ und $\triangle ABD$
jeweils noch die Fläche des Dreiecks $\triangle SBA$ hinzu, sieht man, dass
auch die beiden Dreiecke $\triangle SBC$ und $\triangle SDA$ gleich groß
sind. Also gelten für die Dreiecksflächen unter anderem folgende Verhältnisse:
\begin{equation*}
  \frac{\left|\triangle ABC\right|}{\left|\triangle SBC\right|}
  =
  \frac{\left|\triangle ABD\right|}{\left|\triangle SDA\right|}
  \qquad
  \text{und}
  \qquad
  \frac{\left|\triangle SBA\right|}{\left|\triangle SBC\right|}
  =
  \frac{\left|\triangle SBA\right|}{\left|\triangle SDA\right|}
\end{equation*}

Mit den Höhen $\lineBE$ und $\lineAF$, lassen sich dann die Flächen
auch durch die zugehörigen Produkte aus Grundlinie und Höhe ausdrücken:
\begin{center}
  \begin{tikzpicture}[scale=1.5]
    \begin{scope}
      \punktkoordinaten
      \fill[fill=black!15!white] (S) -- (B) -- (C) -- cycle;
      \pktS
      \pktA[shift={(52.5:4mm)}]
      \pktB
      \pktC
      \pktD
      \pktE
      \draw[thin] (B) -- (C);
      \draw (B) -- (HB);
      \strahlensatz
      \begin{scope}
        \clip (S) -- (B) -- (HB);
        \draw (HB) circle (2mm);
        \fill ([shift={(240:1mm)}]HB) circle (0.5pt);
      \end{scope}
    \end{scope}
    \begin{scope}[xshift=5cm]
      \punktkoordinaten
      \fill[fill=black!15!white] (S) -- (A) -- (D) -- cycle;
      \pktS
      \pktA
      \pktB[shift={(307.5:4mm)}]
      \pktC
      \pktD
      \pktF
      \draw[thin] (A) -- (D);
      \draw (A) -- (HA);
      \strahlensatz
      \begin{scope}
        \clip (S) -- (A) -- (HA);
        \draw (HA) circle (2mm);
        \fill ([shift={(120:1mm)}]HA) circle (0.5pt);
      \end{scope}
    \end{scope}
  \end{tikzpicture}
\end{center}
\begin{alignat*}{3}
  \left|\triangle ABC\right|&=\frac{\lineAC\cdot\lineBE}{2} & \qquad&\qquad & \left|\triangle ABD\right|&=\frac{\lineBD\cdot\lineAF}{2} \\[2ex]
  \left|\triangle SBC\right|&=\frac{\lineSC\cdot\lineBE}{2} & \qquad&\qquad & \left|\triangle SDA\right|&=\frac{\lineSD\cdot\lineAF}{2} \\[2ex]
  \left|\triangle SBA\right|&=\frac{\lineSA\cdot\lineBE}{2} & \qquad&\qquad & \left|\triangle SBA\right|&=\frac{\lineSB\cdot\lineAF}{2}
\end{alignat*}

% AC/SC = BD/SD
\newcommand{\eqIL}{\frac{\left|\triangle ABC\right|}{\left|\triangle SBC\right|}&=\frac{\left|\triangle ABD\right|}{\left|\triangle SDA\right|}}
\newcommand{\eqIM}{\frac{\frac{\lineAC\cdot\lineBE}{2}}{\frac{\lineSC\cdot\lineBE}{2}}&=\frac{\frac{\lineBD\cdot\lineAF}{2}}{\frac{\lineSD\cdot\lineAF}{2}}}
\newcommand{\eqIR}{\frac{\;\lineAC\;}{\lineSC}&=\frac{\;\lineBD\;}{\lineSD}}
% SA/SC = SB/SD
\newcommand{\eqIIL}{\frac{\left|\triangle SBA\right|}{\left|\triangle SBC\right|}&=\frac{\left|\triangle SBA\right|}{\left|\triangle SDA\right|}}
\newcommand{\eqIIM}{\frac{\frac{\lineSA\cdot\lineBE}{2}}{\frac{\lineSC\cdot\lineBE}{2}}&=\frac{\frac{\lineSB\cdot\lineAF}{2}}{\frac{\lineSD\cdot\lineAF}{2}}}
\newcommand{\eqIIR}{\frac{\;\lineSA\;}{\lineSC}&=\frac{\;\lineSB\;}{\lineSD}}
Das Einsetzen in die Flächengleichungen ergibt folgende Zusammenhänge:
\begin{alignat*}{5}
  \eqIL  & \quad&\Rightarrow\quad & \eqIM  & \quad&\Rightarrow\quad & \eqIR \\[2ex]
  \eqIIL & \quad&\Rightarrow\quad & \eqIIM & \quad&\Rightarrow\quad & \eqIIR
\end{alignat*}


Aus diesen Zusammenhängen lässt sich ein Dritter ableiten:
\begin{equation*}
  \left.
  \begin{split}
    \frac{\;\lineAC\;}{\lineSC}&=\frac{\;\lineBD\;}{\lineSD}
    \quad\Rightarrow\quad
    \frac{\;\lineSD\;}{\lineSC}=\frac{\;\lineBD\;}{\lineAC}
    \\[2ex]
    \frac{\;\lineSA\;}{\lineSC}&=\frac{\;\lineSB\;}{\lineSD}
    \quad\Rightarrow\quad
    \frac{\;\lineSD\;}{\lineSC}=\frac{\;\lineSB\;}{\lineSA}
  \end{split}\quad
  \right\}
  \quad\Rightarrow\quad
  \frac{\;\lineBD\;}{\lineAC}=\frac{\;\lineSB\;}{\lineSA}
  \quad\Rightarrow\quad
  \frac{\;\lineSA\;}{\lineAC}=\frac{\;\lineSB\;}{\lineBD}
\end{equation*}

% -----------------------------------------------------------------
\subsection*{Streckenverhältnisse zwischen Strahlen und Parallelen}
% -----------------------------------------------------------------
Ausgehend von der ursprünglichen Strahlensatzfigur verschiebt man den unteren
Strahl, der durch die Punkte $S$ und $D$ gegeben ist, so weit parallel nach
oben, bis er als Gerade genau durch den Punkt $A$, und den oberen
Strahl, der durch die Punkte $S$ und $C$ gegeben ist, so weit parallel nach
unten, bis er als Gerade genau durch den Punkt $B$ verläuft:
\begin{center}
  \begin{tikzpicture}[scale=1.5]
    \begin{scope}
      \punktkoordinaten
      \pktS
      \pktA[shift={(52.5:4mm)}]
      \pktB
      \pktC
      \pktD
      \pktG
      \strahlensatz
      \draw ([yshift=0.931748562cm]S) -- ([yshift=0.931748562cm]S2E);
    \end{scope}
    \begin{scope}[xshift=5cm]
      \punktkoordinaten
      \pktS
      \pktA
      \pktB[shift={(307.5:4mm)}]
      \pktC
      \pktD
      \pktH
      \strahlensatz
      \draw ([yshift=-0.931748562cm]S) -- ([yshift=-0.931748562cm]S1E);
    \end{scope}
  \end{tikzpicture}
\end{center}
Wendet man nun die oben bereits bewiesenen Verhältnisgleichungen auf die
Strahlen an, die von Punkt $C$ bzw. von Punkt $D$ ausgehen, erhält man
unter anderem:
\begin{equation*}
  \frac{\;\lineSA\;}{\lineSC}=\frac{\;\lineGD\;}{\lineCD}
  \qquad
  \text{und}
  \qquad
  \frac{\;\lineSB\;}{\lineSD}=\frac{\;\lineCH\;}{\lineCD}
\end{equation*}

Durch die beiden Parallelverschiebungen gilt außerdem $\lineAB=\lineGD=\lineCH$.
Damit erkennt man, dass die rechten Seiten der Gleichungen übereinstimmen, und
man erhält den Zusammenhang, der die Streckenverhältnisse zwischen Strahlen und
Parallelen beschreibt:
\begin{equation*}
  \frac{\;\lineSA\;}{\lineSC}=\frac{\;\lineAB\;}{\lineCD}
  \qquad
  \text{und}
  \qquad
  \frac{\;\lineSB\;}{\lineSD}=\frac{\;\lineAB\;}{\lineCD}
  \qquad
  \Rightarrow
  \qquad
  \frac{\;\lineAB\;}{\lineCD}=\frac{\;\lineSA\;}{\lineSC}=\frac{\;\lineSB\;}{\lineSD}
\end{equation*}

% ------------------------------------------------------------------------------
\end{document}
% ------------------------------------------------------------------------------


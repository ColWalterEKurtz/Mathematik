\documentclass
[
  draft    = true,
  fontsize = 11pt,
  parskip  = half-,
  BCOR     = 0pt,
  DIV      = 11,
  ngerman
]
{scrartcl}

% default packages
\usepackage[utf8]{inputenc}
\usepackage[T1]{fontenc}
\usepackage{lmodern}
\usepackage[ngerman]{babel}
% extra packages
\usepackage{amsmath}
\usepackage{enumerate}
\usepackage{siunitx}

% ------------------------------------------------------------------------------
\begin{document}
% ------------------------------------------------------------------------------

% ------------------------------------------
\section*{Geraden und Dreiecke in der Ebene}
% ------------------------------------------

% -------------------
\paragraph{Aufgabe 1}
% -------------------
In welchem Punkt schneidet die Gerade $g_1$, die durch die Punkte $P$
und $Q$ festgelegt ist, die Gerade $g_2$, die durch die Punkte $R$ und
$S$ verfäuft?

% -------------------
\paragraph{Aufgabe 2}
% -------------------
Nimmt man die $x$-Achse als dritte Gerade hinzu, ergibt sich zusammen
mit den Geraden $g_1$ und $g_2$ ein Dreieck.
\begin{enumerate}[{2}.1]
  \item Bestimme den Umfang dieses Dreiecks.
  \item Bestimme den Flächeninhalt dieses Dreiecks.
  \item Bestimme die Innenwinkel dieses Dreiecks.
\end{enumerate}

\hrulefill

\begin{equation*}
  \begin{array}{lllll}
    \text{a)} & P\left(\num{7}\;\middle|\;\num{2}\right) & Q\left(\num{1}\;\middle|\;\num{-6}\right) & R\left(\num{-15}\;\middle|\;\num{-16}\right) & S\left(\num{15}\;\middle|\;\num{8}\right) \\[1ex]
    \text{b)} & P\left(\num{1}\;\middle|\;\num{-2}\right) & Q\left(\num{18}\;\middle|\;\num{2}\right) & R\left(\num{5}\;\middle|\;\num{-9}\right) & S\left(\num{15}\;\middle|\;\num{-7}\right) \\[1ex]
    \text{c)} & P\left(\num{19}\;\middle|\;\num{18}\right) & Q\left(\num{-16}\;\middle|\;\num{-3}\right) & R\left(\num{10}\;\middle|\;\num{2}\right) & S\left(\num{-5}\;\middle|\;\num{-4}\right) \\[1ex]
    \text{d)} & P\left(\num{4}\;\middle|\;\num{14}\right) & Q\left(\num{9}\;\middle|\;\num{-16}\right) & R\left(\num{12}\;\middle|\;\num{7}\right) & S\left(\num{8}\;\middle|\;\num{3}\right) \\[1ex]
    \text{e)} & P\left(\num{3}\;\middle|\;\num{2}\right) & Q\left(\num{-8}\;\middle|\;\num{20}\right) & R\left(\num{13}\;\middle|\;\num{-3}\right) & S\left(\num{-20}\;\middle|\;\num{1}\right) \\[1ex]
    \text{f)} & P\left(\num{2}\;\middle|\;\num{13}\right) & Q\left(\num{3}\;\middle|\;\num{9}\right) & R\left(\num{12}\;\middle|\;\num{-9}\right) & S\left(\num{-14}\;\middle|\;\num{11}\right) \\[1ex]
    \text{g)} & P\left(\num{18}\;\middle|\;\num{-8}\right) & Q\left(\num{-2}\;\middle|\;\num{-18}\right) & R\left(\num{10}\;\middle|\;\num{11}\right) & S\left(\num{19}\;\middle|\;\num{14}\right) \\[1ex]
    \text{h)} & P\left(\num{7}\;\middle|\;\num{13}\right) & Q\left(\num{-9}\;\middle|\;\num{-19}\right) & R\left(\num{-2}\;\middle|\;\num{-16}\right) & S\left(\num{-8}\;\middle|\;\num{-13}\right) \\[1ex]
    \text{i)} & P\left(\num{-19}\;\middle|\;\num{8}\right) & Q\left(\num{-15}\;\middle|\;\num{16}\right) & R\left(\num{-3}\;\middle|\;\num{-5}\right) & S\left(\num{9}\;\middle|\;\num{15}\right) \\[1ex]
    \text{j)} & P\left(\num{15}\;\middle|\;\num{14}\right) & Q\left(\num{-8}\;\middle|\;\num{-13}\right) & R\left(\num{6}\;\middle|\;\num{19}\right) & S\left(\num{3}\;\middle|\;\num{10}\right) \\[1ex]
    \text{k)} & P\left(\num{18}\;\middle|\;\num{19}\right) & Q\left(\num{13}\;\middle|\;\num{14}\right) & R\left(\num{7}\;\middle|\;\num{3}\right) & S\left(\num{-10}\;\middle|\;\num{-2}\right) \\[1ex]
    \text{l)} & P\left(\num{-16}\;\middle|\;\num{13}\right) & Q\left(\num{12}\;\middle|\;\num{-1}\right) & R\left(\num{4}\;\middle|\;\num{15}\right) & S\left(\num{-2}\;\middle|\;\num{5}\right) \\[1ex]
    \text{m)} & P\left(\num{18}\;\middle|\;\num{-16}\right) & Q\left(\num{16}\;\middle|\;\num{-10}\right) & R\left(\num{-17}\;\middle|\;\num{-8}\right) & S\left(\num{-18}\;\middle|\;\num{-11}\right) \\[1ex]
    \text{n)} & P\left(\num{17}\;\middle|\;\num{3}\right) & Q\left(\num{12}\;\middle|\;\num{13}\right) & R\left(\num{-13}\;\middle|\;\num{-6}\right) & S\left(\num{-14}\;\middle|\;\num{-5}\right) \\[1ex]
    \text{o)} & P\left(\num{-6}\;\middle|\;\num{7}\right) & Q\left(\num{13}\;\middle|\;\num{-14}\right) & R\left(\num{-13}\;\middle|\;\num{-2}\right) & S\left(\num{4}\;\middle|\;\num{-19}\right) \\[1ex]
    \text{p)} & P\left(\num{19}\;\middle|\;\num{5}\right) & Q\left(\num{-17}\;\middle|\;\num{-4}\right) & R\left(\num{-1}\;\middle|\;\num{10}\right) & S\left(\num{3}\;\middle|\;\num{-11}\right) \\[1ex]
    \text{q)} & P\left(\num{-19}\;\middle|\;\num{-12}\right) & Q\left(\num{-13}\;\middle|\;\num{-2}\right) & R\left(\num{-15}\;\middle|\;\num{3}\right) & S\left(\num{-18}\;\middle|\;\num{1}\right) \\[1ex]
    \text{r)} & P\left(\num{-4}\;\middle|\;\num{6}\right) & Q\left(\num{2}\;\middle|\;\num{-14}\right) & R\left(\num{-5}\;\middle|\;\num{20}\right) & S\left(\num{-1}\;\middle|\;\num{-12}\right) \\[1ex]
    \text{s)} & P\left(\num{-7}\;\middle|\;\num{-10}\right) & Q\left(\num{9}\;\middle|\;\num{4}\right) & R\left(\num{15}\;\middle|\;\num{-1}\right) & S\left(\num{1}\;\middle|\;\num{-3}\right) \\[1ex]
    \text{t)} & P\left(\num{4}\;\middle|\;\num{-6}\right) & Q\left(\num{16}\;\middle|\;\num{-15}\right) & R\left(\num{7}\;\middle|\;\num{-18}\right) & S\left(\num{13}\;\middle|\;\num{18}\right) \\[1ex]
    \text{u)} & P\left(\num{13}\;\middle|\;\num{-7}\right) & Q\left(\num{15}\;\middle|\;\num{1}\right) & R\left(\num{2}\;\middle|\;\num{4}\right) & S\left(\num{6}\;\middle|\;\num{-20}\right)
  \end{array}
\end{equation*}


\clearpage

\paragraph{Lösung a)}
\begin{equation*}
  g_1:f(x)=\frac{\num{4}}{\num{3}}x-\frac{\num{22}}{\num{3}}
  \qquad
  g_2:f(x)=\frac{\num{4}}{\num{5}}x-\num{4}
\end{equation*}

\begin{equation*}
  A\left(\num{5}\;\middle|\;\num{0}\right)
  \qquad
  B\left(\frac{\num{11}}{\num{2}}\;\middle|\;\num{0}\right)
  \qquad
  C\left(\frac{\num{25}}{\num{4}}\;\middle|\;\num{1}\right)
\end{equation*}

\begin{equation*}
  U_D=\num{3.35078}
  \qquad
  A_D=\frac{\num{1}}{\num{4}}
\end{equation*}

\begin{equation*}
  \alpha=\num{38.6598}
  \qquad
  \beta=\num{53.1301}
  \qquad
  \gamma=\num{88.2101}
\end{equation*}

\paragraph{Lösung b)}
\begin{equation*}
  g_1:f(x)=\frac{\num{4}}{\num{17}}x-\frac{\num{38}}{\num{17}}
  \qquad
  g_2:f(x)=\frac{\num{1}}{\num{5}}x-\num{10}
\end{equation*}

\begin{equation*}
  A\left(\frac{\num{19}}{\num{2}}\;\middle|\;\num{0}\right)
  \qquad
  B\left(\num{50}\;\middle|\;\num{0}\right)
  \qquad
  C\left(\num{-220}\;\middle|\;\num{-54}\right)
\end{equation*}

\begin{equation*}
  U_D=\num{551.614}
  \qquad
  A_D=\num{1093.5}
\end{equation*}

\begin{equation*}
  \alpha=\num{13.2405}
  \qquad
  \beta=\num{11.3099}
  \qquad
  \gamma=\num{155.45}
\end{equation*}

\paragraph{Lösung c)}
\begin{equation*}
  g_1:f(x)=\frac{\num{3}}{\num{5}}x+\frac{\num{33}}{\num{5}}
  \qquad
  g_2:f(x)=\frac{\num{2}}{\num{5}}x-\num{2}
\end{equation*}

\begin{equation*}
  A\left(\num{-11}\;\middle|\;\num{0}\right)
  \qquad
  B\left(\num{5}\;\middle|\;\num{0}\right)
  \qquad
  C\left(\num{-43}\;\middle|\;-\frac{\num{96}}{\num{5}}\right)
\end{equation*}

\begin{equation*}
  U_D=\num{105.016}
  \qquad
  A_D=\frac{\num{768}}{\num{5}}
\end{equation*}

\begin{equation*}
  \alpha=\num{30.9638}
  \qquad
  \beta=\num{21.8014}
  \qquad
  \gamma=\num{127.235}
\end{equation*}

\paragraph{Lösung d)}
\begin{equation*}
  g_1:f(x)=-\num{6}x+\num{38}
  \qquad
  g_2:f(x)=x-\num{5}
\end{equation*}

\begin{equation*}
  A\left(\num{5}\;\middle|\;\num{0}\right)
  \qquad
  B\left(\frac{\num{19}}{\num{3}}\;\middle|\;\num{0}\right)
  \qquad
  C\left(\frac{\num{43}}{\num{7}}\;\middle|\;\frac{\num{8}}{\num{7}}\right)
\end{equation*}

\begin{equation*}
  U_D=\num{4.1082}
  \qquad
  A_D=\frac{\num{16}}{\num{21}}
\end{equation*}

\begin{equation*}
  \alpha=\num{45}
  \qquad
  \beta=\num{80.5377}
  \qquad
  \gamma=\num{54.4623}
\end{equation*}

\paragraph{Lösung e)}
\begin{equation*}
  g_1:f(x)=-\frac{\num{18}}{\num{11}}x+\frac{\num{76}}{\num{11}}
  \qquad
  g_2:f(x)=-\frac{\num{4}}{\num{33}}x-\frac{\num{47}}{\num{33}}
\end{equation*}

\begin{equation*}
  A\left(-\frac{\num{47}}{\num{4}}\;\middle|\;\num{0}\right)
  \qquad
  B\left(\frac{\num{38}}{\num{9}}\;\middle|\;\num{0}\right)
  \qquad
  C\left(\frac{\num{11}}{\num{2}}\;\middle|\;-\frac{\num{23}}{\num{11}}\right)
\end{equation*}

\begin{equation*}
  U_D=\num{35.7989}
  \qquad
  A_D=\num{16.6982}
\end{equation*}

\begin{equation*}
  \alpha=\num{6.91123}
  \qquad
  \beta=\num{58.5704}
  \qquad
  \gamma=\num{114.518}
\end{equation*}

\paragraph{Lösung f)}
\begin{equation*}
  g_1:f(x)=-\num{4}x+\num{21}
  \qquad
  g_2:f(x)=-\frac{\num{10}}{\num{13}}x+\frac{\num{3}}{\num{13}}
\end{equation*}

\begin{equation*}
  A\left(\frac{\num{3}}{\num{10}}\;\middle|\;\num{0}\right)
  \qquad
  B\left(\frac{\num{21}}{\num{4}}\;\middle|\;\num{0}\right)
  \qquad
  C\left(\frac{\num{45}}{\num{7}}\;\middle|\;-\frac{\num{33}}{\num{7}}\right)
\end{equation*}

\begin{equation*}
  U_D=\num{17.5414}
  \qquad
  A_D=\num{11.6679}
\end{equation*}

\begin{equation*}
  \alpha=\num{37.5686}
  \qquad
  \beta=\num{75.9638}
  \qquad
  \gamma=\num{66.4677}
\end{equation*}

\paragraph{Lösung g)}
\begin{equation*}
  g_1:f(x)=\frac{\num{1}}{\num{2}}x-\num{17}
  \qquad
  g_2:f(x)=\frac{\num{1}}{\num{3}}x+\frac{\num{23}}{\num{3}}
\end{equation*}

\begin{equation*}
  A\left(\num{-23}\;\middle|\;\num{0}\right)
  \qquad
  B\left(\num{34}\;\middle|\;\num{0}\right)
  \qquad
  C\left(\num{148}\;\middle|\;\num{57}\right)
\end{equation*}

\begin{equation*}
  U_D=\num{364.706}
  \qquad
  A_D=\num{1624.5}
\end{equation*}

\begin{equation*}
  \alpha=\num{18.4349}
  \qquad
  \beta=\num{26.5651}
  \qquad
  \gamma=\num{135}
\end{equation*}

\paragraph{Lösung h)}
\begin{equation*}
  g_1:f(x)=\num{2}x-\num{1}
  \qquad
  g_2:f(x)=-\frac{\num{1}}{\num{2}}x-\num{17}
\end{equation*}

\begin{equation*}
  A\left(\num{-34}\;\middle|\;\num{0}\right)
  \qquad
  B\left(\frac{\num{1}}{\num{2}}\;\middle|\;\num{0}\right)
  \qquad
  C\left(-\frac{\num{32}}{\num{5}}\;\middle|\;-\frac{\num{69}}{\num{5}}\right)
\end{equation*}

\begin{equation*}
  U_D=\num{80.7866}
  \qquad
  A_D=\num{238.05}
\end{equation*}

\begin{equation*}
  \alpha=\num{26.5651}
  \qquad
  \beta=\num{63.4349}
  \qquad
  \gamma=\num{90}
\end{equation*}

\paragraph{Lösung i)}
\begin{equation*}
  g_1:f(x)=\num{2}x+\num{46}
  \qquad
  g_2:f(x)=\frac{\num{5}}{\num{3}}x
\end{equation*}

\begin{equation*}
  A\left(\num{-23}\;\middle|\;\num{0}\right)
  \qquad
  B\left(\num{0}\;\middle|\;\num{0}\right)
  \qquad
  C\left(\num{-138}\;\middle|\;\num{-230}\right)
\end{equation*}

\begin{equation*}
  U_D=\num{548.372}
  \qquad
  A_D=\num{2645}
\end{equation*}

\begin{equation*}
  \alpha=\num{63.4349}
  \qquad
  \beta=\num{59.0362}
  \qquad
  \gamma=\num{57.5288}
\end{equation*}

\paragraph{Lösung j)}
\begin{equation*}
  g_1:f(x)=\frac{\num{27}}{\num{23}}x-\frac{\num{83}}{\num{23}}
  \qquad
  g_2:f(x)=\num{3}x+\num{1}
\end{equation*}

\begin{equation*}
  A\left(-\frac{\num{1}}{\num{3}}\;\middle|\;\num{0}\right)
  \qquad
  B\left(\frac{\num{83}}{\num{27}}\;\middle|\;\num{0}\right)
  \qquad
  C\left(-\frac{\num{53}}{\num{21}}\;\middle|\;-\frac{\num{46}}{\num{7}}\right)
\end{equation*}

\begin{equation*}
  U_D=\num{18.9668}
  \qquad
  A_D=\num{11.1958}
\end{equation*}

\begin{equation*}
  \alpha=\num{71.5651}
  \qquad
  \beta=\num{49.5739}
  \qquad
  \gamma=\num{58.861}
\end{equation*}

\paragraph{Lösung k)}
\begin{equation*}
  g_1:f(x)=x+\num{1}
  \qquad
  g_2:f(x)=\frac{\num{5}}{\num{17}}x+\frac{\num{16}}{\num{17}}
\end{equation*}

\begin{equation*}
  A\left(-\frac{\num{16}}{\num{5}}\;\middle|\;\num{0}\right)
  \qquad
  B\left(\num{-1}\;\middle|\;\num{0}\right)
  \qquad
  C\left(-\frac{\num{1}}{\num{12}}\;\middle|\;\frac{\num{11}}{\num{12}}\right)
\end{equation*}

\begin{equation*}
  U_D=\num{6.74504}
  \qquad
  A_D=\frac{\num{121}}{\num{120}}
\end{equation*}

\begin{equation*}
  \alpha=\num{16.3895}
  \qquad
  \beta=\num{45}
  \qquad
  \gamma=\num{118.61}
\end{equation*}

\paragraph{Lösung l)}
\begin{equation*}
  g_1:f(x)=-\frac{\num{1}}{\num{2}}x+\num{5}
  \qquad
  g_2:f(x)=\frac{\num{5}}{\num{3}}x+\frac{\num{25}}{\num{3}}
\end{equation*}

\begin{equation*}
  A\left(\num{-5}\;\middle|\;\num{0}\right)
  \qquad
  B\left(\num{10}\;\middle|\;\num{0}\right)
  \qquad
  C\left(-\frac{\num{20}}{\num{13}}\;\middle|\;\frac{\num{75}}{\num{13}}\right)
\end{equation*}

\begin{equation*}
  U_D=\num{34.6284}
  \qquad
  A_D=\num{43.2692}
\end{equation*}

\begin{equation*}
  \alpha=\num{59.0362}
  \qquad
  \beta=\num{26.5651}
  \qquad
  \gamma=\num{94.3987}
\end{equation*}

\paragraph{Lösung m)}
\begin{equation*}
  g_1:f(x)=-\num{3}x+\num{38}
  \qquad
  g_2:f(x)=\num{3}x+\num{43}
\end{equation*}

\begin{equation*}
  A\left(-\frac{\num{43}}{\num{3}}\;\middle|\;\num{0}\right)
  \qquad
  B\left(\frac{\num{38}}{\num{3}}\;\middle|\;\num{0}\right)
  \qquad
  C\left(-\frac{\num{5}}{\num{6}}\;\middle|\;\frac{\num{81}}{\num{2}}\right)
\end{equation*}

\begin{equation*}
  U_D=\num{112.381}
  \qquad
  A_D=\num{546.75}
\end{equation*}

\begin{equation*}
  \alpha=\num{71.5651}
  \qquad
  \beta=\num{71.5651}
  \qquad
  \gamma=\num{36.8699}
\end{equation*}

\paragraph{Lösung n)}
\begin{equation*}
  g_1:f(x)=-\num{2}x+\num{37}
  \qquad
  g_2:f(x)=-x-\num{19}
\end{equation*}

\begin{equation*}
  A\left(\num{-19}\;\middle|\;\num{0}\right)
  \qquad
  B\left(\frac{\num{37}}{\num{2}}\;\middle|\;\num{0}\right)
  \qquad
  C\left(\num{56}\;\middle|\;\num{-75}\right)
\end{equation*}

\begin{equation*}
  U_D=\num{227.419}
  \qquad
  A_D=\num{1406.25}
\end{equation*}

\begin{equation*}
  \alpha=\num{45}
  \qquad
  \beta=\num{63.4349}
  \qquad
  \gamma=\num{71.5651}
\end{equation*}

\paragraph{Lösung o)}
\begin{equation*}
  g_1:f(x)=-\frac{\num{21}}{\num{19}}x+\frac{\num{7}}{\num{19}}
  \qquad
  g_2:f(x)=-x-\num{15}
\end{equation*}

\begin{equation*}
  A\left(\num{-15}\;\middle|\;\num{0}\right)
  \qquad
  B\left(\frac{\num{1}}{\num{3}}\;\middle|\;\num{0}\right)
  \qquad
  C\left(\num{146}\;\middle|\;\num{-161}\right)
\end{equation*}

\begin{equation*}
  U_D=\num{460.139}
  \qquad
  A_D=\num{1234.33}
\end{equation*}

\begin{equation*}
  \alpha=\num{45}
  \qquad
  \beta=\num{47.8624}
  \qquad
  \gamma=\num{87.1376}
\end{equation*}

\paragraph{Lösung p)}
\begin{equation*}
  g_1:f(x)=\frac{\num{1}}{\num{4}}x+\frac{\num{1}}{\num{4}}
  \qquad
  g_2:f(x)=-\frac{\num{21}}{\num{4}}x+\frac{\num{19}}{\num{4}}
\end{equation*}

\begin{equation*}
  A\left(\num{-1}\;\middle|\;\num{0}\right)
  \qquad
  B\left(\frac{\num{19}}{\num{21}}\;\middle|\;\num{0}\right)
  \qquad
  C\left(\frac{\num{9}}{\num{11}}\;\middle|\;\frac{\num{5}}{\num{11}}\right)
\end{equation*}

\begin{equation*}
  U_D=\num{4.24162}
  \qquad
  A_D=\frac{\num{100}}{\num{231}}
\end{equation*}

\begin{equation*}
  \alpha=\num{14.0362}
  \qquad
  \beta=\num{79.2157}
  \qquad
  \gamma=\num{86.7481}
\end{equation*}

\paragraph{Lösung q)}
\begin{equation*}
  g_1:f(x)=\frac{\num{5}}{\num{3}}x+\frac{\num{59}}{\num{3}}
  \qquad
  g_2:f(x)=\frac{\num{2}}{\num{3}}x+\num{13}
\end{equation*}

\begin{equation*}
  A\left(-\frac{\num{39}}{\num{2}}\;\middle|\;\num{0}\right)
  \qquad
  B\left(-\frac{\num{59}}{\num{5}}\;\middle|\;\num{0}\right)
  \qquad
  C\left(-\frac{\num{20}}{\num{3}}\;\middle|\;\frac{\num{77}}{\num{9}}\right)
\end{equation*}

\begin{equation*}
  U_D=\num{33.1012}
  \qquad
  A_D=\num{32.9389}
\end{equation*}

\begin{equation*}
  \alpha=\num{33.6901}
  \qquad
  \beta=\num{59.0362}
  \qquad
  \gamma=\num{87.2737}
\end{equation*}

\paragraph{Lösung r)}
\begin{equation*}
  g_1:f(x)=-\frac{\num{10}}{\num{3}}x-\frac{\num{22}}{\num{3}}
  \qquad
  g_2:f(x)=-\num{8}x-\num{20}
\end{equation*}

\begin{equation*}
  A\left(-\frac{\num{5}}{\num{2}}\;\middle|\;\num{0}\right)
  \qquad
  B\left(-\frac{\num{11}}{\num{5}}\;\middle|\;\num{0}\right)
  \qquad
  C\left(-\frac{\num{19}}{\num{7}}\;\middle|\;\frac{\num{12}}{\num{7}}\right)
\end{equation*}

\begin{equation*}
  U_D=\frac{\num{439}}{\num{115}}
  \qquad
  A_D=\frac{\num{9}}{\num{35}}
\end{equation*}

\begin{equation*}
  \alpha=\frac{\num{663}}{\num{8}}
  \qquad
  \beta=\num{73.3008}
  \qquad
  \gamma=\num{23.8243}
\end{equation*}

\paragraph{Lösung s)}
\begin{equation*}
  g_1:f(x)=\frac{\num{7}}{\num{8}}x-\frac{\num{31}}{\num{8}}
  \qquad
  g_2:f(x)=\frac{\num{1}}{\num{7}}x-\frac{\num{22}}{\num{7}}
\end{equation*}

\begin{equation*}
  A\left(\frac{\num{31}}{\num{7}}\;\middle|\;\num{0}\right)
  \qquad
  B\left(\num{22}\;\middle|\;\num{0}\right)
  \qquad
  C\left(\num{1}\;\middle|\;\num{-3}\right)
\end{equation*}

\begin{equation*}
  U_D=\num{43.3404}
  \qquad
  A_D=\frac{\num{369}}{\num{14}}
\end{equation*}

\begin{equation*}
  \alpha=\num{41.1859}
  \qquad
  \beta=\num{8.1301}
  \qquad
  \gamma=\num{130.684}
\end{equation*}

\paragraph{Lösung t)}
\begin{equation*}
  g_1:f(x)=-\frac{\num{3}}{\num{4}}x-\num{3}
  \qquad
  g_2:f(x)=\num{6}x-\num{60}
\end{equation*}

\begin{equation*}
  A\left(\num{-4}\;\middle|\;\num{0}\right)
  \qquad
  B\left(\num{10}\;\middle|\;\num{0}\right)
  \qquad
  C\left(\frac{\num{76}}{\num{9}}\;\middle|\;-\frac{\num{28}}{\num{3}}\right)
\end{equation*}

\begin{equation*}
  U_D=\num{39.0176}
  \qquad
  A_D=\frac{\num{196}}{\num{3}}
\end{equation*}

\begin{equation*}
  \alpha=\num{36.8699}
  \qquad
  \beta=\num{80.5377}
  \qquad
  \gamma=\num{62.5924}
\end{equation*}

\paragraph{Lösung u)}
\begin{equation*}
  g_1:f(x)=\num{4}x-\num{59}
  \qquad
  g_2:f(x)=-\num{6}x+\num{16}
\end{equation*}

\begin{equation*}
  A\left(\frac{\num{8}}{\num{3}}\;\middle|\;\num{0}\right)
  \qquad
  B\left(\frac{\num{59}}{\num{4}}\;\middle|\;\num{0}\right)
  \qquad
  C\left(\frac{\num{15}}{\num{2}}\;\middle|\;\num{-29}\right)
\end{equation*}

\begin{equation*}
  U_D=\num{71.3759}
  \qquad
  A_D=\num{175.208}
\end{equation*}

\begin{equation*}
  \alpha=\num{80.5377}
  \qquad
  \beta=\num{75.9638}
  \qquad
  \gamma=\num{23.4986}
\end{equation*}



% ------------------------------------------------------------------------------
\end{document}
% ------------------------------------------------------------------------------

% -------------------------
\mysec{Allgemeine Dreiecke}
% -------------------------
\begin{tikzpicture}
  \coordinate (A)  at (0, 0);
  \coordinate (B)  at (6, 0);
  \coordinate (Ca) at (60:6cm);
  \coordinate (Cb) at ([xshift=6cm]140:6cm);
  \coordinate (C)  at (intersection of A--Ca and B--Cb);
  \coordinate (Fa) at ([shift={(270:6cm)}]C);
  \coordinate (F)  at (intersection of A--B and C--Fa);
  % Dreiecksseiten
  \draw (A) -- node[shift={(270:3mm)}] {{\footnotesize$c$}} (B);
  \draw (B) -- node[shift={(50:3mm)}]  {{\footnotesize$a$}} (C);
  \draw (C) -- node[shift={(150:3mm)}] {{\footnotesize$b$}} (A);
  % Hoehe
  \draw (C) -- node[shift={(0:3mm)}] {{\footnotesize$h_{c}$}} (F);
  % rechter Winkel
  \begin{scope}
    \clip (F) -- (B) -- (C) -- cycle;
    \draw (F) circle (5mm);
    \fill ([shift={(45:2.75mm)}]F) circle (1.1pt);
  \end{scope}
  % Eckpunkte
  \fill (A) circle (1pt);
  \fill (B) circle (1pt);
  \fill (C) circle (1pt);
  % Beschriftung der Eckpunkte
  \node[shift={(220:3mm)}] at (A) {{\footnotesize$A$}};
  \node[shift={(320:3mm)}] at (B) {{\footnotesize$B$}};
  \node[shift={(90:3mm)}]  at (C) {{\footnotesize$C$}};
  % Winkel
  \begin{scope}
    \clip (A) -- (B) -- (C) -- cycle;
    \draw (A) circle (8mm);
    \draw (B) circle (9mm);
    \draw (C) circle (8mm);
    \node[shift={(30:5mm)}]              at (A) {{\footnotesize$\alpha$}};
    \node[shift={(160:6mm)}]             at (B) {{\footnotesize$\beta$}};
    \node[shift={(280:5mm)}, fill=white] at (C) {{\footnotesize$\gamma$}};
  \end{scope}
\end{tikzpicture}

Fläche:\medskip
\formrow{A=\frac{c\cdot h_{c}}{2}=\frac{a\cdot h_{a}}{2}=\frac{b\cdot h_{b}}{2}}

Innenwinkelsumme:\medskip
\formrow{\alpha+\beta+\gamma=180^{\circ}}

Sinussatz:\medskip\par
\formrow{\frac{a}{\sin\alpha}=\frac{b}{\sin\beta}=\frac{c}{\sin\gamma}=2r_{u}}

Kosinussatz:\medskip\par
\formrow{c^{2}=a^{2}+b^{2}-2ab\cdot\cos\gamma}

Schnittpunkte:\medskip\par
\begingroup\small
\begin{tabular}{ll}
Höhen              &                               \\
Seitenhalbierenden & \itshape (Schwerpunkt)        \\
Winkelhalbierenden & \itshape (Inkreismittelpunkt) \\
Mittelsenkrechten  & \itshape (Umkreismittelpunkt)
\end{tabular}
\endgroup


\begingroup\small
% -------------------------------
\mysec{Wie sucht man die Lösung?}
% -------------------------------

% -----------------------------------------------
\paragraph{Erstens}\textsc{Verstehen der Aufgabe}
% -----------------------------------------------
\begin{itemize}
  \renewcommand{\itemsep}{-0.5ex}
  \item \textit{Was ist unbekannt? Was ist gegeben? Wie lautet die Bedingung?}
  \item Ist es möglich, die Bedingung zu befriedigen? Ist die Bedingung ausreichend,
        um die Unbekannte zu bestimmen? Oder ist sie unzureichend? Oder überbestimmt?
        Oder kontradiktorisch?
  \item Zeichne eine Figur! Führe passende Bezeichnungen ein!
  \item Trenne die verschiedenen Teile der Bedingung! Kannst du sie hinschreiben?
\end{itemize}

% ------------------------------------------------
\paragraph{Zweitens}\textsc{Ausdenken eines Plans}
% ------------------------------------------------
\begin{itemize}
  \renewcommand{\itemsep}{-0.5ex}
  \item Hast Du die Aufgabe schon früher gesehen? Oder hast Du dieselbe Aufgabe in einer
        wenig verschiedenen Form gesehen?
  \item \textit{Kennst du eine verwandte Aufgabe?} Kennst Du einen Lehrsatz, der förderlich
        sein könnte?
  \item \textit{Betrachte die Unbekannte!} Und versuche Dich auf eine Dir bekannte Aufgabe
        zu besinnen, die dieselbe oder eine ähnliche Unbekannte hat.
  \item \textit{Hier ist eine Aufgabe, die der Deinen verwandt und schon gelöst ist. Kannst Du sie
        gebrauchen?} Kannst Du ihr Resultat verwenden? Kannst Du ihre Methode verwenden?
        Würdest Du irgend ein Hilfselement einführen, damit Du sie verwenden kannst?
  \item Kannst Du die Aufgabe anders ausdrücken? Kannst Du sie auf noch verschiedene Weise
        ausdrücken? Geh auf die Definition zurück!
  \item Wenn Du die vorliegende Aufgabe nicht lösen kannst, so versuche, zuerst eine verwandte
        Aufgabe zu lösen. Kannst Du Dir eine zugänglichere verwandte Aufgabe denken? Eine
        allgemeinere Aufgabe? Eine speziellere Aufgabe? Eine analoge Aufgabe? Kannst Du einen Teil
        der Aufgabe lösen? Behalte nur einen Teil der Bedingung bei und lasse den anderen fort; wie
        weit ist die Unbekannte dann bestimmt, wie kann ich sie verändern? Kannst Du etwas
        Förderliches aus den Daten ableiten? Kannst Du Dir andere Daten denken, die geeignet sind,
        die Unbekannte zu bestimmen? Kannst Du die Unbekannte ändern, oder die Daten oder, wenn
        nötig beide, so dass die neue Unbekannte und die neuen Daten einander näher sind?
  \item Hast Du alle Daten benutzt? Hast Du die ganze Bedingung benutzt? Hast Du alle wesentlichen
        Begriffe in Rechnung gezogen, die in der Aufgabe enthalten sind?
\end{itemize}

% ----------------------------------------------
\paragraph{Drittens}\textsc{Ausführen des Plans}
% ----------------------------------------------
\begin{itemize}
  \renewcommand{\itemsep}{-0.5ex}
  \item Wenn Du Deinen Plan der Lösung durchführst, so \textit{kontrolliere jeden Schritt}. Kannst
        Du deutlich sehen, dass der Schritt richtig ist? Kannst Du beweisen, dass er richtig ist?
\end{itemize}

% ------------------------------------
\paragraph{Viertens}\textsc{Rückschau}
% ------------------------------------
\begin{itemize}
  \renewcommand{\itemsep}{-0.5ex}
  \item Kannst Du das \textit{Resultat kontrollieren?} Kannst Du den Beweis kontrollieren?
  \item Kannst Du das Resultat auf verschiedene Weise ableiten? Kannst Du es auf den ersten
        Blick sehen?
  \item Kannst Du das Resultat oder die Methode für irgend eine andere Aufgabe gebrauchen?
\end{itemize}

\hrulefill\\
Aus: \textsc{Polya}, George (1995): \textit{Schule des Denkens. Vom Lösen mathematischer Probleme.}
\endgroup

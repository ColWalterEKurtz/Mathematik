% -----------------
\mysec{Mittelwerte}
% -----------------
\begin{minipage}[t]{0.48\textwidth}
  Arithmetisches Mittel:\medskip
  \formrow{\bar{x}=\frac{1}{n}\left(x_{1}+x_{2}+\dotsb+x_{n}\right)}

  Geometisches Mittel:\medskip
  \formrow{\bar{x}=\sqrt[n]{x_{1}\cdot x_{2}\cdot\dotsb\cdot x_{n}}}
\end{minipage}\hfill
\begin{minipage}[t]{0.48\textwidth}
  Median:\medskip
  \formrow
  {
    \bar{x}=\left\{
            \begin{array}{ll}
              x_{\frac{n+1}{2}}                                         & \text{$n$ ungerade} \\[2ex]
              \frac{1}{2}\left(x_{\frac{n}{2}}+x_{\frac{n}{2}+1}\right) & \text{$n$ gerade}
            \end{array}
            \right.
  }
\end{minipage}

% -------------------
\mysec{Skalenniveaus}
% -------------------
\begin{center}
  \renewcommand{\arraystretch}{1.5}
  \begin{tabular}{|ll|l|l|}
    \hline
    \multicolumn{2}{|l|}{\textbf{Skalenniveau}} & \textbf{Operationen}            & \textbf{Mittelwerte}  \\
    \hline
    \multicolumn{2}{|l|}{Nominalskala}          & $=\;\neq$                       & Modus                 \\
    \hline
    \multicolumn{2}{|l|}{Ordinalskala}          & $=\;\neq\;<\;>$                 & Median                \\
    \hline
    \multirow{2}{*}{Kardinalskala}
      & \multicolumn{1}{|l|}{Intervallskala}    & $=\;\neq\;<\;>\;+\;-$           & arithmetisches Mittel \\
      \cline{2-4}
      & \multicolumn{1}{|l|}{Rationalskala}     & $=\;\neq\;<\;>\;+\;-\;\cdot\;:$ & geometrisches Mittel  \\
    \hline
  \end{tabular}
\end{center}

\formnum \textbf{Nominalskalierte Merkmale:} Deren mögliche Ausprägungen
können zwar unterschieden werden, weisen aber keine natürliche Rangfolge auf.
\begin{itemize}
  \setlength{\itemsep}{-0.5\baselineskip}
  \item Geschlecht: männlich, weiblich
  \item Geburtsort: Hamburg, Berlin, München
  \item Familienstand: ledig, verlobt, verheiratet, geschieden, verwitwet
  \item Religionszugehörigkeit: evangelisch, katholisch, muslimisch
\end{itemize}
\medskip\par

\formnum \textbf{Ordinalskalierte Merkmale:} Deren mögliche Ausprägungen können
unterschieden werden und weisen darüber hinaus eine natürliche Rangfolge auf.
\begin{itemize}
  \setlength{\itemsep}{-0.5\baselineskip}
  \item Schulnoten: sehr gut, gut, befriedigend, ausreichend, mangelhaft, ungenügend
  \item Musik-Charts: Platz 1, Platz 2, \ldots
  \item Schadstoffgruppen für Kraftfahrzeuge: 1, 2, 3, 4
  \item Dienstgrade beim Militär: Admiral, \ldots, Matrose
\end{itemize}
\medskip\par

\formnum \textbf{Intervallskalierte Merkmale:} Deren mögliche Ausprägungen
können so quantifiziert werden, dass gleichen Differenzen der Messwerte immer
gleichgroße Merkmalsunterschiede von je zwei Objekten entsprechen.
\begin{itemize}
  \setlength{\itemsep}{-0.5\baselineskip}
  \item Temperatur in ${}^{\circ}C$ oder ${}^{\circ}\!F$
  \item Kalenderzeit
  \item IQ-Skala
\end{itemize}
\medskip\par

\formnum \textbf{Rationalskalierte Merkmale:} Deren mögliche Ausprägungen
können so quantifiziert werden, dass der Markmalswert exakt dem Abstand zu
einem absoluten Nullpunkt entspricht.
\begin{itemize}
  \setlength{\itemsep}{-0.5\baselineskip}
  \item Temperatur in ${}^{\circ}\!K$
  \item Anzahl, Größe, Gewicht
  \item Preise
\end{itemize}


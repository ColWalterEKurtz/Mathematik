\documentclass
[
  draft    = true,
  fontsize = 11pt,
  parskip  = half-,
  BCOR     = 0pt,
  DIV      = 10,
  dvipsnames
]
{scrartcl}

% Standardpakete
\usepackage{fixltx2e}
\usepackage[utf8]{inputenc}
\usepackage[T1]{fontenc}
\usepackage{lmodern}
\usepackage[ngerman]{babel}
% Zusatzpakete
\usepackage{amsmath}
\usepackage{amssymb}
\usepackage{tikz}


\pagestyle{empty}

% ------------------------------------------------------------------------------
\begin{document}
% ------------------------------------------------------------------------------
\begin{center}
  \usekomafont{sectioning}%
  \usekomafont{section}%
  Hierarchie der (konvexen) Vierecke
\end{center}
\begin{center}
  \begin{tikzpicture}
    \newlength{\dx}
    \newlength{\dy}
    \setlength{\dx}{40mm}
    \setlength{\dy}{-30mm}
    % Pfeile
    %\draw (-5cm, -15cm) grid[xstep=10mm, ystep=10mm] (5cm, 1cm);
    \draw[line width=1pt, ->, >=latex] (-1.00,  -1.00) -- (-2.00,  -2.25);
    \draw[line width=1pt, ->, >=latex] ( 1.00,  -1.00) -- ( 2.00,  -2.25);
    \draw[line width=1pt, ->, >=latex] (-3.00,  -4.00) -- (-3.75,  -5.25);
    \draw[line width=1pt, ->, >=latex] (-1.00,  -4.00) -- (-0.25,  -5.25);
    \draw[line width=1pt, ->, >=latex] ( 3.00,  -4.00) -- ( 3.75,  -5.25);
    \draw[line width=1pt, ->, >=latex] ( 1.00,  -4.00) -- ( 0.25,  -5.25);
    \draw[line width=1pt, ->, >=latex] (-4.00,  -7.50) -- (-4.00,  -8.50);
    \draw[line width=1pt, ->, >=latex] ( 0.00,  -7.50) -- ( 0.00,  -8.75);
    \draw[line width=1pt, ->, >=latex] ( 4.00,  -7.50) -- ( 4.00,  -8.50);
    \draw[line width=1pt, ->, >=latex] ( 3.00,  -7.50) -- ( 0.75,  -9.00);
    \draw[line width=1pt, ->, >=latex] (-4.00, -11.50) -- (-1.25, -13.00);
    \draw[line width=1pt, ->, >=latex] ( 0.00, -11.50) -- ( 0.00, -12.50);
    \draw[line width=1pt, ->, >=latex] ( 4.00, -11.50) -- ( 1.25, -13.00);
    % Beschriftung
    \node[below=0.8cm] at ( 0.0\dx, 0\dy) {Quadrat};
    \node[below=0.8cm] at (-0.5\dx, 1\dy) {Raute};
    \node[below=0.8cm] at ( 0.5\dx, 1\dy) {Rechteck};
    \node[below=0.8cm] at (-1.0\dx, 2\dy) {Drachenviereck};
    \node[below=0.8cm] at ( 0.0\dx, 2\dy) {Parallelogramm};
    \node[below=0.8cm] at ( 1.0\dx, 2\dy) {gls. Trapez};
    \node[below=1.0cm] at (-1.0\dx, 3.25\dy) {Tangentenviereck};
    \node[below=1.0cm] at ( 0.0\dx, 3.25\dy) {Trapez};
    \node[below=1.0cm] at ( 1.0\dx, 3.25\dy) {Sehnenviereck};
    \node[below=1.0cm] at ( 0.0\dx, 4.5\dy) {konvexes Viereck};
    \begin{scope}
      % Quadrat
      \draw (-0.75, -0.5) rectangle (0.75, 1.0);
    \end{scope}
    \begin{scope}[xshift=-0.5\dx, yshift=\dy]
      % Raute
      \draw (0, -0.5) -- (1, 0) -- (0, 0.5) -- (-1, 0) -- cycle;
    \end{scope}
    \begin{scope}[xshift=0.5\dx, yshift=\dy]
      % Rechteck
      \draw (-1, -0.5) rectangle (1, 0.5);
    \end{scope}
    \begin{scope}[xshift=-\dx, yshift=2\dy]
      % Drachenviereck
      \draw (-1, 0) -- (-0.5, -0.5) -- (1, 0) -- (-0.5, 0.5) -- cycle;
    \end{scope}
    \begin{scope}[xshift=0mm, yshift=2\dy]
      % Parallelogramm
      \draw (-1, -0.5) -- (0.5, -0.5) -- (1, 0.5) -- (-0.5, 0.5) -- cycle;
    \end{scope}
    \begin{scope}[xshift=\dx, yshift=2\dy]
      % Trapez (gleichschenklig)
      \draw (-1, -0.5) -- (1, -0.5) -- (0.5, 0.5) -- (-0.5, 0.5) -- cycle;
    \end{scope}
    \begin{scope}[xshift=-\dx, yshift=3.25\dy, scale=0.8]
      % Eckpunkte
      \coordinate (A) at (-1.700000, -0.900000);
      \coordinate (B) at ( 1.290816, -1.079081);
      \coordinate (C) at ( 0.750000,  1.150000);
      \coordinate (D) at (-0.625503,  0.903999);
      % Inkreis
      \draw (0, 0) circle[radius=1];
      % Tangentenviereck
      \draw (A) --(B) -- (C) -- (D) -- cycle;
    \end{scope}
    \begin{scope}[xshift=0mm, yshift=3.25\dy, scale=1.25]
      % Trapez
      \draw (-1, -0.5) -- (1, -0.5) -- (0.25, 0.5) -- (-0.5, 0.5) -- cycle;
    \end{scope}
    \begin{scope}[xshift=\dx, yshift=3.25\dy]
      % Umkreis
      \draw (0, 0) circle[radius=1cm];
      % Eckpunkte
      \coordinate (A) at (340:1cm);
      \coordinate (B) at (50:1cm);
      \coordinate (C) at (160:1cm);
      \coordinate (D) at (255:1cm);
      % Sehnenviereck
      \draw (A) -- (B) -- (C) -- (D) -- cycle;
    \end{scope}
    \begin{scope}[xshift=0mm, yshift=4.5\dy]
      % Eckpunkte
      \coordinate (A) at (  -1, -0.7);
      \coordinate (B) at (   1, -0.7);
      \coordinate (C) at ( 0.7,  0.7);
      \coordinate (D) at (-0.8,  0.1);
      % allgemeines Viereck
      \draw (A) --(B) -- (C) -- (D) -- cycle;
    \end{scope}
  \end{tikzpicture}
\end{center}
\vspace*{\baselineskip}
\begin{center}
  Ein Pfeil
  \glqq$A$\,\raisebox{0.6ex}{\tikz\draw[line width=1pt, ->, >=latex] (0, 0)  -- (1, 0);}\,$B$\grqq{}
  bedeutet: \glqq Jedes $A$ ist auch ein $B$\grqq.
\end{center}

% ------------------------------------------------------------------------------
\end{document}
% ------------------------------------------------------------------------------


% Vorlage und globale Optionen
\documentclass
[
  draft    = true,
  fontsize = 11pt,
  parskip  = half-,
  BCOR     = 0pt,
  DIV      = calc
]
{scrartcl}

% Standardpakete
\usepackage{fixltx2e}
\usepackage[utf8]{inputenc}
\usepackage[T1]{fontenc}
\usepackage{lmodern}
\usepackage[ngerman]{babel}
% Zusatzpakete
\usepackage{amsmath}
\usepackage{enumerate}
\usepackage{graphicx}
\usepackage{tikz}


% ------------------------------------------------------------------------------
\begin{document}
% ------------------------------------------------------------------------------

\allowdisplaybreaks

% -------------------------------------------
\section*{Quadratische Funktionen (Parabeln)}
% -------------------------------------------
\begin{center}
  \newcommand{\tsize}{\footnotesize}
  \begin{tikzpicture}
    \clip[draw] (-2, -3) rectangle (8, 7);
    % Kaestchen
    \draw[ultra thin, xstep=5mm, ystep=5mm] (-2, -3) grid (8, 7);
    % Achsen
    \draw[line width=0.7pt, ->, >=latex] (-1.5, 0) -- (7.5, 0) node[below]{{\normalsize$x$}};
    \draw[line width=0.7pt, ->, >=latex] (0, -2.5) -- (0, 6.5) node[left] {{\normalsize$y$}};
    % Symmetrieachse
    \draw[line width=0.7pt, style=dashed] (3, -2.75) -- (3, 6.75);
    % Parabel
    \draw[line width=0.8pt] plot[smooth] file{fx.table};
    % Scheitelpunkt
    \fill (3, -2) circle (1.5pt);
    % y-Achsenabschnitt
    \fill (0, 2.5) circle (1.5pt);
    % Nullstellen
    \fill (1, 0) circle (1.5pt);
    \fill (5, 0) circle (1.5pt);
    % Beschriftung
    \node[right] at (3,  6.25) {{\tsize Symmetrieachse}};
    \node[right] at (3, -2.25) {{\tsize Scheitelpunkt}};
    \node[right] at (0,  2.75) {{\tsize y-Achsenabschnitt}};
    \node[left]  at (1, -0.25) {{\tsize Nullstelle}};
    \node[right] at (5, -0.25) {{\tsize Nullstelle}};
  \end{tikzpicture}
\end{center}

% ------------------------------------------------
\paragraph{Die verschiedenen Funktionsgleichungen}
% ------------------------------------------------
\begin{itemize}
  \item \makebox[10em][l]{Normalform:}         $y=ax^{2}+bx+c$
  \item \makebox[10em][l]{Scheitelpunktform:}  $y=a(x-d)^{2}+e$
  \item \makebox[10em][l]{Faktorisierte Form:} $y=a(x-s)(x-t)$
\end{itemize}
Aus jeder Darstellung lassen sich bestimmte Merkmale
der Parabel direkt ablesen:
\begin{itemize}
  \item y-Achsenabschnitt: $c$
  \item Koordinaten des Scheitelpunktes: $S(d\mid e)$
  \item Erste und zweite Nullstelle: $s$ und $t$
\end{itemize}
Der Faktor $a$ legt in allen drei Darstellungen fest, ob die Parabel
\emph{nach oben} oder \emph{nach unten} geöffnet ist, und ob sie
gegenüber der Normalparabel \emph{gestreckt} oder \emph{gestaucht}
erscheint.
\begin{equation*}
  \begin{split}
    a<0&\quad\Rightarrow\quad\text{nach unten geöffnet} \\
    a>0&\quad\Rightarrow\quad\text{nach oben geöffnet}  \\
    |a|<1&\quad\Rightarrow\quad\text{gestaucht} \\
    |a|=1&\quad\Rightarrow\quad\text{normal}    \\
    |a|>1&\quad\Rightarrow\quad\text{gestreckt}
  \end{split}
\end{equation*}

% ----------------------
\section*{Die pq-Formel}
% ----------------------
\begin{equation*}
  0=x^{2}+px+q
  \quad\Leftrightarrow\quad
  x_{1,2}=-\frac{p}{2}\pm\sqrt{\left(\frac{p}{2}\right)^{2}-q\;}
\end{equation*}

% --------------------
\paragraph{Beispiel 1}
% --------------------
Gesucht werden die Nullstellen folgender Parabel:
\begin{equation*}
  y=x^{2}-x-2
\end{equation*}

Der Ansatz lautet:
\begin{equation*}
  0=x^{2}-x-2
\end{equation*}

Da hier schon $a=1$ gilt, lassen sich $p$ und $q$ einfach ablesen:
\begin{equation*}
    p=-1\quad\text{und}\quad q=-2
\end{equation*}

Diese Zahlen setzt man in die Formel ein:
\begin{equation*}
  \begin{split}
    x_{1,2}&=-\frac{-1}{2}\pm\sqrt{\left(\frac{-1}{2}\right)^{2}-(-2)} \\[1ex]
           &=\frac{1}{2}\pm\sqrt{\frac{1}{4}+2} \\[1ex]
           &=\frac{1}{2}\pm\sqrt{\frac{1}{4}+\frac{8}{4}} \\[1ex]
           &=\frac{1}{2}\pm\sqrt{\frac{9}{4}}
            =\frac{1}{2}\pm\frac{3}{2}
             \quad\Leftrightarrow\quad (x=-1)\vee(x=2)
  \end{split}
\end{equation*}

Die Nullstellen liegen also bei $-1$ und $2$.

% --------------------
\paragraph{Beispiel 2}
% --------------------
Gesucht werden die Nullstellen folgender Parabel:
\begin{equation*}
  y=2x^{2}-26x+80
\end{equation*}

Der Ansatz lautet:
\begin{equation*}
  0=2x^{2}-26x+80
\end{equation*}

Da hier $a=2$ gilt, müssen beide Seiten der Gleichung erst durch 2 geteilt werden:
\begin{equation*}
  0=x^{2}-13x+40
\end{equation*}

Jetzt lassen sich $p$ und $q$ ablesen:
\begin{equation*}
    p=-13\quad\text{und}\quad q=40
\end{equation*}

Diese Zahlen setzt man in die Formel ein:
\begin{equation*}
  \begin{split}
    x_{1,2}&=-\frac{-13}{2}\pm\sqrt{\left(\frac{-13}{2}\right)^{2}-40} \\[1ex]
           &=\frac{13}{2}\pm\sqrt{\frac{169}{4}-40} \\[1ex]
           &=\frac{13}{2}\pm\sqrt{\frac{169}{4}-\frac{160}{4}} \\[1ex]
           &=\frac{13}{2}\pm\sqrt{\frac{9}{4}}
            =\frac{13}{2}\pm\frac{3}{2}
             \quad\Leftrightarrow\quad (x=5)\vee(x=8)
  \end{split}
\end{equation*}

Die Nullstellen liegen also bei $5$ und $8$.

% --------------------
\paragraph{Beispiel 3}
% --------------------
Gesucht werden die Nullstellen folgender Parabel:
\begin{equation*}
  y=\frac{1}{2}x^{2}-2x+3
\end{equation*}

Der Ansatz lautet:
\begin{equation*}
  0=\frac{1}{2}x^{2}-2x+3
\end{equation*}

Da hier $a=1/2$ gilt, müssen beide Seiten der Gleichung erst durch $1/2$ geteilt werden:
\begin{equation*}
  0=x^{2}-4x+6
\end{equation*}

Jetzt lassen sich $p$ und $q$ ablesen:
\begin{equation*}
    p=-4\quad\text{und}\quad q=6
\end{equation*}

Diese Zahlen setzt man in die Formel ein:
\begin{equation*}
  \begin{split}
    x_{1,2}&=-\frac{-4}{2}\pm\sqrt{\left(\frac{-4}{2}\right)^{2}-6} \\[1ex]
           &=2\pm\sqrt{\frac{16}{4}-6} \\[1ex]
           &=2\pm\sqrt{4-6} \\[1ex]
           &=2\pm\sqrt{-2}
  \end{split}
\end{equation*}
An dieser Stelle müsste man die Wurzel aus $-2$ ziehen, was in den reellen
Zahlen nicht möglich ist. Das bedeutet, das die Parabel keine (reellen)
Nullstellen besitzt.

Sie ist also entweder nach oben geöffnet und besitzt gleichzeitig einen
Scheitelpunkt, der oberhalb der $x$-Achse liegt, oder sie ist nach unten
geöffnet und besitzt einen Scheitelpunkt unterhalb der $x$-Achse.

% -----------------------------------
\section*{Die quadratische Ergänzung}
% -----------------------------------
\begin{equation*}
  % overbrace
  \text
  {%
    \makebox[0pt][l]
    {%
      $\displaystyle
       \hphantom{x^{2}+px+q=x^{2}+px}%
       \overbrace
       {%
         \phantom{+\left(\frac{p}{2}\right)^{2}-\left(\frac{p}{2}\right)^{2}}%
       }
       ^
       {%
         \text{zusammen 0}%
       }%
      $
    }%
  }%
  % underbrace
  \text
  {%
    \makebox[0pt][l]
    {%
      $\displaystyle
       \hphantom{x^{2}+px+q=}%
       \underbrace
       {%
         \phantom{x^{2}+px+\left(\frac{p}{2}\right)^{2}}%
       }
       _
       {%
         \text{\makebox[0pt][c]{vollständiges Binom}}%
       }%
      $
    }%
  }%
  x^{2}+px+q
  =
  x^{2}+px+\left(\frac{p}{2}\right)^{2}-\left(\frac{p}{2}\right)^{2}+q
\end{equation*}

% --------------------
\paragraph{Beispiel 4}
% --------------------
Gesucht wird der Scheitelpunkt folgender Parabel:
\begin{align*}
  y&=x^{2}-x-2
\intertext{Da hier bereits $a=1$ gilt, kann die quadratische Ergänzung
           direkt eingefügt werden:}
   &=x^{2}-x+\left(\frac{-1}{2}\right)^{2}-\left(\frac{-1}{2}\right)^{2}-2\\[1ex]
   &=x^{2}-x+\frac{1}{4}-\frac{1}{4}-\frac{8}{4}\\[1ex]
   &=\left(x-\frac{1}{2}\right)^{2}-\frac{9}{4}
\end{align*}
Also liegt der Scheitelpunkt der Parabel im Punkt
$\displaystyle S\left(\frac{1}{2}\;\bigg\vert\;-\frac{9}{4}\right)$.

% --------------------
\paragraph{Beispiel 5}
% --------------------
Gesucht wird der Scheitelpunkt folgender Parabel:
\begin{align*}
  y&=\frac{1}{2}x^{2}-x-\frac{3}{2}
\intertext{Da hier $\displaystyle a=\frac{1}{2}$ gilt, muss dieser Faktor erst
           ausgeklammert werden:}
   &=\frac{1}{2}\left[x^{2}-2x-3\right]
\intertext{Der Term in den eckigen Klammern lässt sich jetzt quadratisch Ergänzen:}
   &=\frac{1}{2}\left[x^{2}-2x+\left(\frac{-2}{2}\right)^{2}-\left(\frac{-2}{2}\right)^{2}-3\right] \\[1ex]
   &=\frac{1}{2}\left[x^{2}-2x+1-1-3\right] \\[1ex]
   &=\frac{1}{2}\left[x^{2}-2x+1-4\right]
\intertext{Mit der zweiten binomischen Formel gilt:}
   &=\frac{1}{2}\left[(x-1)^{2}-4\right]
\intertext{Die Scheitelpunktform ergibt sich nun durch Ausmultiplizieren:}
   &=\frac{1}{2}(x-1)^{2}-2
\end{align*}
Also liegt der Scheitelpunkt der Parabel im Punkt $S(1\mid-2)$.

\clearpage
% -------------------------------------------------
\section*{Die Faktorisierung der Scheitelpunktform}
% -------------------------------------------------
Falls in der Scheitelpunktform $y=a(x-d)^{2}+e$ die Parameter $a$ und $e$
verschiedene Vorzeichen haben, lässt sich die rechte Seite der Gleichung
mithilfe der dritten binomischen Formel $v^{2}-w^{2}=(v+w)\cdot(v-w)$
als Produkt schreiben.

Denn falls $a$ und $e$ verschiedene Vorzeichen haben, ergibt sich durch
das Ausklammern von $a$ immer eine \emph{Differenz} in den eckigen Klammern:
\begin{align*}
  y&=a(x-d)^{2}+e \\[1ex]
   &=a\left[(x-d)^{2}+\frac{e}{a}\right]=a\left[(x-d)^{2}-\frac{|e|}{|a|}\right]
\intertext{Und diese Differenz lässt sich dann als Teil einer dritten
           binomischen Formel interpretieren:}
   &=a\left[
        \left((x-d)+\sqrt{\frac{|e|}{|a|}}\right)
        \cdot
        \left((x-d)-\sqrt{\frac{|e|}{|a|}}\right)
      \right]
\intertext{Die ekigen Klammern um das Produkt sind nun überflüssig:}
   &=a\left((x-d)+\sqrt{\frac{|e|}{|a|}}\right)
      \left((x-d)-\sqrt{\frac{|e|}{|a|}}\right)
\intertext{Die runden Klammern um die beiden Differenzen auch:}
   &=a\left(x-d+\sqrt{\frac{|e|}{|a|}}\right)
      \left(x-d-\sqrt{\frac{|e|}{|a|}}\right)
\intertext{Um die Parameter besser von den Variablen unterscheiden zu können,
           ließen sich allerdings folgende Klammern wieder einfügen:}
   &=a\left(x-\left(d-\sqrt{\frac{|e|}{|a|}}\right)\right)
      \left(x-\left(d+\sqrt{\frac{|e|}{|a|}}\right)\right)
\end{align*}

Die beiden Nullstellen lassen sich jetzt aus dem Produkt einfach ablesen:
\begin{equation*}
  x_{1}=\left(d-\sqrt{\frac{|e|}{|a|}}\right)
  \quad\text{und}\quad
  x_{2}=\left(d+\sqrt{\frac{|e|}{|a|}}\right)
\end{equation*}

% --------------------
\paragraph{Beispiel 6}
% --------------------
Gesucht werden jetzt die Nullstellen der Parabel aus Beispiel 5:
\begin{align*}
  y&=\frac{1}{2}x^{2}-x-\frac{3}{2}
\intertext{Wir gehen allerdings davon aus, dass wir bereits den Scheitelpunkt
           bestimmt haben, und starten bei folgendem Umformungsschritt:}
   &=\frac{1}{2}\left[(x-1)^{2}-4\right]
\intertext{Die Differenz in den ekigen Klammern wird nun gemäß der dritten
           binomischen Formel als Produkt geschrieben:}
   &=\frac{1}{2}\Big[\big((x-1)+2\big)\cdot\big((x-1)-2\big)\Big]
\intertext{Jetzt verzichtet man auf die überflüssigen Klammern:}
   &=\frac{1}{2}(x-1+2)(x-1-2)
\intertext{Und fasst die Zahlen zusammen:}
   &=\frac{1}{2}(x+1)(x-3)
\intertext{Die Nullstellen liegen also bei:}
  x_{1}&=-1\quad\text{und}\quad x_{2}=3
\end{align*}

% ------------------------------------------------------------------------------
\end{document}
% ------------------------------------------------------------------------------


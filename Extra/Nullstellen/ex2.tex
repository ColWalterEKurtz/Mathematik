\begin{exercise}
      {ID-1efa64e28656dc77ca166967befad745ca0296a0}
      {Polynomdivision}
  \ifproblem\problem
    \begingroup
    \sbox{\dummy}{$\displaystyle\frac{b}{a}$}%
    \setlength{\ruled}{\dp\dummy}%
    \setlength{\rulet}{\ruled}%
    \addtolength{\rulet}{\ht\dummy}%
    \newcommand{\colgap}{\qquad&\rule[-\ruled]{0pt}{\rulet}\qquad}%
    Bestimme die restlichen Nullstellen:
    % use multiple pages if neccessary
    \allowdisplaybreaks
    \begin{alignat*}{3}
      \text{a)}\quad f(x)&=x^{\num{3}}-\num{2}x^{\num{2}}-\num{31}x-\num{28} \colgap \num{7}&\text{ ist eine Nullstelle} \\
      \text{b)}\quad f(x)&=x^{\num{3}}-\num{9}x^{\num{2}}-\num{16}x+\num{60} \colgap \num{-3}&\text{ ist eine Nullstelle} \\
      \text{c)}\quad f(x)&=-\num{2}x^{\num{3}}+\frac{\num{13}}{\num{2}}x^{\num{2}}+\num{13}x-\frac{\num{63}}{\num{8}} \colgap -\frac{\num{7}}{\num{4}}&\text{ ist eine Nullstelle}
    \end{alignat*}
    \endgroup
  \fi
  %\ifoutline\outline
  %\fi
  \ifoutcome\outcome
    \begingroup
    \sbox{\dummy}{$\displaystyle\frac{b}{a}$}%
    \setlength{\ruled}{\dp\dummy}%
    \setlength{\rulet}{\ruled}%
    \addtolength{\rulet}{\ht\dummy}%
    \newcommand{\colgap}{\qquad&\rule[-\ruled]{0pt}{\rulet}\qquad}%
    Bestimme die restlichen Nullstellen:
    % use multiple pages if neccessary
    \allowdisplaybreaks
    \begin{alignat*}{3}
      \text{a)}\quad f(x)&=x^{\num{3}}-\num{2}x^{\num{2}}-\num{31}x-\num{28} \colgap \text{NST}&=\left\{\num{-4}; \num{-1}; \num{7}\right\} \\
      \text{b)}\quad f(x)&=x^{\num{3}}-\num{9}x^{\num{2}}-\num{16}x+\num{60} \colgap \text{NST}&=\left\{\num{-3}; \num{2}; \num{10}\right\} \\
      \text{c)}\quad f(x)&=-\num{2}x^{\num{3}}+\frac{\num{13}}{\num{2}}x^{\num{2}}+\num{13}x-\frac{\num{63}}{\num{8}} \colgap \text{NST}&=\left\{-\frac{\num{7}}{\num{4}}; \frac{\num{1}}{\num{2}}; \frac{\num{9}}{\num{2}}\right\}
    \end{alignat*}
    \endgroup
  \fi
\end{exercise}

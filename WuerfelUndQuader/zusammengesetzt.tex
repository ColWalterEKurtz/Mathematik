\begin{exercise}
      {ID-8f64aa5d7be5483361618aefe563095c0f3b5fb4}
      {Zusammengesetzt}
  \ifproblem\problem
    Der abgebildete Körper besitzt folgende Maße:
    \begin{align*}
      a&=\sicm{10} & c&=\sicm{25} & e&=\sicm{15} \\
      b&=\sicm{25} & d&=\sicm{30} & f&=\sicm{35}
    \end{align*}
    \begin{center}
      \begin{tikzpicture}
        \coordinate (A) at (  0.0000,   0.0000);
        \coordinate (B) at ( -2.2759,  -1.3113);
        \coordinate (C) at ( -1.6552,  -1.4702);
        \coordinate (D) at ( -0.6207,  -0.8742);
        \coordinate (E) at (  0.9310,  -1.2715);
        \coordinate (F) at (  2.1724,  -0.5563);
        \coordinate (G) at (  0.0000,   1.0331);
        \coordinate (H) at ( -1.2414,   0.3179);
        \coordinate (I) at ( -2.2759,  -0.2781);
        \coordinate (J) at ( -1.6552,  -0.4371);
        \coordinate (K) at ( -0.6207,   0.1589);
        \coordinate (L) at (  0.9310,  -0.2384);
        \coordinate (M) at (  2.1724,   0.4768);
        \coordinate (N) at (  0.6207,   0.8742);
        \coordinate (O) at ( -0.0000,   3.4438);
        \coordinate (P) at ( -1.2414,   2.7285);
        \coordinate (Q) at ( -0.6207,   2.5696);
        \coordinate (R) at (  0.6207,   3.2848);
        \coordinate (S) at ( -1.2414,  -0.7152);
        \coordinate (T) at (  0.6207,  -0.1589);
        \coordinate (h1) at ( -2.8966,  -1.6689);
        \coordinate (h2) at ( -2.2759,  -1.8278);
        \coordinate (h3) at ( -0.7241,  -2.2252);
        \coordinate (h4) at (  0.8276,  -2.1060);
        \coordinate (h5) at (  1.8621,  -1.5100);
        \coordinate (h6) at (  3.1034,  -0.7947);
        \coordinate (h7) at (  3.1034,   0.2384);
        \coordinate (h8) at (  3.1034,   2.6490);
        \draw[line width=0.6pt, style=solid, join=bevel] (B) -- (C) -- (D) -- (E) -- (F);
        \draw[line width=0.6pt, style=solid, join=bevel] (K) -- (H) -- (I) -- (J) -- cycle;
        \draw[line width=0.6pt, style=solid, join=bevel] (K) -- (L) -- (M) -- (N) -- cycle;
        \draw[line width=0.6pt, style=solid, join=bevel] (O) -- (P) -- (Q) -- (R) -- cycle;
        \draw[line width=0.6pt, style=solid, join=bevel] (H) -- (P);
        \draw[line width=0.6pt, style=solid, join=bevel] (B) -- (I);
        \draw[line width=0.6pt, style=solid, join=bevel] (C) -- (J);
        \draw[line width=0.6pt, style=solid, join=bevel] (D) -- (Q);
        \draw[line width=0.6pt, style=solid, join=bevel] (E) -- (L);
        \draw[line width=0.6pt, style=solid, join=bevel] (F) -- (M);
        \draw[line width=0.6pt, style=solid, join=bevel] (N) -- (R);
        \draw[line width=0.6pt, style=dotted] (B) -- (h1);
        \draw[line width=0.6pt, style=dotted] (C) -- (h2);
        \draw[line width=0.6pt, style=dotted] (E) -- (h3);
        \draw[line width=0.6pt, style=dotted] (C) -- (h4);
        \draw[line width=0.6pt, style=dotted] (E) -- (h5);
        \draw[line width=0.6pt, style=dotted] (F) -- (h6);
        \draw[line width=0.6pt, style=dotted] (M) -- (h7);
        \draw[line width=0.6pt, style=dotted] (R) -- (h8);
        \draw[line width=0.4pt, style=solid, <->, >=latex] (h1) -- (h2);
        \node at ($(h1)!0.5!(h2)!2mm!270:(h2)$) {{\small\rule[-0.5ex]{0pt}{2.2ex}$a$}};
        \draw[line width=0.4pt, style=solid, <->, >=latex] (h2) -- (h3);
        \node at ($(h2)!0.5!(h3)!2mm!270:(h3)$) {{\small\rule[-0.5ex]{0pt}{2.2ex}$b$}};
        \draw[line width=0.4pt, style=solid, <->, >=latex] (h4) -- (h5);
        \node at ($(h4)!0.5!(h5)!2mm!270:(h5)$) {{\small\rule[-0.5ex]{0pt}{2.2ex}$c$}};
        \draw[line width=0.4pt, style=solid, <->, >=latex] (h5) -- (h6);
        \node at ($(h5)!0.5!(h6)!2mm!270:(h6)$) {{\small\rule[-0.5ex]{0pt}{2.2ex}$d$}};
        \draw[line width=0.4pt, style=solid, <->, >=latex] (h5) -- (h6);
        \node at ($(h5)!0.5!(h6)!2mm!270:(h6)$) {{\small\rule[-0.5ex]{0pt}{2.2ex}$d$}};
        \draw[line width=0.4pt, style=solid, <->, >=latex] (h6) -- (h7);
        \node at ($(h6)!0.5!(h7)!2mm!270:(h7)$) {{\small\rule[-0.5ex]{0pt}{2.2ex}$e$}};
        \draw[line width=0.4pt, style=solid, <->, >=latex] (h7) -- (h8);
        \node at ($(h7)!0.5!(h8)!2mm!270:(h8)$) {{\small\rule[-0.5ex]{0pt}{2.2ex}$f$}};
      \end{tikzpicture}
    \end{center}
    \begin{enumerate}[a)]
      \item Von diesem Körper kann man genau eine Ecke nicht sehen.
            Wie viele Ecken besitzt er also insgesamt?
      \item Berechne die Oberfläche und das Volumen des abgebildeten Körpers.
      \item Gib die Oberfläche zusätzlich in \si{\square\milli\metre} und
            \si{\square\deci\metre} an.
      \item Gib das Volumen zusätzlich in \si{\cubic\milli\metre} und
            \si{\cubic\deci\metre} an.
      \item Wie groß ist die Fläche des Körpers, die man aus der
            aktuelle Perspektive nicht sehen kann?
      \item Welches Volumen hat der kleinstmögliche Quader, in den der
            abgebildete Körper komplett hineinpasst?
      \item Welches Volumen hat der größtmögliche Quader, der komplett in den
            abgebildeten Körper hineinpasst?
      \item Gold besitzt eine Dichte von ca. \SI[per-mode=symbol]{19}{\gram\per\cubic\centi\metre}.
            Eine \emph{Feinunze} Gold wiegt etwa \sig{30} und kostet rund
            \eur{1000}. Wie schwer und wie teuer würde der Körper sein, wenn er
            aus purem Gold wäre?
    \end{enumerate}
  \fi
  %\ifoutline\outline
  %\fi
  \ifoutcome\outcome
    \begin{enumerate}[a)]
      \item Der Körper besitzt 17 Ecken.
      \item Der Körper besitzt eine Oberfläche von
            \sicmm{8100} und ein Volumen von
            \sicmmm{30000}.
      \item $O=\SI{810000}{\square\milli\metre}=\sidmm{81}$
      \item $V=\SI{30000000}{\cubic\milli\metre}=\sidmmm{30}$
      \item Aus der aktuellen Perspektive kann man eine Fläche von
            \sicmm{4050} des Körpers nicht sehen.
      \item Der kleinstmögliche Quader, in den der abgebildete Körper
            komplett hineinpasst, hat ein Volumen von \sicmmm{96250}.
      \item Der größtmögliche Quader, der komplett in den abgebildeten Körper
            hineinpasst, hat ein Volumen von \sicmmm{15750}.
      \item Wenn der Körper aus purem Gold wäre, würde er ca. \sikg{570}
            wiegen und ungefähr \eur{19000000} kosten.
    \end{enumerate}
  \fi
\end{exercise}

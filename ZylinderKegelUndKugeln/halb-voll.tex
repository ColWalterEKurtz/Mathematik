\begin{exercise}
      {ID-dee3efdab93124eeaf3216be19d9e151a7769957}
      {Halb voll}
  \ifproblem\problem
    Der Teil eines Sektglases, der mit Flüssigkeit gefüllt werden kann,
    hat die Form eines Kegels mit dem Durchmesser \sicm{6.6} und der
    Höhe \sicm{9.7}. \xxa{} hat es randvoll mit Orangensaft gefüllt
    und trinkt jetzt von dem Saft. Das Glas kann dabei auf verschiedene
    Arten noch \glqq halb voll\grqq{} sein. Untersuche folgende Fragen:
    \begin{enumerate}[a)]
      \item Wie viel Prozent des Volumens des Glases sind noch gefüllt,
            wenn das Glas bis zur halben Höhe mit Saft gefüllt ist?
      \item Wie hoch steht der Saft im Glas, wenn das halbe Volumen
            des Glases gefüllt ist?
      \item Wie hoch steht der Saft im Glas, wenn der Durchmesser des
            Flüssigkeitsspiegels auf die Hälfte abgenommen hat?
      \item Wie hoch steht der Saft, wenn die halbe Mantelfläche
            von Flüssigkeit bedeckt ist?
    \end{enumerate}
  \fi
  %\ifoutline\outline
  %\fi
  %\ifoutcome\outcome
  %\fi
\end{exercise}

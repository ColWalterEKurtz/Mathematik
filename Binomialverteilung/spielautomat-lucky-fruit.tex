\begin{exercise}
      {ID-531f79a24d8262c9c951f3742b32315b816d5f57}
      {Spielautomat Lucky Fruit}
  \ifthenelse{\isundefined{\introlen}}{\newlength{\introlen}}{\relax}%
  \ifthenelse{\isundefined{\followlen}}{\newlength{\followlen}}{\relax}%
  \ifproblem\problem\par
    \setlength{\introlen}{0.6\textwidth}
    \begin{minipage}[t]{\introlen}
      Der elektronisch gesteuerte Geldspielautomat \textit{Lucky Fruit}
      funktioniert so, dass er auf einem Bildschirm eine der drei Früchte
      Apfel, Birne oder  Zitrone anzeigt. Die Anzeigewahrscheinlichkeit für
      \glqq Apfel\grqq{} beträgt \pc{10}, für \glqq Birne\grqq{} \pc{30}
      und für \glqq Zitrone\grqq{} \pc{60}.
    \end{minipage}%
    \begin{minipage}[t]{0.40\textwidth}
      \flushright
      % define vertical size
      \raisebox{-4.4cm}[0pt][0pt]{%
        \begin{tikzpicture}
          % externe Bilder laden
          \pgfdeclareimage[height=6mm]{apfel}{\subdir fa.pdf};
          \pgfdeclareimage[height=6mm]{birne}{\subdir fb.pdf};
          \pgfdeclareimage[width=6mm]{zitrone}{\subdir fz.pdf};
          % Umriss
          \draw[line width=0.25mm, rounded corners=3mm] (0, 0) rectangle (4, 5);
          % Muenzen
          \draw (1.8, 4.5) circle[radius=3mm];
          \draw (2.5, 4.5) circle[radius=3mm];
          \node at (1.8, 4.5) {{\small10}};
          \node at (2.5, 4.5) {{\small10}};
          % Pfeil zum Einwurf
          \draw[line width=1mm, ->, >=latex] (2.9, 4.5) -- (3.5, 4.5);
          % Einwurf
          \draw (3.6, 4.2) rectangle (3.7, 4.8);
          % Name des Automaten
          \node at (2, 3.8) {{\sffamily\Large\bfseries Lucky Fruit}};
          % grauer Hintergrund
          \fill[fill=black!33!white] (1.65, 1.2) rectangle (2.35, 1.9);
          \fill[fill=black!33!white, yshift=14mm] (1.65, 1.2) rectangle (2.35, 1.9);
          % Pfeile im Kreuz
          \draw[line width=0.6mm, ->, >=latex] (0.95, 2.25) -- (1.55, 2.25);
          \draw[line width=0.6mm, <-, >=latex] (2.45, 2.25) -- (3.05, 2.25);
          % Kreuz in der Mitte
          \draw (1.65, 1.2) rectangle (2.35, 3.3);
          \draw (0.85, 1.9) rectangle (3.15, 2.6);
          % Fruechte
          \pgftext[at=\pgfpoint{2.00cm}{2.95cm}]{\pgfuseimage{birne}};
          \pgftext[at=\pgfpoint{2.00cm}{2.25cm}]{\pgfuseimage{apfel}};
          \pgftext[at=\pgfpoint{2.00cm}{1.55cm}]{\pgfuseimage{zitrone}};
          % Startknopf
          \draw (0.9, 0.95) circle[x radius=7mm, y radius=3mm];
          \node at (0.9, 0.95) {{\small Start}};
          % Stopknopf
          \draw (3.1, 0.95) circle[x radius=7mm, y radius=3mm];
          \node at (3.1, 0.95) {{\small Stop}};
          % Geldausgabe
          \draw[rounded corners] (0.6, 0.15) rectangle (3.4, 0.45);
        \end{tikzpicture}%
      }%
    \end{minipage}
    \begin{enumerate}[1)]
      \setlength   {\followlen}{\introlen}%
      \addtolength {\followlen}{-\leftskip}%
      \addtolength {\followlen}{-\leftmargini}%
      \item \begin{minipage}[t]{\followlen}
              Beim Sonderspiel \textit{Double Play} werden für den Einsatz
              von 20 Cent nacheinander zwei Spiele gestartet. Ein Gewinn
              stellt sich dann ein, wenn zweimal die gleiche Frucht
              angezeigt wird. Zeichnen Sie ein passendes Baumdiagramm.
            \end{minipage}
            \begin{enumerate}[a)]
              \item Bestimmen Sie die Wahrscheinlichkeit für den Fall,
                    dass sich kein Gewinn einstellt.
              \item Ermitteln Sie, wie viele Sonderspiele mindestens
                    gespielt werden müssen, damit die Wahrscheinlichkeit,
                    dass \glqq Apfel-Apfel\grqq{} gar nicht eintrifft,
                    kleiner ist als \pc{10}.
            \end{enumerate}
      \item Im Falle von \glqq Apfel-Apfel\grqq{} spielt der Automat
            75 Cent aus, bei \glqq Birne-Birne\grqq{} 50 Cent und
            bei \glqq Zitrone-Zitrone\grqq{} 25 Cent.
            In allen anderen Spielausfällen gibt es keine Auszahlung.
            \begin{enumerate}[a)]
              \item Berechnen Sie den bei einem Sonderspiel zu
                    erwartenden Auszahlungsbetrag.
              \item Die Einzelwahrscheinlichkeiten lassen sich
                    elektronisch verstellen. Das Gerät soll nun so
                    eingestellt werden, dass die Ereignisse
                    \glqq Birne\grqq{} und \glqq Zitrone\grqq{}
                    gleichwahrscheinlich sind. Ermitteln Sie, wie
                    die Wahrscheinlichkeit $p$ für \glqq Apfel\grqq{}
                    gewählt werden muss, damit der zu erwartende
                    Gewinn minimal wird.
            \end{enumerate}
      \item Die Steuerelektronik enthält für Kontrollzwecke einen
            Speicher, in dem die Ergebnisse der letzten 50 Einzelspiele
            gespeichert sind. Berechnen Sie die Wahrscheinlichkeit dafür,
            dass während dieser 50 Spiele \glqq Apfel\grqq{} und
            \glqq Zitrone\grqq{} zusammengenommen seltener eintrifft als
            \glqq Birne\grqq. Gehen Sie davon aus, dass die anfangs
            genannten Einzelwahrscheinlichkeiten nicht verstellt worden sind.
      \item Aufgrund gesetzlicher Vorgaben darf das Ordnungsamt einen der
            beschriebenen Spielautomaten nicht zulassen, wenn die
            Wahrscheinlichkeit $p$ für das Ereignis \glqq Apfel\grqq{}
            weniger als \num{0.1} beträgt. Durch den Einsatz
            fehlerhafter Bauteile ergab es sich aber, dass die
            Voreinstellung bei einigen Geräten leider doch zu einer
            Wahrscheinlichkeit $p$ führte, die unter \num{0.1} lag.
            Da die Nichtzulassung dieser Geräte drohte, wurde die
            Geräteserie repariert und anschließend in 100 Probespielen
            überprüft, mit welcher Wahrscheinlichkeit \glqq Apfel\grqq{}
            angezeigt wird.\par
            Geben Sie die Null- und die Gegenhypothese an. Bestimmen Sie
            die Entscheidungsregel für einen Test, bei dem die Nullhypothese
            auf einem Signifikanzniveau von \pc{2} überprüft wird und
            erklären Sie die Bedeutung der möglichen Fehlentscheidungen
            bei diesem Test. Begründen Sie die Wahl Ihrer Nullhypothese.
    \end{enumerate}
  \fi
  %\ifoutline\outline\par
  %\fi
  %\ifoutcome\outcome\par
  %\fi
\end{exercise}

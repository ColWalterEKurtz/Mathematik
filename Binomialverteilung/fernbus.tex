% 2022-01-04
\begin{exercise}
      {ID-82cea7dfc25a6b841fe3ace9c03da3d56eb4bebd}
      {Fernbus}
  \ifproblem\problem\par
    % <PROBLEM>
    Reisen mit dem Fernbus werden immer beliebter.
    Reiseanbieter werben mit günstigen Preisen
    und besonderem Komfort.
    \begin{enumerate}[a)]
      \item Für eine Städtereise stellt ein
            Busunternehmen einen Fernbus mit
            \num{59} Plätzen bereit, die vor
            Reiseantritt gebucht und bezahlt
            werden. Im Mittel werden
            \SI{95}{\percent} der Buchungen
            angetreten.
            \begin{enumerate}[(1)]
              \item Erläutern Sie, unter welchen
                    Voraussetzungen die Anzahl
                    der angetretenen Buchungen
                    bei einer Reise als
                    binomialverteilt mit
                    $p=\num{0.95}$ angenommen
                    werden kann.
            \end{enumerate}
            Im Folgenden wird die Anzahl der
            angetretenen Buchungen nun wirklich als
            binomialverteilt mit $p=\num{0.95}$
            vorausgesetzt. Für einen bestimmten
            Reisetermin sind genau \num{59}
            Buchungen vorgenommen worden.
            \begin{enumerate}[(1)]
              \setcounter{enumii}{1}%
              \item Bestimmen Sie jeweils die
                    Wahrscheinlichkeit für folgende
                    Ereignisse:
                    \begin{itemize}
                      %\renewcommand{\itemsep}{-1ex}%
                      \item[$E_1$:]
                        Genau \num{59} Buchungen
                        werden angetreten.
                      \item[$E_2$:]
                        Mindestens \num{55}
                        Buchungen werden angetreten.
                    \end{itemize}
            \end{enumerate}
      \item Da erfahrungsgemäß nicht alle Buchungen
            angetreten werden, verkauft das
            Busunternehmen mehr Plätze als vorhanden
            sind. Für eine Städtereise mit \num{96}
            Plätzen werden \num{99} Buchungen
            vorgenommen (Überbuchung). Es wird
            unverändert angenommen, dass die
            Anzahl der angetretenen Buchungen
            binomialverteilt mit $p=\num{0.95}$ ist.
            \begin{enumerate}[(1)]
              \item Ermitteln Sie die Wahrscheinlichkeit,
                    dass mehr als eine Person ihre Reise
                    wegen Überbuchung nicht antreten kann.
              \item Bestimmen Sie die Anzahl der Buchungen,
                    die das Busunternehmen bei einer Reise
                    mit \num{96} Plätzen höchstens
                    bestätigen darf, um das Risiko, dass
                    mindestens eine Person ihre Reise
                    aufgrund der Überbuchung nicht
                    antreten kann, auf \SI{5}{\percent}
                    zu begrenzen.
            \end{enumerate}
            Kann eine Person die Reise wegen
            Überbuchung nicht antreten, wird vom
            Busunternehmen der komplette
            Reisepreis von \num{20} Euro
            zurückerstattet. Als Entschädigung
            wird zusätzlich ein Betrag von
            \num{300} Euro ausgezahlt.
            \begin{enumerate}[(1)]
              \setcounter{enumii}{2}%
              \item Beurteilen Sie, ob sich aus
                    finanzieller Sicht die Praxis,
                    \num{99} Buchungen für eine
                    Reise mit \num{96} Plätzen zu
                    bestätigen, für das
                    Busunternehmen im Mittel lohnt.
            \end{enumerate}
      \item In der Werbung eines anderen Busunternehmens
            werden bisher Kunden damit gewonnen, dass bis
            kurz vor Reiseantritt eine kostenlose
            Stornierung der Buchung möglich ist. Aktuell
            liegt der Anteil der kurzfristig stornierten
            Buchungen bei \SI{7}{\percent}.
            Das Busunternehmen möchte diesen Anteil
            verringern und ändert die Vertragsbedingungen
            dahingehend, dass bei kurzfristigen
            Stornierungen ein Teil des Fahrpreises gezahlt
            werden muss. Es möchte die Vermutung absichern,
            dass durch diese Maßnahme der Anteil der
            kurzfristig stornierten Buchungen sinkt.
            Die nächsten \num{1000} Buchungen sollen auf
            diese Wirkung hin untersucht werden. Die
            Anzahl der kurzfristig stornierten Buchungen
            wird als binomialverteilt angenommen.
            \begin{enumerate}[(1)]
              \item Geben Sie begründet an, welche
                    Nullhypothese aus der Sicht des
                    Busunternehmens gewählt wird.
              \item Bestimmen Sie die Anzahl der
                    kurzfristig stornierten Buchungen,
                    bis zu der das Busunternehmen auf
                    einem Signifikanzniveau von
                    \SI{5}{\percent} von einem Erfolg
                    der Maßnahme ausgehen wird.
              \item Bei einem Signifikanzniveau von
                    \SI{1}{\percent} lautet die
                    Entscheidungsregel:
                    \begin{quote}
                      \glqq
                      Das Busunternehmen verwirft die
                      Nullhypothese $H_0$, und geht von
                      einem Erfolg der Maßnahme aus,
                      falls die Anzahl der kurzfristig
                      stornierten Buchungen höchstens
                      \num{51} beträgt.
                      \grqq
                    \end{quote}
                    Bestimmen Sie die Wahrscheinlichkeit
                    für den Fehler 2. Art, falls aufgrund
                    der Maßnahme der Anteil der
                    kurzfristig stornierten Buchungen nur
                    noch \SI{4}{\percent} beträgt, und
                    erläutern Sie diesen Fehler im
                    Sachzusammenhang.
            \end{enumerate}
    \end{enumerate}
    % </PROBLEM>
  \fi
  %\ifoutline\outline\par
    % <OUTLINE>
    % </OUTLINE>
  %\fi
  %\ifoutcome\outcome\par
    % <OUTCOME>
    % </OUTCOME>
  %\fi
\end{exercise}

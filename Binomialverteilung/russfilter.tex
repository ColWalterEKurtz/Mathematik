\begin{exercise}
      {ID-b57e34bb13f6c9a075f4fd932ade9b265c218e90}
      {Rußfilter}
  \ifproblem\problem
    Eine Fabrik $F_{1}$ stellt Rußfilter für Dieselmotoren her.
    \pc{25} der hergestellten Filter sind allerdings defekt und
    gelangen als Ausschussware in den Verkauf.
    \begin{enumerate}[1)]
      \item Berechnen Sie die Wahrscheinlichkeit dafür, dass sich
            in einer Stichprobe vom Umfang $n=50$
            \begin{enumerate}[a)]
              \item genau 15 defekte Rußfilter;
              \item mehr als 12 defekte Rußfilter;
              \item weniger als 12 defekte Rußfilter befinden.
            \end{enumerate}
      \item Die Fabriken $F_{2}$ und $F_{3}$ stellen ebenfalls
            Rußfilter her. Fabrik $F_{2}$ produziert mit einem
            Ausschussanteil von \pc{20} und Fabrik $F_{3}$ mit
            einem Ausschussanteil von \pc{30}.\par
            Ein Autohersteller bezieht \pc{50} seiner Rußfilter
            von Fabrik $F_{1}$, \pc{35} seiner Rußfilter von
            Fabrik $F_{2}$ und \pc{15} seiner Rußfilter von
            Fabrik $F_{3}$. Ein Rußfilter wird nun zufällig aus
            dem Lager des Autoherstellers entnommen und überprüft.
            \begin{enumerate}[a)]
              \item Bestimmen Sie die Wahrscheinlichkeit dafür,
                    dass er Ausschuss ist.
              \item Ermitteln Sie die Wahrscheinlichkeit dafür,
                    dass ein defekter Rußfilter von Fabrik $F_{2}$
                    bezogen wurde.
            \end{enumerate}
      \item In Fabrik $F_{1}$ wird ein neues Herstellungsverfahren
            eingeführt, welches den Ausschussanteil auf \pc{20}
            senken soll. Um zu überprüfen, ob der Ausschuss\-anteil
            tatsächlich gesenkt wurde (Nullhypothese: $H_{0}:p=\num{0.25}$),
            wird der Produktion eine Stichprobe vom Umfang $n=100$
            entnommen.
            \begin{enumerate}[a)]
              \item Bestimmen Sie auf dem Signifikanzniveau $\alpha=\num{0.05}$
                    eine Entscheidungsregel.
              \item Ermitteln Sie die Wahrscheinlichkeit dafür,
                    dass die Nullhypothese nicht verworfen wird,
                    obwohl der Ausschussanteil tatsächlich auf \pc{20}
                    gesenkt wurde.
            \end{enumerate}
    \end{enumerate}
  \fi
  %\ifoutline\outline
  %\fi
  %\ifoutcome\outcome
  %\fi
\end{exercise}

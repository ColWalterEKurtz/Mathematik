\begin{exercise}
      {ID-9945d02537df3ea4890a8338091dc506f33a6c0f}
      {Straßenbahn}
  \ifproblem\problem\par
    Nach Angaben der Initiative \glqq Pro-Bahn-Mitteldeutsch\-land\grqq{} beträgt
    der Anteil der Schwarzfahrer im Nahverkehr \SIrange{3}{4}{\percent}.
    Informationen der Magdeburger Verkehrsbetriebe zufolge wurden im
    vergangenen Jahr ca. 62 Millionen Personen befördert, die Anzahl der
    ertappten Schwarzfahrer betrug -- so konnte man in der Tageszeitung
    \glqq Volksstimme\grqq{} lesen -- \num{32000}.
    \begin{enumerate}[1)]
      \item Vergleichen Sie diese Zahlenwerte miteinander und
            kommentieren Sie das Ergebnis.
      \item Zwei Kontrolleure steigen an der Haltestelle \glqq Nicolaiplatz\grqq{}
            in eine Bahn der Linie~10 in Richtung Zentrum und kontrollieren
            alle 42 Fahrgäste. An der Haltestelle \glqq Alter Markt\grqq{} steigen
            sie in eine Bahn der Linie~6 Richtung Diesdorf um, in der sie weitere
            56 Fahrgäste überprüfen.
            \begin{enumerate}[a)]
              \item Berechnen Sie ausgehend von einem Schwarzfahreranteil von ca.
                    \pc{4} am gesamten Fahrgastaufkommen die Wahrscheinlichkeit,
                    dass die Kontrolleure bei diesen beiden Kontrollen mindestens
                    2 Schwarzfahrer ertappen.
              \item Untersuchen Sie, mit wie vielen Schwarzfahrern bei diesen
                    beiden Kontrollen zu rechnen ist.
              \item Berechnen Sie die Wahrscheinlichkeit, dass unter den 42
                    Fahrgästen der Linie~10 kein Schwarzfahrer ist.
              \item Ermitteln Sie rechnerisch, wie viele Personen überprüft
                    werden müssen, um mit einer Wahrscheinlichkeit von mehr als
                    \pc{95} mindestens einen Schwarzfahrer zu erwischen.
            \end{enumerate}
      \item Die Anzahl der Schwarzfahrer hält sich nur in Grenzen, wenn jeder
            Fahrgast mit einer gewissen Wahrscheinlichkeit $p$ kontrolliert wird.
            Ist dieser Wert $p$ zu klein, wird die Anzahl der Schwarzfahrer
            dramatisch ansteigen, ist der Wert zu hoch, gehen z.\,B. Einnahmen
            durch das \glqq erhöhte Beförderungsentgelt\grqq{} in Höhe von
            40~Euro verloren.\par
            Im Folgenden betrachten wir einen Fahrgast, der ca. 52 Fahrten
            im Monat unternimmt und gehen dabei von folgenden Annahmen aus:
            \begin{itemize}
              \item Der durchschnittliche Fahrpreis beträgt pro Fahrgast
                    und Fahrt 1,20 Euro.
              \item Der Schwarzfahreranteil am gesamten Fahrgastaufkommen
                    beträgt ca. \pc{4}.
              \item Ein Schwarzfahrer wird ehrlich, wenn er bei seinen 52
                    Fahrten mindestens zweimal kontrolliert wird.
              \item Ein ehrlicher Fahrgast wird zum Schwarzfahrer, wenn er
                    bei seinen 52 Fahrten überhaupt nicht kontrolliert wird.
            \end{itemize}
            \begin{enumerate}[a)]
              \item Begründen Sie, dass sich die Wahrscheinlichkeit dafür,
                    dass ein Schwarzfahrer ehrlich wird, durch folgenden
                    Term beschreiben lässt:
                    \begin{equation*}
                      1-(1-p)^{52}-52p(1-p)^{51}
                    \end{equation*}
              \item Begründen Sie, dass sich die Wahrscheinlichkeit dafür,
                    dass ein ehrlicher Fahrgast zum Schwarzfahrer wird, durch
                    folgenden Term beschreiben lässt:
                    \begin{equation*}
                      (1-p)^{52}
                    \end{equation*}
              \item Ermitteln Sie mit Hilfe eines zweistufigen Baumdiagramms
                    einen Funktionsterm $P_{1}$ in Abhängigkeit von $p$ für
                    die Wahrscheinlichkeit, dass ein beliebiger Fahrgast
                    am Ende des Monats Schwarzfahrer ist.
              \item Bestimmen Sie einen Funktionsterm $P_{2}$ für die
                    Wahrscheinlichkeit, dass ein beliebiger Fahrgast am
                    Ende des Monats ehrlicher Zahler ist.
              \item Stellen Sie die beiden Funktionsterme mit dem CAS grafisch
                    dar und beschreiben Sie jeweils den Verlauf des Graphen.
              \item Ermitteln Sie, für welches $p$ die Einnahmen pro Fahrgast
                    maximal werden.
              \item Untersuchen Sie, wie sich eine Verdoppelung des
                    \glqq erhöhten Beförderungsentgelts\grqq{} auswirkt.
            \end{enumerate}
    \end{enumerate}
  \fi
  %\ifoutline\outline\par
  %\fi
  %\ifoutcome\outcome\par
  %\fi
\end{exercise}

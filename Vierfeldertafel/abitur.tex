\begin{exercise}
      {ID-5abfb0bebe1996652a7547edfd9b270df68630b3}
      {Abitur}
  \ifproblem\problem\par
    \pc{52.4} der \num{244600} Jugendlichen, die am Ende des vergangenen
    Schuljahres ihre Schule mit der allgemeinen Hochschulreife verließen,
    waren Frauen. In den neuen Ländern und Berlin liegt der Frauenanteil
    mit \pc{59.1} deutlich höher als im früheren Bundesgebiet (\pc{50.8}).
    \begin{enumerate}
      \item Stelle eine Vierfeldertafel auf, die diesen Sachzusammenhang
            beschreibt.
      \item Zeichne ein Baumdiagramm mit \glqq Herkunft\grqq{} als erstem
            und \glqq{} Geschlecht\grqq{} als zweitem Merkmal.
      \item Zeichne ein Baumdiagramm mit \glqq Geschlecht\grqq{} als erstem
            und \glqq{} Herkunft\grqq{} als zweitem Merkmal.
      \item Aus der Gesamtheit aller Abiturientinnen und Abiturienten des
            betrachteten Jahrgangs wurde eine Person zufällig ausgewählt.
            \begin{enumerate}[a)]
              \item Mit welcher Wahrscheinlichkeit stammt diese Person aus
                    Ostdeutschland?
              \item Mit welcher Wahrscheinlichkeit ist die ausgewählte
                    Person eine Frau?
              \item Falls diese Person aus Ostdeutschland kommt, mit welcher
                    Wahrscheinlichkeit ist sie ein Mann?
              \item Falls diese Person eine Frau ist, mit welcher
                    Wahrscheinlichkeit kommt sie aus Westdeutschland?
            \end{enumerate}
    \end{enumerate}
  \fi
  %\ifoutline\outline\par
  %\fi
  %\ifoutcome\outcome\par
  %\fi
\end{exercise}

% 2022-01-04
\begin{exercise}
      {ID-ba5d5b861e1307c6b134becbcf70fe226ecb28c2}
      {Beliebtheit}
  \ifproblem\problem\par
    % <PROBLEM>
    Es soll die Beliebtheit einer Fernsehsendung
    überprüft werden. Eine Blitzumfrage lieferte
    folgendes Ergebnis: \SI{30}{\percent} der
    Zuschauerinnen und Zuschauer, die die Sendung
    gesehen hatten, waren \num{25} Jahre und
    jünger. Von diesen hatten \SI{50}{\percent}
    und von den übrigen Zuschauerinnen und
    Zuschauern (über \num{25} Jahre) hatten
    \SI{80}{\percent} eine positive Meinung.
    \begin{enumerate}[a)]
      \item Stelle den Sachzusammenhang in einer
            Vierfeldertafel dar.
      \item Stelle den Sachzusammenhang in einem
            Baumdiagramm dar und zeichne auch den
            inversen Baum. Bestimme alle
            Pfadwahrscheinlichkeiten.
      \item Wie viel Prozent der Zuschauerinnen
            und Zuschauer, von denen man weiß,
            dass sie eine positive Meinung über
            die Sendung haben, sind älter als
            \num{25} Jahre?
      \item Wie viel Prozent der Zuschauerinnen
            und Zuschauer, von denen man weiß,
            dass sie älter als \num{25} Jahre sind,
            haben keine positive Meinung über die
            Sendung?
      \item Überprüfe durch eine Rechnung, ob in
            diesem Fall Alter und Meinung stochastisch
            unabhängig sind.
    \end{enumerate}
    % </PROBLEM>
  \fi
  %\ifoutline\outline\par
    % <OUTLINE>
    % </OUTLINE>
  %\fi
  %\ifoutcome\outcome\par
    % <OUTCOME>
    % </OUTCOME>
  %\fi
\end{exercise}

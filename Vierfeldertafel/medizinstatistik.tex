\begin{exercise}
      {ID-70925a1da779e3897d6b9c8de86fd74714dee905}
      {Medizinstatistik}
  \ifproblem\problem
    Aufgrund von statistischen Erhebungen weiß man über eine bestimmte Krankheit
    folgendes: In der Bevölkerung ist von 150 Personen jeweils eine Person von
    der Krankheit betroffen.\par
    Der Test zur Diagnose dieser Krankheit zeigt mit einer Wahrscheinlichkeit von
    \num{0.97} die Krankheit an, wenn man tatsächlich krank ist. Ist man nicht
    krank, so zeigt dies der Test mit einer Wahrscheinlichkeit von \num{0.95} an.
    \begin{enumerate}
      \item Erstelle eine Vierfeldertafel.
      \item Mit welcher Wahrscheinlichkeit ist jemand tatsächlich krank,
            bei dem der Test die Krankheit anzeigt?
    \end{enumerate}
  \fi
  %\ifoutline\outline
  %\fi
  %\ifoutcome\outcome
  %\fi
\end{exercise}

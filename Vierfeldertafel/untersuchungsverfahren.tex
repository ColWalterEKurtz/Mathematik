\begin{exercise}
      {ID-2fbc5073213eee16776e59ffe7818b6dfaa1039a}
      {Untersuchungsverfahren}
  \ifproblem\problem
    Das Untersuchungsverfahren zur Diagnose einer bestimmten Krankheit hat
    noch folgende Fehlerquellen: \pc{1} der kranken Personen werden als
    gesund eingestuft, \pc{2} der gesunden Personen werden als krank eingestuft.
    Die Wahrscheinlichkeit, dass eine Person an dieser Krankheit leidet, sei $p$.
    \begin{enumerate}
      \item Mit welcher Wahrscheinlichkeit ist eine Person krank,
            wenn sie als gesund eingestuft wurde?
      \item Mit welcher Wahrscheinlichkeit diagnostiziert das Untersuchungsverfahren
            die Krankheit korrekt?
      \item Wie groß ist $p$, wenn die Wahrscheinlichkeit \num{0.0007} ist, dass
            eine vom Diagnoseverfahren als gesund eingestufte Person krank ist?
    \end{enumerate}
  \fi
  %\ifoutline\outline
  %\fi
  %\ifoutcome\outcome
  %\fi
\end{exercise}

\begin{exercise}
      {ID-c685dcd0fc90860cecb6175ef8f68b17a8de5388}
      {Infektionskrankheit}
  \ifproblem\problem\par
    In einem Land der Dritten Welt leidet \pc{1} der Menschen an einer
    bestimmten Infektionskrankheit. Ein Test zeigt die Krankheit bei den
    tatsächlich erkrankten zu \pc{98} korrekt an. Leider zeigt der Test
    auch \pc{3} der Gesunden als erkrankt an.
    \begin{enumerate}[a)]
      \item Stelle den Sachzusammenhang in einer Vierfeldertafel dar.
      \item Stelle den Sachzusammenhang in einem Baumdiagramm dar und zeichne
            auch den inversen Baum. Bestimme alle Pfadwahrscheinlichkeiten.
      \item Mit welcher Wahrscheinlichkeit zeigt der Test bei einer zufällig
            ausgewählten Person ein positives Ergebnis?
      \item Mit welcher Wahrscheinlichkeit ist eine als positiv getestete
            Person auch tatsächlich krank? Kommentiere das Ergebnis.
      \item Mit welcher Wahrscheinlichkeit ist eine als negativ getestete
            Person gesund? Kommentiere das Ergebnis.
    \end{enumerate}
  \fi
  %\ifoutline\outline\par
  %\fi
  %\ifoutcome\outcome\par
  %\fi
\end{exercise}

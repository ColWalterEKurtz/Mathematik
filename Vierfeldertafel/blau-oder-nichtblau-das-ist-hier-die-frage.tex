\begin{exercise}
      {ID-141a39f9b876d741e156bea9ccd6368c30f71592}
      {Blau oder Nichtblau, das ist hier die Frage}
  \ifproblem\problem\par
    % <PROBLEM>
    An Freitagen fehlen \xya{} und \xxb{} oft
    in der Schule. \xya{} fehlt freitags mit einer
    Wahrscheinlichkeit von \num{0.3} und \xxb{}
    ist mit einer Wahrscheinlichkeit von \num{0.45}
    nicht da. Die Wahrscheinlichkeit dafür,
    dass beide anwesend sind, beträgt nur \num{0.4}.
    Sind die Abwesenheit von \xya{} und die
    Abwesenheit von \xxb{} an Freitagen unabhängige
    Ereignisse?
    % </PROBLEM>
  \fi
  \ifoutline\outline\par
    % <OUTLINE>
    Wenn zwei Ereignisse $A$ und $B$ stochastisch
    unabhängig voneinander sind, gilt folgender
    Zusammenhang:
    \begin{equation*}
      P(A\cap B)=P(A)\cdot P_{A}(B)=P(A)\cdot P(B)
    \end{equation*}
    Das Ereignis $A$ beeinflusst das Eintreten
    von Ereignis $B$ also überhaupt nicht:
    \begin{equation*}
      P(B)
      =
      P_{A}(B)
      =
      \frac{P(A\cap B)}{P(A)}
    \end{equation*}
    % </OUTLINE>
  \fi
  \ifoutcome\outcome\par
    % <OUTCOME>
    Mit den Angaben aus der Aufgabenstellung lässt
    sich folgende Vierfeldertafel aufstellen:\par
    \begin{minipage}{0.4\linewidth}
      \begin{fofotab}%[t]
        % Bezeichnungen
        \lblA{$A$}
        \lbla{$\overline{A}$}
        \lblB{$B$}
        \lblb{$\overline{B}$}
        % Mitte
        \andAB{\num{0.4}}
        \andAb{\num{0.3}}
        \andaB{\num{0.15}}
        \andab{\num{0.15}}
        % Rand
        \sumA {\num{0.7}}
        \suma {\num{0.3}}
        \sumB {\num{0.55}}
        \sumb {\num{0.45}}
        \total{1}
      \end{fofotab}%
    \end{minipage}%
    \begin{minipage}{0.58\linewidth}
      \begin{itemize}
        \renewcommand{\itemsep}{-1ex}%
        \item[$A$:] \glqq\xya{} ist anwesend\grqq
        \item[$\overline{A}$:] \glqq\xya{} ist nicht anwesend\grqq
        \item[$B$:] \glqq\xxb{} ist anwesend\grqq
        \item[$\overline{B}$:] \glqq\xxb{} ist nicht anwesend\grqq
      \end{itemize}
    \end{minipage}\par
    Wenn die An- bzw. Abwesenheit von \xya{} die
    An- bzw. Abwesenheit von \xxb{} nicht
    beeinflussen würde, müsste für die gemeinsame
    An- bzw. Abwesenheit gelten:
    \begin{equation*}
      \begin{split}
        P(A\cap B)
        &=P(A)\cdot P(B)
        \\
        P(A\cap\overline{B})
        &=P(A)\cdot P(\overline{B})
        \\
        P(\overline{A}\cap B)
        &=P(\overline{A})\cdot P(B)
        \\
        P(\overline{A}\cap\overline{B})
        &=P(\overline{A})\cdot P(\overline{B})
      \end{split}
    \end{equation*}
    Das Nachrechnen mit den Werten aus der
    Vierfeldertafel ergibt:
    \begin{equation*}
      P(A)\cdot P(B)=
      \num{0.7}\cdot\num{0.55}=\num{0.385}
      \neq
      \num{0.4}=P(A\cap B)
    \end{equation*}
    Wenn \xya{} und \xxb{} unabhängig voneinander
    in der Schule fehlen würden, dann wären sie nur
    an \pc{38.5} aller Freitage gemeinsam anwesend.
    Da sie aber an \pc{40} aller Freitage gemeinsam
    anwesend sind, scheint die Anwesenheit der bzw.
    des Einen die Anwesenheit des bzw. der Anderen
    positiv zu beeinflussen.
    % </OUTCOME>
  \fi
\end{exercise}

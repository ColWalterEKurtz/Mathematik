% 2021-12-03
\begin{exercise}
      {ID-570ec2b5c46d080cddf6769bb433b0dac809c5e9}
      {Tabakkonsum}
  \ifproblem\problem\par
    % <PROBLEM>
    An einem Berufskolleg wurden alle \num{674}
    Schülerinnen und Schüler befragt, ob sie
    rauchen oder nicht. Das Ergebnis der
    Befragung sieht wie folgt aus:
    \num{82} der insgesamt \num{293} männlichen
    Schüler gaben an zu rauchen.
    \num{250} Schülerinnen gaben an, nicht zu
    rauchen.
    \begin{enumerate}[a)]
      \item Stelle den Sachzusammenhang in einer
            Vierfeldertafel dar.
      \item Mit welcher Wahrscheinlichkeit ist
            eine zufällig ausgewählte Person
            weiblich und Nichtraucherin?
      \item Mit welcher Wahrscheinlichkeit ist
            eine Schülerin Nichtraucherin?
      \item Untersuche ob die Merkmale
            \textit{Geschlecht} und
            \textit{Tabakkonsum} stochastisch
            unabhängig voneinander sind.
    \end{enumerate}
    % </PROBLEM>
  \fi
  %\ifoutline\outline\par
    % <OUTLINE>
    % </OUTLINE>
  %\fi
  %\ifoutcome\outcome\par
    % <OUTCOME>
    % </OUTCOME>
  %\fi
\end{exercise}

\begin{exercise}
      {ID-88a4ca5d6db7e361e14af294394530579ad3b560}
      {Fernsehsendung}
  \ifproblem\problem
    Es soll die Beliebtheit einer Fernsehsendung überprüft werden.
    Eine Blitzumfrage hatte folgendes Ergebnis: \pc{30} der Zuschauer,
    die die Sendung gesehen hatten, waren 25 Jahre und jünger.
    Von diesen hatten \pc{50} und von den übrigen Zuschauern (über 25 Jahre)
    hatten \pc{80} eine positive Meinung.
    \begin{enumerate}
      \item Stelle den Sachzusammenhang in einer Vierfeldertafel dar.
      \item Stelle den Sachzusammenhang in einem Baumdiagramm dar und zeichne
            auch den inversen Baum. Bestimme alle Pfadwahrscheinlichkeiten.
      \item Wie viel \% der Zuschauer, von denen man weiß, dass sie eine
            positive Meinung über die Sendung hatten, waren älter als 25 Jahre?
      \item Wie viel \% der Zuschauer, von denen man weiß, dass sie älter als
            25 Jahre sind, hatten keine positive Meinung über die Sendung?
      \item Überprüfe durch Rechnung ob in diesem Fall \textit{Alter} und
            \textit{Meinung} unabhängig voneinander sind.
    \end{enumerate}
  \fi
  %\ifoutline\outline
  %\fi
  %\ifoutcome\outcome
  %\fi
\end{exercise}

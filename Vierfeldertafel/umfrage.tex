\begin{exercise}
      {ID-61c3a589e641fea087fc6a6dd89fe40d9628a39d}
      {Umfrage}
  \ifproblem\problem\par
    Ein Betreiber eines Eisenbahnunternehmens hat eine Umfrage unter
    seinen Fahrgästen durchgeführt, die ergab, dass \pc{10} der Fahrgäste
    in der ersten Klasse reisen. Außerdem wurde in der Umfrage abgefragt,
    wie zufrieden die Fahrgäste mit dem Service des Unternehmens sind.
    Hoch erfreut stellt das Unternehmen fest, dass $5/6$ der Fahrgäste
    zweiter Klasse zufrieden sind. Alarmierend dagegen sind die
    Zufridenheitszahlen der ersten Klasse: \pc{70} der Fahrgäste erster
    Klasse sind unzufrieden. Betrachtet werden folgende Ereignisse:
    \begin{itemize}
      \item[E:] \glqq Ein Teilnehmer der Umfrage ist 1. Klasse-Fahrer\grqq
      \item[Z:] \glqq Ein Teilnehmer der Umfrage ist mit dem Service
                des Unternehmens zufrieden\grqq
    \end{itemize}
    \begin{enumerate}
      \item Erstelle eine vollständig ausgefüllte Vierfeldertafel.
      \item Als dem Geschäftsführer die Zufriedenheitszahlen der 1. Klasse
            mitgeteilt werden, ist dieser schockiert. Resigniert erklärt er,
            dass das Unternehmen es nicht geschafft habe, den Zufriedenheitswert
            von \pc{77} der Fahrgäste aus dem Vorjahr zu verbessern.
            Hat er Recht?
      \item Tatsächlich stellt er fest, dass im Vorjahr \pc{85} der Fahrgäste
            zweiter Klasse zufrieden waren und immerhin \pc{45} der Fahrgäste
            erster Klasse. Damit haben sich beide Werte in diesem Jahr
            verschlechtert. Stelle diese Werte in Bezug zu deiner Antwort aus
            Teilaufgabe 2. Erstelle dazu auch eine Vierfeldertafel für das Vorjahr.
    \end{enumerate}
  \fi
  %\ifoutline\outline\par
  %\fi
  %\ifoutcome\outcome\par
  %\fi
\end{exercise}

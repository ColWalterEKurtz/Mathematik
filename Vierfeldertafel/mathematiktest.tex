\begin{exercise}
      {ID-5442ca44082f03531796efdd1b3f80a687e8e9aa}
      {Mathematiktest}
  \ifproblem\problem
    Die \num{16} Jungen und \num{14} Mädchen einer Schulklasse nehmen an einem
    Mathematiktest teil. \num{13} Jungen bestehen. Insgesamt bestehen \num{20}
    Schülerinnen und Schüler den Test. Wie viele Mädchen bestehen den
    Test nicht?
  \fi
  \ifoutline\outline
    Die Angaben aus der Aufgabenstellung lassen sich z.\,B. auf folgende
    Weise in einer Vierfeldertafel darstellen:
    \begin{center}
      \begin{fourfoldtable}
        \Apos{Junge}%
        \Aneg{Mädchen}%
        \Bpos{bestanden}%
        \Bneg{$\overline{\text{bestanden}}$}%
        \numbers{13}{}{20}{}{}{}{16}{14}{}%
      \end{fourfoldtable}%
    \end{center}
  \fi
  \ifoutcome\outcome
    Insgesamt bestehen \num{7} Mädchen den Mathematiktest nicht.
    \begin{center}
      \begin{fourfoldtable}
        \Apos{Junge}%
        \Aneg{Mädchen}%
        \Bpos{bestanden}%
        \Bneg{$\overline{\text{bestanden}}$}%
        \numbers{13}{7}{20}{3}{7}{10}{16}{14}{30}%
      \end{fourfoldtable}%
    \end{center}
  \fi
\end{exercise}

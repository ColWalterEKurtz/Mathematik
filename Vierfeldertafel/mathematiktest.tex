\begin{exercise}
      {ID-5442ca44082f03531796efdd1b3f80a687e8e9aa}
      {Mathematiktest}
  \ifproblem\problem
    Die 16 Jungen und 14 Mädchen einer Schulklasse nehmen an einem
    Mathematiktest teil. 13 Jungen bestehen. Insgesamt bestehen 20
    Schülerinnen und Schüler den Test. Wie viele Mädchen bestehen den
    Test nicht?
  \fi
  \ifoutline\outline
    Die Angaben aus der Aufgabenstellung lassen sich z.\,B. auf folgende
    Weise in einer Vierfeldertafel darstellen:
    \begin{center}
      \newcommand{\attribApos}{Junge}%
      \newcommand{\attribAneg}{Mädchen}%
      \newcommand{\attribBpos}{bestanden}%
      \newcommand{\attribBneg}{$\overline{\text{bestanden}}$}%
      \renewcommand{\arraystretch}{1.25}%
      \begin{tabular}{|c||c|c||c|}
        \hline
                    & \attribApos & \attribAneg & Summe \\
        \hline
        \hline
        \attribBpos & 13          &             & 20    \\
        \hline
        \attribBneg &             &             &       \\
        \hline
        \hline
        Summe       & 16          & 14          &       \\
        \hline
      \end{tabular}%
    \end{center}
  \fi
  \ifoutcome\outcome
    Insgesamt bestehen 7 Mädchen den Mathematiktest nicht.
    \begin{center}
      \newcommand{\attribApos}{Junge}%
      \newcommand{\attribAneg}{Mädchen}%
      \newcommand{\attribBpos}{bestanden}%
      \newcommand{\attribBneg}{$\overline{\text{bestanden}}$}%
      \renewcommand{\arraystretch}{1.25}%
      \begin{tabular}{|c||c|c||c|}
        \hline
                    & \attribApos & \attribAneg & Summe \\
        \hline
        \hline
        \attribBpos & 13          & 7           & 20    \\
        \hline
        \attribBneg & 3           & 7           & 10    \\
        \hline
        \hline
        Summe       & 16          & 14          & 30    \\
        \hline
      \end{tabular}%
    \end{center}
  \fi
\end{exercise}

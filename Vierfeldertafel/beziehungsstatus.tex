\begin{exercise}
      {ID-113b26f6913fe7ef7f9df2e4250082027e2ebe8f}
      {Beziehungsstatus}
  \ifproblem\problem\par
    % <PROBLEM>
    \num{324} deiner \num{600} Facebook-Freunde haben
    ihren Beziehungsstatus nicht angegeben. Da du
    natürlich alle persönlich kennst, weißt du, dass
    insgesamt \num{360} deiner Face\-book-Freunde in
    einer Beziehung sind. \num{132} von denen, die
    ihren Beziehungsstatus angegeben haben, sind
    single.
    \begin{enumerate}[a)]
      \item Erstelle mit den Informationen aus dem
            Text eine Vierfeldertafel, und ergänze
            alle fehlenden Werte.
      \item Erstelle mit den Informationen aus der
            Vierfeldertafel ein vollständiges
            Baumdiagramm auf dessen erster Stufe
            der tatsächliche Beziehungsstatus
            unterschieden wird.
      \item Wie wahrscheinlich ist es, dass jemand
            in einer Beziehung ist und dies auch bei
            Facebook angibt?
      \item Wie groß ist die Wahrscheinlichkeit, dass
            jemand, der Single ist, dies auch bei
            Facebook angibt?
    \end{enumerate}
    % </PROBLEM>
  \fi
  %\ifoutline\outline\par
    % <OUTLINE>
    % </OUTLINE>
  %\fi
  \ifoutcome\outcome
    % <OUTCOME>
    \begin{enumerate}[a)]
      \item Mit den absoluten Werten aus der Aufgabenstellung
            lässt sich folgende Vierfeldertafel aufstellen:\par
            \begin{minipage}[c]{0.41\linewidth}
              \begin{fofotab}%[t]
                % Bezeichnungen
                \lblA{$A$}
                \lbla{$\overline{A}$}
                \lblB{$B$}
                \lblb{$\overline{B}$}
                % Mitte
                \andAB{\num{144}}
                \andAb{\num{132}}
                \andaB{\num{216}}
                \andab{\num{108}}
                % Rand
                \sumA {\num{276}}
                \suma {\num{324}}
                \sumB {\num{360}}
                \sumb {\num{240}}
                \total{\num{600}}
              \end{fofotab}
            \end{minipage}%
            \begin{minipage}[c]{0.58\linewidth}
              \begin{itemize}
                \renewcommand{\itemsep}{-1ex}%
                \item[$A$:]\glqq Status angegeben\grqq
                \item[$\overline{A}$:]\glqq Status nicht angegeben\grqq
                \item[$B$:]\glqq in einer Beziehung\grqq
                \item[$\overline{B}$:]\glqq nicht in einer Beziehung\grqq
              \end{itemize}
            \end{minipage}
      \item Um das Baumdiagramm erstellen zu können,
            werden zunächst Wahrscheinlichkeiten,
            also relative Häufigkeiten benötigt.
            Diese erhält man, indem man alle Werte
            der Vierfeldertafel durch die
            Gesamtanzahl \num{600} teilt:
            \begin{center}
              \begin{fofotab}%[t]
                % Bezeichnungen
                \lblA{$A$}
                \lbla{$\overline{A}$}
                \lblB{$B$}
                \lblb{$\overline{B}$}
                % Mitte
                \andAB{\num{0.24}}
                \andAb{\num{0.22}}
                \andaB{\num{0.36}}
                \andab{\num{0.18}}
                % Rand
                \sumA {\num{0.46}}
                \suma {\num{0.54}}
                \sumB {\num{0.6}}
                \sumb {\num{0.4}}
                \total{\num{1}}
              \end{fofotab}
            \end{center}
            Die bedingten Wahrscheinlichkeiten für
            die zweite Stufe des Baumdiagramms
            müssen eigens berechnet werden:
            \begin{align*}
              P_{B}(A)&=\frac{P(A\cap B)}{P(B)}
              =\frac{\num{0.24}}{\num{0.6}}
              =\num{0.4}
              &
              P_{\overline{B}}(A)&=\frac{P(A\cap\overline{B})}{P(\overline{B})}
              =\frac{\num{0.22}}{\num{0.4}}
              =\num{0.55}
              \\[1ex]
              P_{B}(\overline{A})&=\frac{P(\overline{A}\cap B)}{P(B)}
              =\frac{\num{0.36}}{\num{0.6}}
              =\num{0.6}
              &
              P_{\overline{B}}(\overline{A})&=\frac{P(\overline{A}\cap\overline{B})}{P(\overline{B})}
              =\frac{\num{0.18}}{\num{0.4}}
              =\num{0.45}
            \end{align*}\par
            Mit diesen Informationen kann man nun
            ein vollständiges Baumdiagramm zeichnen:
            \begin{center}
              %<OCTAVE>
              \begin{tikzpicture}[line width=0.6pt]
                % tree
                \begin{scope}
                  % size settings
                  \newcommand{\radius}{4mm}%
                  \newcommand{\xscale}{5}%
                  \newcommand{\yscale}{4}%
                  % background color of nodes
                  \newcommand{\colora}{white}%
                  \newcommand{\colorb}{white}%
                  % node text
                  \newcommand{\ntexta}{$B$}%
                  \newcommand{\ntextb}{$\overline{B}$}%
                  \newcommand{\ntextaa}{$A$}%
                  \newcommand{\ntextab}{$\overline{A}$}%
                  \newcommand{\ntextba}{$A$}%
                  \newcommand{\ntextbb}{$\overline{A}$}%
                  % edge text
                  \newcommand{\etexta}{\num{0.6}}%
                  \newcommand{\etextb}{\num{0.4}}%
                  \newcommand{\etextaa}{\num{0.4}}%
                  \newcommand{\etextab}{\num{0.6}}%
                  \newcommand{\etextba}{\num{0.55}}%
                  \newcommand{\etextbb}{\num{0.45}}%
                  % geometry
                  \coordinate (Z)  at ( 1.500*\xscale*\radius,  2.000*\yscale*\radius);
                  \coordinate (A)  at ( 0.500*\xscale*\radius,  1.000*\yscale*\radius);
                  \coordinate (B)  at ( 2.500*\xscale*\radius,  1.000*\yscale*\radius);
                  \coordinate (AA) at ( 0.000*\xscale*\radius,  0.000*\yscale*\radius);
                  \coordinate (AB) at ( 1.000*\xscale*\radius,  0.000*\yscale*\radius);
                  \coordinate (BA) at ( 2.000*\xscale*\radius,  0.000*\yscale*\radius);
                  \coordinate (BB) at ( 3.000*\xscale*\radius,  0.000*\yscale*\radius);
                  % edges
                  \draw (Z) -- (A);
                  \draw (Z) -- (B);
                  \draw (A) -- (AA);
                  \draw (A) -- (AB);
                  \draw (B) -- (BA);
                  \draw (B) -- (BB);
                  % root
                  \fill[fill=black] (Z) circle[radius=2pt];
                  % nodes
                  \filldraw[fill=\colora, draw=black] (A)  circle[radius=\radius] node{\ntexta};
                  \filldraw[fill=\colorb, draw=black] (B)  circle[radius=\radius] node{\ntextb};
                  \filldraw[fill=\colora, draw=black] (AA) circle[radius=\radius] node{\ntextaa};
                  \filldraw[fill=\colorb, draw=black] (AB) circle[radius=\radius] node{\ntextab};
                  \filldraw[fill=\colora, draw=black] (BA) circle[radius=\radius] node{\ntextba};
                  \filldraw[fill=\colorb, draw=black] (BB) circle[radius=\radius] node{\ntextbb};
                  % label macros
                  \newcommand{\rlabel}[4]%
                  {%
                    \coordinate (TEMP) at ($(#1)!0.5!(#2)$);
                    \coordinate (TEMP) at ($(TEMP)!#3!270:(#2)$);
                    \node at (TEMP) {#4};
                  }%
                  \newcommand{\llabel}[4]{\rlabel{#2}{#1}{#3}{#4}};
                  % edge labels
                  \rlabel{Z}{A}{3mm}{\etexta};
                  \llabel{Z}{B}{3mm}{\etextb};
                  \rlabel{A}{AA}{3mm}{\etextaa~~};
                  \llabel{A}{AB}{3mm}{~\etextab};
                  \rlabel{B}{BA}{3mm}{\etextba~~~};
                  \llabel{B}{BB}{3mm}{~~\etextbb};
                \end{scope}
              \end{tikzpicture}
            \end{center}
      \item Sowohl der Beziehungsstatus, als auch
            die Angabe bei Facebook sind in der
            Fragestellung vom Zufall abhängig,
            also ist folgende Wahrscheinlichkeit
            gesucht:
            \begin{equation*}
              P(A\cap B)=\num{0.24}
            \end{equation*}
      \item Da in der Fragestellung der Beziehungsstatus
            vorgegeben wird, ist nur noch die Angabe
            bei Facebook vom Zufall abhängig. Also
            ist folgende bedingte Wahrscheinlichkeit
            gesucht:
            \begin{equation*}
              P_{\overline{B}}(A)=\num{0.55}
            \end{equation*}
    \end{enumerate}
    % </OUTCOME>
  \fi
\end{exercise}

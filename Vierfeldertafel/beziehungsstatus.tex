\begin{exercise}
      {ID-2b08229d39b094fb81c38059c4d9f77e20dc37ed}
      {Beziehungsstatus}
  \ifproblem\problem\par
    \num{196} deiner \num{440} Facebook-Freunde haben ihren Beziehungsstatus nicht angegeben.
    Da du natürlich alle persönlich kennst, weißt du, dass insgesamt \num{288} deiner
    Face\-book-Freunde in einer Beziehung sind. \num{116} von denen, die ihren
    Beziehungsstatus angegeben haben, sind single.
    \begin{enumerate}[a)]
      \item Erstelle mit den Informationen aus dem Text eine Vierfeldertafel,
            und ergänze alle fehlenden Werte.
      \item Erstelle mit den Informationen aus der Vierfeldertafel ein
            vollständiges Baumdiagramm auf dessen erster Stufe der
            tatsächliche Beziehungsstatus unterschieden wird.
      \item Wie wahrscheinlich ist es, dass jemand in einer Beziehung
            ist und dies auch bei Facebook angibt?
      \item Wie groß ist die Wahrscheinlichkeit, dass jemand, der Single ist,
            dies auch bei Facebook angibt?
    \end{enumerate}
  \fi
  \ifoutline\outline\par
    \begin{enumerate}[a)]
      \item Eine Vierfeldertafel mit den Angaben aus dem Text
            könnte z.\,B. so aussehen:
            \begin{center}
              \begin{minipage}{0.36\linewidth}
                \begin{itemize}
                  \setlength{\leftskip}{0pt}%
                  \setlength{\itemsep}{-0.1\baselineskip}%
                  \item[$A$:]            Status angegeben
                  \item[$\overline{A}$:] Status nicht angegeben
                  \item[$B$:]            ist in einer Beziehung
                  \item[$\overline{B}$:] ist nicht in einer Beziehung
                \end{itemize}
              \end{minipage}
              \qquad
              \begin{fourfoldtable}
                \Apos{$A$}%
                \Aneg{$\overline{A}$}%
                \Bpos{$B$}%
                \Bneg{$\overline{B}$}%
                \numbers{}{}{288}{116}{}{}{}{196}{440}%
              \end{fourfoldtable}
            \end{center}
      \item Berechne die bedingten Wahrscheinlichkeiten, indem du die
            jeweiligen Pfadregeln umformst!
      \item Gesucht ist: $P(A\cap B)$
      \item Gesucht ist: $P_{\overline{B}}(A)$
    \end{enumerate}
  \fi
  \ifoutcome\outcome\par
    \begin{enumerate}[a)]
      \item Eine vollständige Vierfeldertafel könnte z.\,B. so aussehen:
            \begin{center}
              \begin{minipage}{0.36\linewidth}
                \begin{itemize}
                  \setlength{\leftskip}{0pt}%
                  \setlength{\itemsep}{-0.1\baselineskip}%
                  \item[$A$:]            Status angegeben
                  \item[$\overline{A}$:] Status nicht angegeben
                  \item[$B$:]            ist in einer Beziehung
                  \item[$\overline{B}$:] ist nicht in einer Beziehung
                \end{itemize}
              \end{minipage}
              \qquad
              \begin{fourfoldtable}
                \Apos{$A$}%
                \Aneg{$\overline{A}$}%
                \Bpos{$B$}%
                \Bneg{$\overline{B}$}%
                \numbers{128}{160}{288}{116}{36}{152}{244}{196}{440}%
              \end{fourfoldtable}
            \end{center}
             Wenn man die relativen Häufigkeiten berechnet, erhält man
             folgende Vierfeldertafel:
             \begin{center}
               \begin{fourfoldtable}
                 \Apos{$A$}%
                 \Aneg{$\overline{A}$}%
                 \Bpos{$B$}%
                 \Bneg{$\overline{B}$}%
                 \numbers{128}{160}{288}{116}{36}{152}{244}{196}{440}%
                 \numbers{0.290909}{0.363636}{0.654545}{0.263636}{0.081818}{0.345455}{0.554545}{0.445455}{1.000000}%
               \end{fourfoldtable}
             \end{center}
      \item Als Baumdiagramm lässt sich der Sachverhalt wie folgt
            darstellen:
            \begin{center}
              \begin{tikzpicture}
                % labels
                \newcommand{\lvlApos}  {\ensuremath{B}}%
                \newcommand{\lvlAneg}  {\ensuremath{\overline{B}}}%
                \newcommand{\lvlBpos}  {\ensuremath{A}}%
                \newcommand{\lvlBneg}  {\ensuremath{\overline{A}}}%
                \newcommand{\pApos}    {\num{0.655}}%
                \newcommand{\pAneg}    {\num{0.345}}%
                \newcommand{\pAposBpos}{\num{0.444}}%
                \newcommand{\pAposBneg}{\num{0.556}}%
                \newcommand{\pAnegBpos}{\num{0.763}}%
                \newcommand{\pAnegBneg}{\num{0.237}}%
                \newcommand{\PAposBpos}{$P(\lvlApos\cap\lvlBpos)\approx\num{0.291}$}%
                \newcommand{\PAposBneg}{$P(\lvlApos\cap\lvlBneg)\approx\num{0.364}$}%
                \newcommand{\PAnegBpos}{$P(\lvlAneg\cap\lvlBpos)\approx\num{0.264}$}%
                \newcommand{\PAnegBneg}{$P(\lvlAneg\cap\lvlBneg)\approx\num{0.082}$}%
                % spacing
                \newcommand{\radius}{5mm}%
                \newcommand{\vstrut}{\vphantom{\ensuremath{\Big(}}}%
                % node positions
                \coordinate (A) at (0,  0);
                \coordinate (B) at (3*\radius,  2.50*\radius);
                \coordinate (C) at (3*\radius, -2.50*\radius);
                \coordinate (D) at (8*\radius,  3.75*\radius);
                \coordinate (E) at (8*\radius,  1.25*\radius);
                \coordinate (F) at (8*\radius, -1.25*\radius);
                \coordinate (G) at (8*\radius, -3.75*\radius);
                % edges
                \draw (A) -- node[above left=1pt]{\pApos}     (B);
                \draw (A) -- node[below left=1pt]{\pAneg}     (C);
                \draw (B) -- node[above=1pt]     {\pAposBpos} (D);
                \draw (B) -- node[below=2pt]     {\pAposBneg} (E);
                \draw (C) -- node[above=1pt]     {\pAnegBpos} (F);
                \draw (C) -- node[below=2pt]     {\pAnegBneg} (G);
                % nodes
                \fill[fill=black]                 (A) circle[radius=2pt];
                \filldraw[fill=white, draw=black] (B) circle[radius=\radius];
                \filldraw[fill=white, draw=black] (C) circle[radius=\radius];
                \filldraw[fill=white, draw=black] (D) circle[radius=\radius];
                \filldraw[fill=white, draw=black] (E) circle[radius=\radius];
                \filldraw[fill=white, draw=black] (F) circle[radius=\radius];
                \filldraw[fill=white, draw=black] (G) circle[radius=\radius];
                % node names
                \node at (B) {\vstrut\lvlApos};
                \node at (C) {\vstrut\lvlAneg};
                \node at (D) {\vstrut\lvlBpos};
                \node at (E) {\vstrut\lvlBneg};
                \node at (F) {\vstrut\lvlBpos};
                \node at (G) {\vstrut\lvlBneg};
                % path probabilities
                \node[right=2*\radius] at (D) {\vstrut\PAposBpos};
                \node[right=2*\radius] at (E) {\vstrut\PAposBneg};
                \node[right=2*\radius] at (F) {\vstrut\PAnegBpos};
                \node[right=2*\radius] at (G) {\vstrut\PAnegBneg};
              \end{tikzpicture}
            \end{center}
      \item Die Wahrscheinlichkeit dafür, dass jemand in einer Beziehung
            ist und dies auch bei Facebook angibt, liegt bei etwa
            \pc{29.1}.
      \item Die Wahrscheinlichkeit dafür, dass jemand, der Single ist,
            dies auch bei Facebook angibt, liegt bei etwa \pc{76.3}.
    \end{enumerate}
  \fi
\end{exercise}

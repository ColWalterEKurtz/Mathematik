\begin{exercise}
      {ID-4f884d9889e14e537d66d20b1700ab4b0f30ddb5}
      {Rot}
  \ifproblem\problem\par
    In drei Urnen befinden sich je zwanzig Kugeln; in der ersten 4 rote und
    16 weiße, in der zweiten 10 rote und 10 weiße und in der dritten nur rote.
    Nun wird eine Urne zufällig ausgewählt und Kugeln mit Zurücklegen gezogen.
    \begin{enumerate}
      \item Es wird eine rote Kugel gezogen. Wie groß ist die Wahrscheinlichkeit,
            dass die erste (zweite, dritte) Urne gewählt wurde?
      \item Es werden nacheinander zwei rote Kugeln gezogen. Wie groß ist die
            Wahrscheinlichkeit, dass die erste (zweite, dritte) Urne gewählt wurde?
      \item Es werden nacheinander drei rote Kugeln gezogen. Wie groß ist die
            Wahrscheinlichkeit, dass die erste (zweite, dritte) Urne gewählt wurde?
      \item Es werden nacheinander drei rote Kugeln und dann eine weiße gezogen.
            Wie groß ist die Wahrscheinlichkeit, dass die erste (zweite, dritte)
            Urne gewählt wurde?
    \end{enumerate}
  \fi
  %\ifoutline\outline\par
  %\fi
  %\ifoutcome\outcome\par
  %\fi
\end{exercise}

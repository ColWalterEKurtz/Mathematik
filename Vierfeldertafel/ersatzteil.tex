\begin{exercise}
      {ID-0c09acfe800d84e4fe27bb9e56e5ef0e3d18936b}
      {Ersatzteil}
  \ifproblem\problem\par
    % <PROBLEM>
    Ein Ersatzteil wird auf zwei Maschinen gefertigt:
    Maschine A liefert pro Tag \num{2000}, Maschine B
    \num{5000} Stück. Bei Maschine A gibt es
    erfahrungsgemäß \pc{3.5}, bei Maschine B \pc{1.5}
    Ausschuss. Mit welcher Wahrscheinlichkeit stammt
    ein defektes Teil von Maschine A?
    % </PROBLEM>
  \fi
  \ifoutline\outline\par
    % <OUTLINE>
    Zunächst erstellt man eine Vierfeldertafel mit
    absoluten Häufigkeiten. Dazu müssen manche
    Angaben aus der Aufgabenstellung erst
    umgerechnet werden:\par
    \begin{minipage}{0.46\linewidth}
      \begin{fofotab}%[t]
        % Bezeichnungen
        \lblA{$A$}
        \lbla{$B$}
        \lblB{$d$}
        \lblb{$\overline{d}$}
        % Mitte
        \andAB{$\num{0.035}\cdot\num{2000}$}
        \andAb{}
        \andaB{$\num{0.015}\cdot\num{5000}$}
        \andab{}
        % Rand
        \sumA {\num{2000}}
        \suma {\num{5000}}
        \sumB {}
        \sumb {}
        \total{}
      \end{fofotab}
    \end{minipage}%
    \begin{minipage}{0.53\linewidth}
      \begin{itemize}
        \renewcommand{\itemsep}{-1ex}%
        \item[$A$:] \glqq das Ersatzteil stammt von Maschine A\grqq
        \item[$B$:] \glqq das Ersatzteil stammt von Maschine B\grqq
        \item[$d$:] \glqq das Ersatzteil ist defekt\grqq
        \item[$\overline{d}$:] \glqq das Ersatzteil ist nicht defekt\grqq
      \end{itemize}
    \end{minipage}\par
    % </OUTLINE>
  \fi
  \ifoutcome\outcome\par
    % <OUTCOME>
    Aus den Angaben der Aufgabenstellung lassen sich
    folgende absoluten Werte ableiten:\par
    \begin{minipage}{0.4\linewidth}
      \begin{fofotab}%[t]
        % Bezeichnungen
        \lblA{$A$}
        \lbla{$B$}
        \lblB{$d$}
        \lblb{$\overline{d}$}
        % Mitte
        \andAB{\num{70}}
        %0.035 * 2000
        \andAb{\num{1930}}
        \andaB{\num{75}}
        %0.015 * 5000
        \andab{\num{4925}}
        % Rand
        \sumA {\num{2000}}
        \suma {\num{5000}}
        \sumB {\num{145}}
        \sumb {\num{6855}}
        %7000 - 145
        \total{\num{7000}}
      \end{fofotab}
    \end{minipage}%
    \begin{minipage}{0.53\linewidth}
      \begin{itemize}
        \renewcommand{\itemsep}{-1ex}%
        \item[$A$:] \glqq das Ersatzteil stammt von Maschine A\grqq
        \item[$B$:] \glqq das Ersatzteil stammt von Maschine B\grqq
        \item[$d$:] \glqq das Ersatzteil ist defekt\grqq
        \item[$\overline{d}$:] \glqq das Ersatzteil ist nicht defekt\grqq
      \end{itemize}
    \end{minipage}\par
    Gesucht ist die (bedingte) Wahrscheinlichkeit
    mit der ein defektes Ersatzteil von Maschine
    A stammt, also:
    \begin{equation*}
      P_{d}(A)=\frac{P(A\cap d)}{P(d)}
      =\frac{\;\frac{70}{\num{7000}}\;}{\;\frac{145}{\num{7000}}\;}
      =\frac{70}{145}
      =\frac{14}{29}
      \approx\num{0.483}
    \end{equation*}
    Ein defektes Ersatzteil wurde also mit
    einer Wahrscheinlichkeit von etwa \pc{48.3}
    von Maschine A hergestellt.
    % </OUTCOME>
  \fi
\end{exercise}

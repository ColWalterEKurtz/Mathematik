\begin{exercise}
      {ID-4f571895f0bdd4bfff0c475224a7e8c573d8a892}
      {Großversuch}
  \ifproblem\problem
    In einem Großversuch wurde an \num{11101} erkrankten Personen ein Medikament
    getestet. Die Ergebnisse sind in einer Tabelle festgehalten. Dabei bedeuten:\par
    \begin{minipage}{0.5\linewidth}
      \begin{itemize}
        \setlength{\itemsep}{-0.1\baselineskip}
        \item[$G$:]            gesund geworden
        \item[$\overline{G}$:] nicht gesund geworden
        \item[$M$:]            Medikament genommen
        \item[$\overline{M}$:] Placebo genommen
      \end{itemize}
    \end{minipage}%
    \begin{minipage}{0.45\linewidth}
      \centering
      \renewcommand{\arraystretch}{1.25}
      \begin{tabular}{|c|c|c|c|}
        \hline
                       & $G$        & $\overline{G}$ & Summe       \\
        \hline
        $M$            & \num{6312} & \num{87}       & \num{6399}  \\
        \hline
        $\overline{M}$ & \num{312}  & \num{4390}     & \num{4702}  \\
        \hline
        Summe          & \num{6624} & \num{4477}     & \num{11101} \\
        \hline
      \end{tabular}
    \end{minipage}\par
    \begin{enumerate}
      \item Stelle die relativen Häufigkeiten in einer Vierfeldertafel dar
            und zeichne das dazugehörige Baumdiagramm.
      \item Wie groß ist die Wahrscheinlichkeit bei einer Person, von der
            man weiß, dass sie das Medikament eingenommen hat, zu gesunden?
      \item Wie groß ist die Wahrscheinlichkeit bei einer Person, von der
            man weiß, dass sie das Placebo eingenommen hat, nicht zu gesunden?
    \end{enumerate}
  \fi
  %\ifoutline\outline
  %\fi
  %\ifoutcome\outcome
  %\fi
\end{exercise}

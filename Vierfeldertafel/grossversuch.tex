\begin{exercise}
      {ID-4f571895f0bdd4bfff0c475224a7e8c573d8a892}
      {Großversuch}
  \ifproblem\problem\par
    In einem Großversuch wurde an \num{11101} erkrankten Personen ein
    Medikament getestet. Die Ergebnisse sind in einer Tabelle festgehalten.
    Dabei bedeuten:
    \begin{center}
      \begin{minipage}{0.45\linewidth}
        \begin{itemize}
          \setlength{\itemsep}{-0.1\baselineskip}
          \item[$M$:]            Medikament genommen
          \item[$P$:]            Placebo genommen
          \item[$G$:]            gesund geworden
          \item[$\overline{G}$:] nicht gesund geworden
        \end{itemize}%
      \end{minipage}%
      \begin{minipage}{0.45\linewidth}
        \centering
        \begin{fourfoldtable}
          \Apos{$M$}%
          \Aneg{$P$}%
          \Bpos{$G$}%
          \Bneg{$\overline{G}$}%
          \numbers{6312}{312}{6624}{87}{4390}{4477}{6399}{4702}{11101}%
        \end{fourfoldtable}
      \end{minipage}%
    \end{center}%
    \begin{enumerate}[a)]
      \item Stelle die relativen Häufigkeiten in einer Vierfeldertafel dar
            und zeichne das dazugehörige Baumdiagramm.
      \item Wie groß ist die Wahrscheinlichkeit bei einer Person, von der
            man weiß, dass sie das Medikament eingenommen hat, zu gesunden?
      \item Wie groß ist die Wahrscheinlichkeit bei einer Person, von der
            man weiß, dass sie das Placebo eingenommen hat, nicht zu gesunden?
    \end{enumerate}
  \fi
  %\ifoutline\outline\par
  %\fi
  \ifoutcome\outcome\par
    \begin{enumerate}[a)]
      \item Wenn man die relativen Häufigkeiten auf vier Nachkommastellen
            rundet, könnte die Vierfeldertafel z.\,B. so aussehen:
            \begin{center}
              \begin{fourfoldtable}
                \Apos{$M$}%
                \Aneg{$P$}%
                \Bpos{$G$}%
                \Bneg{$\overline{G}$}%
                \numbers{0.5686}{0.0281}{0.5967}{0.0078}{0.3955}{0.4033}{0.5764}{0.4236}{1.0000}%
              \end{fourfoldtable}
            \end{center}
            In Form eines Baumdiagramms könnte man den Sachverhalt
            wie folgt darstellen:
            \begin{center}
              \begin{tikzpicture}
                % labels
                \newcommand{\lvlApos}  {$M$}%
                \newcommand{\lvlAneg}  {$P$}%
                \newcommand{\lvlBpos}  {$G$}%
                \newcommand{\lvlBneg}  {$\overline{G}$}%
                \newcommand{\pApos}    {\num{0.576}}%
                \newcommand{\pAneg}    {\num{0.424}}%
                \newcommand{\pAposBpos}{\num{0.986}}%
                \newcommand{\pAposBneg}{\num{0.014}}%
                \newcommand{\pAnegBpos}{\num{0.066}}%
                \newcommand{\pAnegBneg}{\num{0.934}}%
                \newcommand{\PAposBpos}{$P(M\cap G)\approx\num{0.5686}$}%
                \newcommand{\PAposBneg}{$P(M\cap\overline{G})\approx\num{0.0078}$}%
                \newcommand{\PAnegBpos}{$P(P\cap G)\approx\num{0.0281}$}%
                \newcommand{\PAnegBneg}{$P(P\cap\overline{G})\approx\num{0.3955}$}%
                % spacing
                \newcommand{\radius}{5.5mm}%
                \newcommand{\vstrut}{\vphantom{\ensuremath{\Big(}}}%
                % node positions
                \coordinate (A) at (0,  0);
                \coordinate (B) at (3*\radius,  2.50*\radius);
                \coordinate (C) at (3*\radius, -2.50*\radius);
                \coordinate (D) at (8*\radius,  3.75*\radius);
                \coordinate (E) at (8*\radius,  1.25*\radius);
                \coordinate (F) at (8*\radius, -1.25*\radius);
                \coordinate (G) at (8*\radius, -3.75*\radius);
                % edges
                \draw (A) -- node[above left=1pt]{\pApos}     (B);
                \draw (A) -- node[below left=1pt]{\pAneg}     (C);
                \draw (B) -- node[above=1pt]     {\pAposBpos} (D);
                \draw (B) -- node[below=1pt]     {\pAposBneg} (E);
                \draw (C) -- node[above=1pt]     {\pAnegBpos} (F);
                \draw (C) -- node[below=1pt]     {\pAnegBneg} (G);
                % nodes
                \fill[fill=black]                 (A) circle[radius=2pt];
                \filldraw[fill=white, draw=black] (B) circle[radius=\radius];
                \filldraw[fill=white, draw=black] (C) circle[radius=\radius];
                \filldraw[fill=white, draw=black] (D) circle[radius=\radius];
                \filldraw[fill=white, draw=black] (E) circle[radius=\radius];
                \filldraw[fill=white, draw=black] (F) circle[radius=\radius];
                \filldraw[fill=white, draw=black] (G) circle[radius=\radius];
                % node names
                \node at (B) {\vstrut\lvlApos};
                \node at (C) {\vstrut\lvlAneg};
                \node at (D) {\vstrut\lvlBpos};
                \node at (E) {\vstrut\lvlBneg};
                \node at (F) {\vstrut\lvlBpos};
                \node at (G) {\vstrut\lvlBneg};
                % path probabilities
                \node[right=2*\radius] at (D) {\vstrut\PAposBpos};
                \node[right=2*\radius] at (E) {\vstrut\PAposBneg};
                \node[right=2*\radius] at (F) {\vstrut\PAnegBpos};
                \node[right=2*\radius] at (G) {\vstrut\PAnegBneg};
              \end{tikzpicture}
            \end{center}
      \item Die Wahrscheinlichkeit zu gesunden liegt bei einer Person,
            von der man weiß, dass sie das Medikament eingenommen hat,
            bei etwa \pc{98.6}.
      \item Die Wahrscheinlichkeit nicht zu gesunden liegt bei einer Person,
            von der man weiß, dass sie das Placebo eingenommen hat,
            bei etwa \pc{93.4}.
    \end{enumerate}
  \fi
\end{exercise}

\begin{exercise}
      {ID-ac8923a85b40ed61447a49c313d32ec44d42e646}
      {In der Weihnachtsbäckerei}
  \ifproblem\problem\par
    % <PROBLEM>
    Oma hat für ihre Familie insgesamt 80 Plätzchen
    gebacken und in kleine Tütchen verpackt.
    Insgesamt haben 48 Plätzchen einen Überzug aus
    Schokolade, 20 haben eine Füllung mit Erdbeermarmelade.
    Unter diesen 48 bzw. 20 Plätzchen gibt es 12, die
    sogar beides haben: Schokoladenüberzug und
    Marmeladenfüllung.
    \begin{enumerate}[a)]
      \item Erstelle eine Vierfeldertafel zu der
            genannten Situation.
      \item Stelle die Situation in zwei
            Baumdiagrammen dar und gib die
            Wahrscheinlichkeiten aller Ergebnisse an.
      \item Aus einer Tüte wird ein Plätzchen mit
            Schokolade gezogen. Berechne die
            Wahrscheinlichkeit dafür, dass es auch
            zusätzlich mit Marmelade gefüllt ist.
    \end{enumerate}
    % </PROBLEM>
  \fi
  %\ifoutline\outline\par
    % <OUTLINE>
    % </OUTLINE>
  %\fi
  \ifoutcome\outcome%\par
    % <OUTCOME>
    \begin{enumerate}[a)]
      \item Die fettgedruckten Werte in der
            Vierfeldertafel wurden aus der
            Aufgabenstellung übernommen, die
            normalgedruckten Werte sind
            Ergebnisse entsprechender
            Berechnungen.\par
            \begin{minipage}{0.40\linewidth}
              \begin{fofotab}
                \lblA{$S$}
                \lbla{$\overline{S}$}
                \lblB{$M$}
                \lblb{$\overline{M}$}
                % Vorgaben aus der Aufgabenstellung
                \andAB{\textbf{12}}
                \sumA {\textbf{48}}
                \sumB {\textbf{20}}
                \total{\textbf{80}}
                % Berechnungen
                \andAb{36}
                \andaB{8}
                \andab{24}
                \suma{32}
                \sumb{60}
              \end{fofotab}
            \end{minipage}%
            \begin{minipage}{0.38\linewidth}
              \begin{itemize}
                \renewcommand{\itemsep}{-1ex}%
                \item[$S$:] \glqq mit Schokoladenüberzug\grqq
                \item[$\overline{S}$:] \glqq ohne Schokoladenüberzug\grqq
                \item[$M$:] \glqq mit Marmeladenfüllung\grqq
                \item[$\overline{M}$:] \glqq ohne Marmeladenfüllung\grqq
              \end{itemize}%
            \end{minipage}%
      \item Im nächsten Schritt werden die absoluten
            Häufigkeiten der Vierfeldertafel in relative
            Häufigkeiten umgerechnet, indem man jeden
            Wert durch die Gesamtanzahl (80) teilt. Die
            relativen Häufigkeiten sind dann als
            Wahrscheinlichkeiten interpretierbar.
            \begin{center}
              \begin{fofotab}%[t]
                % Bezeichnungen
                \lblA{$S$}
                \lbla{$\overline{S}$}
                \lblB{$M$}
                \lblb{$\overline{M}$}
                % Berechnungen
                %<OCTAVE>
                \andAB{\num{0.15}}
                \andAb{\num{0.45}}
                \andaB{\num{0.1}}
                \andab{\num{0.3}}
                \sumA {\num{0.6}}
                \suma {\num{0.4}}
                \sumB {\num{0.25}}
                \sumb {\num{0.75}}
                \total{\num{1}}
                %</OCTAVE>
                %printf("\\andAB{%s}\n", myn2s(12/80, 3, 0, 0, 0, 1));
                %printf("\\andAb{%s}\n", myn2s(36/80, 3, 0, 0, 0, 1));
                %printf("\\andaB{%s}\n", myn2s( 8/80, 3, 0, 0, 0, 1));
                %printf("\\andab{%s}\n", myn2s(24/80, 3, 0, 0, 0, 1));
                %printf("\\sumA {%s}\n", myn2s(48/80, 3, 0, 0, 0, 1));
                %printf("\\suma {%s}\n", myn2s(32/80, 3, 0, 0, 0, 1));
                %printf("\\sumB {%s}\n", myn2s(20/80, 3, 0, 0, 0, 1));
                %printf("\\sumb {%s}\n", myn2s(60/80, 3, 0, 0, 0, 1));
                %printf("\\total{%s}\n", myn2s(80/80, 3, 0, 0, 0, 1));
              \end{fofotab}
            \end{center}
            Aus dieser Vierfeldertafel kann man bereits die
            Wahrscheinlichkeiten aller (vier) Ergebnisse
            ablesen:
            \begin{equation*}
              P(S\cap M)=\num{0.15}
              \qquad
              P(S\cap\overline{M})=\num{0.45}
              \qquad
              P(\overline{S}\cap M)=\num{0.1}
              \qquad
              P(\overline{S}\cap\overline{M})=\num{0.3}
            \end{equation*}
            Aus dieser Vierfeldertafel lassen sich
            zwei verschiedene Baumdiagramme ableiten:
            \begin{center}
              \begin{tabular}{llcll}
                Stufe 1:
                & Schokoladenüberzug
                & \qquad
                & Stufe 1:
                & Marmeladenfüllung
                \\
                Stufe 2:
                & Marmeladenfüllung
                & \qquad
                & Stufe 2: &
                Schokoladenüberzug
              \end{tabular}
            \end{center}
            Um die beiden Baumdiagramme vollständig
            zeichnen zu können, müssen allerdings
            erst die bedingten Wahrscheinlichkeiten
            für die jeweilige 2. Stufe berechnet
            werden.
            \par
            Wenn man den Schokoladenüberzug für die erste
            Stufe wählt, erhält man:
            \begin{align*}
              P_{S}(M)&=
              \frac{P(S\cap M)}{P(S)}=
              \frac{\num{0.15}}{\num{0.6}}=
              \num{0.25}
              &
              P_{S}(\overline{M})&=
              \frac{P(S\cap\overline{M})}{P(S)}=
              \frac{\num{0.45}}{\num{0.6}}=
              \num{0.75}
              \\
              P_{\overline{S}}(M)&=
              \frac{P(\overline{S}\cap M)}{P(\overline{S})}=
              \frac{\num{0.1}}{\num{0.4}}=
              \num{0.25}
              &
              P_{\overline{S}}(\overline{M})&=
              \frac{P(\overline{S}\cap\overline{M})}{P(\overline{S})}=
              \frac{\num{0.3}}{\num{0.4}}=
              \num{0.75}
            \end{align*}
            Wenn man die Marmeladenfüllung für die erste
            Stufe wählt, erhält man:
            \begin{align*}
              P_{M}(S)&=
              \frac{P(S\cap M)}{P(M)}=
              \frac{\num{0.15}}{\num{0.25}}=
              \num{0.6}
              &
              P_{M}(\overline{S})&=
              \frac{P(\overline{S}\cap M)}{P(M)}=
              \frac{\num{0.1}}{\num{0.25}}=
              \num{0.4}
              \\
              P_{\overline{M}}(S)&=
              \frac{P(S\cap\overline{M})}{P(\overline{M})}=
              \frac{\num{0.45}}{\num{0.75}}=
              \num{0.6}
              &
              P_{\overline{M}}(\overline{S})&=
              \frac{P(\overline{S}\cap\overline{M})}{P(\overline{M})}=
              \frac{\num{0.3}}{\num{0.75}}=
              \num{0.4}
            \end{align*}
            Damit erhält man die beiden Bäume:\bigskip\par
            \hspace*{\fill}%
            \begin{tikzpicture}[line width=0.6pt]
              % tree
              \begin{scope}
                % some default colors
                \newcommand{\colr}{Red};%
                \newcommand{\colg}{ForestGreen};%
                \newcommand{\colb}{Cerulean};%
                \newcommand{\coly}{YellowOrange};%
                \newcommand{\cola}{Black!35!White};%
                \newcommand{\cole}{Black!55!White};%
                % size settings
                \newcommand{\radius}{4mm}%
                \newcommand{\xscale}{4}%
                \newcommand{\yscale}{4}%
                % background color of nodes
                \newcommand{\colora}{white}%
                \newcommand{\colorb}{white}%
                % default node text
                \newcommand{\ntextA}{$S$}%
                \newcommand{\ntexta}{$\overline{S}$}%
                \newcommand{\ntextB}{$M$}%
                \newcommand{\ntextb}{$\overline{M}$}%
                % default edge text
                \newcommand{\etextA} {\num{0.6}}%
                \newcommand{\etexta} {\num{0.4}}%
                \newcommand{\etextAB}{\num{0.25}~~~}%
                \newcommand{\etextAb}{~~\num{0.75}}%
                \newcommand{\etextaB}{\num{0.25}~~~}%
                \newcommand{\etextab}{~~\num{0.75}}%
                % geometry
                \coordinate (Z)  at ( 1.500*\xscale*\radius,  2.000*\yscale*\radius);
                \coordinate (A)  at ( 0.500*\xscale*\radius,  1.000*\yscale*\radius);
                \coordinate (B)  at ( 2.500*\xscale*\radius,  1.000*\yscale*\radius);
                \coordinate (AA) at ( 0.000*\xscale*\radius,  0.000*\yscale*\radius);
                \coordinate (AB) at ( 1.000*\xscale*\radius,  0.000*\yscale*\radius);
                \coordinate (BA) at ( 2.000*\xscale*\radius,  0.000*\yscale*\radius);
                \coordinate (BB) at ( 3.000*\xscale*\radius,  0.000*\yscale*\radius);
                % edges
                \draw (Z) -- (A);
                \draw (Z) -- (B);
                \draw (A) -- (AA);
                \draw (A) -- (AB);
                \draw (B) -- (BA);
                \draw (B) -- (BB);
                % root
                \fill[fill=black] (Z) circle[radius=2pt];
                % nodes
                \filldraw[fill=\colora, draw=black] (A)  circle[radius=\radius] node{\ntextA};
                \filldraw[fill=\colorb, draw=black] (B)  circle[radius=\radius] node{\ntexta};
                \filldraw[fill=\colora, draw=black] (AA) circle[radius=\radius] node{\ntextB};
                \filldraw[fill=\colorb, draw=black] (AB) circle[radius=\radius] node{\ntextb};
                \filldraw[fill=\colora, draw=black] (BA) circle[radius=\radius] node{\ntextB};
                \filldraw[fill=\colorb, draw=black] (BB) circle[radius=\radius] node{\ntextb};
                % label macros
                \newcommand{\rlabel}[4]%
                {%
                  \coordinate (TEMP) at ($(#1)!0.5!(#2)$);
                  \coordinate (TEMP) at ($(TEMP)!#3!270:(#2)$);
                  \node at (TEMP) {#4};
                }%
                \newcommand{\llabel}[4]{\rlabel{#2}{#1}{#3}{#4}};
                % edge labels
                \rlabel{Z}{A}{3mm}{\etextA};
                \llabel{Z}{B}{3mm}{\etexta};
                \rlabel{A}{AA}{3mm}{\etextAB};
                \llabel{A}{AB}{3mm}{\etextAb};
                \rlabel{B}{BA}{3mm}{\etextaB};
                \llabel{B}{BB}{3mm}{\etextab};
              \end{scope}
            \end{tikzpicture}
            \hspace*{\fill}%
            \begin{tikzpicture}[line width=0.6pt]
              % tree
              \begin{scope}
                % some default colors
                \newcommand{\colr}{Red};%
                \newcommand{\colg}{ForestGreen};%
                \newcommand{\colb}{Cerulean};%
                \newcommand{\coly}{YellowOrange};%
                \newcommand{\cola}{Black!35!White};%
                \newcommand{\cole}{Black!55!White};%
                % size settings
                \newcommand{\radius}{4mm}%
                \newcommand{\xscale}{4}%
                \newcommand{\yscale}{4}%
                % background color of nodes
                \newcommand{\colora}{white}%
                \newcommand{\colorb}{white}%
                % default node text
                \newcommand{\ntextA}{$M$}%
                \newcommand{\ntexta}{$\overline{M}$}%
                \newcommand{\ntextB}{$S$}%
                \newcommand{\ntextb}{$\overline{S}$}%
                % default edge text
                \newcommand{\etextA} {\num{0.25}~~~}%
                \newcommand{\etexta} {~~\num{0.75}}%
                \newcommand{\etextAB}{\num{0.6}~~}%
                \newcommand{\etextAb}{~\num{0.4}}%
                \newcommand{\etextaB}{\num{0.6}~~}%
                \newcommand{\etextab}{~\num{0.4}}%
                % geometry
                \coordinate (Z)  at ( 1.500*\xscale*\radius,  2.000*\yscale*\radius);
                \coordinate (A)  at ( 0.500*\xscale*\radius,  1.000*\yscale*\radius);
                \coordinate (B)  at ( 2.500*\xscale*\radius,  1.000*\yscale*\radius);
                \coordinate (AA) at ( 0.000*\xscale*\radius,  0.000*\yscale*\radius);
                \coordinate (AB) at ( 1.000*\xscale*\radius,  0.000*\yscale*\radius);
                \coordinate (BA) at ( 2.000*\xscale*\radius,  0.000*\yscale*\radius);
                \coordinate (BB) at ( 3.000*\xscale*\radius,  0.000*\yscale*\radius);
                % edges
                \draw (Z) -- (A);
                \draw (Z) -- (B);
                \draw (A) -- (AA);
                \draw (A) -- (AB);
                \draw (B) -- (BA);
                \draw (B) -- (BB);
                % root
                \fill[fill=black] (Z) circle[radius=2pt];
                % nodes
                \filldraw[fill=\colora, draw=black] (A)  circle[radius=\radius] node{\ntextA};
                \filldraw[fill=\colorb, draw=black] (B)  circle[radius=\radius] node{\ntexta};
                \filldraw[fill=\colora, draw=black] (AA) circle[radius=\radius] node{\ntextB};
                \filldraw[fill=\colorb, draw=black] (AB) circle[radius=\radius] node{\ntextb};
                \filldraw[fill=\colora, draw=black] (BA) circle[radius=\radius] node{\ntextB};
                \filldraw[fill=\colorb, draw=black] (BB) circle[radius=\radius] node{\ntextb};
                % label macros
                \newcommand{\rlabel}[4]%
                {%
                  \coordinate (TEMP) at ($(#1)!0.5!(#2)$);
                  \coordinate (TEMP) at ($(TEMP)!#3!270:(#2)$);
                  \node at (TEMP) {#4};
                }%
                \newcommand{\llabel}[4]{\rlabel{#2}{#1}{#3}{#4}};
                % edge labels
                \rlabel{Z}{A}{3mm}{\etextA};
                \llabel{Z}{B}{3mm}{\etexta};
                \rlabel{A}{AA}{3mm}{\etextAB};
                \llabel{A}{AB}{3mm}{\etextAb};
                \rlabel{B}{BA}{3mm}{\etextaB};
                \llabel{B}{BB}{3mm}{\etextab};
              \end{scope}
            \end{tikzpicture}
            \hspace*{\fill}%
      \item Gesucht wird die bedingte Wahrscheinlichkeit
            für \glqq Marmelade\grqq{} unter der
            Bedingung \glqq Schokolade\grqq, also:
            \begin{equation*}
              P_{S}(M)=
              \frac{P(S\cap M)}{P(S)}=
              \frac{\num{0.15}}{\num{0.6}}=
              \num{0.25}
            \end{equation*}
    \end{enumerate}
    % </OUTCOME>
  \fi
\end{exercise}

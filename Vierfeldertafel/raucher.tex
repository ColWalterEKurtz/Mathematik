\begin{exercise}
      {ID-66023e9d6f57577f879c9e5e96ee2dd87e2ed481}
      {Raucher}
  \ifproblem\problem\par
    An einem Berufskolleg wurden alle 674 Schülerinnen und Schüler befragt,
    ob sie rauchen oder nicht. Das Ergebnis der Befragung sieht wie folgt aus:
    82 der insgesamt 293 männlichen Schüler gaben an zu rauchen.
    250 Schülerinnen gaben an, nicht zu rauchen.
    \begin{enumerate}
      \item Stelle den Sachzusammenhang in einer Vierfeldertafel dar.
      \item Mit welcher Wahrscheinlichkeit ist eine zufällig ausgewählte
            Person weiblich und Nichtraucherin?
      \item Mit welcher Wahrscheinlichkeit ist eine Schülerin Nichtraucherin?
      \item Untersuche ob in diesem Fall die Merkmale \textit{Geschlecht}
            und \textit{Raucher} unabhängig voneinander sind.
    \end{enumerate}
  \fi
  %\ifoutline\outline\par
  %\fi
  %\ifoutcome\outcome\par
  %\fi
\end{exercise}

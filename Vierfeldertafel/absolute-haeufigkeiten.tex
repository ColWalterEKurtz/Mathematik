\begin{exercise}
      {ID-c3f373b5c46d0a174a387efd7739c3553110a8bf}
      {Absolute Häufigkeiten}
  \ifproblem\problem
    Fülle mit den folgenden Informationen eine Vierfeldertafel mit
    absoluten Häufigkeiten aus.
    \begin{enumerate}
      \item In einer Schulklasse mit 29 Schülern haben 10 Schüler braune Haare
            und 7 Schüler grüne Augen. 5 Schüler haben grüne Augen und braune Haare.
      \item In einer Firma arbeiten 45 Männer und 50 Frauen. 30 weibliche Mitarbeiter
            der Firma sind jünger als 50 Jahre und 27 Männer sind älter als 50 Jahre.
      \item Von 25 Schülern, die am Sportunterricht teilnehmen, sind 15 weiblich.
            Genau 15 von diesen sind gut im Weitwurf.
            10 Mädchen sind nicht gut im Weitwurf.
      \item Bei einer Versuchsreihe nehmen 47 Personen teil. 20 von diesen Personen
            wurden auf eine bestimmte Krankheit positiv getestet. 32 von den
            Testpersonen sind gegen diese Krankheit geimpft, wobei 15 Personen
            positiv getestet wurden und nicht dagegen geimpft sind.
    \end{enumerate}
  \fi
  %\ifoutline\outline
  %\fi
  %\ifoutcome\outcome
  %\fi
\end{exercise}

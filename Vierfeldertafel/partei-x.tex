\begin{exercise}
      {ID-9c4dc8a6c8ab19cb898338d7148a1290338b2dac}
      {Partei X}
  \ifproblem\problem
    Bei einer Meinungsumfrage wurden die Wähler eines bestimmten Wahllokals
    nach der Stimmabgabe befragt, ob sie für Partei X gestimmt hätten.
    Dabei gaben \pc{3} an, sie gewählt zu haben, \pc{97} nannten andere
    Parteien. Nach Auszählung der Stimmen ergab sich in diesem Wahllokal
    ein Stimmenanteil von \pc{8} für Partei X. Wir gehen davon aus,
    dass die Wähler, die sich nach der Wahl zu Partei X bekannten,
    diese auch wirklich gewählt haben.
    \begin{enumerate}
      \item Wie viel Prozent der Befragten haben gelogen?
      \item Wie viel Prozent der Wähler von Partei X haben gelogen?
    \end{enumerate}
  \fi
  %\ifoutline\outline
  %\fi
  %\ifoutcome\outcome
  %\fi
\end{exercise}

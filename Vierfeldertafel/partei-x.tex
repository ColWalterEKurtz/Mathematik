% 2021-12-08
\begin{exercise}
      {ID-34efd2ea792a89d08c0372a8c3e375ac914a7ae7}
      {Partei X}
  \ifproblem\problem\par
    % <PROBLEM>
    Bei einer Meinungsumfrage wurden die Wähler
    eines bestimmten Wahllokals nach der
    Stimmabgabe befragt, ob sie für Partei X
    gestimmt hätten. Dabei gaben \pc{3} an, sie
    gewählt zu haben, \pc{97} nannten andere
    Parteien. Nach Auszählung der Stimmen ergab
    sich in diesem Wahllokal ein Stimmenanteil
    von \pc{8} für Partei X. Wir gehen davon aus,
    dass die Wähler, die sich nach der Wahl zu
    Partei X bekannten, diese auch wirklich
    gewählt haben.
    \begin{enumerate}[a)]
      \item Wie viel Prozent der Befragten haben
            gelogen?
      \item Wie viel Prozent der Wähler von Partei
            X haben gelogen?
    \end{enumerate}
    % </PROBLEM>
  \fi
  %\ifoutline\outline\par
    % <OUTLINE>
    % </OUTLINE>
  %\fi
  \ifoutcome\outcome\par
    % <OUTCOME>
    \begin{minipage}[c]{0.38\linewidth}
      \begin{fofotab}%[t]
        % Bezeichnungen
        \lblA{$X$}
        \lbla{$\overline{X}$}
        \lblB{$B$}
        \lblb{$\overline{B}$}
        % Mitte
        \andAB{\num{0.03}}
        \andAb{\num{0.05}}
        \andaB{\num{0}}
        \andab{\num{0.92}}
        % Rand
        \sumA {\num{0.08}}
        \suma {\num{0.92}}
        \sumB {\num{0.03}}
        \sumb {\num{0.97}}
        \total{1}
      \end{fofotab}
    \end{minipage}%
    \begin{minipage}[c]{0.49\linewidth}
      \begin{itemize}
        \renewcommand{\itemsep}{-1ex}%
        \item[$X$:]\glqq Partei X gewählt\grqq
        \item[$\overline{X}$:]\glqq Partei X nicht gewählt\grqq
        \item[$B$:]\glqq zu Partei X bekannt\grqq
        \item[$\overline{B}$:]\glqq nicht zu Partei X bekannt\grqq
      \end{itemize}
    \end{minipage}\par
    \begin{enumerate}[a)]
      \item Die Befragten können theoretisch auf
            zwei verschiedene Arten lügen:
            \begin{equation*}
              P(\text{\glqq Lüge\grqq})
              =P(X\cap\overline{B})+P(\overline{X}\cap B)
              =\num{0.05}+\num{0}=\SI{5}{\percent}
            \end{equation*}
      \item Sucht man die Unehrlichen nur im Kreis
            der Wähler von Partei X, kann man dies
            als bedingte Wahrscheinlichkeit
            berechnen:
            \begin{equation*}
              P_{X}(\overline{B})
              =\frac{P(X\cap\overline{B})}{P(X)}
              =\frac{\num{0.05}}{\num{0.08}}
              =\SI{62.5}{\percent}
              %5/8*100
            \end{equation*}
    \end{enumerate}
    % </OUTCOME>
  \fi
\end{exercise}

\renewcommand{\subdir}{Zahlen/}

\ifproblem
% --------------
\section{Zahlen}
% --------------
\fi
\input{\subdir zahlenraetsel.tex}
\input{\subdir vierstellig.tex}
\input{\subdir benachbarte-kreise.tex}
\input{\subdir fuenfundsiebzig.tex}
\input{\subdir dreistelligen-zahlen.tex}
\input{\subdir quersumme.tex}
\input{\subdir primzahl-als-summe.tex}
\input{\subdir kopfrechnen.tex}
\input{\subdir zahlen-zaehlen.tex}
\input{\subdir geteilt-durch-7.tex}
\input{\subdir quer.tex}
\input{\subdir erweitern.tex}
\input{\subdir wesentlich-verschieden.tex}
\input{\subdir minimales-produkt.tex}
\input{\subdir quadratzahl.tex}
\input{\subdir kubikzahl.tex}
\input{\subdir quersumme-18.tex}
\input{\subdir 73-stellig.tex}
\input{\subdir vollstaendig-gekuerzt.tex}
\input{\subdir teilbarkeit.tex}
\input{\subdir ein-merkwuerdiges-produkt.tex}
\input{\subdir geteilt-durch-6.tex}
\input{\subdir kann-es-sein-oder-nicht-sein.tex}
\input{\subdir nullen.tex}
\input{\subdir eins-eins-eins.tex}
\input{\subdir zaun.tex}
\input{\subdir schrittlaenge.tex}
\input{\subdir drei-buslinien.tex}


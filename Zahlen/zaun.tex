\begin{exercise}
      {ID-f6984cf366c7211209d720bfe345f0549391d1d3}
      {Zaun}
  \ifproblem\problem\par
    % <PROBLEM>
    Ein rechteckiges Grundstück soll eingezäunt werden.
    Es ist \SI{120}{\metre} lang und \SI{84}{\metre}
    breit. Der Abstand von einem Zaunpfahl zum Nächsten
    soll überall gleich groß sein.
    Wie viele Zaunpfähle werden mindestens benötigt,
    um das Grundstück einzuzäunen?
    % </PROBLEM>
  \fi
  %\ifoutline\outline\par
    % <OUTLINE>
    % </OUTLINE>
  %\fi
  \ifoutcome\outcome\par
    % <OUTCOME>
    Die Anzahl der Zaunpfähle lässt sich bestimmen,
    wenn man den Abstand $x$ kennt, der jeden
    Zaunpfahl von seinem Vorgänger bzw. Nachfolger
    trennt.
    \begin{center}
      \begin{tikzpicture}[scale=0.5]
        % Grundstueck
        \draw[line width=0.6pt] (0, 0) rectangle (120mm, 84mm);
        % Seitenlaengen
        \node[below] at (60mm, 0mm) {\SI{120}{\metre}};
        \node[right] at (120mm, 42mm) {\SI{84}{\metre}};
        % Ecken
        \fill (  0mm,  0mm) circle[radius=2pt];
        \fill (120mm,  0mm) circle[radius=2pt];
        \fill (120mm, 84mm) circle[radius=2pt];
        \fill (  0mm, 84mm) circle[radius=2pt];
        % Beispiele
        \fill ( 12mm, 84mm) circle[radius=2pt];
        \fill ( 24mm, 84mm) circle[radius=2pt];
        \fill ( 36mm, 84mm) circle[radius=2pt];
        \fill (  0mm, 72mm) circle[radius=2pt];
        \fill (  0mm, 60mm) circle[radius=2pt];
        \fill (  0mm, 48mm) circle[radius=2pt];
        \begin{scope}[|<->|, >=stealth]
          \draw (12mm, 72mm) -- (24mm, 72mm);
          \draw (12mm, 72mm) -- (12mm, 60mm);
          \node at (18mm, 66mm) {$x$};
        \end{scope}
        \begin{scope}[style=dashed]
          \draw (12mm, 84mm) -- (12mm, 72mm);
          \draw (24mm, 84mm) -- (24mm, 72mm);
          \draw ( 0mm, 72mm) -- (12mm, 72mm);
          \draw ( 0mm, 60mm) -- (12mm, 60mm);
        \end{scope}
      \end{tikzpicture}
    \end{center}
    Dieser Abstsnd muss so gewählt werden, dass
    sowohl \num{120}, als auch \num{84}
    ganzzahlige Vielfache von $x$ sind.
    Außerdem muss er so groß wie möglich sein,
    damit die Anzahl der benötigten Zaunpfähle
    so gering wie möglich ausfällt.
    Gesucht wird also der größte gemeinsame
    Teiler von \num{120} und \num{84}.
    \begin{equation*}
      \begin{split}
        120&=12\cdot10=2\cdot6\cdot2\cdot5=2\cdot2\cdot3\cdot2\cdot5=2^3\cdot3\cdot5
        \\
        84&=2\cdot42=2\cdot6\cdot7=2\cdot2\cdot3\cdot7=2^2\cdot3\cdot7
        \\
        \operatorname{ggT}(120,84)&=2^2\cdot3=12
      \end{split}
    \end{equation*}
    Je zwei Zaunpfähle sollten also \SI{12}{\metre}
    Abstand voneinander haben.
    \begin{center}
      \begin{tikzpicture}
        % ------
        % clabel
        % ------
        %
        % #1 start point
        % #2 target point
        % #3 line distance (pos: left; neg: right)
        % #4 angle to rotate ([0..360]|auto)
        % #5 text
        %
        % example:
        % \clabel{0, 0}{16, 9}{3mm}{auto}{$d$}
        %
        \newcommand{\clabel}[5]%
        {%
          \begingroup
            \coordinate (TEMP) at ($(#1)!0.5!(#2)$);%
            \coordinate (TEMP) at ($(TEMP)!#3!90:(#2)$);%
            \ifthenelse{\equal{#4}{auto}}%
            {%
              \coordinate (TEMPA) at (#1);%
              \coordinate (TEMPB) at (#2);%
              \pgfmathanglebetweenpoints%
                {\pgfpointanchor{TEMPA}{center}}%
                {\pgfpointanchor{TEMPB}{center}}%
              \node[rotate=\pgfmathresult] at (TEMP) {#5};%
            }%
            {%
              \node[rotate=#4] at (TEMP) {#5};%
            }%
          \endgroup
        }%
        % ------
        % vlabel
        % ------
        %
        % vertex label
        %
        % #1  label
        % #2  label node
        % #3  direction 1 node
        % #4  direction 2 node
        % #5  label distance
        %
        % example:
        % \vlabel{$A$}{A}{B}{C}{0.75em};
        %
        \newcommand{\vlabel}[5]%
        {%
          \begin{scope}%
            \coordinate (S1) at ($(#2)!1cm!0:(#3)$);
            \coordinate (S2) at ($(#2)!1cm!0:(#4)$);
            \coordinate (M)  at ($(S1)!0.5!0:(S2)$);
            \node at ($(#2)!#5!180:(M)$) {#1};
          \end{scope}%
        }%
        \begin{scope}[yshift=6.5mm]
          \coordinate (A) at (0, 0);
          \coordinate (B) at (0:12mm);
          \coordinate (C) at (0:24mm);
          \draw[line width=0.6pt] (A) -- (B) -- (C);
          \fill (A) circle[radius=1pt];
          \fill (B) circle[radius=1pt];
          \fill (C) circle[radius=1pt];
          \clabel{A}{B}{2mm}{0}{$s$};
          \clabel{B}{C}{2mm}{0}{$s$};
          \node[below] at (A) {$p$};
          \node[below] at (B) {$p$};
          \node[below] at (C) {$p$};
        \end{scope}
        \begin{scope}[xshift=5cm]
          \coordinate (A) at (0, 0);
          \coordinate (B) at (0:12mm);
          \coordinate (C) at (60:12mm);
          \draw[line width=0.6pt] (A) -- (B) -- (C) -- cycle;
          \fill (A) circle[radius=1pt];
          \fill (B) circle[radius=1pt];
          \fill (C) circle[radius=1pt];
          \clabel{A}{B}{-2mm}{0}{$s$};
          \clabel{B}{C}{-2mm}{0}{$s$};
          \clabel{C}{A}{-2mm}{0}{$s$};
          \vlabel{$p$}{A}{B}{C}{0.75em};
          \vlabel{$p$}{B}{C}{A}{0.75em};
          \vlabel{$p$}{C}{A}{B}{0.75em};
        \end{scope}
      \end{tikzpicture}
    \end{center}
    Wenn man einen Zaun baut, der an verschiedenen
    Punkten beginnt und endet, benötigt man für
    $n$ Segmente $n+1$ Zaunpfähle (wie in der
    linken Darstellung).
    Wenn der Zaun einen geschlossenen Weg darstellt,
    kann das letzte Segment wieder am ersten
    Zaunpfahl befestigt werden, und man benötigt
    genau so viele Pfähle wie Segmente (wie in der
    rechten Darstellung).\par
    Da der Zaun aus der Aufgabenstellung einen
    geschlossenen Weg darstellt, erhält man die
    Anzahl der Zaunpfähle schon dadurch, dass man
    die Gesamtlänge des Zauns durch die Segmentlänge
    teilt:
    \begin{equation*}
      \frac{2\cdot120+2\cdot84}{12}
      =2\cdot\frac{120}{12}+2\cdot\frac{84}{12}
      =2\cdot10+2\cdot7
      =34
    \end{equation*}
    Man benötigt also mindestes 34 Zaunpfähle, um
    das Grundstück mit einem gleichmäßigen Abstand
    zwischen den Zaunpfählen eizuzäunen.
    % </OUTCOME>
  \fi
\end{exercise}

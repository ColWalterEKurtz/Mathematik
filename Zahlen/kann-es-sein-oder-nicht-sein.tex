\begin{exercise}
      {ID-c738c39d6d000b75fd00deb7f5a0ac183f4381a8}
      {Kann es sein, oder nicht sein?}
  \ifproblem\problem\par
    Gibt es eine natürliche Zahl $n>1$, bei der die Rechnung $n^{4}+4$ eine
    Primzahl als Ergebnis hat?
  \fi
  \ifoutline\outline\par
    Ergänze den Term $n^{4}+4$ zu einem vollständigen Binom:
    \begin{equation*}
      n^{4}+4=n^{4}+4n^{2}+4-4n^{2}
    \end{equation*}
  \fi
  \ifoutcome\outcome\par
    Wir zeigen, dass die Summe $n^{4}+4$ immer als Produkt aus
    zwei verschiedenen Faktoren geschrieben werden kann, von denen
    wegen $n>1$ keiner den Wert 1 besitzt:
    \begin{equation*}
      \begin{split}
        n^{4}+4&=n^{4}+4n^{2}+4-4n^{2}\\
               &=\left(n^{2}+2\right)^{2}-4n^{2}\\
               &=\left(n^{2}+2+2n\right)\left(n^{2}+2-2n\right)
      \end{split}
    \end{equation*}
  \fi
\end{exercise}

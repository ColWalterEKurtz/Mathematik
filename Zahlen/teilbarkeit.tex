\begin{exercise}
      {ID-f948d5285f73d6805256a306f86f1a8c291382ee}
      {Teilbarkeit}
  \ifproblem\problem\par
    Zeige, dass eine Zahl $\overline{abcabc}$ aus den Ziffern $a$, $b$
    und $c$ immer durch 7, 11 und 13 teilbar ist.
  \fi
  \ifoutline\outline\par
    Überlege dir eine Rechnung, bei der immer Zahlen der
    Form $\overline{abcabc}$ entstehen.
    Multipliziere mit Zehnerpotenzen\ldots
  \fi
  \ifoutcome\outcome\par
    Jede Zahl $\overline{abcabc}$ lässt sich durch folgende Rechnung bilden:
    \begin{equation*}
      \overline{abcabc}=1000\cdot\overline{abc}+\overline{abc}
      =(1000+1)\cdot\overline{abc}
      =1001\cdot\overline{abc}
    \end{equation*}
    Also ist jede Zahl $\overline{abcabc}$ durch 1001 teilbar.
    Mit $7\cdot11\cdot13=1001$ ist die Zahl $\overline{abcabc}$
    dann auch durch 7, 11 und 13 teilbar.
  \fi
\end{exercise}

\begin{exercise}
      {ID-c5b39ea16827187dcc3c7a2fdcc009495a327bff}
      {Drei Buslinien}
  \ifproblem\problem
    Um 6 Uhr morgens starten drei Busse gleichzeitig vom Bahnhof einer Stadt.
    Sie fahren bis spätestens 22 Uhr abends.\par
    Für eine Runde benötigt
    der  erste   Bus 1 Stunde und 30 Minuten,
    der  zweite  Bus 1 Stunde und 50 Minuten und
    der  dritte  Bus 1 Stunde und 10 Minuten.
    Nach jeder Runde machen sie 10 Minuten Pause.
    Wann starten
    \begin{enumerate}[a)]
      \squeeze
      \item der erste und der zweite Bus
      \item der zweite und der dritte Bus
      \item alle drei Busse
    \end{enumerate}
    wieder gleichzeitig vom Bahnhof?
  \fi
  \ifoutline\outline
    Bestimme die gemeinsamen Vielfache der jeweiligen Fahrzeiten.
  \fi
  \ifoutcome\outcome
    Wenn man die Fahrzeiten in Minuten umrechnet, und die Pausen berücksichtigt,
    starten die drei Busse in folgenden Abständen jeweils wieder vom Bahnhof:
    \begin{itemize}
      \squeeze
      \item der erste Bus: 100 Minuten
      \item der zweite Bus: 120 Minuten
      \item der dritte Bus: 80 Minuten
    \end{itemize}
    Die gemeinsamen Vielfache der jeweiligen Zeitspannen geben an,
    wann die Busse (wieder) geichzeitig vom Bahnhof starten:
    \begin{equation*}
      \begin{split}
        \text{a) }&\operatorname{kgV}(100, 120)=600\\
        \text{b) }&\operatorname{kgV}(120, 80)=240\\
        \text{c) }&\operatorname{kgV}(80, 100, 120)=1200
      \end{split}
    \end{equation*}
    Umgerechnet in Uhrzeiten ergeben sich folgende gemeinsame Startzeitpunkte:
    \begin{enumerate}[a)]
      \item \makebox[7em][l]{Bus 1 und 2:} 06:00 Uhr und 16:00 Uhr
      \item \makebox[7em][l]{Bus 2 und 3:} 06:00 Uhr, 10:00 Uhr, 14:00 Uhr und 18:00 Uhr
      \item \makebox[7em][l]{Bus 1, 2 und 3:} 06:00 Uhr
    \end{enumerate}
  \fi
\end{exercise}

\begin{exercise}
      {ID-e9db77c9a5bbbfd8e37ea60ee54738c90e3cccfc}
      {Schrittlänge}
  \ifproblem\problem\par
    Vater und Sohn messen mit ihren Schritten den Abstand zwischen zwei Bäumen.
    Zuerst schreitet der Vater die Strecke ab, danach sein Sohn. Als der Sohn
    losgeht, ist auf dem Boden die Spur des Vaters deutlich zu erkennen.
    Der Sohn tritt insgesamt 10 Mal genau auf den Fußabdruck seines Vaters --
    zum Glück auch am Ende der Strecke. Ein Schritt des Vaters ist \sicm{70}
    lang, ein Schritt des Sohnes misst \sicm{56}. Wie weit sind die beiden
    Bäume voneinander entfernt?
  \fi
  \ifoutline\outline\par
    Vielleicht hilft das kleinste gemeinsame Vielfache\ldots
  \fi
  %\ifoutcome\outcome\par
  %\fi
\end{exercise}

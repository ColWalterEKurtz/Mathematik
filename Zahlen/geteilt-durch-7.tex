\begin{exercise}
      {ID-40e39887e6e87d4c4359a077d612f411f48d2ce5}
      {Geteilt durch 7}
  \ifproblem\problem
    Eine positive ganze Zahl soll auf Teilbarkeit durch 7 geprüft werden.
    Betrachtet wird die folgende Regel:\par
    \begin{quote}
      \itshape
      Die Einerziffer der Zahl wird gestrichen und dann wird das Doppelte der
      Einerziffer abgezogen. Wenn die so erhaltene Zahl durch 7 teilbar ist,
      dann ist auch die Ausgangszahl durch 7 teilbar.
    \end{quote}
    \textit{Beispiel:} Ist $539$ durch $7$ teilbar? Aus der vorgegebenen Zahl 539
    entsteht mit der Zwischenrechnung $53\cancel9-2\cdot9=53-18$ die Zahl 35.
    Da 35 durch 7 teilbar ist, ist nach der Regel auch 539 durch 7 teilbar.
    \begin{enumerate}[a)]
      \item Zeige mit Hilfe dieser Regel, dass die Zahl $364$ durch $7$ teilbar ist.
      \item Zeige durch viermalige Anwendung dieser Regel, dass $3\,645\,068$ durch
            $7$ teilbar ist. Notiere hierzu die Zahlen, deren Teilbarkeit durch 7
            zu prüfen ist. Berechne dann auch den Wert des Quotienten $3\,645\,068:7$
            durch schriftliche Division.
      \item Beweise die Regel.
    \end{enumerate}
  \fi
  %\ifoutline\outline
  %\fi
  %\ifoutcome\outcome
  %\fi
\end{exercise}

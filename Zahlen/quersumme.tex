\begin{exercise}
      {ID-57035c885475fb85ec5dea8994b9bc64209e8d40}
      {Quersumme}
  \ifproblem\problem\par
    Eine natürliche Zahl kann die Eigenschaft haben, dass sie durch ihre Quersumme
    teilbar ist. Ein Beispiel ist 12.
    \begin{enumerate}[a)]
      \item Gib zwei zweistellige natürliche Zahlen an, deren größter gemeinsamer
            Teiler 1 ist und die jeweils durch ihre Quersumme teilbar sind.
      \item Untersuche, ob alle durch 9 teilbaren zweistelligen natürlichen Zahlen
            durch ihre Quersumme teilbar sind.
      \item Begründe durch allgemeine Feststellungen, dass alle durch 10 teilbaren
            zweistelligen natürlichen Zahlen durch ihre Quersumme teilbar sind.
    \end{enumerate}
  \fi
  %\ifoutline\outline\par
  %\fi
  %\ifoutcome\outcome\par
  %\fi
\end{exercise}

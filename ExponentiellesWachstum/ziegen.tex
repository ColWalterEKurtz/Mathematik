\begin{exercise}
      {ID-5b9e754f0f44f8f5d57edc2d1914d1e454ba87ad}
      {Ziegen}
  \ifproblem\problem\par
    Auf einer Südatlantikinsel wurde im Jahr 1695 eine
    unbekannte Zahl von Ziegen ausgesetzt. Im Jahr 1705
    zählte man 25 Ziegen und 2 Jahre später
    36 Ziegen.
    \begin{enumerate}[a)]
      \item Bestimme die jährliche Zuwachsrate in Prozent.
      \item Wie viele Ziegen sind im Jahr 1695 ausgesetzt
            worden, wenn man exponentielles Wachstum voraussetzt?
      \item Wie viele Ziegen wären im Jahr 1710 zu erwarten?
      \item Bestimme die monatliche Zuwachsrate in Prozent.
      \item Nach wie vielen Monaten verdoppelt sich der Bestand
            jeweils?
    \end{enumerate}
  \fi
  \ifoutline\outline
    \begin{equation*}
      w(t)=w_0\cdot q^t
      \quad\text{mit}\quad
      q=1\pm\frac{p}{100}
    \end{equation*}
  \fi
  \ifoutcome\outcome
    \begin{enumerate}[a)]
      \item Innerhalb von 2 Jahren ist die Anzahl der Ziegen
            von 25 auf 36 gestiegen. Daraus ergibt sich folgende
            Gleichung:
            \begin{equation*}
              36=25\cdot q^2=25\cdot\left(1+\frac{p}{100}\right)^2
            \end{equation*}
            Diese Gleichung löst man jetzt nach $p$ auf:
            \begin{alignat*}{3}
              \quad&\quad
              &
              36&=25\cdot\left(1+\frac{p}{100}\right)^2
              &
              \quad&|:25
              \\[1ex]
              \Leftrightarrow&\quad
              &
              \frac{36}{25}&=\left(1+\frac{p}{100}\right)^2
              &
              \quad&|\;\sqrt{\,\cdot\,}
              \\[1ex]
              \Leftrightarrow&\quad
              &
              \frac{6}{5}&=1+\frac{p}{100}
              &
              \quad&|-1
              \\[1ex]
              \Leftrightarrow&\quad
              &
              \frac{1}{5}&=\frac{p}{100}
              &
              \quad&|\cdot100
              \\[1ex]
              \Leftrightarrow&\quad
              &
              20&=p
              &
              \quad&\quad
            \end{alignat*}
            Die jährliche Zuwachsrate liegt also bei \pc{20}.
      \item Wie viele Ziegen vermehren sich in 10 Jahren
            (1695--1705) bei einer jährlichen Wachstunsrate von
            \pc{20} auf 25 Exemplare?
            \begin{equation*}
              25=w_0\cdot\num{1.2}^{10}
              \quad\Rightarrow\quad
              w_0=\frac{25}{\num{1.2}^{10}}
              \approx\num{4.038}
            \end{equation*}
            Im Jahr 1695 sind etwa 4 Ziegen ausgesetzt worden.
      \item Von den 36 Tieren im Jahr 1707 kann man 3 Jahre
            in die Zukunft rechnen:
            \begin{equation*}
              w(3)=36\cdot\num{1.2}^3\approx\num{62.208}
            \end{equation*}
            Im Jahr 1710 wären also etwa 62 Ziegen zu erwarten.
      \item Gesucht wird der Wachstumsfaktor $q_m$, der bei
            zwölfmaliger Anwendung dieselbe Vervielfachung
            bewirkt wie der jährliche Wachstumsfaktor $q$.
            \begin{equation*}
              \num{1.2}=q_m^{12}
              \quad\Rightarrow\quad
              q_m=\sqrt[12]{\num{1.2}}
              \approx\num{1.0153}
            \end{equation*}
            Die Anzahl der Ziegen nimmt also monatlich
            um etwa \pc{1.53} zu.
      \item Die Anzahl der Ziegen $w_0$ verdoppelt sich, wenn
            der Faktor $q^t$ den Wert 2 annimmt.
            Gesucht wird also der Exponent, mit dem man $q$
            potenzieren muss, damit 2 herauskommt.
            \begin{equation*}
              2=\big(\sqrt[12]{\num{1.2}}\big)^t
              \quad\Rightarrow\quad
              t=\frac{\ln(2)}{\ln\big(\sqrt[12]{\num{1.2}}\big)}
              \approx\num{45.621}
            \end{equation*}
            Die Anzahl der Ziegen verdoppelt sich also etwa alle \num{45.6} Monate.
    \end{enumerate}
  \fi
\end{exercise}

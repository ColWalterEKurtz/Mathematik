\begin{exercise}
      {ID-5b9e754f0f44f8f5d57edc2d1914d1e454ba87ad}
      {Ziegen}
  \ifproblem\problem
    Auf einer Südatlantikinsel wurde im Jahr 1695 eine
    unbekannte Zahl von Ziegen ausgesetzt. Im Jahr 1705
    zählte man 25 Ziegen und 2 Jahre später
    36 Ziegen.
    \begin{enumerate}[a)]
      \item Bestimme die jährliche Zuwachsrate in Prozent.
      \item Wie viele Ziegen sind im Jahr 1695 ausgesetzt
            worden, wenn man exponentielles Wachstum voraussetzt?
      \item Wie viele Ziegen wären im Jahr 1710 zu erwarten?
      \item Bestimme die monatliche Zuwachsrate in Prozent.
      \item Nach wie vielen Monaten verdoppelt sich der Bestand
            jeweils?
    \end{enumerate}
  \fi
  %\ifoutline\outline
  %\fi
  \ifoutcome\outcome
    \begin{enumerate}[a)]
      \item Die jährliche Zuwachsrate liegt bei etwa \pc{20}.
      \item Im Jahr 1695 sind etwa 4 (\num{4.038}) Ziegen
            ausgesetzt worden.
      \item Im Jahr 1710 wären etwa 62 (\num{62.208}) Ziegen
            zu erwarten.
      \item Die Anzahl der Ziegen nimmt monatlich um etwa
            \pc{1.531} zu.
      \item Die Anzahl der Ziegen verdoppelt sich etwa
            alle \num{3.802} Jahre.
    \end{enumerate}
  \fi
\end{exercise}

\begin{exercise}
      {ID-cbcd5c19eeca66a75d275de58ada01ce967cf7f5}
      {Zellteilung}
  \ifproblem\problem
    Bakterien vermehren sich durch Zellteilung, wobei sich eine
    Bakterienzelle der hier betrachteten Art durchschnittlich
    alle 12 Minuten teilt. Zum Zeitpunkt $t=0$ sei genau eine
    Bakterienzelle vorhanden.
    \begin{enumerate}[a)]
      \item Finde eine Formel für die Anzahl $N=N(t)$
            der Bakterien nach gegebener Zeit $t$ in Stunden.
      \item Wie viele Bakterien sind dann nach 1 Stunde, 2 Stunden,
            6 Stunden, 12 Stunden bzw. 24 Stunden vorhanden?
      \item Eine Bakterienzelle hat ein Volumen von ca.
            \SI{2e-18}{\cubic\metre}. Wie lange dauert es, bis die Bakterienkultur
            ein Volumen von \SI{1}{\cubic\metre} bzw. \SI{1}{\cubic\kilo\metre}
            einnimmt? Beurteile dein Ergebnis kritisch.
    \end{enumerate}
  \fi
  %\ifoutline\outline
  %\fi
  \ifoutcome\outcome
    \begin{enumerate}[a)]
      \item Der (exponentielle) Zusammenhang zwischen der Anzahl der
            Bakterien $N$ und der Zeit $t$ in Stunden lautet:
            \begin{equation*}
              N\left(t_{0}+\num{0.2}\right)=2\cdot N(t_{0})
              \quad\Rightarrow\quad
              q=\num{32}
              \quad\Rightarrow\quad
              N(t)=N_{0}\cdot\num{32}^{t}
            \end{equation*}
            Mit einem Anfangsbestand $N_{0}$ von einer einzigen Zelle
            vereinfacht sich die Funktionsgleichung zu:
            \begin{equation*}
              N(t)=\num{32}^{t}
            \end{equation*}
      \item Anzahl der Bakterien nach\ldots\\
            \makebox[6em][l]{1 Stunde:}%
            \makebox[9em][l]{etwa \num{32}}
            \makebox[6em][l]{12 Stunden:}%
            \makebox[9em][l]{etwa \num{1.1529e+18}}\\
            \makebox[6em][l]{2 Stunden:}%
            \makebox[9em][l]{etwa \num{1024}}
            \makebox[6em][l]{24 Stunden:}%
            \makebox[9em][l]{etwa \num{1.3292e+36}}\\
            \makebox[6em][l]{6 Stunden:}%
            \makebox[9em][l]{etwa \num{1.0737e+09}}
      \item Nach etwa \num{11.759} Stunden besitzen ca. \num{5e+17} Bakterien
            ein Volumen von \SI{1}{\cubic\metre}.\par
            Nach etwa \num{17.738} Stunden besitzen ca. \num{5e+26} Bakterien
            ein Volumen von \SI{1}{\cubic\kilo\metre}.
    \end{enumerate}
  \fi
\end{exercise}

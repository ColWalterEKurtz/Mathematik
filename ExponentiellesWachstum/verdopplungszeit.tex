% 2021-11-19
\begin{exercise}
      {ID-b90871525e32c60a676d6be133f8061a6f7a10b8}
      {Verdopplungszeit}
  \ifproblem\problem\par
    % <PROBLEM>
    Eine Bakterienkultur besitzt eine Verdopplungszeit
    von einer Stunde.
    Zu Beginn besteht die Kultur aus \num{500}
    Bakterien.
    \begin{enumerate}[a)]
      \item Stelle die Funktionsgleichung auf,
            die das exponentielle Wachstum der
            Bakterien in Abhängigkeit der Zeit
            $t$ beschreibt.
      \item Wie viele Bakterien sind es nach \num{3} Stunden?
      \item Wann beträgt die Anzahl der Bakterien das
            Hundertfache des Anfangswertes?
    \end{enumerate}
    % </PROBLEM>
  \fi
  %\ifoutline\outline\par
    % <OUTLINE>
    % </OUTLINE>
  %\fi
  \ifoutcome\outcome
    % <OUTCOME>
    \begin{enumerate}[a)]
      \item Wenn sich die Anzahl der Bakterien
            jede Stunde verdoppelt, und am
            Anfang \num{500} Bakterien
            vorhanden waren, kann das
            exponentielle Wachstum dieser
            Kultur durch folgende Gleichung
            beschrieben werden:
            \begin{equation*}
              w(t)=\num{500}\cdot2^t
            \end{equation*}
      \item Die Anzahl der Bakterien, die nach \num{3}
            Stunden vorhanden sind, erhält man durch
            das Einsetzen der gegebenen Zeitspanne in
            die Wachstumsfunktion $w$:
            \begin{equation*}
              w(3)=\num{500}\cdot2^3
                  =\num{500}\cdot8
                  =\num{4000}
            \end{equation*}
            Nach \num{3} Stunden besteht die
            Bakterienkultur also aus \num{4000}
            Bakterien.
      \item Um die Zeitspanne zu ermitteln, nach
            der der Startwert auf das Hundertfache
            angewachsen ist, bietet es sich an
            die Wachstumsfunktion $w$ nach $t$
            aufzulösen:
            \begin{alignat*}{3}
              \relax&\quad
              &
              100\cdot w_0&=w_0\cdot2^t
              &
              \quad&|:w_0
              \\
              \Leftrightarrow&\quad
              &
              100&=2^t
              &
              \quad&|\ln(\ldots)
              \\
              \Leftrightarrow&\quad
              &
              \ln(100)&=\ln(2^t)
              &
              \quad&\relax
              \\
              \Leftrightarrow&\quad
              &
              \ln(100)&=t\cdot\ln(2)
              &
              \quad&|:\ln(2)
              \\
              \Leftrightarrow&\quad
              &
              \frac{\ln(100)}{\ln(2)}&=t\approx\num{6.644}
              %log(100)/log(2)
              &
              \quad&\relax
            \end{alignat*}
            Also ist die Bakterienkultur nach knapp
            \SI{6}{\hour} und \SI{39}{\minute}
            auf das Hundertfache des
            Startwertes angewachsen.
            %(log(100)/log(2)-6)*60
    \end{enumerate}
    % </OUTCOME>
  \fi
\end{exercise}

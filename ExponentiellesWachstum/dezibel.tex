\begin{exercise}
      {ID-196ea9e5e4ca6b243c37777da9e0eb95f75b5bff}
      {Dezibel}
  \ifproblem\problem
    Die Lautstärke eines Geräusches wird in der Einheit Dezibel (\si{\decibel})
    gemessen. Steigt der Dezibelwert um 1, so wächst die Lautstärke um einen
    bestimmten Faktor $q$. Berechne den Wachstumsfaktor $q$ der Dezibelskala,
    wenn eine Erhöhung um \SI{10}{\decibel} einer Verdopplung der Lautstärke
    entspricht.
  \fi
  %\ifoutline\outline
  %\fi
  \ifoutcome\outcome
    Der Wachstumsfaktor der Dezibelskala beträgt etwa \num{1.072}.
  \fi
\end{exercise}

\begin{exercise}
      {ID-a138a4380458890d4bcad7131f79068409cd3e47}
      {Kapital}
  \ifproblem\problem
    Ein Anfangskapital in Höhe von \eur{3000} wird mit einem Jahreszinssatz
    von \pc{4.5} verzinst.
    \begin{enumerate}[a)]
      \item Stelle die Funktionsgleichung auf, die das Anwachsen dieses
            Kapitals beschreibt.
      \item Auf welchen Wert wächst das Kapital in 12 Jahren?
      \item Ein Anfangskapital, dessen Höhe hier nicht bekannt ist, soll
            bei einem gleichbleibenden Zinssatz $p$ angelegt werden und nach
            13 Jahren eine Verdopplung des Anfangswertes erreichen.
            Wie hoch muss dann der jährliche Zinssatz sein?
    \end{enumerate}
  \fi
  %\ifoutline\outline
  %\fi
  \ifoutcome\outcome
    \begin{enumerate}[a)]
      \item Das Anwachsen des Kapitals lässt sich mit der Gleichung
            $W_{n}=\num{3000}\cdot\num{1.045}^{n}$ beschreiben.
      \item Nach 12 Jahren ist das Kapital auf etwa \eur{5087.64}
            angewachsen.
      \item Bei einem Zinssatz von etwa $p=\pc{5.477}$ würde sich das
            Kapital in 13 Jahren verdoppeln.
    \end{enumerate}
  \fi
\end{exercise}

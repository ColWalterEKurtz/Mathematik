\begin{exercise}
      {ID-c79791299ad53fb6f71c7abcebb7fe23c92fd040}
      {Käse}
  \ifproblem\problem\par
    In einer \glqq steril\grqq{} verpackten Käsepackung wurden
    4 Wochen nach Verpackungsdatum \num{7.2} Millionen
    Bakterien pro Gramm und einen Tag später \num{7.9} Millionen
    Bakterien pro Gramm nachgewiesen.
    \begin{enumerate}[a)]
      \item Bestimme die tägliche Zuwachsrate in Prozent.
      \item Bestimme die wöchentliche Zuwachsrate in Prozent.
      \item Nach wie vielen Tagen verdoppelt sich der Bestand jeweils?
      \item Wie viele Bakterien waren unter der Annahme eines
            exponentiellen Wachstums bei der Verpackung in die
            Käseportion gelangt?
      \item Wie viele Bakterien wären nach 8 Wochen zu erwarten?
    \end{enumerate}
  \fi
  %\ifoutline\outline\par
  %\fi
  \ifoutcome\outcome\par
    \begin{enumerate}[a)]
      \item Die Anzahl der Bakterien wächst täglich um etwa \pc{9.722}.
      \item Die Anzahl der Bakterien wächst wöchentlich um etwa \pc{91.453}.
      \item Die Anzahl der Bakterien verdoppelt sich etwa
            alle \num{7.471} Tage.
      \item Beim Verpacken sind etwa \num{535900} Bakterien mit
            in die Verpackung gelangt.
      \item Bei exponentiellem Wachstum wären nach 8
            Wochen etwa \num{96.734} Millionen Bakterien zu erwarten.
    \end{enumerate}
  \fi
\end{exercise}

\begin{exercise}
      {ID-da4e535d6f4afffde1805068321bdcd0cafc4c8e}
      {Sparbuch}
  \ifproblem\problem
    Peters Großvater hat zur Geburt seines Enkels ein Konto mit \eur{200}
    Startguthaben für ihn eröffnet. Seitdem zahlt er jährlich am Tag
    der Kontoeröffnung weitere \eur{200} ein. Der jährliche Zinssatz $p$
    beträgt \pc{4.2} und bezieht sich auf das Guthaben, das um Mitternacht
    des Tages der Kontoeröffnung auf dem Konto vorhanden ist (d.\,h. die
    jährliche Überweisung von \eur{200} erfolgt jeweils nach der Verzinsung).
    \begin{enumerate}[a)]
      \item Gib für die ersten fünf Jahre in einer Tabelle an, wie hoch
            das Guthaben auf Peters Konto am jeweiligen Stichtag der
            Kontoeröffnung ist (nach den Zinsen um Mitternacht und der
            anschließenden Überweisung von \eur{200}).
      \item Nach wie viel Jahren ist das Kapital auf mehr als
            \eur{2500} angewachsen?
    \end{enumerate}
  \fi
  %\ifoutline\outline
  %\fi
  \ifoutcome\outcome
    \begin{center}
      \renewcommand{\arraystretch}{1.2}
      \begin{tabular}{|c|l|}
        \hline
        Jahre nach Eröffnung & Guthaben in \officialeuro     \\
                             & (nach Zinsen und Überweisung) \\
        \hline
        0 & \num{200} \\
        \hline
        1 & \num{408.40} \\
        \hline
        2 & \num{625.55} \\
        \hline
        3 & \num{851.83} \\
        \hline
        4 & \num{1087.60} \\
        \hline
        5 & \num{1333.28} \\
        \hline
        6 & \num{1589.28} \\
        \hline
        7 & \num{1856.03} \\
        \hline
        8 & \num{2133.98} \\
        \hline
        9 & \num{2423.61} \\
        \hline
        10& \num{2725.40} \\
        \hline
      \end{tabular}
    \end{center}
  \fi
\end{exercise}

\begin{exercise}
      {ID-0849ae0a17c082aab8c41b5272822a8bd8349519}
      {Wald}
  \ifproblem\problem
    Vor 10 Jahren betrug der Holzbestand eines Waldes \SI{7000}{\cubic\metre}.
    Ohne Durchforstungsmaßnahmen ist er inzwischen auf \SI{9880}{\cubic\metre}
    angewachsen. Man darf annehmen, dass das Holzwachstum im
    betrachteten Zeitraum ein exponentieller Vorgang gewesen ist.
    \begin{enumerate}[a)]
      \item Zeige, dass die jährliche Wachstumsrate ca. \pc{3.5} beträgt.
      \item Berechne die Zeitspanne, innerhalb der sich der Holzbestand
            verdoppelt bzw. verdreifacht (unter der Annahme, dass sich
            das Wachstum exponentiell fortsetzt).
      \item Man hat vor, in 3 Jahren \SI{3000}{\cubic\metre} Holz zu schlagen.
            Wann wird dieser Wald den heutigen Holzbestand danach
            wieder erreicht haben?
    \end{enumerate}
  \fi
  \ifoutline\outline
    Wenn bei exponentiellem Wachstum in $m$ Jahren $h$
    \si{\cubic\metre} Holz geschlagen werden sollen, dann
    wird der aktuelle Bestand $W_{0}$ in $n$ Jahren wieder
    erreicht sein:
    \begin{equation*}
      W_{0}=\left(W_{0}\cdot q^{m}-h\right)\cdot q^{n}
    \end{equation*}
  \fi
  \ifoutcome\outcome
    \begin{enumerate}[a)]
      \item Der aktuelle Bestand ergibt sich (annähernd) aus
            den Daten der Aufgabenstellung, wenn man diese in
            die Funktionsgleichung für exponentielles Wachstum
            einsetzt:
            \begin{equation*}
              \num{7000}\cdot\num{1.035}^{10}\approx\num{9874.191}\approx\num{9880}
            \end{equation*}
      \item Wenn man weiterhin von exponentiellem Wachstum ausgeht,
            verdoppelt sich der aktuelle Bestand in etwa \num{20.114}
            Jahren, und in \num{31.881} Jahren wird er sich verdreifacht
            haben.
      \item Wenn man in 3 Jahren \SI{3000}{\cubic\metre} Holz erntet,
            wird der Wald in etwa \num{6.285} Jahren wieder den
            Bestand von heute (\SI{9880}{\cubic\metre}) erreicht haben.
    \end{enumerate}
  \fi
\end{exercise}

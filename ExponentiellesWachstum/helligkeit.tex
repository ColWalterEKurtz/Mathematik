\begin{exercise}
      {ID-ad043c67465915ee1922a48c6d59a99dcf68d539}
      {Helligkeit}
  \ifproblem\problem\par
    Ein Taucher interessiert sich wegen Unterwassseraufnahmen
    dafür, welche Helligkeit in verschiedenen Tiefen herrscht.
    Messungen in einem bestimmten (recht trüben) See ergeben,
    dass die Helligkeit pro Meter Wassertiefe um ca. \pc{17}
    abnimmt.
    \begin{enumerate}[a)]
      \item Wie groß ist die Helligkeit in \simeter{1}, \simeter{2},
            \simeter{5} bzw. \simeter{10} Tiefe, verglichen mit der
            Helligkeit an der Wasseroberfläche?
      \item Beschreibe die Helligkeit $H$ als Funktion der
            Wassertiefe $x$ als Bruchteil der Helligkeit
            $H_{0}$ an der Wasseroberfläche.
      \item In welcher Tiefe beträgt die Helligkeit zum ersten Mal
            weniger als \pc{1} von $H_{0}$?
    \end{enumerate}
  \fi
  %\ifoutline\outline\par
  %\fi
  \ifoutcome\outcome\par
    \begin{enumerate}[a)]
      \item Die Helligkeit $H$ in $x$ Metern Tiefe lässt sich
            durch folgende Funktionsgleichung beschreiben:
            \begin{equation*}
              H(x)=H_{0}\cdot\num{0.83}^{x}
            \end{equation*}
      \item Relativ zur Helligkeit an der Wasseroberfläche beträgt
            die Helligkeit in\\
            \makebox[3em][r]{\simeter{1}}: etwa
            \makebox[5em][l]{\pc{83.000}}
            \makebox[3em][r]{\simeter{2}}: etwa
            \makebox[5em][l]{\pc{39.390}}\\
            \makebox[3em][r]{\simeter{5}}: etwa
            \makebox[5em][l]{\pc{68.890}}
            \makebox[3em][r]{\simeter{10}}: etwa
            \makebox[5em][l]{\pc{15.516}}
      \item Ab etwa \simeter{24.715} Wassertiefe ist nicht mehr als \pc{1}
            von der Helligkeit an der Wasseroberfläche vorhanden.
    \end{enumerate}
  \fi
\end{exercise}

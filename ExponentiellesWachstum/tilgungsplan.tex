\begin{exercise}
      {ID-5aaf870eb6b0fe978439930f3cd869258aca13a6}
      {Tilgungsplan}
  \ifproblem\problem\par
    Für einen Kredit über \eur{20000} wird folgender Tilgungsplan
    erstellt: Die monatlichen Kreditzinsen betragen \pc{0.5}
    der Schuld zu Monatsbeginn. Am Monatsende werden \eur{400}
    für Zinsen und Tilgung zurückgezahlt. Wie hoch ist die
    Schuld nach \num{12} Monaten?
  \fi
  \ifoutline\outline\par
    Bestand nach $n$ Intervallen, wenn sich ein exponentieller Prozess mit
    Wachstumsfaktor $q$ und ein linearer Prozess mit Wachstumssummand $r$
    überlagern:
    \begin{equation*}
      W_{n}=W_{n-1}\cdot q+r
      \quad\Rightarrow\quad
      W_{n}=W_{0}\cdot q^{n}+r\cdot\frac{1-q^n}{1-q}
    \end{equation*}
  \fi
  \ifoutcome\outcome\par
    Unmittelbar nach Zahlung der 12. Rate beträgt die
    Restschuld ca. \eur{14162.28}.
  \fi
\end{exercise}

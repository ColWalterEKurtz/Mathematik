\begin{exercise}
      {ID-8e85f019a45e47dea6a7b50c4ce90d0db0f60dd8}
      {Radiokarbonmethode}
  \ifproblem\problem
    Für eine Altersbestimmung von organischem Material verwendet
    man die Tatsache, dass die Luft schon immer (bzw. sehr lange)
    einen konstanten Anteil des radioaktiven Kohlenstoffisotops
    \isotope[14]{C} enthielt. Über die Atmung wird dieses in den
    lebenden Organismus aufgenommen. Deshalb finden in lebender
    Materie dauernd ca. \num{16000} \isotope[14]{C}-Zerfälle je Minute
    und \si{\kilo\gram} organischen Gewebes statt. Wenn bei
    Eintritt des Todes die Atmung aufhört, werden der toten
    Materie keine neuen \isotope[14]{C}-Isotope mehr zugeführt.
    Der Anteil des radioaktiven \isotope[14]{C}-Isotopes im toten
    Gewebe nimmt deshalb durch Zerfall dauernd weiter ab, die
    Anzahl der Zerfälle geht zurück. \isotope[14]{C} hat eine
    Halbwertszeit von \num{5730} Jahren.
    \begin{enumerate}[a)]
      \item Berechne die Zerfallskonstante.
      \item Wie viele \isotope[14]{C}-Isotope müssen in \sikg{1}
            lebender Materie vorhanden sein, damit \num{16000}
            Zerfälle pro Minute stattfinden?
      \item Messungen an der Mumie Tutanchamuns ergaben
            1985, dass etwa \num{10720} Zerfälle pro Minute
            und \si{\kilo\gram} stattfanden. In welchem Jahr
            starb Tutanchamun?
      \item 1991 wurde in den Alpen die Gletschermumie
            \glqq Ötzi\grqq{} gefunden. Sie enthielt noch \pc{53.3}
            des \isotope[14]{C}-Isotops, das in lebendem Gewebe
            vorhanden ist. Wann starb \glqq Ötzi\grqq?
      \item Bis zu welchem Alter lässt sich die
            \isotope[14]{C}-Methode anwenden, wenn man noch
            \pc{0.1} des ursprünglichen \isotope[14]{C}-Gehalts
            mit hinreichender Genauigkeit messen kann?
    \end{enumerate}
  \fi
  %\ifoutline\outline
  %\fi
  \ifoutcome\outcome
    \begin{enumerate}[a)]
      \item Die Zerfallskonstante lautet: $q=\num{0.999879}$
      \item Damit in \sikg{1} lebender Materie pro Minute \num{16000}
            \isotope[14]{C}-Isotope zerfallen können, müssen
            in ihr etwa \num{6.952e+13} Isotope vorhanden sein.
      \item Tutanchamun starb etwa \num{1326} Jahre vor der
            Geburt Christi.
      \item Ötzi starb etwa \num{3211} Jahre vor der Geburt Christi.
      \item Bei einer Messgenauigkeit von \pc{0.1} des ursprünglichen
            \isotope[14]{C}-Gehalts kann man mit der Radiokarbonmethode
            etwa \num{5.7e+04} Jahre in die Vergangenheit blicken.
    \end{enumerate}
  \fi
\end{exercise}

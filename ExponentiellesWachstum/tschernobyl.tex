% 2021-11-19
\begin{exercise}
      {ID-ee0514b56291c14f49f41bac279c4d09af94beb1}
      {Tschernobyl}
  \ifproblem\problem\par
  %\ifproblem\clearpage\hrulefill\\
    % <PROBLEM>
    Bei der Nuklearkatastrophe von Tschernobyl
    im April 1986 wurden etwa \SI{500}{\gram}
    des radioaktiven Isotops \isotope[131]{I}
    (\glqq Jod 131\grqq) freigesetzt.
    Die Halbwertszeit von \isotope[131]{I}
    beträgt ca. \num{8} Tage.
    \begin{enumerate}[a)]
      \item Stelle die Funktionsgleichung auf, die
            den Jod-Zerfall in Abhängigkeit von der
            Zeit $t$ in Tagen beschreibt.
      \item Wie viel Gramm des ausgetretenen
            \isotope[131]{I} war \num{4} Wochen
            nach der Explosion noch in der Umwelt
            vorhanden?
    \end{enumerate}
    % </PROBLEM>
  \fi
  %\ifoutline\outline\par
    % <OUTLINE>
    % </OUTLINE>
  %\fi
  %\ifoutcome\outcome\par
    % <OUTCOME>
    % </OUTCOME>
  %\fi
\end{exercise}

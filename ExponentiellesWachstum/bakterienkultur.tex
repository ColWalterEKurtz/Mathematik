\begin{exercise}
      {ID-0c905cd04b0787555cd76ecebbb45df9e2677292}
      {Bakterienkultur}
  \ifproblem\problem\par
    Eine Bakterienkultur bedeckt eine Fläche von \SI{0.2}{\square\milli\metre}
    und vermehrt sich jede Stunde um \pc{5}.
    \begin{enumerate}[a)]
      \item Wie groß ist die bedeckte Fläche $A(t)$ nach $t$ Stunden?
      \item Bestimme die tägliche Zuwachsrate in Prozent.
      \item Nach wie vielen Tagen wird eine Fläche von
            \SI{80}{\square\milli\metre} bedeckt sein?
    \end{enumerate}
  \fi
  %\ifoutline\outline\par
  %\fi
  \ifoutcome\outcome\par
    \begin{enumerate}[a)]
      \item Die bedeckte Fläche lässt sich mit folgender
            Funktionsgleichung berechnen ($t$ in Stunden):
            \begin{equation*}
              A(t)=\num{0.2}\cdot\num{1.05}^{t}
            \end{equation*}
      \item Die bedeckte Fläche nimmt täglich um \pc{222.510} zu.
      \item Nach etwa \pc{5.117} Tagen wird eine Fläche von
            \SI{80}{\square\milli\metre} bedeckt sein.
    \end{enumerate}
  \fi
\end{exercise}

\begin{exercise}
      {ID-83c3725a342794a86434b0578b3d60627daff0fe}
      {Bierschaum}
  \ifproblem\problem
    Bei einer schlecht eingeschenkten Maß Bier beträgt die Schaumhöhe
    anfangs \sicm{10}. Um das Bier einigermaßen trinken zu können, wartet
    der Gast eine gewisse Zeit. Nach 3 Minuten ist die Schaumhöhe
    auf die Hälfte zurückgegangen.
    \begin{enumerate}[a)]
      \item Stelle die Zerfallsgleichung für den Bierschaumzerfall auf.
      \item Berechne wann die Schaumhöhe auf \sicm{1} zurückgegangen ist.
      \item Bei einem anderen Gast beträgt die Schaumhöhe nach 3
            Miuten noch \sicm{3}. Wie hoch war die Schaumhöhe nach dem
            Einschenken.
      \item Mache plausibel, wann der Zerfall am stärksten ist.
    \end{enumerate}
  \fi
  %\ifoutline\outline
  %\fi
  \ifoutcome\outcome
    \begin{enumerate}[a)]
      \item Die Zerfallsgleichung ($t$ in Minuten) lautet:
            \begin{equation*}
              h(t)=W_{0}\cdot\left(\frac{1}{2}\right)^{\frac{1}{3}t}\approx W_{0}\cdot\num{0.794}^{t}
            \end{equation*}
      \item Nach etwa \num{10} (\num{9.966}) Minuten ist die Schaumhöhe auf \sicm{1}
            zurückgegangen.
      \item Direkt nach dem Einschenken war der Bierschaum \sicm{6} hoch.
    \end{enumerate}
  \fi
\end{exercise}

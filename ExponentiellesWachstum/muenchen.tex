\begin{exercise}
      {ID-314d5ab5b6794c6ec64f164de2e6ca6997a512d0}
      {München}
  \ifproblem\problem\par
    Zwischen den Jahren 1810 und 1850 nahm die Einwohnerzahl Münchens
    exponentiell zu. Während 1810 die Einwohnerzahl \num{40450} betrug,
    lag diese 40 Jahre später bei \num{96396}. Bestimme den Wachstumsfaktor
    $q$, um den die Einwohnerzahl pro Jahr zunahm.
  \fi
  %\ifoutline\outline\par
  %\fi
  \ifoutcome\outcome\par
    Die Einwohnerzahl Münchens ist in den Jahren von 1810 bis
    1850 um etwa das \num{1.022}-fache pro Jahr angewachsen
    (also: $q\approx\num{1.022}$).
  \fi
\end{exercise}

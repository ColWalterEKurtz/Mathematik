\begin{exercise}
      {ID-bde175356b3da6b16d7bbf255925817bb14782b5}
      {Aspirin}
  \ifproblem\problem\par
    Das schmerzstillende Mittel Acetylsalicylsäure wird im Körper
    exponentiell abgebaut. Seine wirksame Menge im Körper eines
    nierengesunden Menschen halbiert sich alle 3 Stunden.
    \begin{enumerate}[a)]
      \item Stelle eine Funktionsgleichung auf, die den Abbau
            einer Tablette mit \simg{500} Acetylsalicylsäure
            beschreibt. Verwende als unabhängige Größe die Zeit
            $t$ in Stunden.
      \item Berechne die Anzahl der Stunden, bis von einer
            solchen Tablette nur noch \simg{10} im Körper sind.
    \end{enumerate}
  \fi
  %\ifoutline\outline\par
  %\fi
  \ifoutcome\outcome\par
    \begin{enumerate}[a)]
      \item Der Abbau dieser Tablette kann durch folgende
            Funktionsgleichung beschrieben werden ($t$ in Stunden):
            \begin{equation*}
              W(t)=500\cdot\left(\frac{1}{2}\right)^{\frac{1}{3}t}
            \end{equation*}
      \item Nach etwa \num{17} Stunden sind nur noch \simg{10} des
            Wirkstoffes im Körper.
    \end{enumerate}
  \fi
\end{exercise}

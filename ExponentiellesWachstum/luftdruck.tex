\begin{exercise}
      {ID-0692db63195d7275c9a4be09d36024fdc4d64836}
      {Luftdruck}
  \ifproblem\problem\par
    Der Luftdruck $p$ wird in Hektopascal (\si{\hecto\pascal})
    gemessen. Aus einer großen Anzahl an Messungen ist bekannt,
    dass er exponentiell mit der Höhe abnimmt, und zwar um
    durchschnittlich \pc{12} pro Kilometer Höhenzunahme. Gehe
    für folgende Rechnungen davon aus, dass auf Meereshöhe ein
    Luftdruck von \SI{1000}{\hecto\pascal} gemessen worden ist.
    \begin{enumerate}[a)]
      \item Gib eine Funktion an, mit der man für die Höhe $h$
            (in \si{\kilo\metre}) über dem Meeresspiegel den
            Luftdruck $p$ (\si{\hecto\pascal}) berechnen kann.
      \item Wie groß war zum Zeitpunkt der Messung der Luftdruck
            in \SI{4500}{\metre} Höhe über dem Meeresspiegel?
      \item Um wie viel Prozent hat der Druck in \SI{4500}{\metre}
            Höhe gegenüber dem Wert auf Meereshöhe abgenommen?
      \item In welcher Höhe würde ein Wetterballon einen Luftdruck
            von \SI{400}{\hecto\pascal} messen?
    \end{enumerate}
  \fi
  %\ifoutline\outline\par
  %\fi
  \ifoutcome\outcome\par
    \begin{enumerate}[a)]
      \item Mit folgender Funktion kann man den Luftdruck $p$
            (in \si{\hecto\pascal}) bei bekannter Höhe $h$
            (in \si{\kilo\metre}) über dem Meeresspiegel
            berechnen:
            \begin{equation*}
              p(h)=p_{0}\cdot\num{0.88}^{h}
            \end{equation*}
      \item Zum Zeitpunkt der Messung betrug der Luftdruck in
            \SI{4500}{\metre} Höhe über dem Meeresspiegel etwa
            \SI{562.564}{\hecto\pascal}
      \item Insgesamt hat der Druck um \pc{43.744} abgenommen.
      \item Einen Luftdruck von \SI{400}{\hecto\pascal} würde man
            in \SI{7167.852}{\metre} Höhe messen.
    \end{enumerate}
  \fi
\end{exercise}

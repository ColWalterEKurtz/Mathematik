\begin{exercise}
      {ID-ffe9c9a343dc413371412600fd10d32c14437337}
      {Panflöte}
  \ifproblem\problem
    Die Rohre einer Panflöte werden kürzer, je höher der
    erklingende Ton sein soll. Bei einer chromatischen
    Panflöte ist ein Rohr immer um \pc{5.6} kürzer als
    das vorhergehende. Das längste Rohr hat für den Ton
    C eine Länge von \sicm{32.58} Welche Funktion
    beschreibt die Länge der Rohre in Abhängigkeit ihrer
    Position in der Panflöte?
    \begin{alignat*}{2}
      \text{a)}\quad f(x)&=\num{32.58}\cdot\num{0.944}^{x} & \qquad
      \text{c)}\quad f(x)&=\num{32.58}\cdot\num{1.056}^{x} \\
      \text{b)}\quad f(x)&=\num{32.58}-\num{5.6}^{x}       & \qquad
      \text{d)}\quad f(x)&=\num{32.58}\cdot\num{0.56}^{x}
    \end{alignat*}
  \fi
  %\ifoutline\outline
  %\fi
  \ifoutcome\outcome
    Zu den Angaben der Aufgabenstellung passt nur die
    Funktionsgleichung a), also:
    \begin{equation*}
      f(x)=\num{32.58}\cdot\num{0.944}^{x}
    \end{equation*}
  \fi
\end{exercise}

\begin{exercise}
      {ID-c63accbddff1f931eb42a7f318f772cae8aceafa}
      {Pkw}
  \ifproblem\problem\par
    Ein Pkw verliert pro Jahr (etwa) \pc{20} seines Wertes.
    \begin{enumerate}[a)]
      \item Stelle die Preiszerfallsfunktion für $t$ in Jahren auf.
      \item Wann hat er nur noch die Hälfte seines Wertes?
      \item Zu welchem Zinssatz müsste ein Kapital angelegt werden,
            das sich in der gleichen Zeit verdoppelt?
    \end{enumerate}
  \fi
  %\ifoutline\outline\par
  %\fi
  \ifoutcome\outcome\par
    \begin{enumerate}[a)]
      \item Mit einem Anschaffungswert von $W_{0}$ und $t$ in Jahren
            könnte die Preiszerfallsfunktion lauten:
            \begin{equation*}
              f(t)=W_{0}\cdot\num{0.8}^{t}
            \end{equation*}
      \item Nach etwa \num{3.106} Jahren ist der Pkw nur
            noch die Hälfte wert.
      \item Damit sich ein Kapital in der gleichen Zeit verdoppelt,
            müsste es zu einem Zinssatz von \pc{25} angelegt werden.
    \end{enumerate}
  \fi
\end{exercise}

\begin{exercise}
      {ID-6cd0ee95cc08abe67b087f71d716ca770b91afed}
      {Caesium}
  \ifproblem\problem
    Von \isotope[137]{Cs} zerfallen innerhalb eines
    Jahres etwa \pc{2.3} seiner Masse.
    \begin{enumerate}[a)]
      \item Stelle die Zerfallsfunktion auf.
      \item Wie viel Prozent des beim Reaktorunfall in
            Tschernobyl \num{1986} ausgetretenen \isotope[137]{Cs}
            sind im Jahr \num{2016} noch vorhanden?
      \item Bestimme die Halbwertszeit von \isotope[137]{Cs}.
    \end{enumerate}
  \fi
  %\ifoutline\outline
  %\fi
  \ifoutcome\outcome
    \begin{enumerate}[a)]
      \item Die Zerfallsfunktion lautet: $W_{n}=W_{0}\cdot\num{0.977}^{n}$
      \item Im Jahr \num{2016} sind noch etwa \pc{49.755} des
            \num{1986} ausgetretenen Caesiums vorhanden.
      \item Die Halbwertszeit von \isotope[137]{Cs} beträgt etwa
            \num{29.789} Jahre.
    \end{enumerate}
  \fi
\end{exercise}

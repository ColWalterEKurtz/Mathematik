\allowdisplaybreaks
\setcounter{chapter}{-1}
% ------------------------
\chapter{Unsortiert\ldots}
% ------------------------

% --------------------
\section{und ungelöst}
% --------------------

% 2021-11-24
% Grundrechenarten
\begin{exercise}
      {ID-53238ed0ed2da870b432a5f3b735d4752b333171}
      {Sportzeug}
  \ifproblem\problem\par
    % <PROBLEM>
    Als \xyc{} in der Schule ankommt, bemerkt er,
    dass er sein Sportzeug vergessen hat. Schnell
    geht er zu Fuß nach Hause und fährt mit dem
    Fahrrad zur Schule zurück. Er ist nach
    \num{25} Minuten wieder an der Schule.
    Wäre er beide Strecken mit dem Fahrrad gefahren,
    hätte er \num{5} Minuten weniger gebraucht.
    \begin{enumerate}[a)]
      \item Wie viele Minuten braucht er, wenn er
            beide Strecken mit dem Fahrrad fährt?
      \item Wie lange braucht er für eine Strecke
            mit dem Fahrrad?
      \item Wie lange braucht er, wenn er beide
            Strecken zu Fuß geht?
    \end{enumerate}
    % </PROBLEM>
  \fi
  %\ifoutline\outline\par
    % <OUTLINE>
    % </OUTLINE>
  %\fi
  %\ifoutcome\outcome\par
    % <OUTCOME>
    % </OUTCOME>
  %\fi
\end{exercise}

% 2021-11-24
% Grundrechenarten; Multiplikation
\begin{exercise}
      {ID-83fc52e3955aeb68d661c8ef4e4e2d5fb209c4ec}
      {Kantenmodell}
  \ifproblem\problem\par
    % <PROBLEM>
    Aus einem Draht von einem Meter Länge wurde das
    Kantenmodell eines Würfels gebaut. Es blieb ein
    Reststück von \sicm{4}. Gib die Länge einer
    Würfelkante an.
    % </PROBLEM>
  \fi
  %\ifoutline\outline\par
    % <OUTLINE>
    % </OUTLINE>
  %\fi
  %\ifoutcome\outcome\par
    % <OUTCOME>
    % </OUTCOME>
  %\fi
\end{exercise}

% 2021-11-24
% Muster und Figuren
\begin{exercise}
      {ID-9e4453973eae67793ec7ed646e5d43e99ae40e82}
      {Rechtecke}
  \ifproblem\problem\par
    % <PROBLEM>
    Du hast aus einem Blatt Papier viele kleine
    Rechtecke ausgeschnitten. Bei allen Rechtecken
    ist die längere Seite \sicm{3} und die kürzere
    \sicm{2} lang. Aus diesen Rechtecken sollen
    jetzt neue Figuren gelegt werden.
    \begin{enumerate}[a)]
      \item Finde alle unterschiedlichen Rechtecke,
            die man legen kann, wenn man genau fünf
            dieser Rechtecke benutzt. Zeichne deine
            Lösungen.
      \item Wie viele dieser kleinen Rechtecke
            werden mindestens benötigt, um ein
            Quadrat zu legen? Zeichne deine Lösung.
      \item Wie viele dieser Rechtecke werden
            benötigt, um das nächstgrößere Quadrat
            zu legen? Gib die Seitenlänge dieses
            Quadrates an.
    \end{enumerate}
    % </PROBLEM>
  \fi
  %\ifoutline\outline\par
    % <OUTLINE>
    % </OUTLINE>
  %\fi
  %\ifoutcome\outcome\par
    % <OUTCOME>
    % </OUTCOME>
  %\fi
\end{exercise}

% 2021-11-24
\begin{exercise}
      {ID-b31de47a79bb87a552ec9a0926e2b3b4731576d3}
      {Sudoku, oder so\ldots}
  \ifproblem\problem\par
    % <PROBLEM>
    \xxa{} malt ein Quadrat mit $4\times4$ Feldern
    auf. In die Felder dieses Quadrats trägt sie
    Zahlen von 1 bis 4 ein, und zwar so, dass in
    jeder Zeile, in jeder Spalte und in den beiden
    Diagonalen diese Zahlen jeweils genau einmal
    stehen.
    \begin{enumerate}[a)]
      \item Gib ein Beispiel für eine solche
            Verteilung dieser Zahlen an.
      \item \xxa{} gelingt es auch, ein
            $5\times5$-Quadrat nach den gleichen
            Regeln mit den Zahlen von 1 bis 5 zu
            füllen. Gib ebenfalls eine solche
            Verteilung dieser Zahlen an.
      \item Warum kann es \xxa{} nicht gelingen,
            entsprechend ein $3\times3$-Quadrat
            mit Zahlen von 1 bis 3 zu füllen?
    \end{enumerate}
    % </PROBLEM>
  \fi
  %\ifoutline\outline\par
    % <OUTLINE>
    % </OUTLINE>
  %\fi
  %\ifoutcome\outcome\par
    % <OUTCOME>
    % </OUTCOME>
  %\fi
\end{exercise}

% 2021-11-24
\begin{exercise}
      {ID-871b577e0293c52fe0b70234ba09d9541cb3ac17}
      {Kartoffeln}
  \ifproblem\problem\par
    % <PROBLEM>
    Nach langer Dürre wurden Kartoffeln um
    \pc{20} teurer. Etwas später wurde der
    Preis für Kartoffeln wieder um \pc{20}
    gesenkt. Ermittle, ob die Kartoffeln vor
    der Preiserhöhung oder nach der Preissenkung
    billiger waren. Wie viel Prozent beträgt
    der Preisunterschied bezogen auf den
    Preis vor der Dürre?
    % </PROBLEM>
  \fi
  %\ifoutline\outline\par
    % <OUTLINE>
    % </OUTLINE>
  %\fi
  %\ifoutcome\outcome\par
    % <OUTCOME>
    % </OUTCOME>
  %\fi
\end{exercise}

% 2021-11-24
\begin{exercise}
      {ID-cb0d7514e9b7d4604ad80eec2924957b404b15fb}
      {Mirpzahlen}
  \ifproblem\problem\par
    % <PROBLEM>
    In der letzten Mathematikarbeit konnten maximal
    \num{40} Punkte erreicht werden. \xxa{} ist sehr
    an ihrem Ergebnis interessiert und fragt ihren
    Lehrer.  Dieser antwortet:
    \begin{enumerate}[(1)]
      \item \glqq Ein Teiler deiner Gesamtpunktzahl
            ist eine Mirpzahl.\grqq
      \item \glqq Wenn du die Quersumme deiner
            Gesamtpunktzahl verdoppelst und 7
            addierst, erhältst du auch eine
            Mirpzahl.\grqq
    \end{enumerate}
    Zeige, dass aus diesen Angaben \xxa{}
    Gesamtpunktzahl eindeutig bestimmt werden kann,
    und gib diese Gesamtpunktzahl an.\par
    \emph{Hinweis:} Eine \emph{Mirpzahl} ist eine
    Primzahl, die eine andere Primzahl ergibt, wenn
    man die Ziffern von rechts nach links liest.
    Folglich ist 13 die erste Mirpzahl.
    % </PROBLEM>
  \fi
  %\ifoutline\outline\par
    % <OUTLINE>
    % </OUTLINE>
  %\fi
  %\ifoutcome\outcome\par
    % <OUTCOME>
    % </OUTCOME>
  %\fi
\end{exercise}

% 2021-11-24
\begin{exercise}
      {ID-123414deaedc4cd12e00fff1fbe827ae1e3b0090}
      {Folge von Dreiecken}
  \ifproblem\problem\par
    % <PROBLEM>
    Die Figuren der abgebildteten Folge sind aus
    kleinen gleichseitigen Dreiecken mit einer
    Seitenlänge von einem Zentimeter so
    zusammengesetzt, dass die Umrandung der $n$-ten
    Figur ein gleichseitiges Dreieck mit einer
    Seitenlänge von $n$ Zentimetern ist. Mit
    $s_{n}$ wird die Anzahl aller kleinen Dreiecke
    in der $n$-ten Figur bezeichnet. Die Abbildung
    zeigt die erste, zweite, dritte und vierte Figur.
    \begin{center}
      \begin{tikzpicture}
        \newcommand{\dreieck}[1]{ \draw (#1) -- ([shift={(0:1cm)}]#1) -- ([shift={(60:1cm)}]#1) -- cycle; }
        \begin{scope}
          \dreieck{0, 0}
        \end{scope}
        \begin{scope}[xshift=2cm]
          \dreieck{0, 0}
          \dreieck{1, 0}
          \dreieck{60:1}
        \end{scope}
        \begin{scope}[xshift=5cm]
          \dreieck{0, 0}
          \dreieck{1, 0}
          \dreieck{2, 0}
          \begin{scope}[shift={(60:1)}]
            \dreieck{0, 0}
            \dreieck{1, 0}
            \dreieck{60:1}
          \end{scope}
        \end{scope}
        \begin{scope}[xshift=9cm]
          \dreieck{0, 0}
          \dreieck{1, 0}
          \dreieck{2, 0}
          \dreieck{3, 0}
          \begin{scope}[shift={(60:1)}]
            \dreieck{0, 0}
            \dreieck{1, 0}
            \dreieck{2, 0}
            \begin{scope}[shift={(60:1)}]
              \dreieck{0, 0}
              \dreieck{1, 0}
              \dreieck{60:1}
            \end{scope}
          \end{scope}
        \end{scope}
      \end{tikzpicture}
    \end{center}
    \begin{enumerate}[a)]
      \item Gib die Zahlen $s_{1}$, $s_{2}$, $s_{3}$
            und $s_{4}$ an.
      \item Notiere eine Formel für die Berechnung
            von $s_{n}$. Überprüfe deine Formel für
            $n=6$ und für $n=8$.
      \item Beweise deine Formel.
    \end{enumerate}
    % </PROBLEM>
  \fi
  %\ifoutline\outline\par
    % <OUTLINE>
    % </OUTLINE>
  %\fi
  %\ifoutcome\outcome\par
    % <OUTCOME>
    % </OUTCOME>
  %\fi
\end{exercise}

% 2021-11-24
\begin{exercise}
      {ID-362f85a968f925931c7f9ca6a2ae5aa9a6d0671a}
      {Kugelpyramide}
  \ifproblem\problem\par
    % <PROBLEM>
    \xxb{} Spielkiste enthält viele gleich große,
    einfarbige Kugeln in fünf verschiedenen Farben.
    Sie entnimmt der Kiste vier Kugeln und baut
    daraus auf einem Drehteller eine Kugelpyramide.
    Eine solche Pyramide besteht aus drei unteren
    Kugeln, welche auf dem Drehteller liegen und
    einander paarweise berühren, sowie einer oberen
    Kugel, welche die drei unteren berührt. Zwei
    Kugelpyramiden gelten als verschieden, wenn
    sie nicht durch Drehen des Tellers ineinander
    überführt werden können.\par
    Ermittle, wie viele verschiedene Kugelpyramiden
    sich errichten lassen.
    % </PROBLEM>
  \fi
  %\ifoutline\outline\par
    % <OUTLINE>
    % </OUTLINE>
  %\fi
  %\ifoutcome\outcome\par
    % <OUTCOME>
    % </OUTCOME>
  %\fi
\end{exercise}

% 2021-11-24
\begin{exercise}
      {ID-6527cda62988e2d85910e4c307f051d0b7a9bd07}
      {Parallelogramm}
  \ifproblem\problem\par
    % <PROBLEM>
    Über den vier Seiten eines nicht rechtwinkligen
    Parallelogramms werden Quadrate konstruiert.
    Beweise, dass die Diagonalenschnittpunkte dieser
    vier Quadrate Eckpunkte eines Quadrates sind.
    % </PROBLEM>
  \fi
  %\ifoutline\outline\par
    % <OUTLINE>
    % </OUTLINE>
  %\fi
  %\ifoutcome\outcome\par
    % <OUTCOME>
    % </OUTCOME>
  %\fi
\end{exercise}

% 2021-11-24
\begin{exercise}
      {ID-cc2e058d13d51c4d2d86b2068a40d8a453800b95}
      {Galileo Galilei}
  \ifproblem\problem\par
    % <PROBLEM>
    Der berühmte italienische Physiker, Astronom
    und Mathematiker Galileo Galilei (1564--1642)
    hat die folgende bemerkenswerte, unendlich lange
    Gleichungskette entdeckt:
    \begin{equation*}
      \frac{1}{3}
      =\frac{1+3}{5+7}
      =\frac{1+3+5}{7+9+11}
      =\frac{1+3+5+7}{9+11+13+15}
      =\dotsb
    \end{equation*}
    Beweise die Korrektheit dieser Gleichungskette.
    % </PROBLEM>
  \fi
  %\ifoutline\outline\par
    % <OUTLINE>
    % </OUTLINE>
  %\fi
  %\ifoutcome\outcome\par
    % <OUTCOME>
    % </OUTCOME>
  %\fi
\end{exercise}

% 2021-11-24
\begin{exercise}
      {ID-7770a25679ccf8de41efadea68d403f4087edb6a}
      {Quadrat und Kreis}
  \ifproblem\problem\par
    % <PROBLEM>
    Ein Draht der Länge $L$ wird in zwei
    Stücke der Längen $l_{1}=x$ und $l_{2}=L-x$
    zerschnitten. Aus dem Teilstück der Länge
    $l_{1}$ wird ein Quadrat gebogen, aus dem
    Teilstück der Länge $l_{2}$ ein Kreis.
    \begin{enumerate}[a)]
      \item Berechne die Summe der Flächeninhalte
            von Quadrat und Kreis in Abhängigkeit
            von $x$ und $L$.
      \item Zeigen: Die Summe der Flächeninhalte
            von Quadrat und Kreis wird am kleinsten,
            wenn man den Schnitt so durchführt, dass
            die Seitenlänge des Quadrats gleich dem
            Durchmesser des Kreises ist.
    \end{enumerate}
    % </PROBLEM>
  \fi
  %\ifoutline\outline\par
    % <OUTLINE>
    % </OUTLINE>
  %\fi
  %\ifoutcome\outcome\par
    % <OUTCOME>
    % </OUTCOME>
  %\fi
\end{exercise}

% 2021-11-24
\begin{exercise}
      {ID-fcd1bd8743639525e926d297e041d177cd5c7a06}
      {Ortskurve}
  \ifproblem\problem\par
    % <PROBLEM>
    Es sei $k$ ein Halbkreis über einem Durchmesser
    $\overline{AB}$. Auf $k$ werde ein weiterer von
    $A$ und $B$ verschiedener Punkt $C$ gewählt. Wir
    betrachten den aus den Strecken $\overline{AC}$
    und $\overline{CB}$ bestehenden Streckenzug und
    bezeichnen mit $X$ denjenigen Punkt, welcher diesen
    Streckenzug in zwei Teile gleicher Länge zerlegt.
    \par
    Bestimme den geometrischen Ort der Punkte $X$,
    wenn $C$ die von $A$ und $B$ verschiedenen
    Punkte auf dem Halbkreis $k$ durchläuft.
    % </PROBLEM>
  \fi
  %\ifoutline\outline\par
    % <OUTLINE>
    % </OUTLINE>
  %\fi
  %\ifoutcome\outcome\par
    % <OUTCOME>
    % </OUTCOME>
  %\fi
\end{exercise}

% 2021-11-24
\begin{exercise}
      {ID-e411812454910f9994f1bcab31f023f8f5c73c19}
      {Die Mühen der Ebenen}
  \ifproblem\problem\par
    % <PROBLEM>
    Im Raum liegen \num{53} verschiedene Geraden,
    die mindestens \num{1337} verschiedene
    Schnittpunkte haben. Zeige, dass die Geraden
    alle in einer gemeinsamen Ebene liegen.
    % </PROBLEM>
  \fi
  %\ifoutline\outline\par
    % <OUTLINE>
    % </OUTLINE>
  %\fi
  %\ifoutcome\outcome\par
    % <OUTCOME>
    % </OUTCOME>
  %\fi
\end{exercise}

% 2021-11-24
\begin{exercise}
      {ID-0b0d5e6ffbbf65f05804d0ff179f3d6b8f85dccd}
      {Gleichungssystem}
  \ifproblem\problem\par
    % <PROBLEM>
    Bestimme alle reellen Lösungen $(x,y,z)$ des
    Gleichungssystems
    \begin{equation*}
      x-\frac{1}{y}
      =y-\frac{1}{z}
      =z-\frac{1}{x}
    \end{equation*}
    % </PROBLEM>
  \fi
  %\ifoutline\outline\par
    % <OUTLINE>
    % </OUTLINE>
  %\fi
  %\ifoutcome\outcome\par
    % <OUTCOME>
    % </OUTCOME>
  %\fi
\end{exercise}

% 2021-11-24
\begin{exercise}
      {ID-21ee36e4af9ce229343ef0f44bddfaed0bb4d6e3}
      {Schafstall}
  \ifproblem\problem\par
    % <PROBLEM>
    Ein Schafstall besitzt einen quadratischen
    Grundriss mit \simeter{10} Seitenlänge. Auf
    die Ostwand des Stalls stößt senkrecht und
    mittig ein Zaun, durch den das Umkreisen
    des Stalls unmöglich wird. Ein Schaf grast
    südlich dieses Zauns. Es ist mit einem
    \simeter{25} langen (dünnen) Strick genau
    an der Stelle angebunden, an der der Zaun
    auf die Wand stößt.
    \begin{center}
      \begin{tikzpicture}
        % Stall
        \draw (0, 0) rectangle node{{\small Stall}} (2, 2);
        % Zaun
        \begin{scope}
          \clip (20mm, 9mm) rectangle (120mm, 11mm);
          \foreach \x in {15,...,125}
          {
            \begin{scope}[xshift=\x mm, yshift=1cm]
              \draw (-1mm, -1mm) -- (1mm,  1mm);
              \draw (-1mm,  1mm) -- (1mm, -1mm);
            \end{scope}
          }
        \end{scope}
        % Schaf
        \node[above left] at (120mm, 11mm) {{\small Zaun}};
        % Pfosten
        \fill (2, 1) circle (1pt);
        % Seil
        \draw[decorate, decoration=snake] (2, 1) .. controls +(300:30mm) and +(180:5mm) .. (6, 0);
        % Schaf
        \node[right, shape=ellipse, draw=black] at (6, 0) {{\footnotesize Schaf}};
      \end{tikzpicture}
    \end{center}
    Das Schaf frisst sämtliches Gras, das in der
    durch Strick, Zaun und Stall begrenzten
    Reichweite liegt.
    \begin{enumerate}[a)]
      \item Erstelle eine Ergänzung der obigen Skizze
            mit der Fläche, die das Schaf abgrasen
            kann.
      \item Berechne den Inhalt der abgrasbaren
            Fläche.
      \item Der Zaun wird jetzt beseitigt. Dadurch
            kann das Schaf in beiden Richtungen um
            den Stall laufen. Berechne auch für
            diesen Fall die Größe der abgrasbaren
            Fläche.
    \end{enumerate}
    % </PROBLEM>
  \fi
  %\ifoutline\outline\par
    % <OUTLINE>
    % </OUTLINE>
  %\fi
  %\ifoutcome\outcome\par
    % <OUTCOME>
    % </OUTCOME>
  %\fi
\end{exercise}

% 2021-11-24
\begin{exercise}
      {ID-e09babea9a98ee485a897ff297d52477515b665b}
      {Trigonometrie}
  \ifproblem\problem\par
    % <PROBLEM>
    Die Seite $\overline{AB}$ eines Rechtecks
    $ABCD$ ist gleichzeitig Durchmesser eines
    Kreises $k$ mit dem Mittelpunkt $M$. Der
    Kreis $k$ schneidet die Seite $\overline{CD}$
    in den Punkten $E$ und $F$ ($E$ liege dabei
    zwischen $D$ und $F$).
    \begin{enumerate}[a)]
      \item Die Strecken $\overline{MF}$ und
            $\overline{BE}$ schneiden sich in Punkt
            $S$. Drücke die Winkelgröße
            $\delta=\left|\sphericalangle FSB\right|$
            allgemein und vollständig durch die
            Winkelgröße
            $\varepsilon=\left|\sphericalangle EMF\right|$
            aus.
      \item Beweise die Behauptung: Wenn für das
            Seitenverhältnis
            $|\overline{AB}|:|\overline{AD}|=3:\sqrt{2}$
            gilt, dann teilen $E$ und $F$ die Strecke
            $\overline{CD}$ in drei gleiche Teile.
    \end{enumerate}
    \begin{center}
      \begin{tikzpicture}[scale=1.5]
        \newcommand{\dotsize}{0.75pt}
        \coordinate (A) at (0, 0);
        \coordinate (B) at (5, 0);
        \coordinate (C) at (5, 2);
        \coordinate (D) at (0, 2);
        \coordinate (M) at ($(A)!0.5!(B)$);
        \begin{scope}
          \clip (-5mm, -5mm) rectangle (55mm, 26mm);
          \path[name path global=s] (C) -- (D);
          \path[name path global=k] (M) circle (2.5);
        \end{scope}
        \path[name intersections={of=s and k, by={F,E}}];
        \coordinate (S) at (intersection of M--F and B--E);
        % Punkte
        \fill (A) circle (\dotsize);
        \fill (B) circle (\dotsize);
        \fill (C) circle (\dotsize);
        \fill (D) circle (\dotsize);
        \fill (M) circle (\dotsize);
        \fill (E) circle (\dotsize);
        \fill (F) circle (\dotsize);
        \fill (S) circle (\dotsize);
        % Rechteck
        \draw (A) -- (B) -- (C) -- (D) -- cycle;
        % Kreis
        \begin{scope}
          \clip (-5mm, -5mm) rectangle (55mm, 26mm);
          \draw (M) circle (2.5);
        \end{scope}
        % Strecken
        \draw (M) -- (E);
        \draw (M) -- (F);
        \draw (B) -- (E);
        % Winkel epsilon
        \begin{scope}
          \clip (E) -- (M) -- (F) -- cycle;
          \draw (M) circle (5mm);
        \end{scope}
        % Winkel delta
        \begin{scope}
          \clip (F) -- (S) -- (B) -- cycle;
          \draw (S) circle (5mm);
        \end{scope}
        % Beschriftung
        \node[below left]  at (A) {{\small$A$}};
        \node[below]       at (M) {{\small$M$}};
        \node[below right] at (B) {{\small$B$}};
        \node[above right] at (C) {{\small$C$}};
        \node[above left]  at (D) {{\small$D$}};
        \node[above]       at (E) {{\small$E$}};
        \node[above]       at (F) {{\small$F$}};
        \node at ([shift={(95:3mm)}]S) {{\small$S$}};
        \node at ([shift={(90:3mm)}]M) {{\small$\varepsilon$}};
        \node at ([shift={(15:3mm)}]S) {{\small$\delta$}};
      \end{tikzpicture}
    \end{center}
    % </PROBLEM>
  \fi
  %\ifoutline\outline\par
    % <OUTLINE>
    % </OUTLINE>
  %\fi
  %\ifoutcome\outcome\par
    % <OUTCOME>
    % </OUTCOME>
  %\fi
\end{exercise}

% 2021-11-25
\begin{exercise}
      {ID-33d80df6f0f6ef69346f01359b2ad56c086690af}
      {Zufällige Dreiecke}
  \ifproblem\problem\par
    % <PROBLEM>
    Es sollen Dreiecke mit zufällig ausgewählten
    Seitenlängen konstruiert werden. Mit einem
    Spielwürfel werden die Seitenlängen ermittelt,
    wobei die jeweils geworfene Augenzahl die
    Länge einer Seite in \si{\centi\metre} angibt.
    \begin{enumerate}[a)]
      \item Ermittle die Wahrscheinlichkeit dafür,
            dass aus drei nacheinander gewürfelten
            Zahlen $a$, $b$ und $c$ ein Dreieck mit
            den Seitenlängen
            $a$\,\si{\centi\metre},
            $b$\,\si{\centi\metre} und
            $c$\,\si{\centi\metre}
            konstruiert werden kann.
      \item Ermittle die Wahrscheinlichkeit dafür,
            dass bei sechs Würfen aus den ersten
            drei Würfen ein Dreieck und aus den
            drei weiteren Würfen ein dazu ähnliches
            Dreieck erzeugt werden kann.
    \end{enumerate}
    % </PROBLEM>
  \fi
  %\ifoutline\outline\par
    % <OUTLINE>
    % </OUTLINE>
  %\fi
  %\ifoutcome\outcome\par
    % <OUTCOME>
    % </OUTCOME>
  %\fi
\end{exercise}

% 2021-11-25
\begin{exercise}
      {ID-37c6785ef9af9d54af4b7b530f5c39892e88ef83}
      {Würfeln}
  \ifproblem\problem\par
    % <PROBLEM>
    \xya{} und \xyb{} spielen ein Würfelspiel.
    Dabei würfelt \xya{} mit einem mit den Zahlen
    von 1 bis 20 beschrifteten Ikosaeder.
    \xyb{} dagegen würfelt mit einem mit den Zahlen
    von 1 bis 12 beschrifteten Dodekaeder.
    \par
    Beide würfeln abwechselnd jeweils vier Mal.
    \xya{} gewinnt, wenn er bei mindestens drei
    der vier aufeinander folgenden Würfen eine
    höhere Augenzahl erzielt als \xyb. Andernfalls
    gewinnt \xyb.
    \par
    Wer von beiden hat die größere Gewinnchance?
    % </PROBLEM>
  \fi
  %\ifoutline\outline\par
    % <OUTLINE>
    % </OUTLINE>
  %\fi
  %\ifoutcome\outcome\par
    % <OUTCOME>
    % </OUTCOME>
  %\fi
\end{exercise}

% 2021-11-25
\begin{exercise}
      {ID-92e92c92ece38256ec20011d3718f8f038d251df}
      {Minimum}
  \ifproblem\problem\par
    % <PROBLEM>
    Aus den Ziffern 1 bis 9 werden drei dreistellige
    Zahlen gebildet, wobei jede Ziffer genau ein Mal
    verwendet wird. Ermittle den kleinsten Wert,
    den das Produkt der drei dreistelligen Zahlen
    annehmen kann.
    % </PROBLEM>
  \fi
  %\ifoutline\outline\par
    % <OUTLINE>
    % </OUTLINE>
  %\fi
  %\ifoutcome\outcome\par
    % <OUTCOME>
    % </OUTCOME>
  %\fi
\end{exercise}

% 2021-11-25
\begin{exercise}
      {ID-ce33775bf8185ca38182ffa1155d5eb090babaa5}
      {Quadratzahl}
  \ifproblem\problem\par
    % <PROBLEM>
    Ermittle alle positiven ganzen Zahlen $n$, für die
    $n^{2}+2^{n}$ Quadratzahl ist.
    % </PROBLEM>
  \fi
  %\ifoutline\outline\par
    % <OUTLINE>
    % </OUTLINE>
  %\fi
  %\ifoutcome\outcome\par
    % <OUTCOME>
    % </OUTCOME>
  %\fi
\end{exercise}

% 2021-11-26
\begin{exercise}
      {ID-5bc95f3ab3e83d89590848ac4b7e25854bd6d229}
      {Raucher}
  \ifproblem\problem\par
    % <PROBLEM>
    An einem Berufskolleg wurden alle \num{674}
    Schülerinnen und Schüler befragt, ob sie
    rauchen oder nicht. Das Ergebnis der Befragung
    sieht wie folgt aus: \num{82} der insgesamt
    \num{293} männlichen Schüler gaben an zu
    rauchen. \num{250} Schülerinnen gaben an,
    nicht zu rauchen.
    \begin{enumerate}[a)]
      \item Stelle den Sachzusammenhang in einer
            Vierfeldertafel dar.
      \item Mit welcher Wahrscheinlichkeit ist eine
            zufällig ausgewählte Person weiblich
            und Nichtraucherin?
      \item Mit welcher Wahrscheinlichkeit ist eine
            Schülerin Nichtraucherin?
      \item Untersuche ob in diesem Fall die
            Merkmale \emph{Geschlecht} und
            \emph{Raucher} unabhängig voneinander
            sind.
    \end{enumerate}
    % </PROBLEM>
  \fi
  \ifoutline\outline
    % <OUTLINE>
    \begin{enumerate}[a)]
      \item Eine passende Vierfeldertafel könnte man z.\,B. so beginnen:
            \begin{center}
              \begin{fourfoldtable}
                \Apos{m}%
                \Aneg{w}%
                \Bpos{r}%
                \numbers
                {82}{}{}
                {}{250}{}
                {293}{}{674}
                \Bneg{$\overline{\text{r}}$}%
              \end{fourfoldtable}
            \end{center}
      \item Gesucht ist: $P(\text{w}\cap\overline{\text{r}})$
      \item Gesucht ist: $P_{\text{w}}(\overline{\text{r}})$
      \item Untersuche ob in diesem Fall die Merkmale \emph{Geschlecht}
            und \emph{Raucher} unabhängig voneinander sind.
    \end{enumerate}
    % </OUTLINE>
  \fi
  %\ifoutcome\outcome\par
    % <OUTCOME>
    % </OUTCOME>
  %\fi
\end{exercise}


% 2021-11-26
\begin{exercise}
      {ID-a1db7985ac42282ba2776702ad9757ce6daa4340}
      {Maximales Rechteck im gleichseitigen Dreieck}
  \ifproblem\problem\par
    % <PROBLEM>
    \ifthenelse{\isundefined{\linecalc}}{\newlength{\linecalc}}{\relax}%
    \setlength{\linecalc}{\linewidth}%
    \addtolength{\linecalc}{-30mm}%
    \begin{minipage}[b]{\linecalc}
      Einem gleichseitigen Dreieck mit der Seitenlänge
      $a$ soll ein Rechteck einbeschrieben werden. Wie
      lang müssen die Rechteckseiten sein, damit der
      Flächeninhalt des Rechtecks maximal wird?
    \end{minipage}\hfill
    \begin{minipage}[b]{25mm}
      \raggedleft
      \raisebox{0\baselineskip}[0\baselineskip][0pt]{%
      \begin{tikzpicture}[scale=0.8]
        \draw (-1.000,  0.000) -- ( 1.000,  0.000) -- ( 0.000,  1.732) -- cycle;
        \filldraw[fill=black!25!white] (-0.500,  0.000) rectangle ( 0.500,  0.866);
      \end{tikzpicture}}
    \end{minipage}%
    % </PROBLEM>
  \fi
  \ifoutline\outline\par
    % <OUTLINE>
    \ifthenelse{\isundefined{\linecalc}}{\newlength{\linecalc}}{\relax}%
    \setlength{\linecalc}{\linewidth}%
    \addtolength{\linecalc}{-50mm}%
    \begin{minipage}{40mm}
      \begin{tikzpicture}
        \draw (-1.000,  0.000) -- ( 1.000,  0.000) -- ( 0.000,  1.732) -- cycle;
        \filldraw[fill=black!25!white] (-0.500,  0.000) rectangle ( 0.500,  0.866);
        \draw[->, >=stealth] (-1.5, 0) -- (1.5, 0) node[below]{{\small$x$}};
        \draw[->, >=stealth] (0, -0.5) -- (0, 2.5) node[below left]{{\small$y$}};
        % a/2
        \draw (1, 0.1) -- (1, -0.1) node[below]{{\small$\displaystyle\frac{a}{2}$}};
        % sin(60) * a
        \draw (0.1, 1.732) -- (-0.1, 1.732) node[left]{{\small$a\cdot\sin60^\circ$}};
      \end{tikzpicture}
    \end{minipage}\hspace*{\fill}%
    \begin{minipage}{\linecalc}
      \begin{equation*}
        \begin{split}
          m&=-\frac{a\cdot\sin60^\circ}{\frac{a}{2}}=-2\sin60^\circ=-\sqrt{3} \\[2ex]
          g(x)&=-\sqrt{3}\cdot x+a\cdot\frac{\sqrt{3}}{2}
        \end{split}
      \end{equation*}
    \end{minipage}
    % </OUTLINE>
  \fi
  %\ifoutcome\outcome\par
    % <OUTCOME>
    % </OUTCOME>
  %\fi
\end{exercise}

% 2021-11-26
\begin{exercise}
      {ID-9e2a412c527329db74a0c31c203ca12d573d825e}
      {Bunte Gerade}
  \ifproblem\problem\par
    % <PROBLEM>
    Jeder Punkt einer Geraden ist entweder rot
    oder blau gefärbt. Zeige, dass es auf der
    Geraden drei gleichfarbige Punkte $A$,
    $B$ und $C$ gibt, für die $|AB|=|BC|$ gilt.
    % </PROBLEM>
  \fi
  %\ifoutline\outline\par
    % <OUTLINE>
    % </OUTLINE>
  %\fi
  %\ifoutcome\outcome\par
    % <OUTCOME>
    % </OUTCOME>
  %\fi
\end{exercise}

% 2021-11-26
\begin{exercise}
      {ID-7d1794bbe37500832e28192b4f8cf48e6d78ccf2}
      {Bunte Ebene}
  \ifproblem\problem\par
    % <PROBLEM>
    Jeder Punkt einer Ebene ist entweder rot oder
    blau gefärbt.
    \begin{enumerate}[a)]
      \item Zeige, dass es in der Ebene ein
            gleichseitiges Dreieck gibt, dessen
            Eckpunkte alle dieselbe Farbe besitzen.
      \item Zeige, dass es in der Ebene ein
            Rechteck gibt, dessen Eckpunkte alle
            dieselbe Farbe besitzen.
    \end{enumerate}
    % </PROBLEM>
  \fi
  %\ifoutline\outline\par
    % <OUTLINE>
    % </OUTLINE>
  %\fi
  %\ifoutcome\outcome\par
    % <OUTCOME>
    % </OUTCOME>
  %\fi
\end{exercise}

% 2021-11-16
\begin{exercise}
      {ID-bb27f471c7599811aa29472b06f4e88fda0268a9}
      {Rauchmelder}
  \ifproblem\problem\par
    % <PROBLEM>
    In einer Schule wurden neue Rauchmelder
    installiert.
    Zwei kritische Ereignisse von besonderem
    Interesse werden durch folgende Abkürzungen
    bezeichnet:
    \begin{itemize}
      \renewcommand{\itemsep}{-1ex}%
      \item[$R$:]\glqq es hat sich Rauch gebildet\grqq
      \item[$S$:]\glqq es ertönt ein Signal\grqq
    \end{itemize}
    \begin{enumerate}[a)]
      %\setlength{\itemsep}{-1ex}%
      %\setcounter{enumi}{0}%
      \item Beschreiben Sie in eigenen Worten, was die
            bedingten Wahrscheinlichkeiten
            \begin{equation*}
              P_R(S)
              \;,\;
              P_R(\overline{S})
              \;\;\text{und}\;\;
              P_{\overline{R}}(S)
            \end{equation*}
            bedeuten.
      \item Geben Sie außerdem an, ob die
            Werte der Wahrscheinlichkeiten aus
            a) eher groß oder eher klein sein
            sollten.
    \end{enumerate}
    % </PROBLEM>
  \fi
  %\ifoutline\outline\par
    % <OUTLINE>
    % </OUTLINE>
  %\fi
  %\ifoutcome\outcome\par
    % <OUTCOME>
    % </OUTCOME>
  %\fi
\end{exercise}

% 2021-11-19
\begin{exercise}
      {ID-ee0514b56291c14f49f41bac279c4d09af94beb1}
      {Tschernobyl}
  \ifproblem\problem\par
  %\ifproblem\clearpage\hrulefill\\
    % <PROBLEM>
    Bei der Nuklearkatastrophe von Tschernobyl
    im April 1986 wurden etwa \SI{500}{\gram}
    des radioaktiven Isotops \isotope[131]{I}
    (\glqq Jod 131\grqq) freigesetzt.
    Die Halbwertszeit von \isotope[131]{I}
    beträgt ca. \num{8} Tage.
    \begin{enumerate}[a)]
      \item Stelle die Funktionsgleichung auf, die
            den Jod-Zerfall in Abhängigkeit von der
            Zeit $t$ in Tagen beschreibt.
      \item Wie viel Gramm des ausgetretenen
            \isotope[131]{I} war \num{4} Wochen
            nach der Explosion noch in der Umwelt
            vorhanden?
    \end{enumerate}
    % </PROBLEM>
  \fi
  %\ifoutline\outline\par
    % <OUTLINE>
    % </OUTLINE>
  %\fi
  %\ifoutcome\outcome\par
    % <OUTCOME>
    % </OUTCOME>
  %\fi
\end{exercise}

% ------------------------------------------------------------------------------
\end{document}
% ------------------------------------------------------------------------------


\allowdisplaybreaks
\setcounter{chapter}{-1}
% ------------------------
\chapter{Unsortiert\ldots}
% ------------------------

% --------------------
\section{und ungelöst}
% --------------------

\input{\subdir gleichseitig.tex}

% ------------------------------------------------------------------------------
\end{document}
% ------------------------------------------------------------------------------

\begin{exercise}
      {ID-42df963f029694c68d8cf7fcfac225e8c53415e5}
      {Maximales Rechteck im gleichseitigen Dreieck}
  \ifproblem\problem
    \ifthenelse{\isundefined{\linecalc}}{\newlength{\linecalc}}{\relax}%
    \setlength{\linecalc}{\linewidth}%
    \addtolength{\linecalc}{-30mm}%
    \begin{minipage}[b]{\linecalc}
      Einem gleichseitigen Dreieck mit der Seitenlänge $a$ soll ein Rechteck
      einbeschrieben werden. Wie lang müssen die Rechteckseiten sein, damit der
      Flächeninhalt des Rechtecks maximal wird?
    \end{minipage}\hfill
    \begin{minipage}[b]{25mm}
      \raggedleft
      \raisebox{0\baselineskip}[0\baselineskip][0pt]{%
      \begin{tikzpicture}[scale=0.8]
        \draw (-1.000,  0.000) -- ( 1.000,  0.000) -- ( 0.000,  1.732) -- cycle;
        \filldraw[fill=black!25!white] (-0.500,  0.000) rectangle ( 0.500,  0.866);
      \end{tikzpicture}}
    \end{minipage}%
  \fi
  \ifoutline\outline
    \ifthenelse{\isundefined{\linecalc}}{\newlength{\linecalc}}{\relax}%
    \setlength{\linecalc}{\linewidth}%
    \addtolength{\linecalc}{-50mm}%
    \begin{minipage}{40mm}
      \begin{tikzpicture}
        \draw (-1.000,  0.000) -- ( 1.000,  0.000) -- ( 0.000,  1.732) -- cycle;
        \filldraw[fill=black!25!white] (-0.500,  0.000) rectangle ( 0.500,  0.866);
        \draw[->, >=stealth] (-1.5, 0) -- (1.5, 0) node[below]{{\small$x$}};
        \draw[->, >=stealth] (0, -0.5) -- (0, 2.5) node[below left]{{\small$y$}};
        % a/2
        \draw (1, 0.1) -- (1, -0.1) node[below]{{\small$\displaystyle\frac{a}{2}$}};
        % sin(60) * a
        \draw (0.1, 1.732) -- (-0.1, 1.732) node[left]{{\small$a\cdot\sin60^\circ$}};
      \end{tikzpicture}
    \end{minipage}\hspace*{\fill}%
    \begin{minipage}{\linecalc}
      \begin{equation*}
        \begin{split}
          m&=-\frac{a\cdot\sin60^\circ}{\frac{a}{2}}=-2\sin60^\circ=-\sqrt{3} \\[2ex]
          g(x)&=-\sqrt{3}\cdot x+a\cdot\frac{\sqrt{3}}{2}
        \end{split}
      \end{equation*}
    \end{minipage}
  \fi
  %\ifoutcome\outcome
  %\fi
\end{exercise}

\begin{exercise}
      {ID-4db817770162d679b90c8d3642abf57bd5d98029}
      {Bunte Gerade}
  \ifproblem\problem
    Jeder Punkt einer Geraden ist entweder rot oder blau gefärbt.
    Zeige, dass es auf der Geraden drei gleichfarbige Punkte $A$,
    $B$ und $C$ gibt, für die $|AB|=|BC|$ gilt.
  \fi
  %\ifoutline\outline
  %\fi
  %\ifoutcome\outcome
  %\fi
\end{exercise}

\begin{exercise}
      {ID-7f3b47c16105c2a1d982f70fde324c13ac99a72b}
      {Bunte Ebene}
  \ifproblem\problem
    Jeder Punkt einer Ebene ist entweder rot oder blau gefärbt.
    \begin{enumerate}[a)]
      \item Zeige, dass es in der Ebene ein gleichseitiges Dreieck gibt,
            dessen Eckpunkte alle dieselbe Farbe besitzen.
      \item Zeige, dass es in der Ebene ein Rechteck gibt,
            dessen Eckpunkte alle dieselbe Farbe besitzen.
    \end{enumerate}
  \fi
  %\ifoutline\outline
  %\fi
  %\ifoutcome\outcome
  %\fi
\end{exercise}


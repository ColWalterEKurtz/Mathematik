\setcounter{chapter}{-1}
% ------------------------
\chapter{Unsortiert\ldots}
% ------------------------

% --------------------
\section{und ungelöst}
% --------------------

\begin{exercise}
      {ID-42df963f029694c68d8cf7fcfac225e8c53415e5}
      {Maximales Rechteck im gleichseitigen Dreieck}
  \ifproblem\problem
    \ifthenelse{\isundefined{\linecalc}}{\newlength{\linecalc}}{\relax}%
    \setlength{\linecalc}{\linewidth}%
    \addtolength{\linecalc}{-30mm}%
    \begin{minipage}[b]{\linecalc}
      Einem gleichseitigen Dreieck mit der Seitenlänge $a$ soll ein Rechteck
      einbeschrieben werden. Wie lang müssen die Rechteckseiten sein, damit der
      Flächeninhalt des Rechtecks maximal wird?
    \end{minipage}\hfill
    \begin{minipage}[b]{25mm}
      \raggedleft
      \raisebox{0\baselineskip}[0\baselineskip][0pt]{%
      \begin{tikzpicture}
        \draw (-1.000,  0.000) -- ( 1.000,  0.000) -- ( 0.000,  1.732) -- cycle;
        \filldraw[fill=black!25!white] (-0.500,  0.000) rectangle ( 0.500,  0.866);
      \end{tikzpicture}}
    \end{minipage}%
  \fi
  \ifoutline\outline
    \ifthenelse{\isundefined{\linecalc}}{\newlength{\linecalc}}{\relax}%
    \setlength{\linecalc}{\linewidth}%
    \addtolength{\linecalc}{-50mm}%
    \begin{minipage}{40mm}
      \begin{tikzpicture}
        \draw (-1.000,  0.000) -- ( 1.000,  0.000) -- ( 0.000,  1.732) -- cycle;
        \filldraw[fill=black!25!white] (-0.500,  0.000) rectangle ( 0.500,  0.866);
        \draw[->, >=stealth] (-1.5, 0) -- (1.5, 0) node[below]{{\small$x$}};
        \draw[->, >=stealth] (0, -0.5) -- (0, 2.5) node[below left]{{\small$y$}};
        % a/2
        \draw (1, 0.1) -- (1, -0.1) node[below]{{\small$\displaystyle\frac{a}{2}$}};
        % sin(60) * a
        \draw (0.1, 1.732) -- (-0.1, 1.732) node[left]{{\small$a\cdot\sin60^\circ$}};
      \end{tikzpicture}
    \end{minipage}\hspace*{\fill}%
    \begin{minipage}{\linecalc}
      \begin{equation*}
        \begin{split}
          m&=-\frac{a\cdot\sin60^\circ}{\frac{a}{2}}=-2\sin60^\circ=-\sqrt{3} \\[2ex]
          g(x)&=-\sqrt{3}\cdot x+a\cdot\frac{\sqrt{3}}{2}
        \end{split}
      \end{equation*}
    \end{minipage}
  \fi
  %\ifoutcome\outcome
  %\fi
\end{exercise}

\begin{exercise}
      {ID-e9db77c9a5bbbfd8e37ea60ee54738c90e3cccfc}
      {Schrittlänge}
  \ifproblem\problem
    Vater und Sohn messen mit ihren Schritten den Abstand zwischen zwei Bäumen.
    Zuerst schreitet der Vater die Strecke ab, danach sein Sohn. Als der Sohn
    losgeht, ist auf dem Boden die Spur des Vaters deutlich zu erkennen.
    Der Sohn tritt insgesamt 10 Mal genau auf den Fußabdruck seines Vaters --
    zum Glück auch am Ende der Strecke. Ein Schritt des Vaters ist \sicm{70}
    lang, ein Schritt des Sohnes misst \sicm{56}. Wie weit sind die beiden
    Bäume voneinander entfernt?
  \fi
  \ifoutline\outline
    Vielleicht hilft das kleinste gemeinsame Vielfache\ldots
  \fi
  %\ifoutcome\outcome
  %\fi
\end{exercise}

\begin{exercise}
      {ID-c5b39ea16827187dcc3c7a2fdcc009495a327bff}
      {Drei Buslinien}
  \ifproblem\problem
    Um 6 Uhr morgens starten drei Busse gleichzeitig vom Bahnhof einer Stadt.
    Sie fahren bis spätestens 22 Uhr abends.\par
    Für eine Runde benötigt
    der erste Bus eine Stunde und 30 Minuten,
    der zweite Bus eine Stunde und 50 Minuten und
    der dritte Bus eine Stunde und 10 Minuten.
    Nach jeder Runde machen sie 10 Minuten Pause.
    Wann starten
    \begin{enumerate}[a)]
      \item der erste und der zweite Bus
      \item der zweite und der dritte Bus
      \item alle drei Busse
    \end{enumerate}
    wieder gleichzeitig vom Bahnhof?
  \fi
  %\ifoutline\outline
  %\fi
  %\ifoutcome\outcome
  %\fi
\end{exercise}

\begin{exercise}
      {ID-4db817770162d679b90c8d3642abf57bd5d98029}
      {Bunte Gerade}
  \ifproblem\problem
    Jeder Punkt einer Geraden ist entweder rot oder blau gefärbt.
    Zeige, dass es auf der Geraden drei gleichfarbige Punkte $A$,
    $B$ und $C$ gibt, für die $|AB|=|BC|$ gilt.
  \fi
  %\ifoutline\outline
  %\fi
  %\ifoutcome\outcome
  %\fi
\end{exercise}

\begin{exercise}
      {ID-7f3b47c16105c2a1d982f70fde324c13ac99a72b}
      {Bunte Ebene}
  \ifproblem\problem
    Jeder Punkt einer Ebene ist entweder rot oder blau gefärbt.
    \begin{enumerate}[a)]
      \item Zeige, dass es in der Ebene ein gleichseitiges Dreieck gibt,
            dessen Eckpunkte alle dieselbe Farbe besitzen.
      \item Zeige, dass es in der Ebene ein Rechteck gibt,
            dessen Eckpunkte alle dieselbe Farbe besitzen.
    \end{enumerate}
  \fi
  %\ifoutline\outline
  %\fi
  %\ifoutcome\outcome
  %\fi
\end{exercise}

\begin{exercise}
      {ID-20bf24634ef73703f73f55c0ef923f095de91a3d}
      {Gleichfarbig}
  \ifproblem\problem
    In einem Korb liegen 70 Kugeln. 10 sind rot, 20 sind grün, 30 sind blau
    und jeweils 5 sind weiß bzw. schwarz. \xya{} darf mit geschlossenen Augen
    einige Kugeln nehmen. Wie viele Kugeln muss er mindestens nehmen, damit
    unter den gezogenen Kugeln \emph{garantiert} 10 gleichfarbige sind?
  \fi
  %\ifoutline\outline
  %\fi
  %\ifoutcome\outcome
  %\fi
\end{exercise}

% stop here...
%\end{document}


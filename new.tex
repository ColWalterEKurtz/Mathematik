\allowdisplaybreaks
\setcounter{chapter}{-1}
% ------------------------
\chapter{Unsortiert\ldots}
% ------------------------

% --------------------
\section{und ungelöst}
% --------------------

\begin{exercise}
      {ID-5cdd9e1fd079bcbcb6853571087cbe536f94db6c}
      {Laternen}
  \ifproblem\problem
    In einer kleinen Stadt stehen auf einer Straße am linken und am rechten
    Straßenrand insgesamt 47 Laternen.
    Auf jeder Straßenseite beträgt der Abstand zwischen je zwei benachbarten
    Laternen \simeter{35}.
    Am linken Straßenrand steht je eine Laterne genau am Anfang und am Ende
    der Straße.
    Wie lang ist die Straße?
  \fi
  %\ifoutline\outline
  %\fi
  %\ifoutcome\outcome
  %\fi
\end{exercise}

\begin{exercise}
      {ID-e2fd99a7b31e7ac68b0614a71ff9ad70726524cf}
      {Drei Farben}
  \ifproblem\problem
    Kann man die Felder der Abbildung so mit den Farben Blau, Rot und Gelb
    färben, dass jede Farbe eine gleich große Gesamtfläche bedeckt wie jede
    andere Farbe und dass niemals zwei Farben längs einer Strecke zusammenstoßen?
    Wenn das möglich ist, stelle eine solche Färbung her!
    \begin{center}
      \begin{tikzpicture}[scale=3]
        \newcommand{\sqrtwo}{1.41421356237310}%
        \draw (0, 0) rectangle (3*\sqrtwo, \sqrtwo);
        \draw (0, 0) -- (\sqrtwo, \sqrtwo) -- (2*\sqrtwo, 0) -- (3*\sqrtwo, \sqrtwo);
        \draw (0, \sqrtwo) -- (\sqrtwo, 0) -- (2*\sqrtwo, \sqrtwo) -- (3*\sqrtwo, 0);
        \filldraw[fill=white, draw=black, line join=bevel]
                 (0, 0.5*\sqrtwo) --
                 (0.5*\sqrtwo, 0) --
                 (\sqrtwo, 0.5*\sqrtwo) --
                 (0.5*\sqrtwo, \sqrtwo) -- cycle;
        \begin{scope}[xshift=\sqrtwo cm]
          \filldraw[fill=white, draw=black, line join=bevel]
                   (0, 0.5*\sqrtwo) --
                   (0.5*\sqrtwo, 0) --
                   (\sqrtwo, 0.5*\sqrtwo) --
                   (0.5*\sqrtwo, \sqrtwo) -- cycle;
        \end{scope}
        \begin{scope}[xshift=2*\sqrtwo cm]
          \filldraw[fill=white, draw=black, line join=bevel]
                   (0, 0.5*\sqrtwo) --
                   (0.5*\sqrtwo, 0) --
                   (\sqrtwo, 0.5*\sqrtwo) --
                   (0.5*\sqrtwo, \sqrtwo) -- cycle;
        \end{scope}
      \end{tikzpicture}
    \end{center}
  \fi
  %\ifoutline\outline
  %\fi
  %\ifoutcome\outcome
  %\fi
\end{exercise}

\begin{exercise}
      {ID-8e60ccc39b1d62861f1b9832d4409847bfc9f98e}
      {Wohnhaus}
  \ifproblem\problem
    In einem Haus mit Erdgeschoss und drei weiteren Etagen wohnen 72 Personen.
    In der zweiten Etage sind es 7 Personen mehr als in der ersten, in der
    dritten 6 Personen mehr als in der ersten. Da im Erdgeschoss außer
    Wohnungen auch ein Geschäft ist, wohnen dort 12 Personen weniger als in
    der ersten Etage. Wie viele Personen wohnen im Erdgeschoss und in jeder
    der weiteren Etagen?
  \fi
  %\ifoutline\outline
  %\fi
  %\ifoutcome\outcome
  %\fi
\end{exercise}

\begin{exercise}
      {ID-75ded6d84c4b9e4f43b75ffa260305a7cd43bf86}
      {Zerlegungen}
  \ifproblem\problem
    Zeichne fünf Rechtecke! Zu jedem dieser Rechtecke sollen dann zwei Geraden
    gezeichnet werden, die den Rand des Rechtecks schneiden und dabei das
    betreffende Rechteck in folgende Figuren zerlegen:
    \begin{enumerate}[a)]
      \squeeze
      \item Zwei Dreiecke und ein Viereck.
      \item Ein Dreieck und zwei Vierecke.
      \item Ein Dreieck und drei Vierecke.
      \item Ein Dreieck, ein Viereck und ein Fünfeck.
      \item Zwei Dreiecke und ein Sechseck.
    \end{enumerate}
  \fi
  %\ifoutline\outline
  %\fi
  %\ifoutcome\outcome
  %\fi
\end{exercise}

\begin{exercise}
      {ID-d99bcfda1f549ea82d26e17dbb3572f23635c273}
      {Kupferdraht}
  \ifproblem\problem
    Ein \simeter{6} \sicm{30} langer Kupferdraht soll in drei Teile
    unterteilt werden. Der erste Teil soll \sicm{30} länger als der
    zweite Teil sein und der dritte Teil \sicm{60} länger als der zweite.
    Wie lang wird jeder der Teile?
  \fi
  %\ifoutline\outline
  %\fi
  %\ifoutcome\outcome
  %\fi
\end{exercise}

\begin{exercise}
      {ID-98ce975597cebd7528264fa4f8f086746d110954}
      {Zahlenrätsel}
  \ifproblem\problem
    Gesucht ist die größte sechsstellige Zahl, für die folgendes gilt:
    \begin{enumerate}[a)]
      \squeeze
      \item Die Zahl ist gerade.
      \item Die Zehnerziffer stellt eine dreimal so große Zahl dar wie
            die Zehntausenderziffer.
      \item Die Einer- und die Tausenderziffer kann man vertauschen,
            ohne dass sich die sechsstellige Zahl ändert.
      \item Die Hunderterziffer stellt eine halb so große Zahl dar wie
            die Hunderttausenderziffer.
    \end{enumerate}
  \fi
  %\ifoutline\outline
  %\fi
  %\ifoutcome\outcome
  %\fi
\end{exercise}
  
% stop here...
%\end{document}

\begin{exercise}
      {ID-42df963f029694c68d8cf7fcfac225e8c53415e5}
      {Maximales Rechteck im gleichseitigen Dreieck}
  \ifproblem\problem
    \ifthenelse{\isundefined{\linecalc}}{\newlength{\linecalc}}{\relax}%
    \setlength{\linecalc}{\linewidth}%
    \addtolength{\linecalc}{-30mm}%
    \begin{minipage}[b]{\linecalc}
      Einem gleichseitigen Dreieck mit der Seitenlänge $a$ soll ein Rechteck
      einbeschrieben werden. Wie lang müssen die Rechteckseiten sein, damit der
      Flächeninhalt des Rechtecks maximal wird?
    \end{minipage}\hfill
    \begin{minipage}[b]{25mm}
      \raggedleft
      \raisebox{0\baselineskip}[0\baselineskip][0pt]{%
      \begin{tikzpicture}[scale=0.8]
        \draw (-1.000,  0.000) -- ( 1.000,  0.000) -- ( 0.000,  1.732) -- cycle;
        \filldraw[fill=black!25!white] (-0.500,  0.000) rectangle ( 0.500,  0.866);
      \end{tikzpicture}}
    \end{minipage}%
  \fi
  \ifoutline\outline
    \ifthenelse{\isundefined{\linecalc}}{\newlength{\linecalc}}{\relax}%
    \setlength{\linecalc}{\linewidth}%
    \addtolength{\linecalc}{-50mm}%
    \begin{minipage}{40mm}
      \begin{tikzpicture}
        \draw (-1.000,  0.000) -- ( 1.000,  0.000) -- ( 0.000,  1.732) -- cycle;
        \filldraw[fill=black!25!white] (-0.500,  0.000) rectangle ( 0.500,  0.866);
        \draw[->, >=stealth] (-1.5, 0) -- (1.5, 0) node[below]{{\small$x$}};
        \draw[->, >=stealth] (0, -0.5) -- (0, 2.5) node[below left]{{\small$y$}};
        % a/2
        \draw (1, 0.1) -- (1, -0.1) node[below]{{\small$\displaystyle\frac{a}{2}$}};
        % sin(60) * a
        \draw (0.1, 1.732) -- (-0.1, 1.732) node[left]{{\small$a\cdot\sin60^\circ$}};
      \end{tikzpicture}
    \end{minipage}\hspace*{\fill}%
    \begin{minipage}{\linecalc}
      \begin{equation*}
        \begin{split}
          m&=-\frac{a\cdot\sin60^\circ}{\frac{a}{2}}=-2\sin60^\circ=-\sqrt{3} \\[2ex]
          g(x)&=-\sqrt{3}\cdot x+a\cdot\frac{\sqrt{3}}{2}
        \end{split}
      \end{equation*}
    \end{minipage}
  \fi
  %\ifoutcome\outcome
  %\fi
\end{exercise}

\begin{exercise}
      {ID-4db817770162d679b90c8d3642abf57bd5d98029}
      {Bunte Gerade}
  \ifproblem\problem
    Jeder Punkt einer Geraden ist entweder rot oder blau gefärbt.
    Zeige, dass es auf der Geraden drei gleichfarbige Punkte $A$,
    $B$ und $C$ gibt, für die $|AB|=|BC|$ gilt.
  \fi
  %\ifoutline\outline
  %\fi
  %\ifoutcome\outcome
  %\fi
\end{exercise}

\begin{exercise}
      {ID-7f3b47c16105c2a1d982f70fde324c13ac99a72b}
      {Bunte Ebene}
  \ifproblem\problem
    Jeder Punkt einer Ebene ist entweder rot oder blau gefärbt.
    \begin{enumerate}[a)]
      \item Zeige, dass es in der Ebene ein gleichseitiges Dreieck gibt,
            dessen Eckpunkte alle dieselbe Farbe besitzen.
      \item Zeige, dass es in der Ebene ein Rechteck gibt,
            dessen Eckpunkte alle dieselbe Farbe besitzen.
    \end{enumerate}
  \fi
  %\ifoutline\outline
  %\fi
  %\ifoutcome\outcome
  %\fi
\end{exercise}


\begin{exercise}
      {ID-62db2d209a7fa3b09f8cfed6eda81856ed3e524f}
      {Neun Punkte}
  \ifproblem\problem
    Diese neun Punkte sollen mit dem Bleistift in einem Zug miteinander
    verbunden werden, der Stift darf also nicht abgesetzt werden.
    Dabei soll der Linienzug nur aus vier geraden Teilstrichen bestehen.
    \begin{center}
      \begin{tikzpicture}
        \fill (0, 0) circle (2pt);
        \fill (1, 0) circle (2pt);
        \fill (2, 0) circle (2pt);
        \fill (0, 1) circle (2pt);
        \fill (1, 1) circle (2pt);
        \fill (2, 1) circle (2pt);
        \fill (0, 2) circle (2pt);
        \fill (1, 2) circle (2pt);
        \fill (2, 2) circle (2pt);
      \end{tikzpicture}
    \end{center}
  \fi
  \ifoutline\outline
    Halte dich \emph{exakt} an die Bedingungen! An Voraussetzungen, die
    in der Aufgabenstellung nicht stehen, brauchst du dich auch
    nicht halten\ldots
  \fi
  %\ifoutcome\outcome
  %\fi
\end{exercise}

\begin{exercise}
      {ID-3b02cda2482a735aa27ce278c83629ecbcfec0af}
      {Familie}
  \ifproblem\problem
    Vater, Mutter, Tochter und Sohn sind zusammen 73 Jahre alt.
    Der Vater ist drei Jahre älter als die Mutter, die Tochter
    ist 2 Jahre älter als der Sohn. Vor vier Jahren waren
    alle Familienmitglieder zusammen 58 Jahre alt.
    Wie alt sind die Familienmitglieder heute?
  \fi
  \ifoutline\outline
    Vielleicht waren vor vier Jahren noch nicht alle Familienmitglieder geboren\ldots
  \fi
  \ifoutcome\outcome
    Wenn alle vie Familienmitglieder vor vier Jahren zusammen
    58 Jahre alt gewesen sind, dann müssten sie heute zusammen
    $58+16=74$ Jahre alt sein. Aber sie sind zusammen ja nur
    73 Jahre alt. Vor vier Jahren kann also das jüngste
    Familienmitglied (der Sohn) noch nicht mitgezählt worden
    sein, d.\,h. er war noch gar nicht geboren.
    \begin{equation*}
      (v-4)+(m-4)+(t-4)=58\quad\Rightarrow\quad v+m+t=70\quad\Rightarrow\quad s=3
    \end{equation*}
    Der Sohn ist heute also 3 Jahre alt und seine Schwester folglich 5.
    \begin{equation*}
      (m+3)+m+5=70\quad\Rightarrow\quad m=\frac{70-5-3}{2}\quad\Rightarrow\quad m=31
    \end{equation*}
    Damit ist die Mutter heute 31 Jahre alt und der Vater 34.
  \fi
\end{exercise}

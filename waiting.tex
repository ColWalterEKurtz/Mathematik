\begin{exercise}
      {ID-66023e9d6f57577f879c9e5e96ee2dd87e2ed481}
      {Raucher}
  \ifproblem\problem
    An einem Berufskolleg wurden alle \num{674} Schülerinnen und Schüler
    befragt, ob sie rauchen oder nicht. Das Ergebnis der Befragung sieht
    wie folgt aus: \num{82} der insgesamt \num{293} männlichen Schüler
    gaben an zu rauchen. \num{250} Schülerinnen gaben an, nicht zu
    rauchen.
    \begin{enumerate}[a)]
      \item Stelle den Sachzusammenhang in einer Vierfeldertafel dar.
      \item Mit welcher Wahrscheinlichkeit ist eine zufällig ausgewählte
            Person weiblich und Nichtraucherin?
      \item Mit welcher Wahrscheinlichkeit ist eine Schülerin Nichtraucherin?
      \item Untersuche ob in diesem Fall die Merkmale \emph{Geschlecht}
            und \emph{Raucher} unabhängig voneinander sind.
    \end{enumerate}
  \fi
  \ifoutline\outline
    \begin{enumerate}[a)]
      \item Eine passende Vierfeldertafel könnte man z.\,B. so beginnen:
            \begin{center}
              \begin{fourfoldtable}
                \Apos{m}%
                \Aneg{w}%
                \Bpos{r}%
                \numbers
                {82}{}{}
                {}{250}{}
                {293}{}{674}
                \Bneg{$\overline{\text{r}}$}%
              \end{fourfoldtable}
            \end{center}
      \item Gesucht ist: $P(\text{w}\cap\overline{\text{r}})$
      \item Gesucht ist: $P_{\text{w}}(\overline{\text{r}})$
      \item Untersuche ob in diesem Fall die Merkmale \emph{Geschlecht}
            und \emph{Raucher} unabhängig voneinander sind.
    \end{enumerate}
  \fi
  %\ifoutcome\outcome
  %\fi
\end{exercise}

\begin{exercise}
      {ID-42df963f029694c68d8cf7fcfac225e8c53415e5}
      {Maximales Rechteck im gleichseitigen Dreieck}
  \ifproblem\problem
    \ifthenelse{\isundefined{\linecalc}}{\newlength{\linecalc}}{\relax}%
    \setlength{\linecalc}{\linewidth}%
    \addtolength{\linecalc}{-30mm}%
    \begin{minipage}[b]{\linecalc}
      Einem gleichseitigen Dreieck mit der Seitenlänge $a$ soll ein Rechteck
      einbeschrieben werden. Wie lang müssen die Rechteckseiten sein, damit der
      Flächeninhalt des Rechtecks maximal wird?
    \end{minipage}\hfill
    \begin{minipage}[b]{25mm}
      \raggedleft
      \raisebox{0\baselineskip}[0\baselineskip][0pt]{%
      \begin{tikzpicture}[scale=0.8]
        \draw (-1.000,  0.000) -- ( 1.000,  0.000) -- ( 0.000,  1.732) -- cycle;
        \filldraw[fill=black!25!white] (-0.500,  0.000) rectangle ( 0.500,  0.866);
      \end{tikzpicture}}
    \end{minipage}%
  \fi
  \ifoutline\outline
    \ifthenelse{\isundefined{\linecalc}}{\newlength{\linecalc}}{\relax}%
    \setlength{\linecalc}{\linewidth}%
    \addtolength{\linecalc}{-50mm}%
    \begin{minipage}{40mm}
      \begin{tikzpicture}
        \draw (-1.000,  0.000) -- ( 1.000,  0.000) -- ( 0.000,  1.732) -- cycle;
        \filldraw[fill=black!25!white] (-0.500,  0.000) rectangle ( 0.500,  0.866);
        \draw[->, >=stealth] (-1.5, 0) -- (1.5, 0) node[below]{{\small$x$}};
        \draw[->, >=stealth] (0, -0.5) -- (0, 2.5) node[below left]{{\small$y$}};
        % a/2
        \draw (1, 0.1) -- (1, -0.1) node[below]{{\small$\displaystyle\frac{a}{2}$}};
        % sin(60) * a
        \draw (0.1, 1.732) -- (-0.1, 1.732) node[left]{{\small$a\cdot\sin60^\circ$}};
      \end{tikzpicture}
    \end{minipage}\hspace*{\fill}%
    \begin{minipage}{\linecalc}
      \begin{equation*}
        \begin{split}
          m&=-\frac{a\cdot\sin60^\circ}{\frac{a}{2}}=-2\sin60^\circ=-\sqrt{3} \\[2ex]
          g(x)&=-\sqrt{3}\cdot x+a\cdot\frac{\sqrt{3}}{2}
        \end{split}
      \end{equation*}
    \end{minipage}
  \fi
  %\ifoutcome\outcome
  %\fi
\end{exercise}

\begin{exercise}
      {ID-4db817770162d679b90c8d3642abf57bd5d98029}
      {Bunte Gerade}
  \ifproblem\problem
    Jeder Punkt einer Geraden ist entweder rot oder blau gefärbt.
    Zeige, dass es auf der Geraden drei gleichfarbige Punkte $A$,
    $B$ und $C$ gibt, für die $|AB|=|BC|$ gilt.
  \fi
  %\ifoutline\outline
  %\fi
  %\ifoutcome\outcome
  %\fi
\end{exercise}

\begin{exercise}
      {ID-7f3b47c16105c2a1d982f70fde324c13ac99a72b}
      {Bunte Ebene}
  \ifproblem\problem
    Jeder Punkt einer Ebene ist entweder rot oder blau gefärbt.
    \begin{enumerate}[a)]
      \item Zeige, dass es in der Ebene ein gleichseitiges Dreieck gibt,
            dessen Eckpunkte alle dieselbe Farbe besitzen.
      \item Zeige, dass es in der Ebene ein Rechteck gibt,
            dessen Eckpunkte alle dieselbe Farbe besitzen.
    \end{enumerate}
  \fi
  %\ifoutline\outline
  %\fi
  %\ifoutcome\outcome
  %\fi
\end{exercise}


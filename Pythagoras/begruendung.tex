\begin{exercise}
      {ID-bffd2d5cd3fb92babd1a2e0bccb21b4cd9d3139c}
      {Begründung}
  \ifproblem\problem\par
    % <PROBLEM>
    Begründe mithilfe der Abbildung, warum der Satz des Pythagoras richtig ist.
    \begin{center}
      \begin{tikzpicture}[scale=0.5, line width=0.6pt]
        % default colors
        \newcommand{\colr}{Red};%
        \newcommand{\colg}{ForestGreen};%
        \newcommand{\colb}{Cerulean};%
        \newcommand{\coly}{YellowOrange};%
        \newcommand{\cola}{Black!35!White};%
        \newcommand{\cole}{Black!55!White};%
        \begin{scope}
          \fill[fill=\cola] (0, 0) rectangle (3, 4);
          \fill[fill=\cola] (3, 4) rectangle (7, 7);
          \draw (0, 0) rectangle (7, 7);
          \draw (3, 0) -- (3, 7);
          \draw (0, 4) -- (7, 4);
          \draw (3, 7) -- (7, 4);
          \draw (0, 4) -- (3, 0);
        \end{scope}
        \begin{scope}[xshift=10cm]
          \filldraw[fill=\cola] (0, 0) -- (3, 0) -- (0, 4) -- cycle;
          \filldraw[fill=\cola] (0, 4) -- (4, 7) -- (0, 7) -- cycle;
          \filldraw[fill=\cola] (3, 0) -- (7, 0) -- (7, 3) -- cycle;
          \filldraw[fill=\cola] (7, 3) -- (7, 7) -- (4, 7) -- cycle;
        \end{scope}
      \end{tikzpicture}
    \end{center}
    % </PROBLEM>
  \fi
  %\ifoutline\outline\par
    % <OUTLINE>
    % </OUTLINE>
  %\fi
  \ifoutcome\outcome
    % <OUTCOME>
    \begin{description}
      \item[Beobachtung 1:]
      Die Zeichnung besteht aus zwei Quadraten, und
      beide Quadrate sind gleich groß.
      \item[Beobachtung 2:]
      Die vier grauen Dreiecke aus dem linken Quadrat
      finden sich im rechten Quadrat wieder.
      Sie sind dort lediglich anders angeordnet.
      \item[Folgerung 1:]
      Die graue Fläche im linken Quadrat besitzt
      denselben Inhalt wie die graue Fläche im
      rechten Quadrat.
      \item[Folgerung 2:]
      Wenn das linke und das rechte Quadrat dieselbe
      Fläche besitzen und in beiden die graue Fläche
      gleich groß ist, dann ist auch die weiße Fläche
      in beiden Quadraten gleich groß.
      \item[Beobachtung 3:]
      Bezogen auf ein graues Dreieck stellen die weißen
      Flächen im linken Quadrat die Kathetenquadrate dar
      und die weiße Fläche im rechten Quadrat bildet das
      Hypotenusenquadrat eines grauen Dreiecks.
      \item[Folgerung 3:]
      Also sind die Kathetenquadrate zusammen genau
      so groß wie das Hypotenusenquadrat.
    \end{description}
    % </OUTCOME>
  \fi
\end{exercise}

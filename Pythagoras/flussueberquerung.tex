\begin{exercise}
      {ID-fd325e2fe6b847b6ef5b134491116b8315221c14}
      {Flussüberquerung}
  \ifproblem\problem\par
    Ein Fluss ist \simeter{180} breit und hat eine Strömungsgeschwindigkeit
    $v=\sims{0.8}$. Um ihn zu überqueren braucht ein Schwimmer \num{5} Minuten.
    \begin{enumerate}[a)]
      \item Um wie viel Meter wird der Schwimmer dabei seitlich abgetrieben?
      \item Wie viel Meter hat der Schwimmer bei der Überquerung zurückgelegt?
    \end{enumerate}
  \fi
  %\ifoutline\outline\par
  %\fi
  \ifoutcome\outcome\par
    \begin{enumerate}[a)]
      \item In \num{5} Minuten wird ein Schwimmer um \simeter{240} seitlich abgetrieben.
      \item Insgesammt legt der Schwimmer eine Strecke von \simeter{300} zurück.
    \end{enumerate}
  \fi
\end{exercise}

\begin{exercise}
      {ID-1da06e859bd46564b9e0c9a2cbe0089b4d613ced}
      {Rechtwinkligkeit}
  \ifproblem\problem\par
    % <PROBLEM>
    Prüfe, ob die Dreiecke mit den angegebenen
    Seitenlängen rechtwinklig sind.
    \begingroup
      \newcommand{\tri}[4]
      {%
        \text{#1}\;&\;
        &
        a&=\SI{#2}{\centi\metre}
        &
        \;&\;
        &
        b&=\SI{#3}{\centi\metre}
        &
        \;&\;
        &
        c&=\SI{#4}{\centi\metre}
      }%
      \begin{alignat*}{13}
        \tri{a)}{3}{4}{5}
        \quad&\quad
        \tri{d)}{29}{21}{20}
        \\
        \tri{b)}{6}{7}{8}
        \quad&\quad
        \tri{e)}{6.5}{9.7}{7.2}
        \\
        \tri{c)}{8}{17}{15}
        \quad&\quad
        \tri{f)}{5.3}{2.8}{4.6}
      \end{alignat*}
    \endgroup
    % </PROBLEM>
  \fi
  \ifoutline\outline\par
    % <OUTLINE>
    Der Satz des Pythagoras lautet in der
    allgemein bekannten Form:
    \begin{quote}
      Wenn ein Dreieck einen rechten Winkel besitzt,
      dann sind die beiden Kathetenquadrate zusammen
      so groß wie das Hypotenusenquadrat.
    \end{quote}
    Aber der Satz des Pythagoras gilt auch in der
    umgekehrten Richtung:
    \begin{quote}
      Wenn die Quadrate über den beiden kürzeren
      Seiten zusammen so groß sind wie das Quadrat
      über der längsten Seite, dann besitzt das
      Dreieck einen rechten Winkel.
    \end{quote}
    % </OUTLINE>
  \fi
  \ifoutcome\outcome
    % <OUTCOME>
    \begin{enumerate}[a)]
    \item Das Dreieck $ABC$ mit
    $a=\SI{3}{\centi\metre}$,
    $b=\SI{4}{\centi\metre}$ und
    $c=\SI{5}{\centi\metre}$
    besitzt einen rechten Winkel bei $\gamma$, denn:
    \begin{equation*}
      \num{3}^2
      \text{\,\si{\square\centi\metre}}
      +
      \num{4}^2
      \text{\,\si{\square\centi\metre}}
      =
      \SI{9}{\square\centi\metre}
      +
      \SI{16}{\square\centi\metre}
      =
      \SI{25}{\square\centi\metre}
      =
      \num{5}^2
      \text{\,\si{\square\centi\metre}}
    \end{equation*}
    \item Das Dreieck $ABC$ mit
    $a=\SI{6}{\centi\metre}$,
    $b=\SI{7}{\centi\metre}$ und 
    $c=\SI{8}{\centi\metre}$
    ist nicht rechtwinklig, denn:
    \begin{equation*}
      \num{6}^2
      \text{\,\si{\square\centi\metre}}
      +
      \num{7}^2
      \text{\,\si{\square\centi\metre}}
      =
      \SI{36}{\square\centi\metre}
      +
      \SI{49}{\square\centi\metre}
      =
      \SI{85}{\square\centi\metre}
      \neq
      \num{8}^2
      \text{\,\si{\square\centi\metre}}
      =
      \SI{64}{\square\centi\metre}
    \end{equation*}
    \item Das Dreieck $ABC$ mit
    $a=\SI{8}{\centi\metre}$,
    $b=\SI{17}{\centi\metre}$ und
    $c=\SI{15}{\centi\metre}$
    besitzt einen rechten Winkel bei $\beta$, denn:
    \begin{equation*}
      \num{8}^2
      \text{\,\si{\square\centi\metre}}
      +
      \num{15}^2
      \text{\,\si{\square\centi\metre}}
      =
      \SI{64}{\square\centi\metre}
      +
      \SI{225}{\square\centi\metre}
      =
      \SI{289}{\square\centi\metre}
      =
      \num{17}^2
      \text{\,\si{\square\centi\metre}}
    \end{equation*}
    \item Das Dreieck $ABC$ mit
    $a=\SI{29}{\centi\metre}$,
    $b=\SI{21}{\centi\metre}$ und
    $c=\SI{20}{\centi\metre}$
    besitzt einen rechten Winkel bei $\alpha$, denn:
    \begin{equation*}
      \num{21}^2
      \text{\,\si{\square\centi\metre}}
      +
      \num{20}^2
      \text{\,\si{\square\centi\metre}}
      =
      \SI{441}{\square\centi\metre}
      +
      \SI{400}{\square\centi\metre}
      =
      \SI{841}{\square\centi\metre}
      =
      \num{29}^2
      \text{\,\si{\square\centi\metre}}
    \end{equation*}
    \item Das Dreieck $ABC$ mit
    $a=\SI{6.5}{\centi\metre}$,
    $b=\SI{9.7}{\centi\metre}$ und
    $c=\SI{7.2}{\centi\metre}$
    besitzt einen rechten Winkel bei $\beta$, denn:
    \begin{equation*}
      \num{6.5}^2
      \text{\,\si{\square\centi\metre}}
      +
      \num{7.2}^2
      \text{\,\si{\square\centi\metre}}
      =
      \SI{42.25}{\square\centi\metre}
      +
      \SI{51.84}{\square\centi\metre}
      =
      \SI{94.09}{\square\centi\metre}
      =
      \num{9.7}^2
      \text{\,\si{\square\centi\metre}}
    \end{equation*}
    \item Das Dreieck $ABC$ mit
    $a=\SI{5.3}{\centi\metre}$,
    $b=\SI{2.8}{\centi\metre}$ und 
    $c=\SI{4.6}{\centi\metre}$
    ist nicht rechtwinklig, denn:
    \begin{equation*}
      \num{2.8}^2
      \text{\,\si{\square\centi\metre}}
      +
      \num{4.6}^2
      \text{\,\si{\square\centi\metre}}
      =
      \SI{7.84}{\square\centi\metre}
      +
      \SI{21.16}{\square\centi\metre}
      =
      \SI{29}{\square\centi\metre}
      \neq
      \num{5.3}^2
      \text{\,\si{\square\centi\metre}}
      =
      \SI{28.09}{\square\centi\metre}
    \end{equation*}
    \end{enumerate}
    % </OUTCOME>
  \fi
\end{exercise}

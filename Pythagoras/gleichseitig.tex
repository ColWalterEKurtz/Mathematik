\begin{exercise}
      {ID-20e4e01973a96bf9d1952025ddd54429c85a324e}
      {Gleichseitig}
  \ifproblem\problem
    \begin{enumerate}[a)]
      \squeeze
      \item Welchen Flächeninhalt hat ein gleichseitiges Dreieck mit dem Umfang \simeter{1}?
      \item Welchen Flächeninhalt hat ein gleichseitiges Dreieck mit der Höhe \simeter{1}?
      \item Welchen Umfang hat ein gleichseitiges Dreieck mit Flächeninhalt \simm{1}?
      \item Welchen Umfang hat ein gleichseitiges Dreieck mit der Höhe \simeter{1}?
    \end{enumerate}
  \fi
  \ifoutline\outline
    In einem gleichseitigen Dreieck halbiert jede Höhe $h$ die jeweilige
    Grundseite $a$ und es gilt nach dem Satz des Pythagoras folgende Gleichung:
    \begin{equation*}
      a^2=h^2+\left(\frac{a}{2}\right)^2
    \end{equation*}
    Damit gelten auch die Zusammenhänge
    \begin{equation*}
      h(a)=\frac{\sqrt{3}}{2}\cdot a
      \quad,\quad
      a(h)=\frac{2h}{\sqrt{3}}
      \quad,\quad
      a(U)=\frac{U}{3}
      \quad\text{und}\quad
      a(A)=\frac{2}{3}\cdot\sqrt[4]{3A^2}
    \end{equation*}
    aus denen sich sich dann die gefragten Größen zusammensetzen lassen.
    \begin{enumerate}[a)]
      \item Für den Flächeninhalt in Abhängigkeit vom Umfang gilt:
            \begin{equation*}
              A(U)
              =
              \frac{gh}{2}
              =
              \frac{a(U)\cdot h\big(a(U)\big)}{2}
              =
              \frac{\frac{U}{3}\cdot\frac{\sqrt{3}}{2}\cdot\frac{U}{3}}{2}
              =
              \frac{\sqrt{3}}{36}\cdot U^2
            \end{equation*}
      \item Für den Flächeninhalt in Abhängigkeit von der Höhe gilt:
            \begin{equation*}
              A(h)
              =
              \frac{gh}{2}
              =
              \frac{a(h)\cdot h}{2}
              =
              \frac{\frac{2h}{\sqrt{3}}\cdot h}{2}
              =
              \frac{1}{\sqrt{3}}\cdot h^2
            \end{equation*}
      \item Für den Umfang in Abhängigkeit vom Flächeninhalt gilt:
            \begin{equation*}
              U(A)
              =
              a+b+c
              =
              3\cdot a(A)
              =
              2\cdot\sqrt[4]{3A^2}
            \end{equation*}
      \item Für den Umfang in Abhängigkeit von der Höhe gilt:
            \begin{equation*}
              U(h)
              =
              a+b+c
              =
              3\cdot a(h)
              =
              3\cdot\frac{2h}{\sqrt{3}}
              =
              2\cdot\sqrt{3}\cdot h
            \end{equation*}
    \end{enumerate}
  \fi
  \ifoutcome\outcome
    \begin{enumerate}[a)]
      \item Der Flächeninhalt beträgt $\frac{\sqrt{3}}{36}\approx\simm{0.048}$.
      \item Der Flächeninhalt beträgt $\frac{1}{\sqrt{3}}\approx\simm{0.577}$.
      \item Der Umfang beträgt $2\cdot\sqrt[4]{3}\approx\simeter{2.63}$.
      \item Der Umfang beträgt $2\cdot\sqrt{3}\approx\simeter{3.46}$.
    \end{enumerate}
  \fi
\end{exercise}

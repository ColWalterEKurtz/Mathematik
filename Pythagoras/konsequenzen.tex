\begin{exercise}
      {ID-537cec26702ea38df967c58d12076c4eac97f82b}
      {Konsequenzen}
  \ifproblem\problem\par
    \xxf{} behauptet: \glqq Wenn in einem Dreieck $a^{2}+b^{2}=c^{2}$ gilt, dann gilt auch\ldots\par
    \begin{minipage}{0.49\textwidth}
      \centering
      \begin{tikzpicture}
        \coordinate (A)  at ( 0mm,  0mm);
        \coordinate (B)  at (50mm,  0mm);
        \coordinate (C)  at (18mm, 24mm);
        \coordinate (F)  at (18mm,  0mm);
        \coordinate (AB) at ([shift={(143.13:30mm)}]A);
        \coordinate (CB) at ([shift={(143.13:30mm)}]C);
        \coordinate (BA) at ([shift={(53.13:40mm)}]B);
        \coordinate (CA) at ([shift={(53.13:40mm)}]C);
        \coordinate (AC) at ([shift={(270:50mm)}]A);
        \coordinate (BC) at ([shift={(270:50mm)}]B);
        \coordinate (AA) at (intersection of B--CA and C--BA);
        \coordinate (BB) at (intersection of A--CB and C--AB);
        \coordinate (CC) at (intersection of A--BC and B--AC);
        % Hoehe
        \draw (C) -- (F);
        % Beschriftung
        \node[below left]  at (A) {{\small$A$}};
        \node[below right] at (B) {{\small$B$}};
        \node[above]       at (C) {{\small$C$}};
        \path ($(A)!0.5!(F)$) -- node[below=4mm] {{\small$c$}} ($(F)!0.5!(B)$);
        \path (A) -- node[below]       {{\small$q$}} (F);
        \path (F) -- node[below]       {{\small$p$}} (B);
        \path (B) -- node[above right] {{\small$a$}} (C);
        \path (A) -- node[above left]  {{\small$b$}} (C);
        \path (F) -- node[right]       {{\small$h$}} (C);
        % Eckpunkte
        \fill (A) circle (1pt);
        \fill (B) circle (1pt);
        \fill (C) circle (1pt);
        % Dreieck
        \draw (A) -- (B) -- (C) -- cycle;
        % rechter Winkel
        \begin{scope}
          \clip (A) -- (B) -- (C) -- cycle;
          \draw (C) circle (5mm);
          \fill[white] ([shift=(278.13:2.6mm)]C) circle (2pt);
          \fill[black] ([shift=(278.13:2.6mm)]C) circle (1pt);
        \end{scope}
      \end{tikzpicture}
    \end{minipage}%
    \hfill
    \begin{minipage}{0.50\textwidth}
      \begin{enumerate}[a)]
        \item $h^{2}=q\cdot p$.\grqq
        \item $a^{2}=c\cdot p$.\grqq
        \item $b^{2}=c\cdot q$.\grqq
      \end{enumerate}
      Hat \xxf{} recht, oder irrt sie sich?% bei mindestens einer Behauptung?
    \end{minipage}
  \fi
  %\ifoutline\outline\par
  %\fi
  %\ifoutcome\outcome\par
  %\fi
\end{exercise}

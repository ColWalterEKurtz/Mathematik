\begin{exercise}
      {ID-6db33a4bf4f78cc0773018bdd34823822a26f3cf}
      {Steigung}
  \ifproblem\problem
    \begin{enumerate}[a)]
      \item Ein Auto befährt eine Passstraße von \sikm{12.5} Länge und überwindet
            dabei eine Höhe von \simeter{1005}. Berechne die Steigung in Prozent.
      \item Welche konstante Steigung müsste eine Straße haben, die einen
            Höhenunterschied von \simeter{157} auf einer Strecke von
            \simeter{1800} überwindet?
      \item Wie lange wäre eine Straße mindestens, die bei maximal \pc{10}
            Steigung einen Höhenunterschied von \simeter{157} überwindet?
    \end{enumerate}
  \fi
  %\ifoutline\outline
  %\fi
  %\ifoutcome\outcome
  %\fi
\end{exercise}

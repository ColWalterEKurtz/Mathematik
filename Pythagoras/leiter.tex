\begin{exercise}
      {ID-4c066f46154388381cb033049b325d491312b80f}
      {Leiter}
  \ifproblem\problem\par
    % <PROBLEM>
    \begin{enumerate}[a)]
      \item Der Fuß einer Leiter steht \SI{2.5}{\metre} vor einer Hauswand.
            Die Leiter erreicht ein \SI{6}{\metre} hoch gelegenes Fenster.
            Wie lang ist die Leiter?
      \item Eine \SI{12.5}{\metre} lange Leiter lehnt an
            einer Hauswand. Das untere Leiterende steht
            dabei \SI{3.5}{\metre} von der Wand entfernt.
            In welcher Höhe liegt die Leiter an der
            Hauswand an?
    \end{enumerate}
    % </PROBLEM>
  \fi
  %\ifoutline\outline\par
    % <OUTLINE>
    % </OUTLINE>
  %\fi
  \ifoutcome\outcome
    % <OUTCOME>
    \begin{center}
      \begin{tikzpicture}[scale=0.4]
        % default colors
        \newcommand{\colr}{Red};%
        \newcommand{\colg}{ForestGreen};%
        \newcommand{\colb}{Cerulean};%
        \newcommand{\coly}{YellowOrange};%
        \newcommand{\cola}{Black!35!White};%
        \newcommand{\cole}{Black!55!White};%
        % Leiter
        \begin{scope}
          \clip (0, 0) rectangle (3, 8);
          \draw[line width=1.5pt, cap=round]
               (0, 6) -- (2.5, 0);
        \end{scope}
        % Boden
        \draw[line width=0.6pt] (-6, 0) -- (3.5, 0);
        % Dach
        \filldraw[fill=\cola]
                 ( 0, 8.0) --
                 (-1, 9.5) --
                 (-4, 9.5) --
                 (-5, 8.0);
        % Haus
        \filldraw[fill=\cola]
                 ( 0, 0) --
                 ( 0, 8) --
                 (-5, 8) --
                 (-5, 0) --
                 cycle;
        % Fenster
        \begin{scope}[yshift=6.0cm]
          \filldraw[fill=white] (-5.0, 0.0) rectangle (-4.6, 1.5);
          \filldraw[fill=white] (-4.0, 0.0) rectangle (-2.8, 1.5);
          \filldraw[fill=white] (-2.2, 0.0) rectangle (-1.0, 1.5);
          \filldraw[fill=white] (-0.4, 0.0) rectangle ( 0.0, 1.5);
        \end{scope}
        \begin{scope}[yshift=3.5cm]
          \filldraw[fill=white] (-5.0, 0.0) rectangle (-4.6, 1.5);
          \filldraw[fill=white] (-4.0, 0.0) rectangle (-2.8, 1.5);
          \filldraw[fill=white] (-2.2, 0.0) rectangle (-1.0, 1.5);
          \filldraw[fill=white] (-0.4, 0.0) rectangle ( 0.0, 1.5);
        \end{scope}
        \begin{scope}[yshift=1cm]
          \filldraw[fill=white] (-5.0, 0.0) rectangle (-4.6, 1.5);
          \filldraw[fill=white] (-4.0, 0.0) rectangle (-2.8, 1.5);
          \filldraw[fill=white] (-2.2, 0.0) rectangle (-1.0, 1.5);
          \filldraw[fill=white] (-0.4, 0.0) rectangle ( 0.0, 1.5);
        \end{scope}
        % Beschriftung
        \draw[style=dashed] (0, 6) -- (3.5, 6);
        \draw[|<->|, >=stealth]
             (3.5, 0) -- node[right] {$h$}
             (3.5, 6);
        \draw[style=dashed] (0.0, 0) -- (0.0, -1);
        \draw[style=dashed] (2.5, 0) -- (2.5, -1);
        \draw[|<->|, >=stealth]
             (0.0, -1) -- node[below] {$a$}
             (2.5, -1);
        \path (2.5, 0) -- node[right=1pt]{$\ell$} (0, 6);
      \end{tikzpicture}
    \end{center}
    \begin{enumerate}[a)]
      \item Wenn man die Angaben aus der
            Aufgabenstellung mit den Bezeichnungen
            aus der Abbildung verbindet, ergeben
            sich folgende Zusammenhänge:
            \begin{equation*}
              a=\SI{2.5}{\metre}
              \qquad
              h=\SI{6}{\metre}
              \qquad
              \text{gesucht: $\ell$}
            \end{equation*}
            Da die Hauswand senkrecht auf dem Boden
            steht, kann man die Länge der Leiter mit
            dem Satz des Pythagoras berechnen. Es gilt:
            \begin{alignat*}{3}
              \relax&\quad
              &
              a^2+h^2&=\ell^2
              &
              \quad&|\,\sqrt{\ldots}
              \\
              \Leftrightarrow&\quad
              &
              \sqrt{a^2+h^2}&=\ell
              &
              \quad&\relax
            \end{alignat*}
            Mit den Werten aus der Aufgabenstellung
            erhält man:
            \begin{equation*}
              \ell=\sqrt%
                   {
                    \num{2.5}^2\text{\,\si{\square\metre}}
                    +
                    \num{6}^2\text{\,\si{\square\metre}}
                   }
                  =\sqrt%
                   {
                    \SI{6.25}{\square\metre}
                    +
                    \SI{36}{\square\metre}
                   }
                  =\sqrt{\SI{42.25}{\square\metre} }
                  =\SI{6.5}{\metre}
                  %sqrt(2.5^2+6^2)
            \end{equation*}
            Die Leiter ist also \SI{6.5}{\metre} lang.
      \item Wenn man die Angaben aus der
            Aufgabenstellung mit den Bezeichnungen
            aus der Abbildung verbindet, ergeben
            sich folgende Zusammenhänge:
            \begin{equation*}
              a=\SI{3.5}{\metre}
              \qquad
              \ell=\SI{12.5}{\metre}
              \qquad
              \text{gesucht: $h$}
            \end{equation*}
            Auch hier lässt sich der Satz des Pythagoras
            verwenden, um die fehlende Größe zu bestimmen.
            Die Höhe $h$, in der die Leiter an der Mauer
            lehnt, erhält man durch folgende Überlegung:
            \begin{alignat*}{3}
              \relax&\quad
              &
              a^2+h^2&=\ell^2
              &
              \quad&|-a^2
              \\
              \Leftrightarrow&\quad
              &
              h^2&=\ell^2-a^2
              &
              \quad&|\,\sqrt{\ldots}
              \\
              \Leftrightarrow&\quad
              &
              h&=\sqrt{\ell^2-a^2}
              \quad&\relax
            \end{alignat*}
            \par
            Mit den Werten aus der Aufgabenstellung
            erhält man:
            \begin{equation*}
              h=\sqrt%
                {
                 \num{12.5}^2\text{\,\si{\square\metre}}
                 -
                 \num{3.5}^2\text{\,\si{\square\metre}}
                }
               =\sqrt%
                {
                  \SI{156.25}{\square\metre}
                  -
                  \SI{12.25}{\square\metre}
                }
               =\sqrt{\SI{144}{\square\metre}}
               =\SI{12}{\metre}
            \end{equation*}
            Die Leiter lehnt also in einer Höhe
            von \SI{12}{\metre} an der Hauswand.
    \end{enumerate}
    % </OUTCOME>
  \fi
\end{exercise}

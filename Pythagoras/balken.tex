\begin{exercise}
      {ID-94c115002c858377840d3c3da6b3a6ba0f931a1d}
      {Balken}
  \ifproblem\problem
    \begin{minipage}[c]{0.20\linewidth}
      \centering
      \begin{tikzpicture}
        \newcommand{\radius}{1.1}
        \draw[line width=0.5pt, fill=black!15!white] (0, 0) circle (\radius);
        \draw[fill=white] (225:\radius) rectangle (45:\radius);
        \node[above=3mm] at (270:\radius) {{\footnotesize\sicm{16}}};
      \end{tikzpicture}
    \end{minipage}\hfill
    \begin{minipage}[c]{0.79\linewidth}
    Aus einem Baumstamm soll ein Balken mit der abgebildeten quadratischen
    Grundfläche gesägt werden. Berechne den Durchmesser, den der Baumstamm
    mindestens haben muss.
    \end{minipage}\par
    \begin{minipage}[c]{0.20\linewidth}
      \centering
      \begin{tikzpicture}
        \newcommand{\radius}{1.1}
        \draw[line width=0.5pt, fill=black!15!white] (0, 0) circle (\radius);
        \draw[fill=white] (205:\radius) rectangle (25:\radius);
        \node at (270:6.6mm) {{\footnotesize\sicm{18}}};
        \node at (180:6.3mm) {{\footnotesize\sicm{8}}};
      \end{tikzpicture}
    \end{minipage}\hfill
    \begin{minipage}[c]{0.79\linewidth}
    Welchen Durchmesser muss ein Baumstamm mindestens haben, aus dem
    ein Balken mit der abgebildeten rechteckigen Grundfläche geschnitten
    werden soll?
    \end{minipage}
  \fi
  \ifoutline\outline
    Die Diagonale des Balkens entspricht dem Duchmesser des Baumstamms.
  \fi
  \ifoutcome\outcome
    \begin{itemize}
      \item Um den Balken mit der quadratischen Grundfläche herzustellen,
            muss der Baumstamm einen Durchmesser von ca. \sicm{22.63} besitzen.
      \item Um den Balken mit der rechteckigen Grundfläche herzustellen,
            muss der Baumstamm einen Durchmesser von ca. \sicm{19.70} besitzen.
    \end{itemize}
  \fi
\end{exercise}

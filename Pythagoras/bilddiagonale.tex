\begin{exercise}
      {ID-c632feb239d82bb8318754dcb73efe693d5d9aef}
      {Bilddiagonale}
  \ifproblem\problem
    Bildschirme von Fernsehgeräten gibt es in zwei unterschiedlichen Formaten.
    Das Verhältnis von Breite zu Höhe beträgt bei älteren Geräten 4:3, bei
    neuen 16:9. Die Bildschirmgröße wird in der Regel mit der Länge der
    Bilddiagonalen angegeben.
    \begin{enumerate}[a)]
      \item Berechne Höhe und Breite von Bildschirmen mit den Bilddiagonalen
            \sicm{69} und \sicm{89} bei einem 4:3-Format. Um wie viel cm\textsuperscript{2}
            unterscheiden sich die beiden Bildschirmflächen?
      \item Wenn ein Film im 16:9-Format auf einem Bildschirm im 4:3-Format gezeigt
            wird, sieht man oben und unten schwarze Streifen. Wie viel Prozent der
            Bildfläche werden von dem Film eingenommen?
      \item \xxa{} hat einen 89er-Bildschirm im Format 4:3. Beim Format 16:9 bleiben
            bei herkömmlichen Sendungen rechts und links Streifen, falls die Höhe
            voll ausgenutzt wird. \xxa{} möchte einen neuen Fernseher im Format
            16:9 kaufen. Dabei soll eine herkömmliche Sendung dieselbe Größe haben
            wie bisher. Welche Bildschirmdiagonale muss \xxa{} kaufen?
    \end{enumerate}
  \fi
  %\ifoutline\outline
  %\fi
  %\ifoutcome\outcome
  %\fi
\end{exercise}

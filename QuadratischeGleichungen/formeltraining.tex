\begin{exercise}
      {ID-3c72c6c4d7e81e895a7e041936ad91631b0b6cd1}
      {Formeltraining}
  \ifproblem\problem\par
    % <PROBLEM>
    Bestimme den Scheitelpunkt und die Nullstellen der folgenden Parabeln:
    \newcommand{\exchar}[1]{\vphantom{\frac{f}{f}}\text{#1}&\;}%
    \begin{alignat*}{8}
      \exchar{a)}
      &
      f(x)&=x^2-2x-3
      &
      \quad&\quad
      &
      \exchar{b)}
      &
      f(x)&=x^2-10x+24
      &
      \quad&\quad
      &
      \exchar{c)}
      &
      f(x)&=x^2-6x
      \\
      \exchar{d)}
      &
      f(x)&=-x^2-16x-48
      &
      \quad&\quad
      &
      \exchar{e)}
      &
      f(x)&=4x^2-16x-20
      &
      \quad&\quad
      &
      \exchar{f)}
      &
      f(x)&=-3x^2+12x+36
      \\
      \exchar{g)}
      &
      f(x)&=x^2-7x+12
      &
      \quad&\quad
      &
      \exchar{h)}
      &
      f(x)&=-\frac{9}{4}x^2+6x-3
      &
      \quad&\quad
      &
      \exchar{i)}
      &
      f(x)&=\frac{1}{2}x^2+\frac{3}{4}x+\frac{1}{4}
    \end{alignat*}
    % </PROBLEM>
  \fi
  %\ifoutline\outline\par
    % <OUTLINE>
    % </OUTLINE>
  %\fi
  \ifoutcome\outcome\par
    % <OUTCOME>
    \begin{minipage}[t]{0.49\linewidth}
      Scheitelpunkt a)
      \small
      \begin{equation*}
        \begin{split}
          f(x)&=x^{2}-2x-3
          \\
          &=x^2-2x+1-1-3
          \\
          &=(x-1)^2-4
          \\[1ex]
          \Rightarrow\quad&S\left(1\;\middle|\;-4\right)
        \end{split}
      \end{equation*}
    \end{minipage}%
    \hfill
    \begin{minipage}[t]{0.49\linewidth}
      Nullstellen a)
      \small
      %<OCTAVE>
      \begingroup
        \newcommand{\vstrut}{\vphantom{\left(f_0^0\right)}}%
        \newcommand{\noeq}{\phantom{\Leftrightarrow}\vstrut&\quad}%
        \newcommand{\iseq}{\Leftrightarrow\vstrut&\quad}%
        \newcommand{\impl}{\Rightarrow\vstrut&\quad}%
        \newcommand{\nomod}{\quad&\phantom{|}}%
        \newcommand{\domod}[1]{\quad&|#1}%
        \begin{alignat*}{3}
          \noeq
          &
          \num{0}&=x^{2}-\num{2}x-\num{3}
          &
          \domod{\;\text{$pq$-Formel}}
          \\
          \noeq
          &
          p&=-\num{2}
          &
          \nomod
          \\
          \noeq
          &
          q&=-\num{3}
          &
          \nomod
          \\
          \noeq
          &
          x_{1,2}&=-\frac{p}{2}\pm\sqrt{\left(\frac{p}{2}\right)^2-q}
          &
          \nomod
          \\
          \noeq
          &
          &=\num{1}\pm\sqrt{\left(-\num{1}\right)^2-\left(-\num{3}\right)}
          &
          \nomod
          \\
          \noeq
          &
          &=\num{1}\pm\sqrt{\num{4}}
          &
          \nomod
          \\
          \noeq
          &
          &=\num{1}\pm\num{2}
          &
          \nomod
          \\
          \noeq
          &
          x_1&=-\num{1}
          &
          \nomod
          \\
          \noeq
          &
          x_2&=\num{3}
          &
          \nomod
        \end{alignat*}
      \endgroup
      %</OCTAVE>
      %myqsolve(1, -2, -3, 0, "x");
    \end{minipage}\bigskip\par
    \begin{minipage}[t]{0.49\linewidth}
      Scheitelpunkt b)
      \small
      \begin{equation*}
        \begin{split}
          f(x)&=x^{2}-10x+24
          \\
          &=x^2-10x+25-25+24
          \\
          &=(x-5)^2-1
          \\[1ex]
          \Rightarrow\quad&S\left(5\;\middle|\;-1\right)
        \end{split}
      \end{equation*}
    \end{minipage}%
    \hfill
    \begin{minipage}[t]{0.49\linewidth}
      Nullstellen b)
      \small
      %<OCTAVE>
      \begingroup
        \newcommand{\vstrut}{\vphantom{\left(f_0^0\right)}}%
        \newcommand{\noeq}{\phantom{\Leftrightarrow}\vstrut&\quad}%
        \newcommand{\iseq}{\Leftrightarrow\vstrut&\quad}%
        \newcommand{\impl}{\Rightarrow\vstrut&\quad}%
        \newcommand{\nomod}{\quad&\phantom{|}}%
        \newcommand{\domod}[1]{\quad&|#1}%
        \begin{alignat*}{3}
          \noeq
          &
          \num{0}&=x^{2}-\num{10}x+\num{24}
          &
          \domod{\;\text{$pq$-Formel}}
          \\
          \noeq
          &
          p&=-\num{10}
          &
          \nomod
          \\
          \noeq
          &
          q&=\num{24}
          &
          \nomod
          \\
          \noeq
          &
          x_{1,2}&=-\frac{p}{2}\pm\sqrt{\left(\frac{p}{2}\right)^2-q}
          &
          \nomod
          \\
          \noeq
          &
          &=\num{5}\pm\sqrt{\left(-\num{5}\right)^2-\num{24}}
          &
          \nomod
          \\
          \noeq
          &
          &=\num{5}\pm\sqrt{\num{1}}
          &
          \nomod
          \\
          \noeq
          &
          &=\num{5}\pm\num{1}
          &
          \nomod
          \\
          \noeq
          &
          x_1&=\num{4}
          &
          \nomod
          \\
          \noeq
          &
          x_2&=\num{6}
          &
          \nomod
        \end{alignat*}
      \endgroup
      %</OCTAVE>
      %myqsolve(1, -10, 24, 0, "x");
    \end{minipage}\bigskip\par
    \begin{minipage}[t]{0.49\linewidth}
      Scheitelpunkt c)
      \small
      \begin{equation*}
        \begin{split}
          f(x)&=x^{2}-6x
          \\
          &=x^{2}-6x+9-9
          \\
          &=(x-3)^2-9
          \\[1ex]
          \Rightarrow\quad&S\left(3\;\middle|\;-9\right)
        \end{split}
      \end{equation*}
    \end{minipage}%
    \hfill
    \begin{minipage}[t]{0.49\linewidth}
      Nullstellen c)
      \small
      %<OCTAVE>
      \begingroup
        \newcommand{\vstrut}{\vphantom{\left(f_0^0\right)}}%
        \newcommand{\noeq}{\phantom{\Leftrightarrow}\vstrut&\quad}%
        \newcommand{\iseq}{\Leftrightarrow\vstrut&\quad}%
        \newcommand{\impl}{\Rightarrow\vstrut&\quad}%
        \newcommand{\nomod}{\quad&\phantom{|}}%
        \newcommand{\domod}[1]{\quad&|#1}%
        \begin{alignat*}{3}
          \noeq
          &
          \num{0}&=x^{2}-\num{6}x
          &
          \domod{\;\text{$pq$-Formel}}
          \\
          \noeq
          &
          p&=-\num{6}
          &
          \nomod
          \\
          \noeq
          &
          q&=\num{0}
          &
          \nomod
          \\
          \noeq
          &
          x_{1,2}&=-\frac{p}{2}\pm\sqrt{\left(\frac{p}{2}\right)^2-q}
          &
          \nomod
          \\
          \noeq
          &
          &=\num{3}\pm\sqrt{\left(-\num{3}\right)^2-\num{0}}
          &
          \nomod
          \\
          \noeq
          &
          &=\num{3}\pm\sqrt{\num{9}}
          &
          \nomod
          \\
          \noeq
          &
          &=\num{3}\pm\num{3}
          &
          \nomod
          \\
          \noeq
          &
          x_1&=\num{0}
          &
          \nomod
          \\
          \noeq
          &
          x_2&=\num{6}
          &
          \nomod
        \end{alignat*}
      \endgroup
      %</OCTAVE>
      %myqsolve(1, -6, 0, 0, "x");
    \end{minipage}\bigskip\par
    \begin{minipage}[t]{0.49\linewidth}
      Scheitelpunkt d)
      \small
      \begin{equation*}
        \begin{split}
          f(x)&=-x^{2}-16x-48
          \\
          &=-\left(x^2+16x+48\right)
          \\
          &=-\left(x^2+16x+64-64+48\right)
          \\
          &=-\left((x+8)^2-16\right)
          \\
          &=-(x+8)^2+16
          \\[1ex]
          \Rightarrow\quad&S\left(-8\;\middle|\;16\right)
        \end{split}
      \end{equation*}
    \end{minipage}%
    \hfill
    \begin{minipage}[t]{0.49\linewidth}
      Nullstellen d)
      \small
      %<OCTAVE>
      \begingroup
        \newcommand{\vstrut}{\vphantom{\left(f_0^0\right)}}%
        \newcommand{\noeq}{\phantom{\Leftrightarrow}\vstrut&\quad}%
        \newcommand{\iseq}{\Leftrightarrow\vstrut&\quad}%
        \newcommand{\impl}{\Rightarrow\vstrut&\quad}%
        \newcommand{\nomod}{\quad&\phantom{|}}%
        \newcommand{\domod}[1]{\quad&|#1}%
        \begin{alignat*}{3}
          \noeq
          &
          \num{0}&=-x^{2}-\num{16}x-\num{48}
          &
          \domod{:(-\num{1})}
          \\
          \iseq
          &
          \num{0}&=x^{2}+\num{16}x+\num{48}
          &
          \domod{\;\text{$pq$-Formel}}
          \\
          \noeq
          &
          p&=\num{16}
          &
          \nomod
          \\
          \noeq
          &
          q&=\num{48}
          &
          \nomod
          \\
          \noeq
          &
          x_{1,2}&=-\frac{p}{2}\pm\sqrt{\left(\frac{p}{2}\right)^2-q}
          &
          \nomod
          \\
          \noeq
          &
          &=-\num{8}\pm\sqrt{\num{8}^2-\num{48}}
          &
          \nomod
          \\
          \noeq
          &
          &=-\num{8}\pm\sqrt{\num{16}}
          &
          \nomod
          \\
          \noeq
          &
          &=-\num{8}\pm\num{4}
          &
          \nomod
          \\
          \noeq
          &
          x_1&=-\num{12}
          &
          \nomod
          \\
          \noeq
          &
          x_2&=-\num{4}
          &
          \nomod
        \end{alignat*}
      \endgroup
      %</OCTAVE>
      %myqsolve(-1, -16, -48, 0, "x");
    \end{minipage}\bigskip\par
    \begin{minipage}[t]{0.49\linewidth}
      Scheitelpunkt e)
      \small
      \begin{equation*}
        \begin{split}
          f(x)&=4x^2-16x-20
          \\
          &=4\left(x^2-4x-5\right)
          \\
          &=4\left(x^2-4x+4-4-5\right)
          \\
          &=4\left((x-2)^2-9\right)
          \\
          &=4(x-2)^2-36
          \\[1ex]
          \Rightarrow\quad&S\left(2\;\middle|\;-36\right)
        \end{split}
      \end{equation*}
    \end{minipage}%
    \hfill
    \begin{minipage}[t]{0.49\linewidth}
      Nullstellen e)
      \small
      %<OCTAVE>
      \begingroup
        \newcommand{\vstrut}{\vphantom{\left(f_0^0\right)}}%
        \newcommand{\noeq}{\phantom{\Leftrightarrow}\vstrut&\quad}%
        \newcommand{\iseq}{\Leftrightarrow\vstrut&\quad}%
        \newcommand{\impl}{\Rightarrow\vstrut&\quad}%
        \newcommand{\nomod}{\quad&\phantom{|}}%
        \newcommand{\domod}[1]{\quad&|#1}%
        \begin{alignat*}{3}
          \noeq
          &
          \num{0}&=\num{4}x^{2}-\num{16}x-\num{20}
          &
          \domod{:\num{4}}
          \\
          \iseq
          &
          \num{0}&=x^{2}-\num{4}x-\num{5}
          &
          \domod{\;\text{$pq$-Formel}}
          \\
          \noeq
          &
          p&=-\num{4}
          &
          \nomod
          \\
          \noeq
          &
          q&=-\num{5}
          &
          \nomod
          \\
          \noeq
          &
          x_{1,2}&=-\frac{p}{2}\pm\sqrt{\left(\frac{p}{2}\right)^2-q}
          &
          \nomod
          \\
          \noeq
          &
          &=\num{2}\pm\sqrt{\left(-\num{2}\right)^2-\left(-\num{5}\right)}
          &
          \nomod
          \\
          \noeq
          &
          &=\num{2}\pm\sqrt{\num{9}}
          &
          \nomod
          \\
          \noeq
          &
          &=\num{2}\pm\num{3}
          &
          \nomod
          \\
          \noeq
          &
          x_1&=-\num{1}
          &
          \nomod
          \\
          \noeq
          &
          x_2&=\num{5}
          &
          \nomod
        \end{alignat*}
      \endgroup
      %</OCTAVE>
      %myqsolve(4, -16, -20, 0, "x");
    \end{minipage}\bigskip\par
    \begin{minipage}[t]{0.49\linewidth}
      Scheitelpunkt f)
      \small
      \begin{equation*}
        \begin{split}
          f(x)&=-3x^2+12x+36
          \\
          &=-3\left(x^2-4x-12\right)
          \\
          &=-3\left(x^2-4x+4-4-12\right)
          \\
          &=-3\left((x-2)^2-16\right)
          \\
          &=-3(x-2)^2+48
          \\[1ex]
          \Rightarrow\quad&S\left(2\;\middle|\;48\right)
        \end{split}
      \end{equation*}
    \end{minipage}%
    \hfill
    \begin{minipage}[t]{0.49\linewidth}
      Nullstellen f)
      \small
      %<OCTAVE>
      \begingroup
        \newcommand{\vstrut}{\vphantom{\left(f_0^0\right)}}%
        \newcommand{\noeq}{\phantom{\Leftrightarrow}\vstrut&\quad}%
        \newcommand{\iseq}{\Leftrightarrow\vstrut&\quad}%
        \newcommand{\impl}{\Rightarrow\vstrut&\quad}%
        \newcommand{\nomod}{\quad&\phantom{|}}%
        \newcommand{\domod}[1]{\quad&|#1}%
        \begin{alignat*}{3}
          \noeq
          &
          \num{0}&=-\num{3}x^{2}+\num{12}x+\num{36}
          &
          \domod{:(-\num{3})}
          \\
          \iseq
          &
          \num{0}&=x^{2}-\num{4}x-\num{12}
          &
          \domod{\;\text{$pq$-Formel}}
          \\
          \noeq
          &
          p&=-\num{4}
          &
          \nomod
          \\
          \noeq
          &
          q&=-\num{12}
          &
          \nomod
          \\
          \noeq
          &
          x_{1,2}&=-\frac{p}{2}\pm\sqrt{\left(\frac{p}{2}\right)^2-q}
          &
          \nomod
          \\
          \noeq
          &
          &=\num{2}\pm\sqrt{\left(-\num{2}\right)^2-\left(-\num{12}\right)}
          &
          \nomod
          \\
          \noeq
          &
          &=\num{2}\pm\sqrt{\num{16}}
          &
          \nomod
          \\
          \noeq
          &
          &=\num{2}\pm\num{4}
          &
          \nomod
          \\
          \noeq
          &
          x_1&=-\num{2}
          &
          \nomod
          \\
          \noeq
          &
          x_2&=\num{6}
          &
          \nomod
        \end{alignat*}
      \endgroup
      %</OCTAVE>
      %myqsolve(-3, 12, 36, 0, "x");
    \end{minipage}\bigskip\par
    \begin{minipage}[t]{0.49\linewidth}
      Scheitelpunkt g)
      \small
      \begin{equation*}
        \begin{split}
          f(x)&=x^2-7x+12
          \\
          &=x^2-7x+\left(\frac{7}{2}\right)^2-\left(\frac{7}{2}\right)^2+12
          \\
          &=\left(x-\frac{7}{2}\right)^2-\frac{49}{4}+\frac{48}{4}
          \\
          &=\left(x-\frac{7}{2}\right)^2-\frac{1}{4}
          \\[1ex]
          \Rightarrow\quad&S\left(\frac{7}{2}\;\middle|\;-\frac{1}{4}\right)
        \end{split}
      \end{equation*}
    \end{minipage}%
    \hfill
    \begin{minipage}[t]{0.49\linewidth}
      Nullstellen g)
      \small
      %<OCTAVE>
      \begingroup
        \newcommand{\vstrut}{\vphantom{\left(f_0^0\right)}}%
        \newcommand{\noeq}{\phantom{\Leftrightarrow}\vstrut&\quad}%
        \newcommand{\iseq}{\Leftrightarrow\vstrut&\quad}%
        \newcommand{\impl}{\Rightarrow\vstrut&\quad}%
        \newcommand{\nomod}{\quad&\phantom{|}}%
        \newcommand{\domod}[1]{\quad&|#1}%
        \begin{alignat*}{3}
          \noeq
          &
          \num{0}&=x^{2}-\num{7}x+\num{12}
          &
          \domod{\;\text{$pq$-Formel}}
          \\
          \noeq
          &
          p&=-\num{7}
          &
          \nomod
          \\
          \noeq
          &
          q&=\num{12}
          &
          \nomod
          \\
          \noeq
          &
          x_{1,2}&=-\frac{p}{2}\pm\sqrt{\left(\frac{p}{2}\right)^2-q}
          &
          \nomod
          \\
          \noeq
          &
          &=\frac{\num{7}}{\num{2}}\pm\sqrt{\left(-\frac{\num{7}}{\num{2}}\right)^2-\num{12}}
          &
          \nomod
          \\
          \noeq
          &
          &=\frac{\num{7}}{\num{2}}\pm\sqrt{\frac{\num{1}}{\num{4}}}
          &
          \nomod
          \\
          \noeq
          &
          &=\frac{\num{7}}{\num{2}}\pm\frac{\num{1}}{\num{2}}
          &
          \nomod
          \\
          \noeq
          &
          x_1&=\num{3}
          &
          \nomod
          \\
          \noeq
          &
          x_2&=\num{4}
          &
          \nomod
        \end{alignat*}
      \endgroup
      %</OCTAVE>
      %myqsolve(1, -7, 12, 0, "x");
    \end{minipage}\bigskip\par
    \begin{minipage}[t]{0.49\linewidth}
      Scheitelpunkt h)
      \small
      \begin{equation*}
        \begin{split}
          f(x)&=-\frac{9}{4}x^2+6x-3
          \\[1ex]
          &=-\frac{9}{4}\left(x^2-\frac{8}{3}x+\frac{4}{3}\right)
          \\[1ex]
          &=-\frac{9}{4}\left(x^2-\frac{8}{3}x+\frac{16}{9}-\frac{16}{9}+\frac{4}{3}\right)
          \\[1ex]
          &=-\frac{9}{4}\left(\left(x-\frac{4}{3}\right)^2-\frac{4}{9}\right)
          \\[1ex]
          &=-\frac{9}{4}\left(x-\frac{4}{3}\right)^2+1
          \\[2ex]
          \Rightarrow\quad&S\left(\frac{4}{3}\;\middle|\;1\right)
        \end{split}
      \end{equation*}
      %polyval([-9/4 6 -3], 4/3)
    \end{minipage}%
    \hfill
    \begin{minipage}[t]{0.49\linewidth}
      Nullstellen h)
      \small
      %<OCTAVE>
      \begingroup
        \newcommand{\vstrut}{\vphantom{\left(f_0^0\right)}}%
        \newcommand{\noeq}{\phantom{\Leftrightarrow}\vstrut&\quad}%
        \newcommand{\iseq}{\Leftrightarrow\vstrut&\quad}%
        \newcommand{\impl}{\Rightarrow\vstrut&\quad}%
        \newcommand{\nomod}{\quad&\phantom{|}}%
        \newcommand{\domod}[1]{\quad&|#1}%
        \begin{alignat*}{3}
          \noeq
          &
          \num{0}&=-\frac{\num{9}}{\num{4}}x^{2}+\num{6}x-\num{3}
          &
          \domod{:\left(-\frac{\num{9}}{\num{4}}\right)}
          \\
          \iseq
          &
          \num{0}&=x^{2}-\frac{\num{8}}{\num{3}}x+\frac{\num{4}}{\num{3}}
          &
          \domod{\;\text{$pq$-Formel}}
          \\
          \noeq
          &
          p&=-\frac{\num{8}}{\num{3}}
          &
          \nomod
          \\
          \noeq
          &
          q&=\frac{\num{4}}{\num{3}}
          &
          \nomod
          \\
          \noeq
          &
          x_{1,2}&=-\frac{p}{2}\pm\sqrt{\left(\frac{p}{2}\right)^2-q}
          &
          \nomod
          \\
          \noeq
          &
          &=\frac{\num{4}}{\num{3}}\pm\sqrt{\left(-\frac{\num{4}}{\num{3}}\right)^2-\frac{\num{4}}{\num{3}}}
          &
          \nomod
          \\
          \noeq
          &
          &=\frac{\num{4}}{\num{3}}\pm\sqrt{\frac{\num{4}}{\num{9}}}
          &
          \nomod
          \\
          \noeq
          &
          &=\frac{\num{4}}{\num{3}}\pm\frac{\num{2}}{\num{3}}
          &
          \nomod
          \\
          \noeq
          &
          x_1&=\frac{\num{2}}{\num{3}}
          &
          \nomod
          \\
          \noeq
          &
          x_2&=\num{2}
          &
          \nomod
        \end{alignat*}
      \endgroup
      %</OCTAVE>
      %myqsolve(-9/4, 6, -3, 0, "x");
    \end{minipage}\bigskip\par
    \begin{minipage}[t]{0.49\linewidth}
      Scheitelpunkt i)
      \small
      \begin{equation*}
        \begin{split}
          f(x)&=\frac{1}{2}x^2+\frac{3}{4}x+\frac{1}{4}
          \\[1ex]
          &=\frac{1}{2}\left(x^2+\frac{3}{2}x+\frac{1}{2}\right)
          \\[1ex]
          &=\frac{1}{2}\left(x^2+\frac{3}{2}x+\frac{9}{16}-\frac{9}{16}+\frac{1}{2}\right)
          \\[1ex]
          &=\frac{1}{2}\left(\left(x+\frac{3}{4}\right)^2-\frac{1}{16}\right)
          \\[1ex]
          &=\frac{1}{2}\left(x+\frac{3}{4}\right)^2-\frac{1}{32}
          \\[2ex]
          \Rightarrow\quad&S\left(-\frac{3}{4}\;\middle|\;-\frac{1}{32}\right)
        \end{split}
      \end{equation*}
      %rats(polyval([1/2 3/4 1/4], -3/4))
    \end{minipage}%
    \hfill
    \begin{minipage}[t]{0.49\linewidth}
      Nullstellen i)
      \small
      %<OCTAVE>
      \begingroup
        \newcommand{\vstrut}{\vphantom{\left(f_0^0\right)}}%
        \newcommand{\noeq}{\phantom{\Leftrightarrow}\vstrut&\quad}%
        \newcommand{\iseq}{\Leftrightarrow\vstrut&\quad}%
        \newcommand{\impl}{\Rightarrow\vstrut&\quad}%
        \newcommand{\nomod}{\quad&\phantom{|}}%
        \newcommand{\domod}[1]{\quad&|#1}%
        \begin{alignat*}{3}
          \noeq
          &
          \num{0}&=\frac{\num{1}}{\num{2}}x^{2}+\frac{\num{3}}{\num{4}}x+\frac{\num{1}}{\num{4}}
          &
          \domod{:\frac{\num{1}}{\num{2}}}
          \\
          \iseq
          &
          \num{0}&=x^{2}+\frac{\num{3}}{\num{2}}x+\frac{\num{1}}{\num{2}}
          &
          \domod{\;\text{$pq$-Formel}}
          \\
          \noeq
          &
          p&=\frac{\num{3}}{\num{2}}
          &
          \nomod
          \\
          \noeq
          &
          q&=\frac{\num{1}}{\num{2}}
          &
          \nomod
          \\
          \noeq
          &
          x_{1,2}&=-\frac{p}{2}\pm\sqrt{\left(\frac{p}{2}\right)^2-q}
          &
          \nomod
          \\
          \noeq
          &
          &=-\frac{\num{3}}{\num{4}}\pm\sqrt{\left(\frac{\num{3}}{\num{4}}\right)^2-\frac{\num{1}}{\num{2}}}
          &
          \nomod
          \\
          \noeq
          &
          &=-\frac{\num{3}}{\num{4}}\pm\sqrt{\frac{\num{1}}{\num{16}}}
          &
          \nomod
          \\
          \noeq
          &
          &=-\frac{\num{3}}{\num{4}}\pm\frac{\num{1}}{\num{4}}
          &
          \nomod
          \\
          \noeq
          &
          x_1&=-\num{1}
          &
          \nomod
          \\
          \noeq
          &
          x_2&=-\frac{\num{1}}{\num{2}}
          &
          \nomod
        \end{alignat*}
      \endgroup
      %</OCTAVE>
      %myqsolve(1/2, 3/4, 1/4, 0, "x");
    \end{minipage}%
    % </OUTCOME>
  \fi
\end{exercise}

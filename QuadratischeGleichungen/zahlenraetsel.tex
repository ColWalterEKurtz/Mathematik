\begin{exercise}
      {ID-8e59e6b9cb5139e6d1a7458593db8ed58dd852d4}
      {Zahlenrätsel}
  \ifproblem\problem
    \begin{enumerate}[a)]
      \item Multipliziert man eine Zahl mit der Hälfte dieser Zahl, so erhält man
            $162$. Wie lautet die gesuchte Zahl?
      \item Multipliziert man das Dreifache einer Zahl mit einem Viertel dieser
            Zahl, so erhält man $108$. Wie lautet die gesuchte Zahl?
      \item Das Produkt zweier aufeinander folgender ganzer Zahlen ist um $55$ größer
            als ihre Summe. Wie lauten die gesuchten Zahlen?
      \item Die Quadrate dreier aufeinander folgender Zahlen ergeben zusammen
            \num{1202}. Wie lauten die gesuchten Zahlen?
    \end{enumerate}
  \fi
  \ifoutline\outline
    \newcommand{\exnum}[1]{\text{\makebox[2em][r]{#1}}&\quad}
    \begin{align*}
      \exnum{a)}x\cdot\frac{x}{2}=162  & \exnum{b)}3x\cdot\frac{x}{4}=108 \\[2ex]
      \exnum{c)}x\cdot(x+1)=x+(x+1)+55 & \exnum{d)}(x-1)^2+x^2+(x+1)^2=1202
    \end{align*}
  \fi
  \ifoutcome\outcome
    \small
    \begin{alignat*}{3}
            \text{a)}&\qquad & x\cdot\frac{x}{2}&=162 & \qquad&|\cdot2             \\[2ex]
      \Leftrightarrow&\qquad &               x^2&=324 & \qquad&|\;\sqrt{\;\cdot\;} \\[2ex]
      \Leftrightarrow&\qquad &      \vert x\vert&=18  & \qquad&
    \end{alignat*}
    Sowohl 18, als auch -18 besitzen die gewünschte Eigenschaft.

    \begin{alignat*}{3}
            \text{b)}&\qquad &   3x\cdot\frac{x}{4}&=108   & \qquad&|:3\quad|\cdot4     \\[2ex]
      \Leftrightarrow&\qquad &                  x^2&=144   & \qquad&|\;\sqrt{\;\cdot\;} \\[2ex]
      \Leftrightarrow&\qquad &         \vert x\vert&=12    & \qquad&
    \end{alignat*}
    Sowohl 12, als auch -12 besitzen die gewünschte Eigenschaft.

    \newcommand{\vstrut}{\text{\rule[-3.5ex]{0pt}{8ex}}}%
    \begin{alignat*}{2}
            \text{c)}&\vstrut\qquad & x\cdot(x+1)&=x+(x+1)+55                                                                     \\
      \Leftrightarrow&\vstrut\qquad &       x^2+x&=2x+56\qquad\qquad|-2x\quad|-56                                                 \\
      \Leftrightarrow&\vstrut\qquad &           0&=x^2-x-56
                                                  =x^2-x+\frac{1}{4}-\frac{1}{4}-56                                               \\
                     &\vstrut\qquad &            &=\left(x-\frac{1}{2}\right)^2-\frac{225}{4}
                                                  =\left(x-\frac{1}{2}+\frac{15}{2}\right)\left(x-\frac{1}{2}-\frac{15}{2}\right) \\
                     &\vstrut\qquad &            &=\left(x+7\right)\left(x-8\right)
    \end{alignat*}
    Also besitzen sowohl die Zahlen -7 und -6, als auch
    die Zahlen 8 und 9 die gewünschte Eigenschaft.

    \begin{alignat*}{3}
            \text{d)}&\qquad &   (x-1)^2+x^2+(x+1)^2&      & \qquad&                    \\[2ex]
                    =&\qquad & x^2-2x+1+x^2+x^2+2x+1&      & \qquad&                    \\[2ex]
                    =&\qquad &                3x^2+2&=1202 & \qquad&|-2\quad|:3         \\[2ex]
      \Leftrightarrow&\qquad &                 x^2  &=400  & \qquad&|\;\sqrt{\;\cdot\;} \\[2ex]
      \Leftrightarrow&\qquad &        \vert x\vert  &=20   & \qquad&
    \end{alignat*}
    Also besitzen sowohl die Zahlen -21, -20 und -19, als auch
    die Zahlen 19, 20 und 21 die gewünschte Eigenschaft.
  \fi
\end{exercise}

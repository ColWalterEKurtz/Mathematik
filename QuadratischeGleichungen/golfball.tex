\begin{exercise}
      {ID-49712269135e7d4b30bf0d643149cc3fa2bb48c2}
      {Golfball}
  \ifproblem\problem
    Der Flug eines Golfballs kann näherungsweise durch eine Parabel modelliert werden.
    Der Graph ist in folgendem Koordinatensystem abgebildet.
    \begin{center}
      \newcommand{\xmark}[2]{\draw (#1, 2) -- (#1, -2) node[below]{\vphantom{\ensuremath{(}}#2};}%
      \newcommand{\ymark}[2]{\draw (2, #1) -- (-2, #1) node[left]{\vphantom{\ensuremath{(}}#2};}%
      \begin{tikzpicture}[scale=0.0625]
        \draw[line width=0.66pt, ->, >=stealth] (-20, 0) -- (150, 0) node[below] {$x$};;
        \xmark{-10}{}
        \xmark{10}{10}
        \xmark{20}{}
        \xmark{30}{}
        \xmark{40}{}
        \xmark{50}{50}
        \xmark{60}{}
        \xmark{70}{}
        \xmark{80}{}
        \xmark{90}{}
        \xmark{100}{100}
        \xmark{110}{}
        \xmark{120}{}
        \xmark{130}{}
        \xmark{140}{}
        \draw[line width=0.66pt, ->, >=stealth] (0, -20) -- (0, 50) node[left] {$y$};
        \ymark{-10}{}
        \ymark{10}{10}
        \ymark{20}{20}
        \ymark{30}{30}
        \ymark{40}{40}
        \begin{scope}
          \draw plot[smooth] coordinates
          {
            (-10.000,  -9.700)  ( -9.000,  -8.667)  ( -8.000,  -7.648)  ( -7.000,  -6.643)
            ( -6.000,  -5.652)  ( -5.000,  -4.675)  ( -4.000,  -3.712)  ( -3.000,  -2.763)
            ( -2.000,  -1.828)  ( -1.000,  -0.907)  (  0.000,   0.000)  (  1.000,   0.893)
            (  2.000,   1.772)  (  3.000,   2.637)  (  4.000,   3.488)  (  5.000,   4.325)
            (  6.000,   5.148)  (  7.000,   5.957)  (  8.000,   6.752)  (  9.000,   7.533)
            ( 10.000,   8.300)  ( 11.000,   9.053)  ( 12.000,   9.792)  ( 13.000,  10.517)
            ( 14.000,  11.228)  ( 15.000,  11.925)  ( 16.000,  12.608)  ( 17.000,  13.277)
            ( 18.000,  13.932)  ( 19.000,  14.573)  ( 20.000,  15.200)  ( 21.000,  15.813)
            ( 22.000,  16.412)  ( 23.000,  16.997)  ( 24.000,  17.568)  ( 25.000,  18.125)
            ( 26.000,  18.668)  ( 27.000,  19.197)  ( 28.000,  19.712)  ( 29.000,  20.213)
            ( 30.000,  20.700)  ( 31.000,  21.173)  ( 32.000,  21.632)  ( 33.000,  22.077)
            ( 34.000,  22.508)  ( 35.000,  22.925)  ( 36.000,  23.328)  ( 37.000,  23.717)
            ( 38.000,  24.092)  ( 39.000,  24.453)  ( 40.000,  24.800)  ( 41.000,  25.133)
            ( 42.000,  25.452)  ( 43.000,  25.757)  ( 44.000,  26.048)  ( 45.000,  26.325)
            ( 46.000,  26.588)  ( 47.000,  26.837)  ( 48.000,  27.072)  ( 49.000,  27.293)
            ( 50.000,  27.500)  ( 51.000,  27.693)  ( 52.000,  27.872)  ( 53.000,  28.037)
            ( 54.000,  28.188)  ( 55.000,  28.325)  ( 56.000,  28.448)  ( 57.000,  28.557)
            ( 58.000,  28.652)  ( 59.000,  28.733)  ( 60.000,  28.800)  ( 61.000,  28.853)
            ( 62.000,  28.892)  ( 63.000,  28.917)  ( 64.000,  28.928)  ( 65.000,  28.925)
            ( 66.000,  28.908)  ( 67.000,  28.877)  ( 68.000,  28.832)  ( 69.000,  28.773)
            ( 70.000,  28.700)  ( 71.000,  28.613)  ( 72.000,  28.512)  ( 73.000,  28.397)
            ( 74.000,  28.268)  ( 75.000,  28.125)  ( 76.000,  27.968)  ( 77.000,  27.797)
            ( 78.000,  27.612)  ( 79.000,  27.413)  ( 80.000,  27.200)  ( 81.000,  26.973)
            ( 82.000,  26.732)  ( 83.000,  26.477)  ( 84.000,  26.208)  ( 85.000,  25.925)
            ( 86.000,  25.628)  ( 87.000,  25.317)  ( 88.000,  24.992)  ( 89.000,  24.653)
            ( 90.000,  24.300)  ( 91.000,  23.933)  ( 92.000,  23.552)  ( 93.000,  23.157)
            ( 94.000,  22.748)  ( 95.000,  22.325)  ( 96.000,  21.888)  ( 97.000,  21.437)
            ( 98.000,  20.972)  ( 99.000,  20.493)  (100.000,  20.000)  (101.000,  19.493)
            (102.000,  18.972)  (103.000,  18.437)  (104.000,  17.888)  (105.000,  17.325)
            (106.000,  16.748)  (107.000,  16.157)  (108.000,  15.552)  (109.000,  14.933)
            (110.000,  14.300)  (111.000,  13.653)  (112.000,  12.992)  (113.000,  12.317)
            (114.000,  11.628)  (115.000,  10.925)  (116.000,  10.208)  (117.000,   9.477)
            (118.000,   8.732)  (119.000,   7.973)  (120.000,   7.200)  (121.000,   6.413)
            (122.000,   5.612)  (123.000,   4.797)  (124.000,   3.968)  (125.000,   3.125)
            (126.000,   2.268)  (127.000,   1.397)  (128.000,   0.512)  (129.000,  -0.387)
            (130.000,  -1.300)  (131.000,  -2.227)  (132.000,  -3.168)  (133.000,  -4.123)
            (134.000,  -5.092)  (135.000,  -6.075)  (136.000,  -7.072)  (137.000,  -8.083)
            (138.000,  -9.108)  (139.000, -10.147)  (140.000, -11.200)
          };
        \end{scope}
      \end{tikzpicture}
    \end{center}
    \begin{enumerate}[a)]
      \item Durch welche der folgenden Funktionsgleichungen kann die Flugbahn
            des Golfballs beschrieben werden? Begründe deine Antwort.
            \begin{equation*}
                y_1=\num{-0.007}x^2+\num{0.9}x
                \quad,\quad
                y_2=\num{0.007}x^2+\num{0.9}x
                \quad,\quad
                y_3=\num{-0.07}x
            \end{equation*}
      \item Gib die maximale Höhe des Golfballs an.
      \item Fliegt ein Ball, dessen Flugbahn durch die Punkte
            $P(0\mid0)$, $Q(10\mid\num{10.3})$ und $R(20\mid\num{19.2})$
            verläuft, höher bzw. weiter als der oben beschriebene Golfball?
            Begründe deine Antwort.
    \end{enumerate}
  \fi
  %\ifoutline\outline
  %\fi
  %\ifoutcome\outcome
  %\fi
\end{exercise}

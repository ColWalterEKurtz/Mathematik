\begin{exercise}
      {ID-e063e9b4f683984a9083d9c4e7d63ec894b533a6}
      {Fallhöhe}
  \ifproblem\problem\par
    \xya{} und \xyb{} stehen auf einer Klippe in Australien. Beide schätzen
    die Höhe der Klippe. \xya{} erinnert sich, dass sich die Höhe $h$ bestimmen
    lässt, indem man einen Stein fallen lässt und die Fallzeit $t$ misst. Die
    Fallhöhe des Steins in Metern entspricht dann etwa fünfmal der Fallzeit in
    Sekunden zum Quadrat.
    \begin{enumerate}[a)]
      \item Stelle eine Funktionsgleichung auf, mit der man die Höhe der Klippe
            bestimmen kann.
      \item \xya{} lässt einen Stein fallen und \xyb{} misst eine Fallzeit
            von \num{2} Sekunden. Wie hoch ist die Klippe ungefähr?
      \item Eine andere Klippe soll laut Reiseführer \SI{45}{\metre} hoch sein.
            Wie lange müsste ein fallengelassener Stein fliegen?
    \end{enumerate}
  \fi
  %\ifoutline\outline\par
  %\fi
  \ifoutcome\outcome
    \begin{enumerate}[a)]
      \item Mit folgender Funktionsgleichung kann man die Höhe der Klippe
            bestimmen:
            \begin{equation*}
              h(t)=5t^2
            \end{equation*}
      \item Die Höhe der Klippe erhält man, wenn man \num{2} in die Funktion aus a)
            einsetzt:
            \begin{equation*}
              h(\num{2})=5\cdot\num{2}^2=\num{20}
            \end{equation*}
            Die Klippe ist also etwa \SI{20}{\metre} hoch.
      \item Zunächst bietet es sich an, die
            Gleichung nach der gesuchten Größe
            aufzulösen:
            \begin{equation*}
              h(t)=5t^2\quad\Rightarrow\quad t=\sqrt{\frac{h(t)}{5}}
            \end{equation*}
            Das Einsetzen der gegebenen Höhe $h(t)=\SI{45}{\metre}$
            liefert dann die zugehörige Fallzeit:
            \begin{equation*}
              t=\sqrt{\frac{\num{45}}{5}}=\num{3}
            \end{equation*}
            Der Stein würde also etwa \num{3} Sekunden lang fallen.
    \end{enumerate}
  \fi
\end{exercise}

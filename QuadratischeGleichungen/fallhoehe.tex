\begin{exercise}
      {ID-e063e9b4f683984a9083d9c4e7d63ec894b533a6}
      {Fallhöhe}
  \ifproblem\problem
    \xya{} und \xyb{} stehen auf einer Klippe in Australien. Beide schätzen
    die Höhe der Klippe. \xya{} erinnert sich, dass sich die Höhe $h$ bestimmen
    lässt, indem man einen Stein fallen lässt und die Fallzeit $t$ misst. Die
    Fallhöhe des Steins in m entspricht dann etwa fünfmal der Fallzeit in
    $s$ zum Quadrat.
    \begin{enumerate}[a)]
      \item Stelle eine Funktionsgleichung auf, mit der man die Höhe der Klippe
            bestimmen kann.
      \item \xya{} lässt einen Stein fallen und \xyb{} misst eine Fallzeit
            von 2 Sekunden. Wie hoch ist die Klippe ungefähr?
      \item Eine andere Klippe soll laut Reiseführer \simeter{32} hoch sein.
            Wie lange müsste ein fallengelassener Stein fliegen?
    \end{enumerate}
  \fi
  %\ifoutline\outline
  %\fi
  %\ifoutcome\outcome
  %\fi
\end{exercise}

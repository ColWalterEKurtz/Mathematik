\begin{exercise}
      {ID-94961b67802fcce2d16892299ee60f9d439001d3}
      {Kaninchenstall}
  \ifproblem\problem\par
    % <PROBLEM>
    \xxa{} möchte für ihre Kaninchen im Garten einen rechteckigen Stall
    bauen. Sie möchte den Stall so bauen, dass er auf einer Seite von der
    Hausmauer begrenzt wird. Die anderen drei Seiten werden durch Maschendraht
    begrenzt.
    \begin{enumerate}[a)]
      \item Sie hat \SI{16.8}{\metre} Maschendraht zur Verfügung. Wie lang muss
            \xxa{} die Seitenlängen des Stalls wählen, damit die Auslauffläche für
            die Kaninchen möglichst groß wird?
      \item Wie lang muss \xxa{} die Seitenlängen wählen, wenn sie den
            Kaninchenstall in die Ecke von zwei Mauern baut, also nur zwei
            Seiten mit dem Zaun begrenzen muss?
    \end{enumerate}
    % </PROBLEM>
  \fi
  \ifoutline\outline\par
    % <OUTLINE>
    Je nachdem ob \xxa{} eine oder zwei Mauern nutzen kann,
    ergibt sich für den Stall folgender Grundriss:
    \begin{center}
      \begin{tikzpicture}[line width=0.6pt]
        % default colors
        \newcommand{\colr}{Red};%
        \newcommand{\colg}{ForestGreen};%
        \newcommand{\colb}{Cerulean};%
        \newcommand{\coly}{YellowOrange};%
        \newcommand{\cola}{Black!35!White};%
        \newcommand{\cole}{Black!55!White};%
        % eine Mauer
        \begin{scope}
          % Mauer
          \filldraw[fill=\cola, draw=black]
            (0, 0) rectangle (5cm, 5mm);
          % Zaun
          \draw (1, 0)
                -- (1, -2)
                -- node[below]{$a$} (4, -2)
                -- node[right]{$b$} (4,  0);
          % Formel
          \node at (2.5, -1) {$A=a\cdot b$};
        \end{scope}
        % zwei Mauern
        \begin{scope}[xshift=8cm]
          % Mauer
          \filldraw[fill=\cola, draw=black]
               ( 0.0,  0.0)
            -- ( 4.5,  0.0)
            -- ( 4.5,  0.5)
            -- (-0.5,  0.5)
            -- (-0.5, -3.0)
            -- ( 0.0, -3.0)
            -- cycle;
          % Zaun
          \draw (0, -2)
                -- node[below]{$a$} (3, -2)
                -- node[right]{$b$} (3,  0);
          % Formel
          \node at (1.5, -1) {$A=a\cdot b$};
        \end{scope}
      \end{tikzpicture}
    \end{center}
    Da man weiß, wie viel Meter Zaun zum Bau des
    Kaninchenstalls zur Verfügung steht, kann man
    aus der Formel zur Berechnung des Flächeninhalts
    eine Variable entfernen:
    \begin{equation*}
      a+2b=\num{16.8}
      \quad\Rightarrow\quad
      a=\num{16.8}-2b
      \quad\Rightarrow\quad
      A=\left(\num{16.8}-2b\right)\cdot b
      =-2b^2+\num{16.8}b
    \end{equation*}
    bzw.
    \begin{equation*}
      a+b=\num{16.8}
      \quad\Rightarrow\quad
      a=\num{16.8}-b
      \quad\Rightarrow\quad
      A=\left(\num{16.8}-b\right)\cdot b
      =-b^2+\num{16.8}b
    \end{equation*}
    Der Scheitelpunkt von $A(b)$ liefert
    nun den maximalen Flächeninhalt und die
    zugehörige Seitenlänge $b$.
    % </OUTLINE>
  \fi
  %\ifoutcome\outcome\par
    % <OUTCOME>
    % </OUTCOME>
  %\fi
\end{exercise}

\begin{exercise}
      {ID-e008dd06c5f6b097726911fa560770bd99f8b5f4}
      {Bogenbrücke}
  \ifproblem\problem
    Eine parabelförmige Bogenbrücke hat eine Spannweite von \simeter{223}.
    Ein Wanderer will die Höhe der Brücke bestimmen. Im Abstand von
    \simeter{1.2} zum Fußpunkt der Brücke (durch Fußschrittmessung) ist der
    Brückenbogen \simeter{2.0} hoch (durch Vergleich mit der Körpergröße).
    \begin{enumerate}[a)]
      \item Welche Höhe hat der Brückenbogen an seiner höchsten Stelle?
      \item Um wie viel Prozent ändert sich die ermittelte Brückenhöhe, wenn
            der Wanderer bei der Fußschrittmessung \sicm{10} weniger gemessen
            hätte?
    \end{enumerate}
  \fi
  %\ifoutline\outline
  %\fi
  %\ifoutcome\outcome
  %\fi
\end{exercise}

\begin{exercise}
      {ID-e008dd06c5f6b097726911fa560770bd99f8b5f4}
      {Bogenbrücke}
  \ifproblem\problem
    Eine parabelförmige Bogenbrücke hat eine Spannweite von \simeter{223}.
    Ein Wanderer will die Höhe der Brücke bestimmen. Im Abstand von
    \simeter{1.2} zum Fußpunkt der Brücke (durch Fußschrittmessung) ist der
    Brückenbogen \simeter{2} hoch (durch Vergleich mit der Körpergröße).
    \begin{enumerate}[a)]
      \item Welche Höhe hat der Brückenbogen an seiner höchsten Stelle?
      \item Um wie viel Prozent ändert sich die ermittelte Brückenhöhe, wenn
            der Wanderer bei der Fußschrittmessung \sicm{10} weniger gemessen
            hätte?
    \end{enumerate}
  \fi
  \ifoutline\outline
    \begin{enumerate}[a)]
      \item Gesucht ist die $y$-Koordinate des Scheitelpunkts einer nach
            unten geöffneten Parabel, deren Nullstellen bei \num{0} und \num{223}
            liegen, und die durch den Punkt $(\num{1.2}\mid\num{2})$ verläuft.
            Bekanntlich liegt die $x$-Koordinate des Scheitelpunkts genau in
            der Mitte zwischen den beiden Nullstellen.
      \item Wie hoch wäre der Brückenbogen, wenn die Parabel aus a) durch den Punkt
            $(\num{1.1}\mid\num{2})$ verliefe?
    \end{enumerate}
  \fi
  \ifoutcome\outcome
    \begin{enumerate}[a)]
      \newcommand{\vstrut}{\rule[-2.75ex]{0pt}{7ex}}%
      \setlength{\abovedisplayskip}{0pt}%
      \item \begin{equation*}
                \begin{split}
                                            f(0)=f(\num{223})=0 \quad&\vstrut\Rightarrow\quad f(x)=ax(x-\num{223})  \\
                                           f(\num{1.2})=\num{2} \quad&\vstrut\Rightarrow\quad a\approx\num{-0.0075} \\
                f(x)=\num{-0.0075}\cdot x^2+\num{1.6757}\cdot x \quad&\vstrut\Rightarrow\quad f\left(\frac{223}{2}\right)\approx\num{93.42}
              \end{split}
            \end{equation*}
            Der Brückenbogen ist an seiner höchsten Stelle also etwa \simeter{93.4} hoch.

      \item \begin{equation*}
                \begin{split}
                                            f(0)=f(\num{223})=0 \quad&\vstrut\Rightarrow\quad f(x)=ax(\num{223})    \\
                                           f(\num{1.1})=\num{2} \quad&\vstrut\Rightarrow\quad a\approx\num{-0.0082} \\
                f(x)=\num{-0.0082}\cdot x^2+\num{1.8272}\cdot x \quad&\vstrut\Rightarrow\quad f\left(\frac{223}{2}\right)\approx\num{101.87}
              \end{split}
            \end{equation*}
            Der Brückenbogen wäre bei dieser Messung an seiner höchsten Stelle
            etwa \simeter{101.9} hoch und relativ gesehen etwa \pc{9} höher als bei
            der ersten Messung.
    \end{enumerate}
  \fi
\end{exercise}

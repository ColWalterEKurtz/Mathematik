\begin{exercise}
      {ID-59056644d772a331e4e82c9e9fddf7cd8f51371a}
      {Gateway-Arch}
  \ifproblem\problem\par
    Der Gateway-Arch (1959--1965 gebaut) in St.\,Louis, Missouri, ist laut
    Angaben eines Reiseführers ein parabelförmiger Bogen aus rostfreiem
    Stahl. Er ist 630 Fuß (\emph{ft}) hoch und an seiner breitesten Stelle
    ebenso breit. Er soll als \glqq{}Tor zum Westen\grqq{} an den nach 1800
    einsetzenden Siedlerstrom nach Westen in den USA erinnern.
    \begin{enumerate}[a)]
      \item Wie breit ist der Bogen in 300\,\emph{ft} Höhe (in \emph{ft})?
      \item 1\,\emph{ft} entspricht \simeter{0.3048}. Bestimme eine
            Funktionsgleichung mit der man die Höhe des Gateway-Arch in Metern
            beschreiben kann.
      \item Bei einer Flugshow soll ein Flugzeug mit einer Flügelspannweite
            von \simeter{18} unter dem Bogen hindurch fliegen. Welche maximale
            Flughöhe muss der Pilot einhalten, wenn in vertikaler und horizontaler
            Richtung ein Sicherheitsabstand zum Bogen von \simeter{10} eingehalten
            werden muss?
    \end{enumerate}
  \fi
  %\ifoutline\outline\par
  %\fi
  %\ifoutcome\outcome\par
  %\fi
\end{exercise}

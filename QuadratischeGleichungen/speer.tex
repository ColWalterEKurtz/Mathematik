\begin{exercise}
      {ID-1025b37f79d7ba6a6d8720f146482e3a42cddabc}
      {Speer}
  \ifproblem\problem
    Mithilfe der Funktion $h(x)=\num{-0.02}x^{2}+\num{0.8}x+\num{1.8}$ kann die Flugkurve
    eines Speers beschrieben werden ($x$ und $h(x)$ in $m$).
    \begin{enumerate}[a)]
      \item Was bedeutet $h(0)$ im Anwendungskontext?
      \item Wie weit fliegt der Speer?
      \item Wie hoch ist der Speer am höchsten Punkt seiner Flugbahn?
    \end{enumerate}
  \fi
  \ifoutline\outline
    \begin{enumerate}[a)]
      \setcounter{enumi}{1}
      \item Gesucht ist die positive Nullstelle der Parabel. ($pq$-Formel)
      \item Gesucht ist der Scheitelpunkt der Parabel. (Scheitelpunktform)
    \end{enumerate}
  \fi
  \ifoutcome\outcome
    \begin{enumerate}[a)]
      \item $h(0)$ ist die Höhe des Speeres in dem Moment, in dem der Sportler ihn loslässt.
      \item Der Speer fliegt ca. \simeter{42.14} weit.
      \item Nach ca. \simeter{20} erreicht der Speer eine maximale Höhe von ca. \simeter{9.80}.
    \end{enumerate}
  \fi
\end{exercise}

\begin{exercise}
      {ID-778c2e6e9f7cb0440d8411132ce0875fabe3e6c8}
      {Speer}
  \ifproblem\problem\par
    % <PROBLEM>
    Mithilfe der folgenden Funktion kann die Flugkurve
    eines Speers beschrieben werden:
    \begin{equation*}
      h(x)=-\frac{\num{1}}{\num{50}}x^{\num{2}}+\frac{\num{4}}{\num{5}}x+\frac{\num{9}}{\num{5}}
    \end{equation*}
    Alle Werte von $x$ und $h(x)$ stellen Maßzahlen in der
    Maßeinheit Meter dar.
    \begin{enumerate}[a)]
      \squeeze
      \item Was bedeutet $h(0)$ im Anwendungskontext?
      \item Wie weit fliegt der Speer?
      \item Wie hoch ist der Speer am höchsten Punkt seiner Flugbahn?
    \end{enumerate}
    % </PROBLEM>
  \fi
  %\ifoutline\outline\par
    % <OUTLINE>
    % </OUTLINE>
  %\fi
  \ifoutcome\outcome
    % <OUTCOME>
    \begin{enumerate}[a)]
      \item In diesem Anwendungskontext kann $h(0)$
            als die Höhe interpretiert werden,
            in der der Speer abgeworfen wird
            (\SI{1.80}{\metre}).
      \item Die Entfernung, in der der Speer wieder auf
            dem Boden auftrifft, entspricht der positiven
            Nullstelle der Funktion $h$.
            %<OCTAVE>
            \begingroup
              \newcommand{\vstrut}{\vphantom{\left(f_0^0\right)}}%
              \newcommand{\noeq}{\phantom{\Leftrightarrow}\vstrut&\quad}%
              \newcommand{\iseq}{\Leftrightarrow\vstrut&\quad}%
              \newcommand{\impl}{\Rightarrow\vstrut&\quad}%
              \newcommand{\nomod}{\quad&\phantom{|}}%
              \newcommand{\domod}[1]{\quad&|#1}%
              \begin{alignat*}{3}
                \noeq
                &
                \num{0}&=-\frac{\num{1}}{\num{50}}x^{2}+\frac{\num{4}}{\num{5}}x+\frac{\num{9}}{\num{5}}
                &
                \domod{:\left(-\frac{\num{1}}{\num{50}}\right)}
                \\
                \iseq
                &
                \num{0}&=x^{2}-\num{40}x-\num{90}
                &
                \domod{\;\text{$pq$-Formel}}
                \\
                \noeq
                &
                p&=-\num{40}
                &
                \nomod
                \\
                \noeq
                &
                q&=-\num{90}
                &
                \nomod
                \\
                \noeq
                &
                x_{1,2}&=-\frac{p}{2}\pm\sqrt{\left(\frac{p}{2}\right)^2-q}
                &
                \nomod
                \\
                \noeq
                &
                &=\num{20}\pm\sqrt{\left(-\num{20}\right)^2-\left(-\num{90}\right)}
                &
                \nomod
                \\
                \noeq
                &
                &=\num{20}\pm\sqrt{\num{490}}
                &
                \nomod
                \\
                \noeq
                &
                x_1&\approx-\num{2.135944}
                &
                \nomod
                \\
                \noeq
                &
                x_2&\approx\num{42.135944}
                &
                \nomod
              \end{alignat*}
            \endgroup
            %</OCTAVE>
            %myqsolve(-1/50, 4/5, 9/5, 0, "x");
            Der Speer fliegt also erwa \SI{42.14}{\metre} weit.
      \item Die maximale Flughöhe erreicht der Speer
            im Scheitelpunkt der Funktion $h$.
            \begin{equation*}
              \begin{split}
                h(x)&=-\frac{\num{1}}{\num{50}}x^{2}+\frac{\num{4}}{\num{5}}x+\frac{\num{9}}{\num{5}}
                =-\frac{\num{1}}{\num{50}}\cdot\left(x^{2}-\num{40}x-\num{90}\right)
                \\[1ex]
                &=-\frac{\num{1}}{\num{50}}\cdot\left(x^{2}-\num{40}x+\num{400}-\num{400}-\num{90}\right)
                \\[1ex]
                &=-\frac{\num{1}}{\num{50}}\cdot\left(\left(x-20\right)^{2}-\num{490}\right)
                =-\frac{\num{1}}{\num{50}}\cdot\left(x-20\right)^{2}+\frac{\num{49}}{\num{5}}
                %p  = [-1/50 4/5 9/5];
                %ps = mypolystr(p, "x", [0 0 0 0 1]);
                %printf("h(x)&=%s\n", ps);
                %a  = p(1);
                %as = myn2s(a, 0,0,0,0,1);
                %p  = 1/a .* [-1/50 4/5 9/5];
                %ps = mypolystr(p, "x", [0 0 0 0 1]);
                %printf("=%s\\cdot\\left(%s\\right)\n", as, ps);
              \end{split}
            \end{equation*}
            Nach \SI{20}{\metre} erreicht der Speer
            eine maximale Höhe von \SI{9.80}{\metre}.
    \end{enumerate}
    % </OUTCOME>
  \fi
\end{exercise}

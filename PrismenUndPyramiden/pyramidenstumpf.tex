\begin{exercise}
      {ID-90e44bba82cbd1b5985b3aba6158140367a7856e}
      {Pyramidenstumpf}
  \ifproblem\problem
    Aus dem Stumpf einer quadratischen Pyramide wurde eine auf dem Kopf
    stehende kleine quadratische Pyramide ausgeschnitten. Ihre Spitze
    reicht bis auf den Boden des Pyramidenstumpfs. Berechne das verbleibende
    Volumen dieses Pyramidenstumpfs.\par
    \begin{minipage}{0.45\linewidth}%
      \begin{tikzpicture}[scale=0.4]
        % Koordinaten
        \coordinate (A) at ( -6.8750,  -1.2500);
        \coordinate (B) at (  3.1250,  -1.2500);
        \coordinate (C) at (  6.8750,   1.2500);
        \coordinate (D) at ( -3.1250,   1.2500);
        \coordinate (E) at ( -4.1250,   4.2500);
        \coordinate (F) at (  1.8750,   4.2500);
        \coordinate (G) at (  4.1250,   5.7500);
        \coordinate (H) at ( -1.8750,   5.7500);
        \coordinate (I) at (  0.0000,   0.0000);
        % Hilfspunkte
        \coordinate (AB) at ($(A)!0.5!(B)$);
        \coordinate (CD) at ($(C)!0.5!(D)$);
        \coordinate (BC) at ($(B)!0.5!(C)$);
        \coordinate (AD) at ($(A)!0.5!(D)$);
        \coordinate (FG) at ($(F)!0.5!(G)$);
        \coordinate (GH) at ($(G)!0.5!(H)$);
        \coordinate (EH) at ($(E)!0.5!(H)$);
        \coordinate (VBT) at ([xshift=5cm]FG);
        \coordinate (VBB) at ([xshift=3cm]BC);
        \begin{scope}[line width=0.6pt]
          % sichtbare Linien
          \begin{scope}
            \draw (A) -- (B) -- (C);
            \draw (E) -- (F) -- (G) -- (H) -- cycle;
            \draw (A) -- (E);
            \draw (B) -- (F);
            \draw (C) -- (G);
          \end{scope}
          % verdeckte Linien
          \begin{scope}[style=dashed]
            \draw (C) -- (D) -- (A);
            \draw (D) -- (H);
            \draw (E) -- (I);
            \draw (F) -- (I);
            \draw (G) -- (I);
            \draw (H) -- (I);
          \end{scope}
          % Hilfslinien
          \begin{scope}[style=dotted]
            \draw (AB) -- (CD);
            \draw (BC) -- (AD);
            \draw (FG) -- (EH);
            \draw (BC) -- (VBB);
            \draw (FG) -- (VBT);
          \end{scope}
          \draw[decorate, decoration=brace] (VBT) -- node[right=2pt]{$h$} (VBB);
        \end{scope}
        \node[below] at (AB) {$a$};
        \node[above] at (GH) {$b$};
      \end{tikzpicture}
    \end{minipage}%
    \hfill
    \begin{minipage}{0.54\linewidth}%
      \flushright
      \newcommand{\cbox}[1]{\vphantom{\ensuremath{\displaystyle\Big(}}\makebox[2.4em][r]{#1}}%
      \begin{tabular}{|c|r|r|r|r|r|}
        \hline
        $a$ & \num{6} & \num{8} & \num{9} & \num{15} & \num{20} \\
        $b$ & \num{4} & \num{7} & \num{6} & \num{9} & \num{14} \\
        $h$ & \num{7} & \num{3} & \num{5} & \num{8} & \num{6} \\
        \hline
        $V$ & \cbox{} & \cbox{} & \cbox{} & \cbox{} & \cbox{} \\
        \hline
      \end{tabular}%
    \end{minipage}
  \fi
  %\ifoutline\outline
  %\fi
  \ifoutcome\outcome
    \begin{center}
      \newcommand{\cbox}[1]{\vphantom{\ensuremath{\displaystyle\Big(}}\makebox[2.4em][r]{#1}}%
      \begin{tabular}{|c|r|r|r|r|r|}
        \hline
        $a$ & \num{6} & \num{8} & \num{9} & \num{15} & \num{20} \\
        $b$ & \num{4} & \num{7} & \num{6} & \num{9} & \num{14} \\
        $h$ & \num{7} & \num{3} & \num{5} & \num{8} & \num{6} \\
        \hline
        $V$ & \cbox{\num{140}} & \cbox{\num{120}} & \cbox{\num{225}} & \cbox{\num{960}} & \cbox{\num{1360}} \\
        \hline
      \end{tabular}%
    \end{center}
  \fi
\end{exercise}

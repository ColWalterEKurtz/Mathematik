\begin{exercise}
      {ID-cf95ff86e4da344a1add414b8d2b1bb0f95fcbe4}
      {Rautendach}
  \ifproblem\problem\par
    Beim rheinischen Rautendach bestehen die Dachflächen eines Gebäudes mit
    quadratischer Grundfläche aus kongruenten Rauten.\par
    \begin{minipage}{0.35\textwidth}
      \centering
      \begin{tikzpicture}[scale=0.69]
        % Punktkoordinaten
        \coordinate (A) at ( -0.3750,  -0.1875);
        \coordinate (B) at ( -1.8750,  -0.9375);
        \coordinate (C) at (  2.1250,  -0.9375);
        \coordinate (D) at (  3.6250,  -0.1875);
        \coordinate (E) at ( -1.1250,   2.4375);
        \coordinate (F) at (  0.1250,   2.0625);
        \coordinate (G) at (  2.8750,   2.4375);
        \coordinate (H) at (  1.6250,   2.8125);
        \coordinate (I) at (  0.8750,   5.4375);
        \coordinate (J) at ( -0.3750,  -1.1875);
        \coordinate (K) at ( -1.8750,  -1.9375);
        \coordinate (L) at (  2.1250,  -1.9375);
        \coordinate (M) at (  3.6250,  -1.1875);
        \coordinate (N) at (  0.8750,  -0.5625);
        \coordinate (O) at (  2.8750,  -0.5625);
        % Hilfspunkte fuer die Beschriftung
        \coordinate (HPEA) at ($(E)!0.23!(F)$);
        \coordinate (HPEE) at ([shift={(210:1.5cm)}]HPEA);
        \coordinate (HPFA) at ($(B)!0.6955!(I)$);
        \coordinate (HPFE) at ([shift={(150:1.5cm)}]HPFA);
        \coordinate (HPGA) at ($(O)!0.4!(G)$);
        \coordinate (HPGE) at ([shift={(30:1.5cm)}]HPGA);
        \coordinate (HPHA) at ($(N)!0.4!(I)$);
        \coordinate (HPHE) at ([shift={(30:3.5cm)}]HPHA);
        \coordinate (HPSA) at ($(G)!0.5!(I)$);
        \coordinate (HPSE) at ([shift={(30:1.5cm)}]HPSA);
        % Flaechen
        \fill[fill=black!25!white] (B) -- (E) -- (I) -- (F) -- cycle;
        \fill[fill=black!25!white] (C) -- (G) -- (I) -- (F) -- cycle;
        \fill[fill=black!15!white] (C) -- (D) -- (G) -- cycle;
        \fill[fill=black!15!white] (B) -- (C) -- (F) -- cycle;
        \fill[fill=black!15!white] (B) -- (K) -- (L) -- (C) -- cycle;
        \fill[fill=black!15!white] (C) -- (L) -- (M) -- (D) -- cycle;
        % Kanten unten
        \draw[dashed] (A) -- (B);
        \draw (B) -- (C);
        \draw (C) -- (D);
        \draw[dashed] (D) -- (A);
        % Dreiecke
        \draw[dashed] (A) -- (E);
        \draw (B) -- (E);
        \draw (B) -- (F);
        \draw (C) -- (F);
        \draw (C) -- (G);
        \draw (D) -- (G);
        \draw[dashed] (D) -- (H);
        \draw[dashed] (A) -- (H);
        % Kanten zur Spitze
        \draw (E) -- (I);
        \draw (F) -- (I);
        \draw (G) -- (I);
        \draw[dashed] (H) -- (I);
        % Kanten zum Gebaeude
        \draw[dashed] (A) -- (J);
        \draw (B) -- (K);
        \draw (C) -- (L);
        \draw (D) -- (M);
        % Diagonalen
        \draw[line width=0.7pt, dotted] (A) -- (C);
        \draw[line width=0.7pt, dotted] (B) -- (D);
        \draw[line width=0.7pt, dotted] (E) -- (F);
        \draw[line width=0.7pt, dotted] (B) -- (I);
        % Hoehen
        \draw[line width=0.7pt, dotted] (I) -- (N);
        \draw[line width=0.7pt, dotted] (G) -- (O);
        % Pfeile
        \draw[<->, >=latex] ([yshift=2mm]K) -- node[below]       {{\small$a$}} ([yshift=2mm]L);
        \draw[<->, >=latex] ([yshift=2mm]L) -- node[below right] {{\small$a$}} ([yshift=2mm]M);
        % Beschriftung
        \draw[very thin] (HPEA) -- (HPEE) node[shift={(210:2mm)}] {{\small$e$}};
        \draw[very thin] (HPFA) -- (HPFE) node[shift={(150:2mm)}] {{\small$f$}};
        \draw[very thin] (HPGA) -- (HPGE) node[shift={(30:2mm)}]  {{\small$g$}};
        \draw[very thin] (HPHA) -- (HPHE) node[shift={(30:2mm)}]  {{\small$h$}};
        \draw[very thin] (HPSA) -- (HPSE) node[shift={(30:2mm)}]  {{\small$s$}};
      \end{tikzpicture}
    \end{minipage}%
    \hfill
    \begin{minipage}{0.60\textwidth}
      \begin{enumerate}[a)]
        \item Berechne in Abhängigkeit von $a$ und $g$ die Längen
              $s$, $e$ und $f$ und zeige, dass $h=2g$ gilt.
        \item Weise nach, dass für den Inhalt der Dachfläche die Gleichung
              $A=\sqrt{a^{2}+8g^{2}}$ gilt.
        \item Können die vier Rauten der Dachfläche Quadrate sein?
      \end{enumerate}
    \end{minipage}
  \fi
  %\ifoutline\outline\par
  %\fi
  %\ifoutcome\outcome\par
  %\fi
\end{exercise}
